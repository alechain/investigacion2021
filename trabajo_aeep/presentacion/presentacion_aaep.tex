\documentclass[11pt]{beamer}
\usetheme{Luebeck}
\usepackage[utf8]{inputenc}
\usepackage{amsmath}
\usepackage{amsfonts}
\usepackage[spanish]{babel}
\usepackage{amssymb}
\author{Alejandro Chain}
\title{Factores determinantes de la migración interregional en Argentina entre 2016 y 2019}
%\setbeamercovered{transparent} 
%\setbeamertemplate{navigation symbols}{} 
%\logo{} 
\institute{Asociación Argentina de Economía Política} 
%\date{} 
%\subject{} 
\begin{document}

\begin{frame}
\titlepage
\end{frame}

%\begin{frame}
%\tableofcontents
%\end{frame}

%%%%%%%%%%%%%%%%%%%%%%%%%%%%%%%%%%%%%%%%%%%%%%%%%%%%%%%%%%%%%%%%%%%%%%%%%%%%%%%%%%%%%%%%%%%%%%%%%%%%%%%%%%%%%%%

\begin{frame}[t]{Esquema de la presentación}
%\frametitle{¿Cúal es el objetivo de este investigación?}

\begin{enumerate}

\item ¿Cómo surge la pregunta central de la investigación?
\vspace{0.5 cm}
\item Regiones: concepto y caracterización social, económica y geográfica.
\vspace{0.5 cm}
\item Migrantes: concepto y análisis comparativo con respecto a los nativos.
 \vspace{0.5 cm}
 \item Estimación e interpretación de los factores socioeconómicos determinantes de la migración interregional.
 \vspace{0.5 cm}
 \item Conclusión.
\end{enumerate}


\end{frame}

%%%%%%%%%%%%%%%%%%%%%%%%%%%%%%%%%%%%%%%%%%%%%%%%%%%%%%%%%%%%%%%%%%%%%%%%%%%%%%%%%%%%%%%%%%%%%%%%%%%%%%%%%%%%%%%



%%%%%%%%%%%%%%%%%%%%%%%%%%%%%%%%%%%%%%%%%%%%%%%%%%%%%%%%%%%%%%%%%%%%%%%%%%%%%%%%%%%%%%%%%%%%%%%%%%%%%%%%%%%%%%%
\begin{frame}[t]{¿Cómo surge la pregunta central de la investigación?}
%\frametitle{¿Cúal es el objetivo de este investigación?}
\begin{itemize}
\item Pregunta recurrente en el entorno: `\textit{`¿A dónde te vas a ir cuando te recibas?''}
\vspace{0.5 cm}
\item Estrecha vinculación de fondo entre: 
        \begin{center}
         \textbf{ Migrar de la provincia $\rightleftarrows$ Mejor calidad de vida }
         \end{center}
 \vspace{0.5 cm}
 \item Disparador de pregunta más general: \textit{``¿Existen características socioeconómicas que causen que ciertas personas sean mas propensas a abandonar su provincia de origen para migrar hacia otra provincia? ''}
\end{itemize}

\end{frame}
%%%%%%%%%%%%%%%%%%%%%%%%%%%%%%%%%%%%%%%%%%%%%%%%%%%%%%%%%%%%%%%%%%%%%%%%%%%%%%%%%%%%%%%%%%%%%%%%%%%%%%%%%%%%%%%

\begin{frame}[t]{¿Cómo surge la pregunta central de la investigación?}
%\frametitle{¿Cúal es el objetivo de este investigación?}
\textbf{Limitaciones del análisis provincial:}
\begin{itemize}
\item Elevada combinación de flujos migratorios (24 localidades).
\vspace{0.5 cm}
\item Análisis pormenorizado tedioso y desorganizado.
 \vspace{0.5 cm}
 \item Disponibilidad y volumen de las bases de datos.
  \vspace{0.5 cm}
 \item Heterogeneidad de algunas provincias en torno a factores sociales, económicos y geográficos.
 
\end{itemize}
 \vspace{0.5 cm}

\textbf{Solución:}
 \begin{itemize}
\item Reformular la pregunta hacia un enfoque de \textcolor{blue}{migraciones interregionales}.
\end{itemize}


\end{frame}

%%%%%%%%%%%%%%%%%%%%%%%%%%%%%%%%%%%%%%%%%%%%%%%%%%%%%%%%%%%%%%%%%%%%%%%%%%%%%%%%%%%%%%%%%%%%%%%%%%%%%%%%%%%%%%%

%%%%%%%%%%%%%%%%%%%%%%%%%%%%%%%%%%%%%%%%%%%%%%%%%%%%%%%%%%%%%%%%%%%%%%%%%%%%%%%%%%%%%%%%%%%%%%%%%%%%%%%%%%%%%%%

\begin{frame}[t]{Regiones}
%\frametitle{¿Cúal es el objetivo de este investigación?}
\begin{block}{Concepto}
Zona territorial de un país que comparte determinadas características homogéneas.
\end{block}
\begin{center}
 División regional tradicional $\Longrightarrow$ \textcolor{red}{Sesgo geográfico}
\end{center}
\vspace{0.5 cm}
\textbf{¿Con que criterio dividimos regionalmente a las provincias?}
\begin{itemize}
\item Factores sociodemográficos.
\vspace{0.5 cm}
\item Fatores económicos.
 \vspace{0.5 cm}
 \item Factores geográficos.
 
\end{itemize}
 \vspace{0.5 cm}

\end{frame}

%%%%%%%%%%%%%%%%%%%%%%%%%%%%%%%%%%%%%%%%%%%%%%%%%%%%%%%%%%%%%%%%%%%%%%%%%%%%%%%%%%%%%%%%%%%%%%%%%%%%%%%%%%%%%%%

%%%%%%%%%%%%%%%%%%%%%%%%%%%%%%%%%%%%%%%%%%%%%%%%%%%%%%%%%%%%%%%%%%%%%%%%%%%%%%%%%%%%%%%%%%%%%%%%%%%%%%%%%%%%%%%

\begin{frame}[t]{Regiones}
%\frametitle{¿Cúal es el objetivo de este investigación?}

\textbf{Factores sociodemográficos para la división regional:}
\begin{itemize}
\item Exportaciones per cápita promedio.
\item Demanda de energía eléctrica en MwH per cápita.
 \item Tasa de actividad promedio.
 \item Cantidad de empresas C/100.000 habitantes.
 \item Remuneración real de los trabajadores registrados del sector privado.
 \item Porcentaje de empleados en los distintos sectores de actividad.
 \item Participación en el total de RON y TOP.
 
 
\end{itemize}

\end{frame}

%%%%%%%%%%%%%%%%%%%%%%%%%%%%%%%%%%%%%%%%%%%%%%%%%%%%%%%%%%%%%%%%%%%%%%%%%%%%%%%%%%%%%%%%%%%%%%%%%%%%%%%%%%%%%%%

%%%%%%%%%%%%%%%%%%%%%%%%%%%%%%%%%%%%%%%%%%%%%%%%%%%%%%%%%%%%%%%%%%%%%%%%%%%%%%%%%%%%%%%%%%%%%%%%%%%%%%%%%%%%%%%

\begin{frame}[t]{Regiones}
%\frametitle{¿Cúal es el objetivo de este investigación?}

\textbf{Factores económicos para la división regional:}
\begin{itemize}
\item Exportaciones per cápita promedio.
\item Demanda de energía eléctrica en MwH per cápita.
 \item Tasa de actividad promedio.
 \item Cantidad de empresas C/100.000 habitantes.
 \item Remuneración real de los trabajadores registrados del sector privado.
 \item Porcentaje de empleados en los distintos sectores de actividad.
 \item Participación en el total de RON y TOP.
 
 
\end{itemize}

\end{frame}

%%%%%%%%%%%%%%%%%%%%%%%%%%%%%%%%%%%%%%%%%%%%%%%%%%%%%%%%%%%%%%%%%%%%%%%%%%%%%%%%%%%%%%%%%%%%%%%%%%%%%%%%%%%%%%%









\begin{frame}[t]{¿Cúal es el objetivo de este investigación?}
%\frametitle{¿Cúal es el objetivo de este investigación?}

\center \textit{ {\large ``Identificar los principales determinantes socioeconómicos de la migración interregional en Argentina.''}}
\vspace{1 cm}
\begin{itemize}
\pause
\item \textbf {Migrantes}
\pause
\item \textbf {Regiones}
\pause
\item \textbf {Factores determinantes}
\end{itemize}
\end{frame}



\begin{frame}[t]{Migrantes}
\begin{block}{Concepto}
Toda persona que hace cinco años vivia en una provincia distinta de aquella en la que nació.
\end{block}
\vspace{1 cm}

\pause
\textbf {¿Desde donde?}
\begin{itemize}
\item Localidad de nacimiento.
\end{itemize}

\pause

\textbf {¿Hacia donde?}
\begin{itemize}
\item Migrantes internos.
\end{itemize}
\pause
\textbf {¿Hace cuanto?}
\begin{itemize}
\item Últimos 5 años.
\end{itemize}

\end{frame}


\end{document}