\documentclass[12pt,a4paper]{article}
\usepackage[utf8]{inputenc}
\usepackage[spanish]{babel}
\usepackage{graphicx}
\usepackage{amsmath} 
\usepackage[left=2cm,right=2cm,top=2cm,bottom=2cm]{geometry}
\usepackage{tikz}
\usepackage[babel]{csquotes}
\usepackage[backend=biber,style=apa]{biblatex}
%\DefineBibliographyStrings{english}
\DeclareLanguageMapping{spanish}{spanish-apa}
\addbibresource{Bibliography.bib}
\renewcommand{\baselinestretch}{1.2}
\usepackage{caption}
\newcommand{\source}[1]{\caption*{\scriptsize{Fuente: {#1}}} }
\setlength{\parskip}{4mm}
%\setcounter{secnumdepth}{0}
\author{Alejandro Chaín}
\title{Borrador determinantes de migraciones interregionales en Argentina}
\begin{document}
\begin{titlepage}
\centering
\vspace{3cm}
{\bfseries\LARGE Universidad Nacional del Nordeste\par}
\vspace{1cm}
{\includegraphics[width=0.3\textwidth]{logounne.png}\par}
{\scshape\Large Facultad de Ciencias Econ\'omicas\par}
\vspace{2cm}
{\scshape\Huge Borrador del proyecto investigación \par}
\vspace{2cm}
{\itshape\Large ``Identificación de los factores socioeconómicos determinantes en la migración
interregional en Argentina entre 2016 y 2019'' \par}
\vfill
{\Large Autor: \par}
{\Large Alejandro Chaín\par}
\vfill
{\Large 4 de Junio del 2021 \par}
\end{titlepage}

%\maketitle
\newpage
\tableofcontents
\newpage
\section{Introducción}
A lo largo de la historia en Argentina se evidenció un éxodo de habitantes de provincias o
regiones menos desarrolladas hacia centros urbanos con mayores niveles de prosperidad económica y social, empero de la relevancia de estos movimientos internos, el estudio de la literatura vigente indica una carencia en la propuesta de un estudio de identificación empírica de los determinantes socioeconómicos de dichos flujos migratorios al interior de la Argentina.

Esta situación provoca un desconocimiento a la hora de analizar la ponderación de los factores que impulsan a los individuos a emigrar hacia determinadas zonas, y, por ende, de poder comprender cuales son los factores mayor peso a la hora de considerar las heterogeneidades en el desarrollo regional.

Es de suma importancia identificar cuáles son los factores que juegan un rol predominante a la hora de incentivar a las personas a migrar desde su localidad de origen hacia otra localidad con un perfil de desarrollo totalmente distinto,  considerando que el crecimiento y la calidad de vida de un espacio geográfico determinado impacta en las decisiones de emigración de individuos que comparten ciertas características socioeconómicas homogéneas.

\section{Marco teórico y antecedentes}
A lo largo de los años han surgido multiples teorías y acercamientos empíricos que buscaron explicar los procesos migratorios. Son multiples y muy disímiles los enfoques que se han utilizado al rededor del mundo, aqui se intentará resumir las principales obras que sirvieron de base para la elaboración de este trabajo.

Uno de los primeros artículos que abordó los determinantes de la migración surge del cartógrafo y geógrafo alemán \textcite{ravenstein_laws_1885}. Este esbozó las “Leyes de la Migración”, a través de un estudio realizado en base a un análisis cuantitativo de un censo poblacional. Estas leyes se pueden resumir a través de los siguientes puntos: (1) A mayor distancia entre el lugar de origen y de destino, menor será el flujo migratiorio; (2) En general los desplazamientos migratorios de larga distancia tienen como destino a una ciudad grande; (3) la migración es un proceso por etapas, en primer lugar los migrantes se dirigen a ciudades cercanas, y posteriormente migran hacia las grandes ciudades; (4) es más frecuente la migración en ambientes rurales que en ambientes urbanos; (5) el principal motivo de migración es la esperanza de obtener un mayor ingreso y un mejor nivel de vida; (6) el nivel de desarrollo tecnológico y económico tiene un impacto positivo en la atracción de los migrantes. Estos factores que se cumplían en  los individuos y los procesos migratorios de esa época, salvando la distancia temporal y el avance en los métodos de estimación actuales, marcan un punto de partida para esta investigación.

En cuanto a los trabajos que exploran los determinantes de las migraciones desde un enfoque individual, se encuentra algunos como el de \textcite{sjaastad_costs_1962} y el de \textcite{todaro_model_1969}. En \textit{''Returns and cost of migration``}\parencite{sjaastad_costs_1962} se considera al proceso migratorio como una inversión en capital humano, en donde el individuo decidirá realizar el éxodo si el valor actual de la inversión (migrar) desde el area de origen al area de destino es positivo.  Siguiendo esta linea, un individuo calcula el valor actual descontado del flujo de ganancias de por vida esperado en su región de origen y en la región a la que proyecta migrar, y tomará la decisión del éxodo solamente si el retorno neto de los ``costos de migrar'' son mayores en la localidad de destino que en su localidad de origen \parencite{zaiceva_impact_2014}. De acuerdo con esta teoría, mientras más joven es el migrante, mas largo es el horizonte de vida proyectado y por ende sus ganancias esperadas, y mientras más viejo sea el migrante, no solo este horizonte esperado en el que podra realizar sus ganancias es menor, sino que también seran mayores  los ``costos de migrar''. Algunos de estos son los costos psicológicos de separarse de familia y amigos, el mayor capital social que deberán dejar atras, la dificultad de integrarse en una nueva cultura, entre otros.

El abordaje en \textit{''A Model of Labor Migration and Urban Unemployment in Less Developed Countries``} \parencite{todaro_model_1969} parte del enfoque clásico de determinantes de migraciones, que se basa en los diferenciales de ingresos y una absorción inmediata por parte del mercado laboral de destino, pero la modifica agregando el diferencial potencial de ingresos y la probabilidad de ser absorbido por el mercado laboral de la localida de destino. 

Ampliando la visión individualista de las estrategias de migración, existen estudios de estrategias familiares de la migración como el del economista \textcite{mincer_family_1978}. La situación familiar y relacional de las personas afecta directamente las ganancias y los costos de migrar, la teoría indica que las ganancias provenientes de la migración se incrementan en menor proporción que los costos cuando existe una familia vinculada a la decisión del éxodo. 

El trabajo de \textcite{kuhnt_literature_2019} aplica un marco teórico para el estudio de las migraciones en el cual divide los determinantes de la migración en tres niveles de especificidad. Este abordaje de macrodeterminantes, mesodeterminantes y microdeterminantes explica como la generalidad de los determinantes impactan de distinta manera en el proceso migratorio. Los macrodeterminantes son aquellos que son comunes a toda la población de una determinada comunidad, estan conformados por aspectos políticos, sociales, económicos y demográficos. Los mesodeterminantes estan conformados por factores locales, tales como las redes migratorias, aspectos culturales de los migrantes o la tecnología. Por último, los microdeterminantes son características individuales o delo hogar que actúan como “mediadores” en la decisión migratoria, como ser el género, la edad, el nivel educativo, el estatus social, la aversión al riesgo, entre otros. 

En el ámbito de la nueva geografía económica se enmarca el trabajo de \textcite{krugman_increasing_1991}, este  propone que la movilidad de las personas (factor trabajo) convergen a la aglomeración de la producción y el consumo en unas pocas regiones con dinámicas productivas similares. Esta teoría de aglomeración espacial surge de que  zonas altamente pobladas generan una mayor atracción de los migrantes a causa de una mayor diversidad de bienes y de salarios reales mas elevados. Sumado a esto, las empresas se ven altamente beneficiadas debido al amplio mercado local que se genera, lo que permite que incurran en menores costos de transporte, al mismo tiempo que se benefician de las economías de escala y de los encadenamientos hacia adelante y hacia atrás que son producidos por la concentración industrial. El proceso descripto anteriormente conlleva a que se generen procesos migratorios hacia áreas determinadas que comparten características en términos de estructuras productivas, funcionamiento del mercado laboral, distribución del factor trabajo entre sectores de actividad económica, costos de transporte, salario real, y por ende, otros determinantes sociales y culturales que son consecuencia de los efectos de la aglomeración espacial.

Uno de los principales factores económicos de la migración es la busqueda de un estandar de vida más elevado del que se podría costear en la localidad de origen \parencite{simpson_demographic_2017}, este factor económico de expulsión  puede verse reflejado en los ingresos que percibe una persona, laborales y no laborales(subsidios y transferencias), como en la posibilidad de acumulación de activos patrimoniales, como por ejemplo el acceso a la vivienda propia. 

El impacto de la distribución del ingreso y los salarios en la autoselección de los migrantes fue estudiado por el economista estadounidense \textcite{borjas_self-selection_1987}. Este plantea que en los países menos desarrollados, en donde la dispersión de los salarios y los retornos a la educación tienden a ser relativamente altos, existirá una ''selección negativa`` en la expulsión de migrantes, en donde solo migrarán las personas menos calificadas. situación contraría se dará en los paises desarrollados, los cuales expulsarán migrantes con un nivel de calificación más elevado, esto se conoce como una ''selección positiva``de migrantes. El trabajo de \textcite{stark_migration_1991} explica que en ciertos casos los mercados laborales de las localidades receptoras de migrantes solamente están abiertas a personas con bajo nivel de calificación. Esto provoca que las personas más calificadas vean como poco atractiva la idea de realizar un éxodo migratorio.

La educación de las personas es un componente del capital humano que juega un  papel importante en la decisión migratoria, sobre todo considerando la posibilidad de transferencia de este capital humano a otras localidades en donde pueda conseguir mayores retornos. Es de esperar que mientras más factible sea esta transferencia, mayores serán los incentivos a migrar. Esta relación no está exenta de ambigüedades, debido a que en la localidad de origen también puede enfrentar mejores retornos o mayores facilidades para conseguir un empleo a causa de poseer niveles educativos más elevados, actuando como un incentivo a no realizar el éxodo migratorio \parencite{danzer_economic_2008}.

En el análisis del impacto de las migraciones en los mercados laborales, el crecimiento económico y en las finanzas públicas se encuentra el antecedente del análisis empírico por parte del área de estudio de migraciones de la OCDE \textcite{dumont_is_2014}. Ampliando en este mismo espectro \textcite{liebig_fiscal_2013} realizo una investigación para los paises de la \textit{``OCDE''} en donde se llegó a la conclusión de que en aquellos en donde la cuota de migrantes de edad avanzada no es tan elevada, estos contribuyen en mayor medida en impuestos y contribuciones sociales de lo que reciben en forma de beneficios individuales por parte del estado. Esto indica una ganancia neta de recursos ficales por parte de las localidades receptoras de migrantes.

Las dimensiones del género en la migración son tratadas en la investigación hecha por el Profesor \textcite{carling_gender_2005}. Uno de los fenómenos que resalta en este trabajo es el concepto de la "feminización de la migración", el cual hace encuentra su fundamento en la particiación más activa que tomó la mujer en el mercado laboral en las últimas décadas. Esta facilidad en el acceso al mercado laboral posibilito la opción de migrar para formar parte de mercados laborales en donde la valoración de su fuerza de trabajo sea mayor.



\section{Regiones}

Una región es una zona territorial de un país que comparten determinadas características homogéneas. Las características elegidas para realizar la división van a determinar directamente que provincias integran cada una de las regiones.

Cuando se estudian a los determinantes de las migraciones interprovinciales se pueden encontrar distintos niveles de factores según la generalidad de su impacto. Por un lado se encuentran factores que son comunes a toda la población de una determinada comunidad, que son conocidos como macrofactores y estan conformados por aspectos políticos, sociales, económicos y demográficos. Por el otro se puden encontrar los microfactores, estos son características individuales que actúan como “mediadores” en la decisión migratoria e impactan de manera distinta dependiendo de los macrofactores que configuran los potenciales destinos u orígenes de migración.

El proceso de decisión migratoria consta de una interrelación constante entre los macrofactores y microfactores, que deben ser considerados a la hora de definir los determinantes que llevan a la decisión del éxodo interprovincial.

Existen ciertas limitaciones para realizar un análisis pormenorizado de las migraciones interprovinciales en la Argentina debido a la elevada cantidad de provincias de origen y destino que surgen de la combinación de los movimientos migratorios. Una de las principales limitaciones es la dificultad de encontrar determinantes aislados para cada una de las provincias, tanto por la disponibilidad y volumen de los datos, como por el hecho de que se volvería tedioso y desordenado el análisis pormenorizado de ciertas combinaciones migratorias.

Este problema trae ligado el cuestionante sobre cuales son los  determinantes óptimos compartidos por las  provincias  que permiten configurar un macro-entorno en donde se pueda  observar dinámicas migratorias similares. Una primera respuesta podría ser la  división regional del país propuesta por el Instituto Nacional de Estadísticas y Censos (INDEC), pero esta misma peca de ponderar con mayor peso a cuestiones geográficas, y no brinda tanta importancia a las consideraciones desde el punto de vista socio-económico.
 
Tomando como base la teoría de la aglomeración espacial de la nueva geografía económica esta teoría, la problemática del análisis pormenorizado de la migraciones interprovinciales puede ser morigerado a través de una aglomeración de las provincias en regiones, considerando determinados macrofactores socio-económicos que serán definidos posteriormente en este trabajo. 

\subsection{Macrofactores}
Los macrofactores son aquellos determinantes que impactan de igual manera a toda la población que habite el territorio en cuestión.

Para la selección de los macrofactores de aglomeración de las provincias se optó por clasificarlos en aspectos sociodemográficos, económicos y geográficos. Para los primeros dos se tomaron como base las características seleccionadas en  el trabajo de \textcite{cicowiez_caracterizacion_2003} y se agregaron algunas otras en pos de reflejar las variables compartidas por las regiones según las teorías de aglomeración productiva de la nueva geografía económica. En representación de los macrofactores geográficos se optó por incorporar una división regional del Instituto Nacional de Estadísticas y Censos, con el fin de captar similitudes geográficas y territoriales.


A continuación se brinda una breve descripción de los macrofactores seleccionados:

\textbf{Factores sociodemográficos de macrolocalización:}
\begin{itemize}
%\item Cantidad de Habitantes: Esta variable indica la cantidad de habitantes de cada provincia, tiene como fuente el Censo Nacional de Población y Vivienda del INDEC realizado en el año 2010, e intenta representar el tamaño de las distintas provincias en términos poblacionales.

\item Tasa de promoción efectiva secundaria: En esta variable se busca encontrar una aproximación a la calidad y efectividad del aparato educativo en cada provincia. No solo es importante el acceso a la educación media en términos de tasa de escolarización neta secundaria, sino también la posibilidad de sortear los obstáculos para lograr completar el nivel secundario. La fuente de estos datos es el Ministerio de Educación, Cultura, Ciencia y Tecnología de la Nación.

\item Mortalidad infantil: Se calcula como la mortalidad infantil por cada 1.000 nacidos vivos según la provincia de residencia de la madre. Estos datos tienen como fuente la secretaría de acceso a la salud del Ministerio de Salud de la Nación Argentina. Lo que se intenta mostrar son las diferencias y similitudes en el acceso a la salud y a las posibilidades del desarrollo de un proyecto familiar.

\item Homicidios, robos, robos agravados, violaciones y muertes en accidentes de tránsito cada 100.000 habitantes: estas variables tienen como fuente el Sistema Nacional de Información Criminal del ministerio de seguridad de la nación. La utilidad buscada en su inclusión es la de aproximar el nivel de seguridad social y jurídica de cada una de las provincias.
\end{itemize}

\textbf{Factores económicos de macrolocalización:}
\begin{itemize}
\item Exportaciones per cápita promedio: Este indicador obtenido del Origen Provincial de las Exportaciones Argentinas (OPEX) INDEC permite exponer el nivel de integración al comercio internacional de las provincias y su capacidad de producir bienes transables, con el efecto que tiene en la definición de los salarios reales en los mercados laborales provinciales.

\item Demanda de energía eléctrica en MwH per-cápita: Este indicador obtenido de CAMMESA permite aproximar el nivel de actividad de cada provincia, siendo que una gran parte de este consumo se da por parte de industrias y comercios.

\item Tasa de actividad promedio: Este indicador, obtenido del INDEC, permite vislumbrar la fuerza de trabajo que existe en cada provincia, siendo que indica el porcentaje de la población total que forma parte de la población económicamente activa.

\item Cantidad de empresas cada 100.000 habitantes: Esta variable, obtenida del GPS de empresas del ministerio de Desarrollo productivo, permite conocer el nivel de concentración de firmas que existe en cada una de las provincias Esta concentración afecta directamente al nivel de demanda de trabajo y produce efectos en términos de encadenamientos productivos que pueden beneficiar a las economías de las provincias.

\item Remuneración real de los trabajadores registrados del sector privado: Este dato, que tiene como fuente al Observatorio de Empleo y Dinámicas Empresariales del Ministerio de Trabajo, Empleo y Seguridad Social, indica el valor relativo que tiene el trabajo de las personas que trabajan en las distintas provincias.

\item Porcentaje de empleados en los distintos sectores de actividad: esta variable que tiene como fuente al Observatorio de Empleo y Dinámicas Empresariales del Ministerio de Trabajo, Empleo y Seguridad Social demuestra el peso de los distintos sectores de actividad como generadores de empleo en las provincias. Esto demarca las diferencias en las estructuras productivas que existen entre ellas, lo que significa que en algunas puede existir un mayor porcentaje de empleo proveniente de la Agricultura, ganadería, pesca y actividades extractivas (sector primario extractivo), en otras del sector de Comercio, servicios, electricidad, gas, agua y construcción (sector terciario de servicios) y en otras del sector industrial (sector secundario).

\item Porcentaje en el total de Recursos Tributarios de Origen Nacional (RON) y Provincial (TOP): este indicador que tiene como fuente la Dirección de Asuntos Provinciales dependiente de la Secretaría de Hacienda en la órbita del Ministerio de Economía de la Nación. La importancia de incluir esta variable radica en las diferencias en los patrones de gastos de los recursos públicos, por parte de los estados provinciales, dependiendo de si son obtenidos por recaudación propia o ajena.
\end{itemize}

\textbf{Factores geográficos de macrolocalización:}
\begin{itemize}
\item Región  Noroeste: Esta región está compuesta por las provincias de Jujuy, Catamarca, La Rioja, Salta, Santiago del Estero y Tucumán.
\item Región  Noreste: Esta región está conformada por las provincias de Chacho, Corrientes, Formosa y Misiones.
\item Región Cuyo: Región compuesta por las provincias de Mendoza, San Luis y San Juan.
\item Región Centro y Buenos Aires: Esta región esta conformada por las provincias de Buenos Aires, Córdoba, Entre Rios, Santa Fe y la Ciudad Autónoma de Buenos Aires.
\item Región Patagonia: Esta región esta compuesta por las provincias de Chubut, La Pampa, Neuquén, Río Negro, Santa Cruz y Tierra del Fuego.
\end{itemize}



\subsection{Aglomeración}

Para comenzar con la división en regiones de las provincias argentinas, es necesario definir el nivel óptimo de aglomeración de acuerdo a los macrofactores elegidos, es decir, definir el número de regiones en que serán divididas las provincias argentinas.

Para esta aglomeración se utilizará el algoritmo de K-medias \parencite{macqueen_methods_1967}, particularmente el desarrollado por \textcite{hartigan_k-means_1979} que utiliza la distancia euclidiana para la medición de las discrepancias entre los objetos y los grupos. El indicador a utilizar para definir el numero de regiones óptimas será el \textit{Within-cluster sum of square (WSS)}.

\vspace{5mm}
\begin {center}
\begin{equation}\label{eq:wss}
WSS=\sum_{i=1}^{N_{C}} \sum_{\textbf{x}\in Ci} distancia(\textbf{x},\bar{\textbf{x}}_{\tiny{\textit{Ci}}})^{2}
\end{equation}
\end {center}
\vspace{5mm}


El WSS (\ref{eq:wss}) es un indicador que mide la suma de las distancias al cuadrado entre las variables dentro de  los cluster (regiones) y sus centroides($\bar{\textbf{x}}_{\tiny{\textit{Ci}}}$). El objetivo es minimizar la discrepancia dentro de cada grupo para distintos números de clusters $k$, teniendo en cuenta el \textit{trade-off} que implica cuando éste disminuye. 

Este indicador encuentra su mínimo en el caso en que el número de clusters (regiones) es igual al número de objetos (provincias) a clusterizar, en donde si tenemos $j$ objetos, el numero de clusters $k$ sería tal que $j=k$, y no se estaría dando ninguna información relevante a los efectos de poder resumir caracteristicas comunes entre los grupos. 

Este \textit{trade-off} provoca que se tenga que tomar una decisión en el numero de clusters óptimos (regiones óptimas), tal que las provincias dentro de cada cluster (región) sean lo más similares posibles en términos de los factores de macrolocalización, pero teniendo en cuenta que una división muy extensa en clusters (regiones) pierde utilidad analítica a los efectos de definir zonas con caracterísiticas similares para un posterior análisis de movilidad regional.

\begin{figure}[h!]
\begin{center}
	\caption{\\Número de regiones óptimas}
 	% Created by tikzDevice version 0.12.3.1 on 2021-08-12 10:20:36
% !TEX encoding = UTF-8 Unicode
\begin{tikzpicture}[x=1pt,y=1pt]
\definecolor{fillColor}{RGB}{255,255,255}
\path[use as bounding box,fill=fillColor,fill opacity=0.00] (0,0) rectangle (433.62,216.81);
\begin{scope}
\path[clip] (  0.00,  0.00) rectangle (433.62,216.81);
\definecolor{drawColor}{RGB}{255,255,255}
\definecolor{fillColor}{RGB}{255,255,255}

\path[draw=drawColor,line width= 0.6pt,line join=round,line cap=round,fill=fillColor] (  0.00,  0.00) rectangle (433.62,216.81);
\end{scope}
\begin{scope}
\path[clip] ( 36.13, 30.70) rectangle (428.12,194.13);
\definecolor{drawColor}{RGB}{70,130,180}

\path[draw=drawColor,line width= 0.6pt,line join=round] ( 59.18,186.71) --
	( 97.62,149.29) --
	(136.05,124.60) --
	(174.48,101.43) --
	(212.91, 86.88) --
	(251.34, 73.72) --
	(289.77, 63.86) --
	(328.20, 59.54) --
	(366.63, 40.71) --
	(405.06, 38.13);
\definecolor{fillColor}{RGB}{70,130,180}

\path[draw=drawColor,line width= 0.4pt,line join=round,line cap=round,fill=fillColor] ( 59.18,186.71) circle (  2.18);

\path[draw=drawColor,line width= 0.4pt,line join=round,line cap=round,fill=fillColor] ( 97.62,149.29) circle (  2.18);

\path[draw=drawColor,line width= 0.4pt,line join=round,line cap=round,fill=fillColor] (136.05,124.60) circle (  2.18);

\path[draw=drawColor,line width= 0.4pt,line join=round,line cap=round,fill=fillColor] (174.48,101.43) circle (  2.18);

\path[draw=drawColor,line width= 0.4pt,line join=round,line cap=round,fill=fillColor] (212.91, 86.88) circle (  2.18);

\path[draw=drawColor,line width= 0.4pt,line join=round,line cap=round,fill=fillColor] (251.34, 73.72) circle (  2.18);

\path[draw=drawColor,line width= 0.4pt,line join=round,line cap=round,fill=fillColor] (289.77, 63.86) circle (  2.18);

\path[draw=drawColor,line width= 0.4pt,line join=round,line cap=round,fill=fillColor] (328.20, 59.54) circle (  2.18);

\path[draw=drawColor,line width= 0.4pt,line join=round,line cap=round,fill=fillColor] (366.63, 40.71) circle (  2.18);

\path[draw=drawColor,line width= 0.4pt,line join=round,line cap=round,fill=fillColor] (405.06, 38.13) circle (  2.18);
\definecolor{drawColor}{RGB}{255,0,0}

\path[draw=drawColor,line width= 0.6pt,dash pattern=on 1pt off 3pt ,line join=round] (136.05, 30.70) -- (136.05,194.13);
\end{scope}
\begin{scope}
\path[clip] (  0.00,  0.00) rectangle (433.62,216.81);
\definecolor{drawColor}{RGB}{0,0,0}

\path[draw=drawColor,line width= 0.6pt,line join=round] ( 36.13, 30.70) --
	( 36.13,194.13);
\end{scope}
\begin{scope}
\path[clip] (  0.00,  0.00) rectangle (433.62,216.81);
\definecolor{drawColor}{RGB}{0,0,0}

\node[text=drawColor,anchor=base east,inner sep=0pt, outer sep=0pt, scale=  0.88] at ( 31.18, 53.13) {200};

\node[text=drawColor,anchor=base east,inner sep=0pt, outer sep=0pt, scale=  0.88] at ( 31.18, 90.21) {300};

\node[text=drawColor,anchor=base east,inner sep=0pt, outer sep=0pt, scale=  0.88] at ( 31.18,127.30) {400};

\node[text=drawColor,anchor=base east,inner sep=0pt, outer sep=0pt, scale=  0.88] at ( 31.18,164.39) {500};
\end{scope}
\begin{scope}
\path[clip] (  0.00,  0.00) rectangle (433.62,216.81);
\definecolor{drawColor}{gray}{0.20}

\path[draw=drawColor,line width= 0.6pt,line join=round] ( 33.38, 56.16) --
	( 36.13, 56.16);

\path[draw=drawColor,line width= 0.6pt,line join=round] ( 33.38, 93.24) --
	( 36.13, 93.24);

\path[draw=drawColor,line width= 0.6pt,line join=round] ( 33.38,130.33) --
	( 36.13,130.33);

\path[draw=drawColor,line width= 0.6pt,line join=round] ( 33.38,167.42) --
	( 36.13,167.42);
\end{scope}
\begin{scope}
\path[clip] (  0.00,  0.00) rectangle (433.62,216.81);
\definecolor{drawColor}{RGB}{0,0,0}

\path[draw=drawColor,line width= 0.6pt,line join=round] ( 36.13, 30.70) --
	(428.12, 30.70);
\end{scope}
\begin{scope}
\path[clip] (  0.00,  0.00) rectangle (433.62,216.81);
\definecolor{drawColor}{gray}{0.20}

\path[draw=drawColor,line width= 0.6pt,line join=round] ( 59.18, 27.95) --
	( 59.18, 30.70);

\path[draw=drawColor,line width= 0.6pt,line join=round] ( 97.62, 27.95) --
	( 97.62, 30.70);

\path[draw=drawColor,line width= 0.6pt,line join=round] (136.05, 27.95) --
	(136.05, 30.70);

\path[draw=drawColor,line width= 0.6pt,line join=round] (174.48, 27.95) --
	(174.48, 30.70);

\path[draw=drawColor,line width= 0.6pt,line join=round] (212.91, 27.95) --
	(212.91, 30.70);

\path[draw=drawColor,line width= 0.6pt,line join=round] (251.34, 27.95) --
	(251.34, 30.70);

\path[draw=drawColor,line width= 0.6pt,line join=round] (289.77, 27.95) --
	(289.77, 30.70);

\path[draw=drawColor,line width= 0.6pt,line join=round] (328.20, 27.95) --
	(328.20, 30.70);

\path[draw=drawColor,line width= 0.6pt,line join=round] (366.63, 27.95) --
	(366.63, 30.70);

\path[draw=drawColor,line width= 0.6pt,line join=round] (405.06, 27.95) --
	(405.06, 30.70);
\end{scope}
\begin{scope}
\path[clip] (  0.00,  0.00) rectangle (433.62,216.81);
\definecolor{drawColor}{RGB}{0,0,0}

\node[text=drawColor,anchor=base,inner sep=0pt, outer sep=0pt, scale=  0.88] at ( 59.18, 19.69) {1};

\node[text=drawColor,anchor=base,inner sep=0pt, outer sep=0pt, scale=  0.88] at ( 97.62, 19.69) {2};

\node[text=drawColor,anchor=base,inner sep=0pt, outer sep=0pt, scale=  0.88] at (136.05, 19.69) {3};

\node[text=drawColor,anchor=base,inner sep=0pt, outer sep=0pt, scale=  0.88] at (174.48, 19.69) {4};

\node[text=drawColor,anchor=base,inner sep=0pt, outer sep=0pt, scale=  0.88] at (212.91, 19.69) {5};

\node[text=drawColor,anchor=base,inner sep=0pt, outer sep=0pt, scale=  0.88] at (251.34, 19.69) {6};

\node[text=drawColor,anchor=base,inner sep=0pt, outer sep=0pt, scale=  0.88] at (289.77, 19.69) {7};

\node[text=drawColor,anchor=base,inner sep=0pt, outer sep=0pt, scale=  0.88] at (328.20, 19.69) {8};

\node[text=drawColor,anchor=base,inner sep=0pt, outer sep=0pt, scale=  0.88] at (366.63, 19.69) {9};

\node[text=drawColor,anchor=base,inner sep=0pt, outer sep=0pt, scale=  0.88] at (405.06, 19.69) {10};
\end{scope}
\begin{scope}
\path[clip] (  0.00,  0.00) rectangle (433.62,216.81);
\definecolor{drawColor}{RGB}{0,0,0}

\node[text=drawColor,anchor=base,inner sep=0pt, outer sep=0pt, scale=  0.7] at (232.12,  7.64) {\bfseries Número de cluster};
\end{scope}
\begin{scope}
\path[clip] (  0.00,  0.00) rectangle (433.62,216.81);
\definecolor{drawColor}{RGB}{0,0,0}

\node[text=drawColor,rotate= 90.00,anchor=base,inner sep=0pt, outer sep=0pt, scale=  0.7] at ( 13.09,112.42) {\bfseries WSS};
\end{scope}
\end{tikzpicture}

	\label{figure:optimas}
\end{center}
\end{figure}
\newpage

Como se puede ver en la Figura \ref{figure:optimas}, se obtiene que luego de realizar el cálculo del \textit{WSS} el numero óptimo de regiones se encuentra en torno a \textbf{tres}. Esta conclusión es debido a que la contribución marginal de aumentar el número de  regiones a cuatro no aportaría una reducción muy elevada al \textbf{WSS} y se seguiría perdiendo generalidad en la regionalización de las provincias sin una ganancia significativa de similitud dentro de las regiones.

Una vez obtenido el número de regiones óptimas se procede a asignar a las provincias a las regiones que pertenecen, dependiendo de la similitud que poseen con respecto a los macrofactores definidos anteriormente.

Para la división regional se utilizará el algoritmo de K-Medias, en donde se definen aleatoriamente los centroides iniciales del algoritmo y se procede a realizar el cálculo de los clusters a través de 1.000 iteraciones, siendo seleccionada aquella  que arroje un menor numero del indicador de la suma de las distancias al cuadrado entre las variables dentro de  los cluster y sus centroides, tambien conocido como \textit{WSS}.

Una vez ejecutado el algoritmo, se obtiene la division de las provincias de Argentina en tres regiones bien definidas (Figura \ref{figure:reg_resultantes}), las cuales comparten similitudes en los parámetros definidos en la macrolocalización.


\begin{figure}[ht!]
\begin{center}
\caption{\\Regiones resultantes de la aglomeración}
% Created by tikzDevice version 0.12.3.1 on 2021-06-29 20:57:30
% !TEX encoding = UTF-8 Unicode
´\begin{tikzpicture}[x=1pt,y=1pt]
\definecolor{fillColor}{RGB}{255,255,255}
\path[use as bounding box,fill=fillColor,fill opacity=0.00] (0,0) rectangle (433.62,289.08);
\begin{scope}
\path[clip] (  0.00,  0.00) rectangle (433.62,289.08);
\definecolor{drawColor}{RGB}{255,255,255}
\definecolor{fillColor}{RGB}{255,255,255}

\path[draw=drawColor,line width= 0.6pt,line join=round,line cap=round,fill=fillColor] (  0.00, -0.00) rectangle (433.62,289.08);
\end{scope}
\begin{scope}
\path[clip] ( 25.68, 58.49) rectangle (428.12,266.40);
\definecolor{drawColor}{RGB}{255,255,255}

\path[draw=drawColor,line width= 0.3pt,line join=round] ( 25.68, 98.33) --
	(428.12, 98.33);

\path[draw=drawColor,line width= 0.3pt,line join=round] ( 25.68,144.59) --
	(428.12,144.59);

\path[draw=drawColor,line width= 0.3pt,line join=round] ( 25.68,190.85) --
	(428.12,190.85);

\path[draw=drawColor,line width= 0.3pt,line join=round] ( 25.68,237.12) --
	(428.12,237.12);

\path[draw=drawColor,line width= 0.3pt,line join=round] (102.34, 58.49) --
	(102.34,266.40);

\path[draw=drawColor,line width= 0.3pt,line join=round] (200.75, 58.49) --
	(200.75,266.40);

\path[draw=drawColor,line width= 0.3pt,line join=round] (299.15, 58.49) --
	(299.15,266.40);

\path[draw=drawColor,line width= 0.3pt,line join=round] (397.55, 58.49) --
	(397.55,266.40);

\path[draw=drawColor,line width= 0.6pt,line join=round] ( 25.68, 75.19) --
	(428.12, 75.19);

\path[draw=drawColor,line width= 0.6pt,line join=round] ( 25.68,121.46) --
	(428.12,121.46);

\path[draw=drawColor,line width= 0.6pt,line join=round] ( 25.68,167.72) --
	(428.12,167.72);

\path[draw=drawColor,line width= 0.6pt,line join=round] ( 25.68,213.98) --
	(428.12,213.98);

\path[draw=drawColor,line width= 0.6pt,line join=round] ( 25.68,260.25) --
	(428.12,260.25);

\path[draw=drawColor,line width= 0.6pt,line join=round] ( 53.14, 58.49) --
	( 53.14,266.40);

\path[draw=drawColor,line width= 0.6pt,line join=round] (151.54, 58.49) --
	(151.54,266.40);

\path[draw=drawColor,line width= 0.6pt,line join=round] (249.95, 58.49) --
	(249.95,266.40);

\path[draw=drawColor,line width= 0.6pt,line join=round] (348.35, 58.49) --
	(348.35,266.40);
\definecolor{fillColor}{RGB}{77,175,74}

\path[fill=fillColor] (407.86,180.09) --
	(411.79,180.09) --
	(411.79,184.01) --
	(407.86,184.01) --
	cycle;

\path[fill=fillColor] (281.50,254.99) --
	(285.42,254.99) --
	(285.42,258.92) --
	(281.50,258.92) --
	cycle;
\definecolor{fillColor}{RGB}{55,126,184}

\path[fill=fillColor] (125.97,174.94) --
	(128.61,170.36) --
	(123.32,170.36) --
	cycle;

\path[fill=fillColor] ( 74.56,202.23) --
	( 77.20,197.65) --
	( 71.92,197.65) --
	cycle;
\definecolor{fillColor}{RGB}{228,26,28}

\path[fill=fillColor] (193.25, 87.50) circle (  1.96);
\definecolor{fillColor}{RGB}{77,175,74}

\path[fill=fillColor] (224.66,198.94) --
	(228.58,198.94) --
	(228.58,202.87) --
	(224.66,202.87) --
	cycle;
\definecolor{fillColor}{RGB}{55,126,184}

\path[fill=fillColor] ( 71.24,201.95) --
	( 73.88,197.38) --
	( 68.60,197.38) --
	cycle;

\path[fill=fillColor] (149.74,191.17) --
	(152.38,186.60) --
	(147.10,186.60) --
	cycle;

\path[fill=fillColor] ( 43.97,203.61) --
	( 46.61,199.04) --
	( 41.33,199.04) --
	cycle;

\path[fill=fillColor] ( 98.53,165.97) --
	(101.17,161.40) --
	( 95.88,161.40) --
	cycle;
\definecolor{fillColor}{RGB}{228,26,28}

\path[fill=fillColor] (142.17,153.35) circle (  1.96);
\definecolor{fillColor}{RGB}{55,126,184}

\path[fill=fillColor] ( 98.68,174.65) --
	(101.32,170.07) --
	( 96.04,170.07) --
	cycle;

\path[fill=fillColor] (171.76,163.48) --
	(174.41,158.90) --
	(169.12,158.90) --
	cycle;

\path[fill=fillColor] ( 83.97,184.71) --
	( 86.61,180.13) --
	( 81.33,180.13) --
	cycle;
\definecolor{fillColor}{RGB}{228,26,28}

\path[fill=fillColor] (192.19,114.62) circle (  1.96);

\path[fill=fillColor] (141.61,137.12) circle (  1.96);
\definecolor{fillColor}{RGB}{55,126,184}

\path[fill=fillColor] (103.02,167.38) --
	(105.66,162.80) --
	(100.38,162.80) --
	cycle;

\path[fill=fillColor] (129.78,173.54) --
	(132.42,168.96) --
	(127.14,168.96) --
	cycle;

\path[fill=fillColor] (133.39,181.46) --
	(136.03,176.88) --
	(130.74,176.88) --
	cycle;
\definecolor{fillColor}{RGB}{228,26,28}

\path[fill=fillColor] (170.82, 67.94) circle (  1.96);
\definecolor{fillColor}{RGB}{77,175,74}

\path[fill=fillColor] (219.97,184.60) --
	(223.90,184.60) --
	(223.90,188.53) --
	(219.97,188.53) --
	cycle;
\definecolor{fillColor}{RGB}{55,126,184}

\path[fill=fillColor] ( 64.55,198.07) --
	( 67.19,193.49) --
	( 61.91,193.49) --
	cycle;
\definecolor{fillColor}{RGB}{228,26,28}

\path[fill=fillColor] (199.21,145.91) circle (  1.96);
\definecolor{fillColor}{RGB}{55,126,184}

\path[fill=fillColor] (106.82,151.92) --
	(109.46,147.35) --
	(104.18,147.35) --
	cycle;
\definecolor{drawColor}{RGB}{228,26,28}
\definecolor{fillColor}{RGB}{228,26,28}

\path[draw=drawColor,line width= 0.6pt,line join=round,line cap=round,fill=fillColor,fill opacity=0.20] (204.37,117.74) --
	(204.13,119.90) --
	(203.43,122.03) --
	(202.27,124.09) --
	(200.66,126.06) --
	(198.64,127.90) --
	(196.24,129.58) --
	(193.48,131.09) --
	(190.42,132.40) --
	(187.10,133.48) --
	(183.57,134.32) --
	(179.88,134.91) --
	(176.08,135.25) --
	(172.25,135.31) --
	(168.43,135.11) --
	(164.68,134.65) --
	(161.06,133.93) --
	(157.63,132.97) --
	(154.43,131.77) --
	(151.52,130.36) --
	(148.93,128.76) --
	(146.72,127.00) --
	(144.90,125.09) --
	(143.51,123.07) --
	(142.58,120.97) --
	(142.11,118.82) --
	(142.11,116.66) --
	(142.58,114.51) --
	(143.51,112.41) --
	(144.90,110.39) --
	(146.72,108.49) --
	(148.93,106.72) --
	(151.52,105.12) --
	(154.43,103.71) --
	(157.63,102.52) --
	(161.06,101.55) --
	(164.68,100.83) --
	(168.43,100.37) --
	(172.25,100.17) --
	(176.08,100.24) --
	(179.88,100.57) --
	(183.57,101.16) --
	(187.10,102.00) --
	(190.42,103.09) --
	(193.48,104.39) --
	(196.24,105.90) --
	(198.64,107.58) --
	(200.66,109.42) --
	(202.27,111.39) --
	(203.43,113.45) --
	(204.13,115.58) --
	(204.37,117.74) --
	cycle;
\definecolor{drawColor}{RGB}{55,126,184}
\definecolor{fillColor}{RGB}{55,126,184}

\path[draw=drawColor,line width= 0.6pt,line join=round,line cap=round,fill=fillColor,fill opacity=0.20] (135.16,178.03) --
	(134.92,180.19) --
	(134.22,182.31) --
	(133.06,184.38) --
	(131.45,186.34) --
	(129.43,188.18) --
	(127.03,189.87) --
	(124.27,191.38) --
	(121.21,192.68) --
	(117.89,193.76) --
	(114.35,194.61) --
	(110.66,195.20) --
	(106.87,195.53) --
	(103.04,195.60) --
	( 99.22,195.40) --
	( 95.47,194.94) --
	( 91.85,194.22) --
	( 88.42,193.25) --
	( 85.22,192.06) --
	( 82.31,190.65) --
	( 79.72,189.05) --
	( 77.50,187.28) --
	( 75.69,185.37) --
	( 74.30,183.36) --
	( 73.37,181.26) --
	( 72.90,179.11) --
	( 72.90,176.94) --
	( 73.37,174.80) --
	( 74.30,172.70) --
	( 75.69,170.68) --
	( 77.50,168.77) --
	( 79.72,167.01) --
	( 82.31,165.41) --
	( 85.22,164.00) --
	( 88.42,162.80) --
	( 91.85,161.84) --
	( 95.47,161.12) --
	( 99.22,160.65) --
	(103.04,160.45) --
	(106.87,160.52) --
	(110.66,160.85) --
	(114.35,161.45) --
	(117.89,162.29) --
	(121.21,163.37) --
	(124.27,164.68) --
	(127.03,166.18) --
	(129.43,167.87) --
	(131.45,169.71) --
	(133.06,171.68) --
	(134.22,173.74) --
	(134.92,175.87) --
	(135.16,178.03) --
	cycle;
\definecolor{drawColor}{RGB}{77,175,74}
\definecolor{fillColor}{RGB}{77,175,74}

\path[draw=drawColor,line width= 0.6pt,line join=round,line cap=round,fill=fillColor,fill opacity=0.20] (316.62,206.62) --
	(316.39,208.78) --
	(315.68,210.91) --
	(314.52,212.97) --
	(312.91,214.94) --
	(310.90,216.78) --
	(308.49,218.46) --
	(305.74,219.97) --
	(302.67,221.27) --
	(299.35,222.36) --
	(295.82,223.20) --
	(292.13,223.79) --
	(288.34,224.12) --
	(284.50,224.19) --
	(280.68,223.99) --
	(276.93,223.53) --
	(273.32,222.81) --
	(269.88,221.84) --
	(266.68,220.65) --
	(263.77,219.24) --
	(261.18,217.64) --
	(258.97,215.87) --
	(257.15,213.97) --
	(255.77,211.95) --
	(254.83,209.85) --
	(254.36,207.70) --
	(254.36,205.54) --
	(254.83,203.39) --
	(255.77,201.29) --
	(257.15,199.27) --
	(258.97,197.36) --
	(261.18,195.60) --
	(263.77,194.00) --
	(266.68,192.59) --
	(269.88,191.39) --
	(273.32,190.43) --
	(276.93,189.71) --
	(280.68,189.25) --
	(284.50,189.05) --
	(288.34,189.11) --
	(292.13,189.45) --
	(295.82,190.04) --
	(299.35,190.88) --
	(302.67,191.96) --
	(305.74,193.27) --
	(308.49,194.78) --
	(310.90,196.46) --
	(312.91,198.30) --
	(314.52,200.27) --
	(315.68,202.33) --
	(316.39,204.46) --
	(316.62,206.62) --
	cycle;
\definecolor{fillColor}{RGB}{228,26,28}

\path[fill=fillColor] (173.21,117.74) circle (  3.57);
\definecolor{fillColor}{RGB}{55,126,184}

\path[fill=fillColor] (104.00,183.58) --
	(108.80,175.25) --
	( 99.19,175.25) --
	cycle;
\definecolor{fillColor}{RGB}{77,175,74}

\path[fill=fillColor] (281.89,203.05) --
	(289.03,203.05) --
	(289.03,210.19) --
	(281.89,210.19) --
	cycle;
\definecolor{drawColor}{RGB}{228,26,28}

\path[draw=drawColor,line width= 0.6pt,line join=round] (173.21,117.74) -- (193.25, 87.50);

\path[draw=drawColor,line width= 0.6pt,line join=round] (173.21,117.74) -- (142.17,153.35);

\path[draw=drawColor,line width= 0.6pt,line join=round] (173.21,117.74) -- (192.19,114.62);

\path[draw=drawColor,line width= 0.6pt,line join=round] (173.21,117.74) -- (141.61,137.12);

\path[draw=drawColor,line width= 0.6pt,line join=round] (173.21,117.74) -- (170.82, 67.94);

\path[draw=drawColor,line width= 0.6pt,line join=round] (173.21,117.74) -- (199.21,145.91);
\definecolor{drawColor}{RGB}{55,126,184}

\path[draw=drawColor,line width= 0.6pt,line join=round] (104.00,178.03) -- (125.97,171.89);

\path[draw=drawColor,line width= 0.6pt,line join=round] (104.00,178.03) -- ( 74.56,199.18);

\path[draw=drawColor,line width= 0.6pt,line join=round] (104.00,178.03) -- ( 71.24,198.90);

\path[draw=drawColor,line width= 0.6pt,line join=round] (104.00,178.03) -- (149.74,188.12);

\path[draw=drawColor,line width= 0.6pt,line join=round] (104.00,178.03) -- ( 43.97,200.56);

\path[draw=drawColor,line width= 0.6pt,line join=round] (104.00,178.03) -- ( 98.53,162.92);

\path[draw=drawColor,line width= 0.6pt,line join=round] (104.00,178.03) -- ( 98.68,171.60);

\path[draw=drawColor,line width= 0.6pt,line join=round] (104.00,178.03) -- (171.76,160.43);

\path[draw=drawColor,line width= 0.6pt,line join=round] (104.00,178.03) -- ( 83.97,181.66);

\path[draw=drawColor,line width= 0.6pt,line join=round] (104.00,178.03) -- (103.02,164.33);

\path[draw=drawColor,line width= 0.6pt,line join=round] (104.00,178.03) -- (129.78,170.49);

\path[draw=drawColor,line width= 0.6pt,line join=round] (104.00,178.03) -- (133.39,178.41);

\path[draw=drawColor,line width= 0.6pt,line join=round] (104.00,178.03) -- ( 64.55,195.02);

\path[draw=drawColor,line width= 0.6pt,line join=round] (104.00,178.03) -- (106.82,148.87);
\definecolor{drawColor}{RGB}{77,175,74}

\path[draw=drawColor,line width= 0.6pt,line join=round] (285.46,206.62) -- (409.83,182.05);

\path[draw=drawColor,line width= 0.6pt,line join=round] (285.46,206.62) -- (283.46,256.95);

\path[draw=drawColor,line width= 0.6pt,line join=round] (285.46,206.62) -- (226.62,200.91);

\path[draw=drawColor,line width= 0.6pt,line join=round] (285.46,206.62) -- (221.93,186.56);
\definecolor{drawColor}{RGB}{55,126,184}

\path[draw=drawColor,line width= 0.2pt,line join=round,line cap=round] ( 90.44,214.22) -- ( 71.92,199.44);
\definecolor{drawColor}{RGB}{77,175,74}

\node[text=drawColor,anchor=base,inner sep=0pt, outer sep=0pt, scale=  0.57] at (361.53,180.11) {CABA};

\node[text=drawColor,anchor=base,inner sep=0pt, outer sep=0pt, scale=  0.57] at (272.57,249.36) {Buenos Aires};
\definecolor{drawColor}{RGB}{55,126,184}

\node[text=drawColor,anchor=base,inner sep=0pt, outer sep=0pt, scale=  0.57] at (112.55,177.68) {Catamarca};

\node[text=drawColor,anchor=base,inner sep=0pt, outer sep=0pt, scale=  0.57] at ( 85.46,202.86) {Chaco};
\definecolor{drawColor}{RGB}{228,26,28}

\node[text=drawColor,anchor=base,inner sep=0pt, outer sep=0pt, scale=  0.57] at (182.11, 79.90) {Chubut};
\definecolor{drawColor}{RGB}{77,175,74}

\node[text=drawColor,anchor=base,inner sep=0pt, outer sep=0pt, scale=  0.57] at (215.72,193.32) {Córdoba};
\definecolor{drawColor}{RGB}{55,126,184}

\node[text=drawColor,anchor=base,inner sep=0pt, outer sep=0pt, scale=  0.57] at (104.70,212.78) {Corrientes};

\node[text=drawColor,anchor=base,inner sep=0pt, outer sep=0pt, scale=  0.57] at (160.62,191.79) {Entre Ríos};

\node[text=drawColor,anchor=base,inner sep=0pt, outer sep=0pt, scale=  0.57] at ( 39.23,204.22) {Formosa};

\node[text=drawColor,anchor=base,inner sep=0pt, outer sep=0pt, scale=  0.57] at ( 87.86,155.43) {Jujuy};
\definecolor{drawColor}{RGB}{228,26,28}

\node[text=drawColor,anchor=base,inner sep=0pt, outer sep=0pt, scale=  0.57] at (125.14,152.90) {La Pampa};
\definecolor{drawColor}{RGB}{55,126,184}

\node[text=drawColor,anchor=base,inner sep=0pt, outer sep=0pt, scale=  0.57] at ( 83.48,166.27) {La Rioja};

\node[text=drawColor,anchor=base,inner sep=0pt, outer sep=0pt, scale=  0.57] at (182.67,164.10) {Mendoza};

\node[text=drawColor,anchor=base,inner sep=0pt, outer sep=0pt, scale=  0.57] at ( 93.75,187.72) {Misiones};
\definecolor{drawColor}{RGB}{228,26,28}

\node[text=drawColor,anchor=base,inner sep=0pt, outer sep=0pt, scale=  0.57] at (181.29,107.03) {Neuquén};

\node[text=drawColor,anchor=base,inner sep=0pt, outer sep=0pt, scale=  0.57] at (130.68,129.50) {Río Negro};
\definecolor{drawColor}{RGB}{55,126,184}

\node[text=drawColor,anchor=base,inner sep=0pt, outer sep=0pt, scale=  0.57] at (113.46,167.66) {Salta};

\node[text=drawColor,anchor=base,inner sep=0pt, outer sep=0pt, scale=  0.57] at (140.65,162.92) {San Juan};

\node[text=drawColor,anchor=base,inner sep=0pt, outer sep=0pt, scale=  0.57] at (142.97,181.38) {San Luis};
\definecolor{drawColor}{RGB}{228,26,28}

\node[text=drawColor,anchor=base,inner sep=0pt, outer sep=0pt, scale=  0.57] at (189.05, 69.90) {Santa Cruz};
\definecolor{drawColor}{RGB}{77,175,74}

\node[text=drawColor,anchor=base,inner sep=0pt, outer sep=0pt, scale=  0.57] at (211.12,179.00) {Santa Fe};
\definecolor{drawColor}{RGB}{55,126,184}

\node[text=drawColor,anchor=base,inner sep=0pt, outer sep=0pt, scale=  0.57] at ( 52.89,187.44) {Santiago del Estero};
\definecolor{drawColor}{RGB}{228,26,28}

\node[text=drawColor,anchor=base,inner sep=0pt, outer sep=0pt, scale=  0.57] at (216.32,149.57) {Tierra del Fuego};
\definecolor{drawColor}{RGB}{55,126,184}

\node[text=drawColor,anchor=base,inner sep=0pt, outer sep=0pt, scale=  0.57] at (117.72,141.29) {Tucumán};
\end{scope}
\begin{scope}
\path[clip] (  0.00,  0.00) rectangle (433.62,289.08);
\definecolor{drawColor}{RGB}{0,0,0}

\path[draw=drawColor,line width= 0.6pt,line join=round] ( 25.68, 58.49) --
	( 25.68,266.40);
\end{scope}
\begin{scope}
\path[clip] (  0.00,  0.00) rectangle (433.62,289.08);
\definecolor{drawColor}{RGB}{0,0,0}

\node[text=drawColor,anchor=base east,inner sep=0pt, outer sep=0pt, scale=  0.50] at ( 20.73, 73.47) {-5.0};

\node[text=drawColor,anchor=base east,inner sep=0pt, outer sep=0pt, scale=  0.50] at ( 20.73,119.74) {-2.5};

\node[text=drawColor,anchor=base east,inner sep=0pt, outer sep=0pt, scale=  0.50] at ( 20.73,166.00) {0.0};

\node[text=drawColor,anchor=base east,inner sep=0pt, outer sep=0pt, scale=  0.50] at ( 20.73,212.26) {2.5};

\node[text=drawColor,anchor=base east,inner sep=0pt, outer sep=0pt, scale=  0.50] at ( 20.73,258.52) {5.0};
\end{scope}
\begin{scope}
\path[clip] (  0.00,  0.00) rectangle (433.62,289.08);
\definecolor{drawColor}{RGB}{0,0,0}

\path[draw=drawColor,line width= 0.6pt,line join=round] ( 22.93, 75.19) --
	( 25.68, 75.19);

\path[draw=drawColor,line width= 0.6pt,line join=round] ( 22.93,121.46) --
	( 25.68,121.46);

\path[draw=drawColor,line width= 0.6pt,line join=round] ( 22.93,167.72) --
	( 25.68,167.72);

\path[draw=drawColor,line width= 0.6pt,line join=round] ( 22.93,213.98) --
	( 25.68,213.98);

\path[draw=drawColor,line width= 0.6pt,line join=round] ( 22.93,260.25) --
	( 25.68,260.25);
\end{scope}
\begin{scope}
\path[clip] (  0.00,  0.00) rectangle (433.62,289.08);
\definecolor{drawColor}{RGB}{0,0,0}

\path[draw=drawColor,line width= 0.6pt,line join=round] ( 25.68, 58.49) --
	(428.12, 58.49);
\end{scope}
\begin{scope}
\path[clip] (  0.00,  0.00) rectangle (433.62,289.08);
\definecolor{drawColor}{RGB}{0,0,0}

\path[draw=drawColor,line width= 0.6pt,line join=round] ( 53.14, 55.74) --
	( 53.14, 58.49);

\path[draw=drawColor,line width= 0.6pt,line join=round] (151.54, 55.74) --
	(151.54, 58.49);

\path[draw=drawColor,line width= 0.6pt,line join=round] (249.95, 55.74) --
	(249.95, 58.49);

\path[draw=drawColor,line width= 0.6pt,line join=round] (348.35, 55.74) --
	(348.35, 58.49);
\end{scope}
\begin{scope}
\path[clip] (  0.00,  0.00) rectangle (433.62,289.08);
\definecolor{drawColor}{RGB}{0,0,0}

\node[text=drawColor,anchor=base,inner sep=0pt, outer sep=0pt, scale=  0.50] at ( 53.14, 50.10) {-3};

\node[text=drawColor,anchor=base,inner sep=0pt, outer sep=0pt, scale=  0.50] at (151.54, 50.10) {0};

\node[text=drawColor,anchor=base,inner sep=0pt, outer sep=0pt, scale=  0.50] at (249.95, 50.10) {3};

\node[text=drawColor,anchor=base,inner sep=0pt, outer sep=0pt, scale=  0.50] at (348.35, 50.10) {6};
\end{scope}
\begin{scope}
\path[clip] (  0.00,  0.00) rectangle (433.62,289.08);
\definecolor{drawColor}{RGB}{0,0,0}

\node[text=drawColor,anchor=base,inner sep=0pt, outer sep=0pt, scale=  0.50] at (226.90, 42.93) {\bfseries Dim1 (24.1{\%})};
\end{scope}
\begin{scope}
\path[clip] (  0.00,  0.00) rectangle (433.62,289.08);
\definecolor{drawColor}{RGB}{0,0,0}

\node[text=drawColor,rotate= 90.00,anchor=base,inner sep=0pt, outer sep=0pt, scale=  0.50] at (  8.95,162.45) {\bfseries Dim2 (18{\%})};
\end{scope}
\begin{scope}
\path[clip] (  0.00,  0.00) rectangle (433.62,289.08);
\definecolor{fillColor}{RGB}{255,255,255}

\path[fill=fillColor] (158.08,  5.50) rectangle (295.71, 30.95);
\end{scope}
\begin{scope}
\path[clip] (  0.00,  0.00) rectangle (433.62,289.08);
\definecolor{drawColor}{RGB}{0,0,0}

\node[text=drawColor,anchor=base west,inner sep=0pt, outer sep=0pt, scale=  0.7] at (163.58, 14.43) {\bfseries Regiones:};
\end{scope}
\begin{scope}
\path[clip] (  0.00,  0.00) rectangle (433.62,289.08);
\definecolor{fillColor}{gray}{0.95}

\path[fill=fillColor] (206.15, 11.00) rectangle (220.61, 25.45);
\end{scope}
\begin{scope}
\path[clip] (  0.00,  0.00) rectangle (433.62,289.08);
\definecolor{fillColor}{RGB}{228,26,28}

\path[fill=fillColor] (213.38, 18.23) circle (  1.96);
\end{scope}
\begin{scope}
\path[clip] (  0.00,  0.00) rectangle (433.62,289.08);
\definecolor{drawColor}{RGB}{228,26,28}
\definecolor{fillColor}{RGB}{228,26,28}

\path[draw=drawColor,line width= 0.6pt,line cap=rect,fill=fillColor,fill opacity=0.20] (206.87, 11.71) rectangle (219.90, 24.74);
\end{scope}
\begin{scope}
\path[clip] (  0.00,  0.00) rectangle (433.62,289.08);
\definecolor{fillColor}{RGB}{228,26,28}

\path[fill=fillColor] (213.38, 18.23) circle (  3.57);
\end{scope}
\begin{scope}
\path[clip] (  0.00,  0.00) rectangle (433.62,289.08);
\definecolor{drawColor}{RGB}{228,26,28}

\path[draw=drawColor,line width= 0.6pt,line join=round] (207.60, 18.23) -- (219.16, 18.23);
\end{scope}
\begin{scope}
\path[clip] (  0.00,  0.00) rectangle (433.62,289.08);
\definecolor{drawColor}{RGB}{228,26,28}

\node[text=drawColor,anchor=base,inner sep=0pt, outer sep=0pt, scale=  0.57] at (213.38, 16.27) {a};
\end{scope}
\begin{scope}
\path[clip] (  0.00,  0.00) rectangle (433.62,289.08);
\definecolor{fillColor}{gray}{0.95}

\path[fill=fillColor] (236.01, 11.00) rectangle (250.46, 25.45);
\end{scope}
\begin{scope}
\path[clip] (  0.00,  0.00) rectangle (433.62,289.08);
\definecolor{fillColor}{RGB}{55,126,184}

\path[fill=fillColor] (243.23, 21.28) --
	(245.88, 16.70) --
	(240.59, 16.70) --
	cycle;
\end{scope}
\begin{scope}
\path[clip] (  0.00,  0.00) rectangle (433.62,289.08);
\definecolor{drawColor}{RGB}{55,126,184}
\definecolor{fillColor}{RGB}{55,126,184}

\path[draw=drawColor,line width= 0.6pt,line cap=rect,fill=fillColor,fill opacity=0.20] (236.72, 11.71) rectangle (249.75, 24.74);
\end{scope}
\begin{scope}
\path[clip] (  0.00,  0.00) rectangle (433.62,289.08);
\definecolor{fillColor}{RGB}{55,126,184}

\path[fill=fillColor] (243.23, 23.78) --
	(248.04, 15.45) --
	(238.43, 15.45) --
	cycle;
\end{scope}
\begin{scope}
\path[clip] (  0.00,  0.00) rectangle (433.62,289.08);
\definecolor{drawColor}{RGB}{55,126,184}

\path[draw=drawColor,line width= 0.6pt,line join=round] (237.45, 18.23) -- (249.02, 18.23);
\end{scope}
\begin{scope}
\path[clip] (  0.00,  0.00) rectangle (433.62,289.08);
\definecolor{drawColor}{RGB}{55,126,184}

\node[text=drawColor,anchor=base,inner sep=0pt, outer sep=0pt, scale=  0.57] at (243.23, 16.27) {a};
\end{scope}
\begin{scope}
\path[clip] (  0.00,  0.00) rectangle (433.62,289.08);
\definecolor{fillColor}{gray}{0.95}

\path[fill=fillColor] (265.86, 11.00) rectangle (280.31, 25.45);
\end{scope}
\begin{scope}
\path[clip] (  0.00,  0.00) rectangle (433.62,289.08);
\definecolor{fillColor}{RGB}{77,175,74}

\path[fill=fillColor] (271.13, 16.26) --
	(275.05, 16.26) --
	(275.05, 20.19) --
	(271.13, 20.19) --
	cycle;
\end{scope}
\begin{scope}
\path[clip] (  0.00,  0.00) rectangle (433.62,289.08);
\definecolor{drawColor}{RGB}{77,175,74}
\definecolor{fillColor}{RGB}{77,175,74}

\path[draw=drawColor,line width= 0.6pt,line cap=rect,fill=fillColor,fill opacity=0.20] (266.57, 11.71) rectangle (279.60, 24.74);
\end{scope}
\begin{scope}
\path[clip] (  0.00,  0.00) rectangle (433.62,289.08);
\definecolor{fillColor}{RGB}{77,175,74}

\path[fill=fillColor] (269.52, 14.66) --
	(276.66, 14.66) --
	(276.66, 21.80) --
	(269.52, 21.80) --
	cycle;
\end{scope}
\begin{scope}
\path[clip] (  0.00,  0.00) rectangle (433.62,289.08);
\definecolor{drawColor}{RGB}{77,175,74}

\path[draw=drawColor,line width= 0.6pt,line join=round] (267.31, 18.23) -- (278.87, 18.23);
\end{scope}
\begin{scope}
\path[clip] (  0.00,  0.00) rectangle (433.62,289.08);
\definecolor{drawColor}{RGB}{77,175,74}

\node[text=drawColor,anchor=base,inner sep=0pt, outer sep=0pt, scale=  0.57] at (273.09, 16.27) {a};
\end{scope}
\begin{scope}
\path[clip] (  0.00,  0.00) rectangle (433.62,289.08);
\definecolor{drawColor}{RGB}{0,0,0}

\node[text=drawColor,anchor=base west,inner sep=0pt, outer sep=0pt, scale=  0.5] at (208.11, 6) {Sur};
\end{scope}
\begin{scope}
\path[clip] (  0.00,  0.00) rectangle (433.62,289.08);
\definecolor{drawColor}{RGB}{0,0,0}

\node[text=drawColor,anchor=base west,inner sep=0pt, outer sep=0pt, scale=  0.5] at (237.96, 6) {Norte};
\end{scope}
\begin{scope}
\path[clip] (  0.00,  0.00) rectangle (433.62,289.08);
\definecolor{drawColor}{RGB}{0,0,0}

\node[text=drawColor,anchor=base west,inner sep=0pt, outer sep=0pt, scale=  0.5] at (267.81, 6) {Centro};
\end{scope}
\end{tikzpicture}

\label{figure:reg_resultantes}
\end{center}
\end{figure}
\newpage

Estas regiones las denominaremos de aquí en adelante Región Norte, Región Centro y Región Sur.

Se puede ver claramente la diferencia en la similitud de los tres grupos, calculada a través del análisis de componentes principales de los macrofactores, siendo solamente la Ciudad Autónoma de Buenos Aires la única que presenta mayor disimilitud con respecto a su región.
Esto indica que CABA  lleva una dinámica socio-económica muy peculiar con respecto al promedio de las provincias argentinas, inclusive considerablemente distinta a las provincias con la cual ostenta mayor similitud.

En el Cuadro \ref{cuadro:indicadores} se presenta un resumen de los macrofactores que representan en promedio a cada región, es decir, cuales son las características compartidas entre las provincias que provocaron que sean parte de un mismo aglomerado.

La \textbf{región Sur} se caracteriza en términos sociales por una tasa de promoción secundaria moderada, en comparación con las otras regiones, una leve mortalidad infantil y una elevada tasa de violaciones cada 100.000 habitantes. En términos económicos, esta región se caracteriza por tener la segunda mayor integración a los mercados internacionales con respecto a la cantidad de exportaciones per-cápita en millones de dolares. Las provincias de estas región tienen un nivel de participación considerablemente similar en la percepción de recursos de origen Nacional y Provincial, siendo la región que menor participación en promedio posee de los RON.

En cuanto al nivel de actividad, posee elevados niveles de consumo  de energía electrica per-cápita, la tasa de actividad promedio y la cantidad relativa de empresas es la segunda más elevada de las tres regiones, el salario real es el más alto, con amplia diferencia, de las tres regiones y la tasa de pobreza es la más baja de las tres. Por último, se caracteriza por una predominancia del sector terciario, seguido del sector primario (Agricultura, ganadería, pesca y actividades extractivas) y por último, con un porcentaje considerablemente menor, el sector secundario (industrial).

La \textbf{región Norte} se caracteriza en términos sociales por una baja tasa de promoción secundaria, en comparación con las otras regiones, una elevada tasa mortalidad infantil, una baja tasa de robos (agravados y no agravados) y homocidios dolosos cada 100.000 habitantes.
En cuanto a factores económicos, esta región se caracteriza por tener una pésima integración a los mercados internacionales con la más baja cantidad de exportaciones per-cápita, en cuanto a los niveles de actividad, posee los menores niveles de consumo  de energía electrica per-cápita, la menor tasa de actividad promedio, la menor cantidad relativa de empresas, el salario real es el más bajo de las tres regiones y la tasa de pobreza es la más alta de las tres, dejandola en el último puesto en términos de desempeño económico en relación a las otras dos regiones. 
En esta región la participación en la percepción de los RON es casi cuatro veces superior a la participación que poseen en el total de recursos tributarios de origen provincial, lo que denota un elevado desequilibrio vertical por parte de estas provincias, sumado a esto es la región que menor participación en los recursos TOP posee.
Por último, se caracteriza, al igual que las otras dos regiones, por una predominancia del sector terciario, seguido del sector secundario (industrial) y dejando en último lugar al sector primario (Agricultura, ganadería, pesca y actividades extractivas).

La \textbf{región Centro} se caracteriza en términos sociales por una elevada tasa de promoción secundaria, en comparación con las otras regiones, una moderada tasa mortalidad infantil, muy similar a la Región Sur , y una elevada tasa de robos agravados y no agravados. Las provincias de esta región tienen una mayor participación en los recursos tributarios de origen provincial que en los de origen nacional, siendo la región que mayor participación ostenta tanto en los RON como en los TOP.

En cuanto a factores económicos, esta región se caracteriza por tener un desempeño intermedio en términos de integración a los mercados internacionales, con una cantidad de exportaciones per-cápita en dolares que se encuentra en un término medio entre las dos regiones restantes, en cuanto a los niveles de actividad, posee el segundo mayor nivel de consumo  de energía electrica per-cápita, la mayor tasa de actividad promedio, la mayor cantidad relativa de empresas cada 100.000 habitantes, un nivel intermedio de salario real  y una tasa de pobreza que se encuentra como la segunda más baja de las tres regiones. Por último, se caracteriza, al igual que las otras dos regiones, por una predominancia del sector terciario, seguido del sector secundario (industrial) y dejando en último lugar, con amplia diferencia, al sector primario (Agricultura, ganadería, pesca y actividades extractivas).


\begin{table}[!htbp] 
\center
\scriptsize
\raggedleft
  \caption{\\Resumen de indicadores por regiones} 
  \label{cuadro:indicadores} 
\begin{tabular}{@{\extracolsep{5pt}} lccc} 
\\[-1.8ex]\hline 
\hline \\[-1.8ex] 
 Indicadores & Región Sur & Región Norte & Región Centro \\ 
\hline \\[-1.8ex] 
Tasa promocion efectiva secundaria (2017) & 78.75 & 78 & 82.05 \\ 
Mortalidad infantil promedio 2016-2019 & 8.10 & 9.75 & 8.50 \\ 
Homicidios dolosos C/ 100.000 hab. (2016-2019) & 4.70 & 4.40 & 5.35 \\ 
Muertes en Accidentes Viales C/ 100.000  hab.  (2016-2019) & 11.50 & 16.10 & 8.75 \\ 
Robos (no agravados) C/ 100.000 hab. (2016-2019) & 889.45 & 793.15 & 1$,$629.35 \\ 
Robos agravados C/ 100.000 hab. (2016-2019) & 8.80 & 5.35 & 9.50 \\ 
Violaciones  C/ 100.000 hab. (2016-2019) & 14.05 & 12.60 & 7.35 \\ 
Exportaciones per-cápita en USD (2016-2019) & 1$,$522.55 & 785.95 & 1$,$917.65 \\ 
Demanda de MwH energía elec. Per cápita (2016) & 3.75 & 2.05 & 2.90 \\ 
Pobreza Promedio (2017-2019) & 26.05 & 32.60 & 29.35 \\ 
Tasa actividad Promedio (2017-2019) & 43.35 & 42.80 & 46.70 \\ 
Empresas  C/ 100.000 hab. (2016-2017) & 1$,$740.30 & 841.55 & 1$,$771.95 \\ 
Remuneracion  real de trabajadores reg. (priv.) (2016-2019) & 24$,$013.10 & 14$,$997.52 & 18$,$136.06 \\ 
Porcentaje de empleados en sector primario(2016-2019) & 20.61 & 12.98 & 4.54 \\ 
Porcentaje de empleados en sector terciario (2016-2019) & 67.08 & 68.39 & 72.45 \\ 
Porcentaje de empleados en sector secundario (2016-2019) & 11.19 & 17.38 & 22.38 \\ 
Porcentaje de percepción de Recursos de Origen Nacional (2016-2019) & 1.59 & 3.45 & 8.78 \\ 
Porcentaje de percepción de Recursos de Origen Provincial  (2016-2019) & 1.22 & 0.90 & 14.83 \\ 
\\ 
\hline \\[-1.8ex] 

\end{tabular} 
\end{table} 

\newpage
Para finalizar el análisis de la aglomeración por macrofactores en la Figura \ref{figure:mapa_reg} se muestra la distribución geográfica de las provincias que componen las regiones definidas anteriormente. En la misma figura se puede observar la comparación con las regiones geográficas definidas en los macrofactores, con las cuales ostenta cierta similitud.
\begin{figure}[ht!]
\begin{center}
\caption{\\Mapa de las regiones resultantes de la macrolocalización}
\includegraphics[scale=0.7]{./graficos/mapa_regiones.pdf}
\label{figure:mapa_reg}
\end{center}
\end{figure}
\newpage

\subsection{Flujos migratorios}

Dentro de las regiones existen provincias de las cuales salen una gran cantidad de emigrantes, al igual que existen provincias que son receptoras de estos mismos, estos configuran los flujos migratorios internos del país. En primer lugar se caracterizará a las regiones según el nivel de expulsión de migrantes interprovinciales.
\begin{table}[!htbp] \centering 
\footnotesize
  \caption{\\Regiones de origen de los migrantes} 
  \label{cuadro:origen_mig} 
\begin{tabular}{@{\extracolsep{5pt}} ccc} 
\\[-1.8ex]\hline 
\hline \\[-1.8ex] 
Región & Porcentaje \\ 
\\[-1.8ex]\hline 
\hline \\[-1.8ex] 

 Sur & 16.68\%\\ 
 Norte & 49.87\%\\ 
 Centro & 33.45\%\\ 
\hline \\[-1.8ex] 
\end{tabular} 
\begin{flushleft}
\begin{scriptsize}
Fuente: Elaboración propia en base a EPH.\\
Nota: Los migrantes están definidos como personas que vivían hace cinco años en otra provincia.\\
Las estimaciones corresponden al período desde el segundo trimestre de 2016 hasta el cuarto trimestre de 2019.
\end{scriptsize}
\end{flushleft}
\end{table} 

En el Cuadro \ref{cuadro:origen_mig} se puede notar que las provincias de la  \textbf{Región Norte} son las que mayor nivel de expulsión poseen, seguidas por las provincias pertenecientes a la Región Centro, y por úlitmo se encuentran las pertenecientes a la Región Sur.

\begin{table}[!htbp] \centering 
\footnotesize
  \caption{\\Regiones de destino de los migrantes} 
  \label{cuadro:destino_mig} 
\begin{tabular}{@{\extracolsep{5pt}} ccc} 
\\[-1.8ex]\hline 
\hline \\[-1.8ex] 
Región & Porcentaje \\ 
\\[-1.8ex]\hline 
\hline \\[-1.8ex]
Sur & 14.93\%\\ 
Norte & 29.79\%\\ 
Centro & 55.28\%\\ 
\hline \\[-1.8ex] 
\end{tabular} 
\begin{flushleft}
\begin{scriptsize}
Fuente: Elaboración propia en base a EPH.\\
Nota: Los migrantes están definidos como personas que vivían hace cinco años en otra provincia.\\
Las estimaciones corresponden al período desde el segundo trimestre de 2016 hasta el cuarto trimestre de 2019.
\end{scriptsize}
\end{flushleft}
\end{table} 

Sin embargo, analizando cuales son las regiones de destino  con mayor porcentaje  de migrantes, se puede encontrar en el Cuadro \ref{cuadro:destino_mig} que la \textbf{Región Centro} es la que mayor nivel de atracción posee por elevada diferencia (55.28\%), seguida por la Región Norte y en último lugar la Región Sur.

Esta relación entre regiones con mayor atracción y expulsión de migrantes tambien puede ser vista a nivel provincial. Para ello se definen la tasa de emigración, que es el cociente entre la cantidad de emigrantes nativos de la provincia ``$x$'' sobre la cantidad de residentes de la provincia ``$x$'' y la tasa de inmigración que indica la cantidad de inmigrantes que habitan la provincia ``$x$'' sobre la cantidad de residentes de la provincia ``$x$''.

\begin{figure}[h!]
\begin{center}
\caption{\\Tasa de inmigración y emigración por provincia}
% Created by tikzDevice version 0.12.3.1 on 2021-05-21 22:18:04
% !TEX encoding = UTF-8 Unicode
\begin{tikzpicture}[x=1pt,y=1pt]
\definecolor{fillColor}{RGB}{255,255,255}
\path[use as bounding box,fill=fillColor,fill opacity=0.00] (0,0) rectangle (433.62,289.08);
\begin{scope}
\path[clip] (  0.00,  0.00) rectangle (433.62,289.08);
\definecolor{drawColor}{RGB}{255,255,255}
\definecolor{fillColor}{RGB}{255,255,255}

\path[draw=drawColor,line width= 0.6pt,line join=round,line cap=round,fill=fillColor] (  0.00,  0.00) rectangle (433.62,289.08);
\end{scope}
\begin{scope}
\path[clip] ( 26.79, 22.04) rectangle (364.14,266.40);
\definecolor{drawColor}{RGB}{255,255,255}

\path[draw=drawColor,line width= 0.3pt,line join=round] ( 26.79, 62.77) --
	(364.14, 62.77);

\path[draw=drawColor,line width= 0.3pt,line join=round] ( 26.79,122.01) --
	(364.14,122.01);

\path[draw=drawColor,line width= 0.3pt,line join=round] ( 26.79,181.25) --
	(364.14,181.25);

\path[draw=drawColor,line width= 0.3pt,line join=round] ( 26.79,240.49) --
	(364.14,240.49);

\path[draw=drawColor,line width= 0.3pt,line join=round] ( 83.01, 22.04) --
	( 83.01,266.40);

\path[draw=drawColor,line width= 0.3pt,line join=round] (164.79, 22.04) --
	(164.79,266.40);

\path[draw=drawColor,line width= 0.3pt,line join=round] (246.58, 22.04) --
	(246.58,266.40);

\path[draw=drawColor,line width= 0.3pt,line join=round] (328.36, 22.04) --
	(328.36,266.40);

\path[draw=drawColor,line width= 0.6pt,line join=round] ( 26.79, 33.15) --
	(364.14, 33.15);

\path[draw=drawColor,line width= 0.6pt,line join=round] ( 26.79, 92.39) --
	(364.14, 92.39);

\path[draw=drawColor,line width= 0.6pt,line join=round] ( 26.79,151.63) --
	(364.14,151.63);

\path[draw=drawColor,line width= 0.6pt,line join=round] ( 26.79,210.87) --
	(364.14,210.87);

\path[draw=drawColor,line width= 0.6pt,line join=round] ( 42.12, 22.04) --
	( 42.12,266.40);

\path[draw=drawColor,line width= 0.6pt,line join=round] (123.90, 22.04) --
	(123.90,266.40);

\path[draw=drawColor,line width= 0.6pt,line join=round] (205.69, 22.04) --
	(205.69,266.40);

\path[draw=drawColor,line width= 0.6pt,line join=round] (287.47, 22.04) --
	(287.47,266.40);
\definecolor{drawColor}{RGB}{77,175,74}
\definecolor{fillColor}{RGB}{77,175,74}

\path[draw=drawColor,line width= 0.4pt,line join=round,line cap=round,fill=fillColor] ( 82.21, 38.37) circle (  1.96);
\definecolor{drawColor}{RGB}{55,126,184}
\definecolor{fillColor}{RGB}{55,126,184}

\path[draw=drawColor,line width= 0.4pt,line join=round,line cap=round,fill=fillColor] ( 76.90,111.06) circle (  1.96);

\path[draw=drawColor,line width= 0.4pt,line join=round,line cap=round,fill=fillColor] ( 63.50,247.09) circle (  1.96);
\definecolor{drawColor}{RGB}{228,26,28}
\definecolor{fillColor}{RGB}{228,26,28}

\path[draw=drawColor,line width= 0.4pt,line join=round,line cap=round,fill=fillColor] (113.40, 61.30) circle (  1.96);
\definecolor{drawColor}{RGB}{77,175,74}
\definecolor{fillColor}{RGB}{77,175,74}

\path[draw=drawColor,line width= 0.4pt,line join=round,line cap=round,fill=fillColor] ( 80.08, 37.28) circle (  1.96);

\path[draw=drawColor,line width= 0.4pt,line join=round,line cap=round,fill=fillColor] ( 78.84, 63.57) circle (  1.96);
\definecolor{drawColor}{RGB}{55,126,184}
\definecolor{fillColor}{RGB}{55,126,184}

\path[draw=drawColor,line width= 0.4pt,line join=round,line cap=round,fill=fillColor] ( 73.72,220.69) circle (  1.96);

\path[draw=drawColor,line width= 0.4pt,line join=round,line cap=round,fill=fillColor] ( 66.53,195.39) circle (  1.96);

\path[draw=drawColor,line width= 0.4pt,line join=round,line cap=round,fill=fillColor] ( 58.75,132.92) circle (  1.96);

\path[draw=drawColor,line width= 0.4pt,line join=round,line cap=round,fill=fillColor] ( 73.94,111.78) circle (  1.96);

\path[draw=drawColor,line width= 0.4pt,line join=round,line cap=round,fill=fillColor] (104.00,111.07) circle (  1.96);

\path[draw=drawColor,line width= 0.4pt,line join=round,line cap=round,fill=fillColor] ( 93.60, 80.40) circle (  1.96);
\definecolor{drawColor}{RGB}{77,175,74}
\definecolor{fillColor}{RGB}{77,175,74}

\path[draw=drawColor,line width= 0.4pt,line join=round,line cap=round,fill=fillColor] ( 71.56, 62.83) circle (  1.96);
\definecolor{drawColor}{RGB}{55,126,184}
\definecolor{fillColor}{RGB}{55,126,184}

\path[draw=drawColor,line width= 0.4pt,line join=round,line cap=round,fill=fillColor] ( 77.61,148.97) circle (  1.96);
\definecolor{drawColor}{RGB}{228,26,28}
\definecolor{fillColor}{RGB}{228,26,28}

\path[draw=drawColor,line width= 0.4pt,line join=round,line cap=round,fill=fillColor] (124.03, 56.87) circle (  1.96);
\definecolor{drawColor}{RGB}{55,126,184}
\definecolor{fillColor}{RGB}{55,126,184}

\path[draw=drawColor,line width= 0.4pt,line join=round,line cap=round,fill=fillColor] ( 92.65,212.40) circle (  1.96);

\path[draw=drawColor,line width= 0.4pt,line join=round,line cap=round,fill=fillColor] ( 75.42, 87.58) circle (  1.96);

\path[draw=drawColor,line width= 0.4pt,line join=round,line cap=round,fill=fillColor] ( 58.18, 76.60) circle (  1.96);

\path[draw=drawColor,line width= 0.4pt,line join=round,line cap=round,fill=fillColor] (112.49, 76.36) circle (  1.96);
\definecolor{drawColor}{RGB}{228,26,28}
\definecolor{fillColor}{RGB}{228,26,28}

\path[draw=drawColor,line width= 0.4pt,line join=round,line cap=round,fill=fillColor] (143.72, 83.37) circle (  1.96);
\definecolor{drawColor}{RGB}{77,175,74}
\definecolor{fillColor}{RGB}{77,175,74}

\path[draw=drawColor,line width= 0.4pt,line join=round,line cap=round,fill=fillColor] ( 77.75, 60.23) circle (  1.96);
\definecolor{drawColor}{RGB}{55,126,184}
\definecolor{fillColor}{RGB}{55,126,184}

\path[draw=drawColor,line width= 0.4pt,line join=round,line cap=round,fill=fillColor] ( 60.79,201.99) circle (  1.96);
\definecolor{drawColor}{RGB}{77,175,74}
\definecolor{fillColor}{RGB}{77,175,74}

\path[draw=drawColor,line width= 0.4pt,line join=round,line cap=round,fill=fillColor] (204.97, 49.35) circle (  1.96);
\definecolor{drawColor}{RGB}{228,26,28}
\definecolor{fillColor}{RGB}{228,26,28}

\path[draw=drawColor,line width= 0.4pt,line join=round,line cap=round,fill=fillColor] ( 67.48, 98.80) circle (  1.96);
\definecolor{drawColor}{RGB}{0,0,0}

\path[draw=drawColor,line width= 0.6pt,line join=round] ( 26.79, 22.04) -- (364.14,266.40);
\definecolor{drawColor}{RGB}{77,175,74}

\node[text=drawColor,anchor=base,inner sep=0pt, outer sep=0pt, scale=  0.57] at ( 90.17, 42.22) {Buenos Aires};
\definecolor{drawColor}{RGB}{55,126,184}

\node[text=drawColor,anchor=base,inner sep=0pt, outer sep=0pt, scale=  0.57] at ( 89.81,103.30) {Catamarca};

\node[text=drawColor,anchor=base,inner sep=0pt, outer sep=0pt, scale=  0.57] at ( 71.59,250.96) {Chaco};
\definecolor{drawColor}{RGB}{228,26,28}

\node[text=drawColor,anchor=base,inner sep=0pt, outer sep=0pt, scale=  0.57] at (105.42, 53.52) {Chubut};
\definecolor{drawColor}{RGB}{77,175,74}

\node[text=drawColor,anchor=base,inner sep=0pt, outer sep=0pt, scale=  0.57] at ( 73.93, 29.53) {Ciudad Autónoma de Buenos Aires};

\node[text=drawColor,anchor=base,inner sep=0pt, outer sep=0pt, scale=  0.57] at ( 89.16, 67.42) {Córdoba};
\definecolor{drawColor}{RGB}{55,126,184}

\node[text=drawColor,anchor=base,inner sep=0pt, outer sep=0pt, scale=  0.57] at ( 65.65,224.55) {Corrientes};

\node[text=drawColor,anchor=base,inner sep=0pt, outer sep=0pt, scale=  0.57] at ( 74.51,187.64) {Entre Ríos};

\node[text=drawColor,anchor=base,inner sep=0pt, outer sep=0pt, scale=  0.57] at ( 64.29,135.11) {Formosa};

\node[text=drawColor,anchor=base,inner sep=0pt, outer sep=0pt, scale=  0.57] at ( 66.01,115.59) {Jujuy};

\node[text=drawColor,anchor=base,inner sep=0pt, outer sep=0pt, scale=  0.57] at (111.99,114.90) {La Pampa};

\node[text=drawColor,anchor=base,inner sep=0pt, outer sep=0pt, scale=  0.57] at (101.69, 84.31) {La Rioja};
\definecolor{drawColor}{RGB}{77,175,74}

\node[text=drawColor,anchor=base,inner sep=0pt, outer sep=0pt, scale=  0.57] at ( 61.33, 66.71) {Mendoza};
\definecolor{drawColor}{RGB}{55,126,184}

\node[text=drawColor,anchor=base,inner sep=0pt, outer sep=0pt, scale=  0.57] at ( 72.00,142.91) {Misiones};
\definecolor{drawColor}{RGB}{228,26,28}

\node[text=drawColor,anchor=base,inner sep=0pt, outer sep=0pt, scale=  0.57] at (132.05, 49.09) {Neuquén};
\definecolor{drawColor}{RGB}{55,126,184}

\node[text=drawColor,anchor=base,inner sep=0pt, outer sep=0pt, scale=  0.57] at (100.66,204.63) {Río Negro};

\node[text=drawColor,anchor=base,inner sep=0pt, outer sep=0pt, scale=  0.57] at ( 65.08, 91.01) {Salta};

\node[text=drawColor,anchor=base,inner sep=0pt, outer sep=0pt, scale=  0.57] at ( 50.13, 80.49) {San Juan};

\node[text=drawColor,anchor=base,inner sep=0pt, outer sep=0pt, scale=  0.57] at (120.56, 68.58) {San Luis};
\definecolor{drawColor}{RGB}{228,26,28}

\node[text=drawColor,anchor=base,inner sep=0pt, outer sep=0pt, scale=  0.57] at (151.77, 87.27) {Santa Cruz};
\definecolor{drawColor}{RGB}{77,175,74}

\node[text=drawColor,anchor=base,inner sep=0pt, outer sep=0pt, scale=  0.57] at ( 69.80, 52.48) {Santa Fe};
\definecolor{drawColor}{RGB}{55,126,184}

\node[text=drawColor,anchor=base,inner sep=0pt, outer sep=0pt, scale=  0.57] at ( 53.96,205.85) {Santiago del Estero};
\definecolor{drawColor}{RGB}{77,175,74}

\node[text=drawColor,anchor=base,inner sep=0pt, outer sep=0pt, scale=  0.57] at (214.31, 41.49) {Tierra del Fuego, Antártida e Islas del Atlántico Sur};
\definecolor{drawColor}{RGB}{228,26,28}

\node[text=drawColor,anchor=base,inner sep=0pt, outer sep=0pt, scale=  0.57] at ( 58.51,102.70) {Tucumán};
\end{scope}
\begin{scope}
\path[clip] (  0.00,  0.00) rectangle (433.62,289.08);
\definecolor{drawColor}{RGB}{0,0,0}

\path[draw=drawColor,line width= 0.6pt,line join=round] ( 26.79, 22.04) --
	( 26.79,266.40);
\end{scope}
\begin{scope}
\path[clip] (  0.00,  0.00) rectangle (433.62,289.08);
\definecolor{drawColor}{RGB}{0,0,0}

\node[text=drawColor,anchor=base east,inner sep=0pt, outer sep=0pt, scale=  0.50] at ( 21.84, 31.42) {0{\%}};

\node[text=drawColor,anchor=base east,inner sep=0pt, outer sep=0pt, scale=  0.50] at ( 21.84, 90.66) {20{\%}};

\node[text=drawColor,anchor=base east,inner sep=0pt, outer sep=0pt, scale=  0.50] at ( 21.84,149.90) {40{\%}};

\node[text=drawColor,anchor=base east,inner sep=0pt, outer sep=0pt, scale=  0.50] at ( 21.84,209.14) {60{\%}};
\end{scope}
\begin{scope}
\path[clip] (  0.00,  0.00) rectangle (433.62,289.08);
\definecolor{drawColor}{gray}{0.20}

\path[draw=drawColor,line width= 0.6pt,line join=round] ( 24.04, 33.15) --
	( 26.79, 33.15);

\path[draw=drawColor,line width= 0.6pt,line join=round] ( 24.04, 92.39) --
	( 26.79, 92.39);

\path[draw=drawColor,line width= 0.6pt,line join=round] ( 24.04,151.63) --
	( 26.79,151.63);

\path[draw=drawColor,line width= 0.6pt,line join=round] ( 24.04,210.87) --
	( 26.79,210.87);
\end{scope}
\begin{scope}
\path[clip] (  0.00,  0.00) rectangle (433.62,289.08);
\definecolor{drawColor}{RGB}{0,0,0}

\path[draw=drawColor,line width= 0.6pt,line join=round] ( 26.79, 22.04) --
	(364.14, 22.04);
\end{scope}
\begin{scope}
\path[clip] (  0.00,  0.00) rectangle (433.62,289.08);
\definecolor{drawColor}{gray}{0.20}

\path[draw=drawColor,line width= 0.6pt,line join=round] ( 42.12, 19.29) --
	( 42.12, 22.04);

\path[draw=drawColor,line width= 0.6pt,line join=round] (123.90, 19.29) --
	(123.90, 22.04);

\path[draw=drawColor,line width= 0.6pt,line join=round] (205.69, 19.29) --
	(205.69, 22.04);

\path[draw=drawColor,line width= 0.6pt,line join=round] (287.47, 19.29) --
	(287.47, 22.04);
\end{scope}
\begin{scope}
\path[clip] (  0.00,  0.00) rectangle (433.62,289.08);
\definecolor{drawColor}{RGB}{0,0,0}

\node[text=drawColor,anchor=base,inner sep=0pt, outer sep=0pt, scale=  0.50] at ( 42.12, 13.64) {0{\%}};

\node[text=drawColor,anchor=base,inner sep=0pt, outer sep=0pt, scale=  0.50] at (123.90, 13.64) {20{\%}};

\node[text=drawColor,anchor=base,inner sep=0pt, outer sep=0pt, scale=  0.50] at (205.69, 13.64) {40{\%}};

\node[text=drawColor,anchor=base,inner sep=0pt, outer sep=0pt, scale=  0.50] at (287.47, 13.64) {60{\%}};
\end{scope}
\begin{scope}
\path[clip] (  0.00,  0.00) rectangle (433.62,289.08);
\definecolor{drawColor}{RGB}{0,0,0}

\node[text=drawColor,anchor=base,inner sep=0pt, outer sep=0pt, scale=  0.50] at (195.46,  6.47) {\bfseries Tasa de inmigración};
\end{scope}
\begin{scope}
\path[clip] (  0.00,  0.00) rectangle (433.62,289.08);
\definecolor{drawColor}{RGB}{0,0,0}

\node[text=drawColor,rotate= 90.00,anchor=base,inner sep=0pt, outer sep=0pt, scale=  0.50] at (  8.95,144.22) {\bfseries Tasa de emigración};
\end{scope}
\begin{scope}
\path[clip] (  0.00,  0.00) rectangle (433.62,289.08);
\definecolor{fillColor}{RGB}{255,255,255}

\path[fill=fillColor] (375.14,109.43) rectangle (428.12,179.02);
\end{scope}
\begin{scope}
\path[clip] (  0.00,  0.00) rectangle (433.62,289.08);
\definecolor{drawColor}{RGB}{0,0,0}

\node[text=drawColor,anchor=base west,inner sep=0pt, outer sep=0pt, scale=  1.10] at (380.64,164.86) {\bfseries Región:};
\end{scope}
\begin{scope}
\path[clip] (  0.00,  0.00) rectangle (433.62,289.08);
\definecolor{drawColor}{RGB}{228,26,28}
\definecolor{fillColor}{RGB}{228,26,28}

\path[draw=drawColor,line width= 0.4pt,line join=round,line cap=round,fill=fillColor] (387.87,151.06) circle (  1.96);
\end{scope}
\begin{scope}
\path[clip] (  0.00,  0.00) rectangle (433.62,289.08);
\definecolor{drawColor}{RGB}{55,126,184}
\definecolor{fillColor}{RGB}{55,126,184}

\path[draw=drawColor,line width= 0.4pt,line join=round,line cap=round,fill=fillColor] (387.87,136.61) circle (  1.96);
\end{scope}
\begin{scope}
\path[clip] (  0.00,  0.00) rectangle (433.62,289.08);
\definecolor{drawColor}{RGB}{77,175,74}
\definecolor{fillColor}{RGB}{77,175,74}

\path[draw=drawColor,line width= 0.4pt,line join=round,line cap=round,fill=fillColor] (387.87,122.15) circle (  1.96);
\end{scope}
\begin{scope}
\path[clip] (  0.00,  0.00) rectangle (433.62,289.08);
\definecolor{drawColor}{RGB}{0,0,0}

\node[text=drawColor,anchor=base west,inner sep=0pt, outer sep=0pt, scale=  0.88] at (400.59,148.03) {1};
\end{scope}
\begin{scope}
\path[clip] (  0.00,  0.00) rectangle (433.62,289.08);
\definecolor{drawColor}{RGB}{0,0,0}

\node[text=drawColor,anchor=base west,inner sep=0pt, outer sep=0pt, scale=  0.88] at (400.59,133.58) {2};
\end{scope}
\begin{scope}
\path[clip] (  0.00,  0.00) rectangle (433.62,289.08);
\definecolor{drawColor}{RGB}{0,0,0}

\node[text=drawColor,anchor=base west,inner sep=0pt, outer sep=0pt, scale=  0.88] at (400.59,119.12) {3};
\end{scope}
\end{tikzpicture}

\label{figure:emig_inmig_prov}
\end{center}
\begin{flushleft}
\begin{scriptsize}
Fuente: Elaboración propia en base a EPH.\\
Nota: Los migrantes están definidos como personas que vivían hace cinco años en otra provincia.\\
Las estimaciones corresponden al período desde el segundo trimestre de 2016 hasta el cuarto trimestre de 2019.\\
\end{scriptsize}
\end{flushleft}
\end{figure}

En la Figura \ref{figure:emig_inmig_prov} se puede observar la relación entre la tasa de emigración e inmigración de las provincias, diferenciadas por la región a la cual pertenecen.

La bisectriz divide el plano en dos zonas, todas las provincias que se encuentran por encima de ella son aquellas en la que la tasa de emigración es superior a la tasa de inmigración, mientras que las que se encuentran por debajo son quellas en la que existe una mayor tasa de inmigración que de emigración. 

Las provincias pertenecientes a la Región Norte se ubican en casi su totalidad por encima de la bisectriz, indicando que son provincias en donde la expulsión de migrantes es mucho mayor que la atracción de los mismos. Las provincias de la Región Sur tienen una mayor tendencia a ubicarse por encima de la linea diagonal, con excepción de Neuquén y Tierra del Fuego. Por último, en la Región Centro todas las provincias se encuentran por debajo de la diagonal, en donde la atracción de migrantes es mayor que su nivel de expulsión.

Estas diferencias entre tasa de emigración e inmigración pueden ser vistas desde un punto de vista geográfico en la Figura \ref{figure:emig_inmig_prov_mapa}. Se observa la diferencia en la distribución de las provincias con mayor tasa de migración e inmigración, las provincias de Rio Negro, Santa Cruz y La Pampa son las que mayores tasa de emigración poseen, mientras que Tierra del Fuego, Santa Cruz y Neuquén son  las provincias en donde la tasa de inmigración es más elevada.
\begin{figure}[h!]
\begin{center}
\caption{\\Mapa de tasas de emigración e inmigración por provincia}
\includegraphics[scale=1.1]{./graficos/emig_inmig_por_prov.pdf}
\label{figure:emig_inmig_prov_mapa}
\end{center}
\begin{flushleft}
\begin{scriptsize}
Fuente: Elaboración propia en base a EPH.\\
Nota: Los migrantes están definidos como personas que vivían hace cinco años en otra provincia.\\
Las estimaciones corresponden al período desde el segundo trimestre de 2016 hasta el cuarto trimestre de 2019.
\end{scriptsize}
\end{flushleft}
\end{figure}
\newpage
\section{Migrantes}
Un migrante es toda persona que se traslada fuera de su localidad de residencia habitual de manera temporal o permanente. En este trabajo se tendra en cuenta solamente los casos en los que éste movimiento migratorio se realice dentro del mismo país, siempre y cuando la migración haya sido hace cinco años o menos. El último punto es importante para que las características que se tomen como determinantes de la migración tengan un correlato con la decisión migratoria.

Las características individuales o familiares de las personas que toman la decisión de emprender un proceso migratorio forman parte de los microdeterminantes de las migraciones. Si bien estos no deben ser tomados como los principales impulsores de la decisión del éxodo, funcionan como mediadores en la decisión migratoria y tienen una elevada influencia en la auto-selección de los migrantes para cada una de las regiones.

En las siguientes subsecciones se analizará si existen características que provoquen que ciertas personas posean una mayor propensión a migrar hacia determinadas localizaciones. Para ello se indagará en las diferencia entre los migrantes y los nativos de las distintas regiones con respecto a los microdeterminantes que se consideran relevantes, los cuales serán explicitados y desarrollados en los siguientes apartados.


\subsection{Género y edad}
El género y la edad de las personas son considerados dentro de los microfactores que pueden afectar considerablemente a la decisión migratoria, es por ello que se analizan estas características para los nativos y migrantes de cada región.

En las últimas decadas la dimensión del género en la migración ha tomado un sendero particular, en gran parte por la participación más activa de la mujer en el mercado laboral y el acceso a la posibilidad de una mayor independencia económica. Todo esto provocó que la mujer tome un rol más activo en la decisión migratoria individual, permitiendole desligarse de la decisión migratoria respondiendo a estrategias  familiares.

En la Figura \ref{figure:sexo_mig_} se puede ver el género de los migrantes y nativos. En las tres regiones los migrantes  son en su mayoría mujeres, siendo la región sur la que mayor cantidad de migrantes mujeres posee con un $51.07\%$, y en donde la menor cantidad de mujeres migrantes habitan en la Región Norte, con un $50.47\%$ de migrantes mujeres.

\begin{figure}[ht!]
\begin{center}
 	\caption{\\Sexo de los nativos y migrantes por regiones}
% Created by tikzDevice version 0.12.3.1 on 2021-07-01 11:40:12
% !TEX encoding = UTF-8 Unicode
\begin{tikzpicture}[x=1pt,y=1pt]
\definecolor{fillColor}{RGB}{255,255,255}
\path[use as bounding box,fill=fillColor,fill opacity=0.00] (0,0) rectangle (433.62,289.08);
\begin{scope}
\path[clip] (  0.00,  0.00) rectangle (433.62,289.08);
\definecolor{drawColor}{RGB}{255,255,255}
\definecolor{fillColor}{RGB}{255,255,255}

\path[draw=drawColor,line width= 0.6pt,line join=round,line cap=round,fill=fillColor] (  0.00,  0.00) rectangle (433.62,289.08);
\end{scope}
\begin{scope}
\path[clip] ( 61.11,215.10) rectangle (411.55,283.58);
\definecolor{drawColor}{RGB}{255,255,255}

\path[draw=drawColor,line width= 0.3pt,line join=round] (116.86,215.10) --
	(116.86,283.58);

\path[draw=drawColor,line width= 0.3pt,line join=round] (196.51,215.10) --
	(196.51,283.58);

\path[draw=drawColor,line width= 0.3pt,line join=round] (276.15,215.10) --
	(276.15,283.58);

\path[draw=drawColor,line width= 0.3pt,line join=round] (355.80,215.10) --
	(355.80,283.58);

\path[draw=drawColor,line width= 0.6pt,line join=round] ( 61.11,233.78) --
	(411.55,233.78);

\path[draw=drawColor,line width= 0.6pt,line join=round] ( 61.11,264.90) --
	(411.55,264.90);

\path[draw=drawColor,line width= 0.6pt,line join=round] ( 77.04,215.10) --
	( 77.04,283.58);

\path[draw=drawColor,line width= 0.6pt,line join=round] (156.69,215.10) --
	(156.69,283.58);

\path[draw=drawColor,line width= 0.6pt,line join=round] (236.33,215.10) --
	(236.33,283.58);

\path[draw=drawColor,line width= 0.6pt,line join=round] (315.97,215.10) --
	(315.97,283.58);

\path[draw=drawColor,line width= 0.6pt,line join=round] (395.62,215.10) --
	(395.62,283.58);
\definecolor{fillColor}{RGB}{228,26,28}

\path[fill=fillColor] (239.03,224.44) rectangle (395.62,243.11);
\definecolor{fillColor}{RGB}{55,126,184}

\path[fill=fillColor] ( 77.04,224.44) rectangle (239.03,243.11);
\definecolor{fillColor}{RGB}{228,26,28}

\path[fill=fillColor] (241.07,255.57) rectangle (395.62,274.24);
\definecolor{fillColor}{RGB}{55,126,184}

\path[fill=fillColor] ( 77.04,255.57) rectangle (241.07,274.24);
\definecolor{drawColor}{RGB}{0,0,0}

\node[text=drawColor,anchor=base,inner sep=0pt, outer sep=0pt, scale=  0.85] at (301.66,230.84) {49.15{\%}};

\node[text=drawColor,anchor=base,inner sep=0pt, outer sep=0pt, scale=  0.85] at (141.84,230.84) {50.85{\%}};

\node[text=drawColor,anchor=base,inner sep=0pt, outer sep=0pt, scale=  0.85] at (302.89,261.96) {48.51{\%}};

\node[text=drawColor,anchor=base,inner sep=0pt, outer sep=0pt, scale=  0.85] at (142.65,261.96) {51.49{\%}};
\end{scope}
\begin{scope}
\path[clip] ( 61.11,141.12) rectangle (411.55,209.60);
\definecolor{drawColor}{RGB}{255,255,255}

\path[draw=drawColor,line width= 0.3pt,line join=round] (116.86,141.12) --
	(116.86,209.60);

\path[draw=drawColor,line width= 0.3pt,line join=round] (196.51,141.12) --
	(196.51,209.60);

\path[draw=drawColor,line width= 0.3pt,line join=round] (276.15,141.12) --
	(276.15,209.60);

\path[draw=drawColor,line width= 0.3pt,line join=round] (355.80,141.12) --
	(355.80,209.60);

\path[draw=drawColor,line width= 0.6pt,line join=round] ( 61.11,159.80) --
	(411.55,159.80);

\path[draw=drawColor,line width= 0.6pt,line join=round] ( 61.11,190.92) --
	(411.55,190.92);

\path[draw=drawColor,line width= 0.6pt,line join=round] ( 77.04,141.12) --
	( 77.04,209.60);

\path[draw=drawColor,line width= 0.6pt,line join=round] (156.69,141.12) --
	(156.69,209.60);

\path[draw=drawColor,line width= 0.6pt,line join=round] (236.33,141.12) --
	(236.33,209.60);

\path[draw=drawColor,line width= 0.6pt,line join=round] (315.97,141.12) --
	(315.97,209.60);

\path[draw=drawColor,line width= 0.6pt,line join=round] (395.62,141.12) --
	(395.62,209.60);
\definecolor{fillColor}{RGB}{228,26,28}

\path[fill=fillColor] (237.84,150.46) rectangle (395.62,169.13);
\definecolor{fillColor}{RGB}{55,126,184}

\path[fill=fillColor] ( 77.04,150.46) rectangle (237.84,169.13);
\definecolor{fillColor}{RGB}{228,26,28}

\path[fill=fillColor] (242.93,181.59) rectangle (395.62,200.26);
\definecolor{fillColor}{RGB}{55,126,184}

\path[fill=fillColor] ( 77.04,181.59) rectangle (242.93,200.26);
\definecolor{drawColor}{RGB}{0,0,0}

\node[text=drawColor,anchor=base,inner sep=0pt, outer sep=0pt, scale=  0.85] at (300.95,156.86) {49.53{\%}};

\node[text=drawColor,anchor=base,inner sep=0pt, outer sep=0pt, scale=  0.85] at (141.36,156.86) {50.47{\%}};

\node[text=drawColor,anchor=base,inner sep=0pt, outer sep=0pt, scale=  0.85] at (304.01,187.98) {47.93{\%}};

\node[text=drawColor,anchor=base,inner sep=0pt, outer sep=0pt, scale=  0.85] at (143.40,187.98) {52.07{\%}};
\end{scope}
\begin{scope}
\path[clip] ( 61.11, 67.14) rectangle (411.55,135.62);
\definecolor{drawColor}{RGB}{255,255,255}

\path[draw=drawColor,line width= 0.3pt,line join=round] (116.86, 67.14) --
	(116.86,135.62);

\path[draw=drawColor,line width= 0.3pt,line join=round] (196.51, 67.14) --
	(196.51,135.62);

\path[draw=drawColor,line width= 0.3pt,line join=round] (276.15, 67.14) --
	(276.15,135.62);

\path[draw=drawColor,line width= 0.3pt,line join=round] (355.80, 67.14) --
	(355.80,135.62);

\path[draw=drawColor,line width= 0.6pt,line join=round] ( 61.11, 85.82) --
	(411.55, 85.82);

\path[draw=drawColor,line width= 0.6pt,line join=round] ( 61.11,116.94) --
	(411.55,116.94);

\path[draw=drawColor,line width= 0.6pt,line join=round] ( 77.04, 67.14) --
	( 77.04,135.62);

\path[draw=drawColor,line width= 0.6pt,line join=round] (156.69, 67.14) --
	(156.69,135.62);

\path[draw=drawColor,line width= 0.6pt,line join=round] (236.33, 67.14) --
	(236.33,135.62);

\path[draw=drawColor,line width= 0.6pt,line join=round] (315.97, 67.14) --
	(315.97,135.62);

\path[draw=drawColor,line width= 0.6pt,line join=round] (395.62, 67.14) --
	(395.62,135.62);
\definecolor{fillColor}{RGB}{228,26,28}

\path[fill=fillColor] (239.74, 76.48) rectangle (395.62, 95.15);
\definecolor{fillColor}{RGB}{55,126,184}

\path[fill=fillColor] ( 77.04, 76.48) rectangle (239.74, 95.15);
\definecolor{fillColor}{RGB}{228,26,28}

\path[fill=fillColor] (242.02,107.61) rectangle (395.62,126.28);
\definecolor{fillColor}{RGB}{55,126,184}

\path[fill=fillColor] ( 77.04,107.61) rectangle (242.02,126.28);
\definecolor{drawColor}{RGB}{0,0,0}

\node[text=drawColor,anchor=base,inner sep=0pt, outer sep=0pt, scale=  0.85] at (302.09, 82.88) {48.93{\%}};

\node[text=drawColor,anchor=base,inner sep=0pt, outer sep=0pt, scale=  0.85] at (142.12, 82.88) {51.07{\%}};

\node[text=drawColor,anchor=base,inner sep=0pt, outer sep=0pt, scale=  0.85] at (303.46,114.00) {48.21{\%}};

\node[text=drawColor,anchor=base,inner sep=0pt, outer sep=0pt, scale=  0.85] at (143.03,114.00) {51.79{\%}};
\end{scope}
\begin{scope}
\path[clip] (411.55,215.10) rectangle (428.12,283.58);
\definecolor{drawColor}{gray}{0.10}

\node[text=drawColor,rotate=-90.00,anchor=base,inner sep=0pt, outer sep=0pt, scale=  0.70] at (416.80,249.34) {\textbf{Región Centro}};
\end{scope}
\begin{scope}
\path[clip] (411.55,141.12) rectangle (428.12,209.60);
\definecolor{drawColor}{gray}{0.10}

\node[text=drawColor,rotate=-90.00,anchor=base,inner sep=0pt, outer sep=0pt, scale=  0.70] at (416.80,175.36) {\textbf{Región Norte}};
\end{scope}
\begin{scope}
\path[clip] (411.55, 67.14) rectangle (428.12,135.62);
\definecolor{drawColor}{gray}{0.10}

\node[text=drawColor,rotate=-90.00,anchor=base,inner sep=0pt, outer sep=0pt, scale=  0.70] at (416.80,101.38) {\textbf{Región Sur}};
\end{scope}
\begin{scope}
\path[clip] (  0.00,  0.00) rectangle (433.62,289.08);
\definecolor{drawColor}{gray}{0.20}

\path[draw=drawColor,line width= 0.6pt,line join=round] ( 77.04, 64.39) --
	( 77.04, 67.14);

\path[draw=drawColor,line width= 0.6pt,line join=round] (156.69, 64.39) --
	(156.69, 67.14);

\path[draw=drawColor,line width= 0.6pt,line join=round] (236.33, 64.39) --
	(236.33, 67.14);

\path[draw=drawColor,line width= 0.6pt,line join=round] (315.97, 64.39) --
	(315.97, 67.14);

\path[draw=drawColor,line width= 0.6pt,line join=round] (395.62, 64.39) --
	(395.62, 67.14);
\end{scope}
\begin{scope}
\path[clip] (  0.00,  0.00) rectangle (433.62,289.08);
\definecolor{drawColor}{RGB}{0,0,0}

\node[text=drawColor,anchor=base,inner sep=0pt, outer sep=0pt, scale=  0.88] at ( 77.04, 56.13) {0{\%}};

\node[text=drawColor,anchor=base,inner sep=0pt, outer sep=0pt, scale=  0.88] at (156.69, 56.13) {25{\%}};

\node[text=drawColor,anchor=base,inner sep=0pt, outer sep=0pt, scale=  0.88] at (236.33, 56.13) {50{\%}};

\node[text=drawColor,anchor=base,inner sep=0pt, outer sep=0pt, scale=  0.88] at (315.97, 56.13) {75{\%}};

\node[text=drawColor,anchor=base,inner sep=0pt, outer sep=0pt, scale=  0.88] at (395.62, 56.13) {100{\%}};
\end{scope}
\begin{scope}
\path[clip] (  0.00,  0.00) rectangle (433.62,289.08);
\definecolor{drawColor}{RGB}{0,0,0}

\node[text=drawColor,anchor=base east,inner sep=0pt, outer sep=0pt, scale=  0.88] at ( 56.16,230.75) {Migrantes};

\node[text=drawColor,anchor=base east,inner sep=0pt, outer sep=0pt, scale=  0.88] at ( 56.16,261.87) {Nativos};
\end{scope}
\begin{scope}
\path[clip] (  0.00,  0.00) rectangle (433.62,289.08);
\definecolor{drawColor}{gray}{0.20}

\path[draw=drawColor,line width= 0.6pt,line join=round] ( 58.36,233.78) --
	( 61.11,233.78);

\path[draw=drawColor,line width= 0.6pt,line join=round] ( 58.36,264.90) --
	( 61.11,264.90);
\end{scope}
\begin{scope}
\path[clip] (  0.00,  0.00) rectangle (433.62,289.08);
\definecolor{drawColor}{RGB}{0,0,0}

\node[text=drawColor,anchor=base east,inner sep=0pt, outer sep=0pt, scale=  0.88] at ( 56.16,156.77) {Migrantes};

\node[text=drawColor,anchor=base east,inner sep=0pt, outer sep=0pt, scale=  0.88] at ( 56.16,187.89) {Nativos};
\end{scope}
\begin{scope}
\path[clip] (  0.00,  0.00) rectangle (433.62,289.08);
\definecolor{drawColor}{gray}{0.20}

\path[draw=drawColor,line width= 0.6pt,line join=round] ( 58.36,159.80) --
	( 61.11,159.80);

\path[draw=drawColor,line width= 0.6pt,line join=round] ( 58.36,190.92) --
	( 61.11,190.92);
\end{scope}
\begin{scope}
\path[clip] (  0.00,  0.00) rectangle (433.62,289.08);
\definecolor{drawColor}{RGB}{0,0,0}

\node[text=drawColor,anchor=base east,inner sep=0pt, outer sep=0pt, scale=  0.88] at ( 56.16, 82.79) {Migrantes};

\node[text=drawColor,anchor=base east,inner sep=0pt, outer sep=0pt, scale=  0.88] at ( 56.16,113.91) {Nativos};
\end{scope}
\begin{scope}
\path[clip] (  0.00,  0.00) rectangle (433.62,289.08);
\definecolor{drawColor}{gray}{0.20}

\path[draw=drawColor,line width= 0.6pt,line join=round] ( 58.36, 85.82) --
	( 61.11, 85.82);

\path[draw=drawColor,line width= 0.6pt,line join=round] ( 58.36,116.94) --
	( 61.11,116.94);
\end{scope}
\begin{scope}
\path[clip] (  0.00,  0.00) rectangle (433.62,289.08);
\definecolor{fillColor}{RGB}{255,255,255}

\path[fill=fillColor] (173.29,  5.50) rectangle (299.37, 30.95);
\end{scope}
\begin{scope}
\path[clip] (  0.00,  0.00) rectangle (433.62,289.08);
\definecolor{fillColor}{gray}{0.95}

\path[fill=fillColor] (184.29, 11.00) rectangle (198.74, 25.45);
\end{scope}
\begin{scope}
\path[clip] (  0.00,  0.00) rectangle (433.62,289.08);
\definecolor{fillColor}{RGB}{228,26,28}

\path[fill=fillColor] (185.00, 11.71) rectangle (198.03, 24.74);
\end{scope}
\begin{scope}
\path[clip] (  0.00,  0.00) rectangle (433.62,289.08);
\definecolor{fillColor}{gray}{0.95}

\path[fill=fillColor] (243.54, 11.00) rectangle (257.99, 25.45);
\end{scope}
\begin{scope}
\path[clip] (  0.00,  0.00) rectangle (433.62,289.08);
\definecolor{fillColor}{RGB}{55,126,184}

\path[fill=fillColor] (244.25, 11.71) rectangle (257.28, 24.74);
\end{scope}
\begin{scope}
\path[clip] (  0.00,  0.00) rectangle (433.62,289.08);
\definecolor{drawColor}{RGB}{0,0,0}

\node[text=drawColor,anchor=base west,inner sep=0pt, outer sep=0pt, scale=  0.70] at (204.24, 15.20) {Hombres};
\end{scope}
\begin{scope}
\path[clip] (  0.00,  0.00) rectangle (433.62,289.08);
\definecolor{drawColor}{RGB}{0,0,0}

\node[text=drawColor,anchor=base west,inner sep=0pt, outer sep=0pt, scale=  0.70] at (263.49, 15.20) {Mujeres};
\end{scope}
\end{tikzpicture}
 
 	\label{figure:sexo_mig_}
\begin{flushleft}
\begin{scriptsize}
Fuente: Elaboración propia en base a EPH.\\
Nota: Los migrantes están definidos como personas que vivían hace cinco años en otra provincia. Los nativos están definidos como personas que nacieron y viven en la misma provincia. Las estimaciones corresponden al período desde el segundo trimestre de 2016 hasta el cuarto trimestre de 2019.
\end{scriptsize}
\end{flushleft}
\end{center}
\end{figure}

\newpage
La edad de las personas no solo afecta el horizonte temporal en el que toman las decisiones, sino que también condiciona el nivel de aversión al riesgo, escalas de incentivos e inclusive determina preferencias y hábitos.

En la Figura \ref{figure:edad_mig} se puede observar la función de densidad de la edad de los nativos y migrantes de cada una de las tres regiones. La densidad de la edad de los migrantes está distribuida  en valores de edad considerablemente menores que la de los nativos, y esta relación se cumple para las tres regiones. Donde es más notoria para la región centro en donde la edad de la población migrante difiere considerablemente de las características etarias de la población nativa, la cual deja entrever una población más avejentada en términos relativos.

La mayoría de los migrantes se encuentran dentro del rango etario de la Población Económicamente Activa. Las personas que se encuentran dentro de la edad laboral tienen una mayor probabilidad de sortear los obstáculos de conseguir trabajo una vez que deciden migrar.

\begin{figure}[ht!]
\begin{center}
\caption{\\Edad de los nativos y migrantes por regiones}
% Created by tikzDevice version 0.12.3.1 on 2021-07-01 11:54:32
% !TEX encoding = UTF-8 Unicode
\begin{tikzpicture}[x=1pt,y=1pt]
\definecolor{fillColor}{RGB}{255,255,255}
\path[use as bounding box,fill=fillColor,fill opacity=0.00] (0,0) rectangle (433.62,289.08);
\begin{scope}
\path[clip] (  0.00,  0.00) rectangle (433.62,289.08);
\definecolor{drawColor}{RGB}{255,255,255}
\definecolor{fillColor}{RGB}{255,255,255}

\path[draw=drawColor,line width= 0.6pt,line join=round,line cap=round,fill=fillColor] (  0.00, -0.00) rectangle (433.62,289.08);
\end{scope}
\begin{scope}
\path[clip] ( 41.49, 67.14) rectangle (166.70,267.01);
\definecolor{drawColor}{RGB}{255,255,255}

\path[draw=drawColor,line width= 0.3pt,line join=round] ( 41.49,104.00) --
	(166.70,104.00);

\path[draw=drawColor,line width= 0.3pt,line join=round] ( 41.49,159.56) --
	(166.70,159.56);

\path[draw=drawColor,line width= 0.3pt,line join=round] ( 41.49,215.11) --
	(166.70,215.11);

\path[draw=drawColor,line width= 0.3pt,line join=round] ( 52.87, 67.14) --
	( 52.87,267.01);

\path[draw=drawColor,line width= 0.3pt,line join=round] ( 87.02, 67.14) --
	( 87.02,267.01);

\path[draw=drawColor,line width= 0.3pt,line join=round] (121.17, 67.14) --
	(121.17,267.01);

\path[draw=drawColor,line width= 0.3pt,line join=round] (155.32, 67.14) --
	(155.32,267.01);

\path[draw=drawColor,line width= 0.6pt,line join=round] ( 41.49, 76.22) --
	(166.70, 76.22);

\path[draw=drawColor,line width= 0.6pt,line join=round] ( 41.49,131.78) --
	(166.70,131.78);

\path[draw=drawColor,line width= 0.6pt,line join=round] ( 41.49,187.33) --
	(166.70,187.33);

\path[draw=drawColor,line width= 0.6pt,line join=round] ( 41.49,242.89) --
	(166.70,242.89);

\path[draw=drawColor,line width= 0.6pt,line join=round] ( 69.94, 67.14) --
	( 69.94,267.01);

\path[draw=drawColor,line width= 0.6pt,line join=round] (104.09, 67.14) --
	(104.09,267.01);

\path[draw=drawColor,line width= 0.6pt,line join=round] (138.24, 67.14) --
	(138.24,267.01);
\definecolor{fillColor}{RGB}{228,26,28}

\path[fill=fillColor,fill opacity=0.50] ( 47.18, 76.25) --
	( 47.40, 76.26) --
	( 47.62, 76.28) --
	( 47.85, 76.30) --
	( 48.07, 76.33) --
	( 48.29, 76.37) --
	( 48.52, 76.42) --
	( 48.74, 76.49) --
	( 48.96, 76.58) --
	( 49.18, 76.69) --
	( 49.41, 76.84) --
	( 49.63, 77.04) --
	( 49.85, 77.29) --
	( 50.07, 77.60) --
	( 50.30, 77.99) --
	( 50.52, 78.46) --
	( 50.74, 79.04) --
	( 50.97, 79.76) --
	( 51.19, 80.64) --
	( 51.41, 81.69) --
	( 51.63, 82.93) --
	( 51.86, 84.39) --
	( 52.08, 86.09) --
	( 52.30, 88.06) --
	( 52.53, 90.38) --
	( 52.75, 93.03) --
	( 52.97, 96.03) --
	( 53.19, 99.38) --
	( 53.42,103.12) --
	( 53.64,107.25) --
	( 53.86,111.79) --
	( 54.08,116.80) --
	( 54.31,122.20) --
	( 54.53,127.99) --
	( 54.75,134.13) --
	( 54.98,140.62) --
	( 55.20,147.40) --
	( 55.42,154.47) --
	( 55.64,161.77) --
	( 55.87,169.21) --
	( 56.09,176.74) --
	( 56.31,184.30) --
	( 56.53,191.82) --
	( 56.76,199.25) --
	( 56.98,206.48) --
	( 57.20,213.43) --
	( 57.43,220.07) --
	( 57.65,226.32) --
	( 57.87,232.13) --
	( 58.09,237.46) --
	( 58.32,242.27) --
	( 58.54,246.43) --
	( 58.76,249.93) --
	( 58.99,252.81) --
	( 59.21,255.05) --
	( 59.43,256.64) --
	( 59.65,257.60) --
	( 59.88,257.92) --
	( 60.10,257.54) --
	( 60.32,256.56) --
	( 60.54,255.05) --
	( 60.77,253.04) --
	( 60.99,250.58) --
	( 61.21,247.71) --
	( 61.44,244.49) --
	( 61.66,240.91) --
	( 61.88,237.12) --
	( 62.10,233.15) --
	( 62.33,229.07) --
	( 62.55,224.92) --
	( 62.77,220.74) --
	( 62.99,216.58) --
	( 63.22,212.51) --
	( 63.44,208.54) --
	( 63.66,204.70) --
	( 63.89,201.02) --
	( 64.11,197.50) --
	( 64.33,194.16) --
	( 64.55,191.03) --
	( 64.78,188.10) --
	( 65.00,185.36) --
	( 65.22,182.79) --
	( 65.44,180.38) --
	( 65.67,178.11) --
	( 65.89,175.99) --
	( 66.11,173.99) --
	( 66.34,172.09) --
	( 66.56,170.27) --
	( 66.78,168.52) --
	( 67.00,166.81) --
	( 67.23,165.14) --
	( 67.45,163.49) --
	( 67.67,161.87) --
	( 67.90,160.25) --
	( 68.12,158.63) --
	( 68.34,157.02) --
	( 68.56,155.42) --
	( 68.79,153.82) --
	( 69.01,152.22) --
	( 69.23,150.65) --
	( 69.45,149.10) --
	( 69.68,147.58) --
	( 69.90,146.09) --
	( 70.12,144.65) --
	( 70.35,143.26) --
	( 70.57,141.92) --
	( 70.79,140.65) --
	( 71.01,139.44) --
	( 71.24,138.31) --
	( 71.46,137.24) --
	( 71.68,136.25) --
	( 71.90,135.32) --
	( 72.13,134.47) --
	( 72.35,133.69) --
	( 72.57,132.97) --
	( 72.80,132.31) --
	( 73.02,131.70) --
	( 73.24,131.14) --
	( 73.46,130.62) --
	( 73.69,130.13) --
	( 73.91,129.66) --
	( 74.13,129.21) --
	( 74.36,128.77) --
	( 74.58,128.33) --
	( 74.80,127.88) --
	( 75.02,127.42) --
	( 75.25,126.93) --
	( 75.47,126.42) --
	( 75.69,125.88) --
	( 75.91,125.31) --
	( 76.14,124.69) --
	( 76.36,124.05) --
	( 76.58,123.36) --
	( 76.81,122.63) --
	( 77.03,121.86) --
	( 77.25,121.07) --
	( 77.47,120.24) --
	( 77.70,119.39) --
	( 77.92,118.52) --
	( 78.14,117.63) --
	( 78.36,116.74) --
	( 78.59,115.85) --
	( 78.81,114.96) --
	( 79.03,114.08) --
	( 79.26,113.22) --
	( 79.48,112.39) --
	( 79.70,111.58) --
	( 79.92,110.81) --
	( 80.15,110.07) --
	( 80.37,109.38) --
	( 80.59,108.72) --
	( 80.82,108.10) --
	( 81.04,107.53) --
	( 81.26,107.00) --
	( 81.48,106.51) --
	( 81.71,106.06) --
	( 81.93,105.65) --
	( 82.15,105.27) --
	( 82.37,104.92) --
	( 82.60,104.60) --
	( 82.82,104.30) --
	( 83.04,104.02) --
	( 83.27,103.77) --
	( 83.49,103.52) --
	( 83.71,103.29) --
	( 83.93,103.06) --
	( 84.16,102.84) --
	( 84.38,102.62) --
	( 84.60,102.40) --
	( 84.82,102.17) --
	( 85.05,101.94) --
	( 85.27,101.71) --
	( 85.49,101.47) --
	( 85.72,101.21) --
	( 85.94,100.95) --
	( 86.16,100.68) --
	( 86.38,100.39) --
	( 86.61,100.09) --
	( 86.83, 99.78) --
	( 87.05, 99.46) --
	( 87.28, 99.13) --
	( 87.50, 98.78) --
	( 87.72, 98.43) --
	( 87.94, 98.07) --
	( 88.17, 97.70) --
	( 88.39, 97.32) --
	( 88.61, 96.95) --
	( 88.83, 96.57) --
	( 89.06, 96.19) --
	( 89.28, 95.82) --
	( 89.50, 95.45) --
	( 89.73, 95.10) --
	( 89.95, 94.76) --
	( 90.17, 94.43) --
	( 90.39, 94.11) --
	( 90.62, 93.82) --
	( 90.84, 93.54) --
	( 91.06, 93.29) --
	( 91.28, 93.06) --
	( 91.51, 92.84) --
	( 91.73, 92.65) --
	( 91.95, 92.47) --
	( 92.18, 92.32) --
	( 92.40, 92.18) --
	( 92.62, 92.05) --
	( 92.84, 91.93) --
	( 93.07, 91.83) --
	( 93.29, 91.73) --
	( 93.51, 91.64) --
	( 93.73, 91.54) --
	( 93.96, 91.45) --
	( 94.18, 91.36) --
	( 94.40, 91.26) --
	( 94.63, 91.16) --
	( 94.85, 91.06) --
	( 95.07, 90.95) --
	( 95.29, 90.83) --
	( 95.52, 90.71) --
	( 95.74, 90.59) --
	( 95.96, 90.46) --
	( 96.19, 90.33) --
	( 96.41, 90.20) --
	( 96.63, 90.07) --
	( 96.85, 89.94) --
	( 97.08, 89.82) --
	( 97.30, 89.69) --
	( 97.52, 89.57) --
	( 97.74, 89.45) --
	( 97.97, 89.34) --
	( 98.19, 89.23) --
	( 98.41, 89.13) --
	( 98.64, 89.03) --
	( 98.86, 88.94) --
	( 99.08, 88.85) --
	( 99.30, 88.76) --
	( 99.53, 88.68) --
	( 99.75, 88.61) --
	( 99.97, 88.54) --
	(100.19, 88.48) --
	(100.42, 88.42) --
	(100.64, 88.37) --
	(100.86, 88.33) --
	(101.09, 88.29) --
	(101.31, 88.26) --
	(101.53, 88.23) --
	(101.75, 88.21) --
	(101.98, 88.20) --
	(102.20, 88.18) --
	(102.42, 88.18) --
	(102.65, 88.18) --
	(102.87, 88.18) --
	(103.09, 88.18) --
	(103.31, 88.19) --
	(103.54, 88.20) --
	(103.76, 88.22) --
	(103.98, 88.23) --
	(104.20, 88.25) --
	(104.43, 88.28) --
	(104.65, 88.31) --
	(104.87, 88.34) --
	(105.10, 88.38) --
	(105.32, 88.43) --
	(105.54, 88.48) --
	(105.76, 88.53) --
	(105.99, 88.59) --
	(106.21, 88.65) --
	(106.43, 88.71) --
	(106.65, 88.78) --
	(106.88, 88.84) --
	(107.10, 88.89) --
	(107.32, 88.94) --
	(107.55, 88.97) --
	(107.77, 88.99) --
	(107.99, 88.99) --
	(108.21, 88.97) --
	(108.44, 88.93) --
	(108.66, 88.87) --
	(108.88, 88.78) --
	(109.11, 88.66) --
	(109.33, 88.52) --
	(109.55, 88.35) --
	(109.77, 88.16) --
	(110.00, 87.94) --
	(110.22, 87.71) --
	(110.44, 87.46) --
	(110.66, 87.19) --
	(110.89, 86.91) --
	(111.11, 86.62) --
	(111.33, 86.33) --
	(111.56, 86.04) --
	(111.78, 85.75) --
	(112.00, 85.46) --
	(112.22, 85.19) --
	(112.45, 84.92) --
	(112.67, 84.67) --
	(112.89, 84.42) --
	(113.11, 84.19) --
	(113.34, 83.98) --
	(113.56, 83.78) --
	(113.78, 83.60) --
	(114.01, 83.43) --
	(114.23, 83.28) --
	(114.45, 83.14) --
	(114.67, 83.01) --
	(114.90, 82.90) --
	(115.12, 82.80) --
	(115.34, 82.71) --
	(115.56, 82.63) --
	(115.79, 82.56) --
	(116.01, 82.50) --
	(116.23, 82.44) --
	(116.46, 82.41) --
	(116.68, 82.38) --
	(116.90, 82.36) --
	(117.12, 82.35) --
	(117.35, 82.35) --
	(117.57, 82.36) --
	(117.79, 82.39) --
	(118.02, 82.42) --
	(118.24, 82.46) --
	(118.46, 82.51) --
	(118.68, 82.56) --
	(118.91, 82.62) --
	(119.13, 82.68) --
	(119.35, 82.75) --
	(119.57, 82.81) --
	(119.80, 82.86) --
	(120.02, 82.91) --
	(120.24, 82.95) --
	(120.47, 82.98) --
	(120.69, 83.00) --
	(120.91, 83.01) --
	(121.13, 83.00) --
	(121.36, 82.97) --
	(121.58, 82.93) --
	(121.80, 82.88) --
	(122.02, 82.81) --
	(122.25, 82.72) --
	(122.47, 82.62) --
	(122.69, 82.51) --
	(122.92, 82.39) --
	(123.14, 82.26) --
	(123.36, 82.13) --
	(123.58, 81.98) --
	(123.81, 81.83) --
	(124.03, 81.68) --
	(124.25, 81.53) --
	(124.48, 81.38) --
	(124.70, 81.23) --
	(124.92, 81.08) --
	(125.14, 80.94) --
	(125.37, 80.80) --
	(125.59, 80.66) --
	(125.81, 80.53) --
	(126.03, 80.41) --
	(126.26, 80.30) --
	(126.48, 80.19) --
	(126.70, 80.08) --
	(126.93, 79.99) --
	(127.15, 79.90) --
	(127.37, 79.83) --
	(127.59, 79.76) --
	(127.82, 79.70) --
	(128.04, 79.65) --
	(128.26, 79.60) --
	(128.48, 79.57) --
	(128.71, 79.55) --
	(128.93, 79.53) --
	(129.15, 79.53) --
	(129.38, 79.53) --
	(129.60, 79.55) --
	(129.82, 79.57) --
	(130.04, 79.60) --
	(130.27, 79.63) --
	(130.49, 79.68) --
	(130.71, 79.72) --
	(130.94, 79.77) --
	(131.16, 79.82) --
	(131.38, 79.88) --
	(131.60, 79.93) --
	(131.83, 79.97) --
	(132.05, 80.02) --
	(132.27, 80.05) --
	(132.49, 80.08) --
	(132.72, 80.10) --
	(132.94, 80.10) --
	(133.16, 80.09) --
	(133.39, 80.07) --
	(133.61, 80.04) --
	(133.83, 79.99) --
	(134.05, 79.93) --
	(134.28, 79.85) --
	(134.50, 79.77) --
	(134.72, 79.67) --
	(134.94, 79.56) --
	(135.17, 79.44) --
	(135.39, 79.32) --
	(135.61, 79.19) --
	(135.84, 79.05) --
	(136.06, 78.92) --
	(136.28, 78.78) --
	(136.50, 78.65) --
	(136.73, 78.52) --
	(136.95, 78.39) --
	(137.17, 78.26) --
	(137.39, 78.14) --
	(137.62, 78.03) --
	(137.84, 77.92) --
	(138.06, 77.82) --
	(138.29, 77.73) --
	(138.51, 77.64) --
	(138.73, 77.56) --
	(138.95, 77.48) --
	(139.18, 77.41) --
	(139.40, 77.34) --
	(139.62, 77.28) --
	(139.85, 77.22) --
	(140.07, 77.16) --
	(140.29, 77.10) --
	(140.51, 77.05) --
	(140.74, 77.00) --
	(140.96, 76.95) --
	(141.18, 76.90) --
	(141.40, 76.85) --
	(141.63, 76.80) --
	(141.85, 76.76) --
	(142.07, 76.71) --
	(142.30, 76.67) --
	(142.52, 76.63) --
	(142.74, 76.59) --
	(142.96, 76.55) --
	(143.19, 76.52) --
	(143.41, 76.49) --
	(143.63, 76.45) --
	(143.85, 76.43) --
	(144.08, 76.40) --
	(144.30, 76.38) --
	(144.52, 76.35) --
	(144.75, 76.33) --
	(144.97, 76.32) --
	(145.19, 76.30) --
	(145.41, 76.29) --
	(145.64, 76.28) --
	(145.86, 76.27) --
	(146.08, 76.26) --
	(146.31, 76.25) --
	(146.53, 76.25) --
	(146.75, 76.24) --
	(146.97, 76.24) --
	(147.20, 76.24) --
	(147.42, 76.23) --
	(147.64, 76.23) --
	(147.86, 76.23) --
	(148.09, 76.23) --
	(148.31, 76.23) --
	(148.53, 76.23) --
	(148.76, 76.23) --
	(148.98, 76.23) --
	(149.20, 76.23) --
	(149.42, 76.23) --
	(149.65, 76.23) --
	(149.87, 76.23) --
	(150.09, 76.22) --
	(150.31, 76.22) --
	(150.54, 76.22) --
	(150.76, 76.22) --
	(150.98, 76.22) --
	(151.21, 76.22) --
	(151.43, 76.22) --
	(151.65, 76.22) --
	(151.87, 76.22) --
	(152.10, 76.22) --
	(152.32, 76.22) --
	(152.54, 76.22) --
	(152.77, 76.22) --
	(152.99, 76.22) --
	(153.21, 76.22) --
	(153.43, 76.22) --
	(153.66, 76.22) --
	(153.88, 76.22) --
	(154.10, 76.22) --
	(154.32, 76.22) --
	(154.55, 76.22) --
	(154.77, 76.22) --
	(154.99, 76.22) --
	(155.22, 76.22) --
	(155.44, 76.22) --
	(155.66, 76.22) --
	(155.88, 76.22) --
	(156.11, 76.22) --
	(156.33, 76.22) --
	(156.55, 76.22) --
	(156.77, 76.22) --
	(157.00, 76.22) --
	(157.22, 76.22) --
	(157.44, 76.22) --
	(157.67, 76.22) --
	(157.89, 76.22) --
	(158.11, 76.22) --
	(158.33, 76.22) --
	(158.56, 76.22) --
	(158.78, 76.22) --
	(159.00, 76.22) --
	(159.22, 76.22) --
	(159.45, 76.22) --
	(159.67, 76.22) --
	(159.89, 76.22) --
	(160.12, 76.22) --
	(160.34, 76.22) --
	(160.56, 76.22) --
	(160.78, 76.22) --
	(161.01, 76.22) --
	(161.01, 76.22) --
	(160.78, 76.22) --
	(160.56, 76.22) --
	(160.34, 76.22) --
	(160.12, 76.22) --
	(159.89, 76.22) --
	(159.67, 76.22) --
	(159.45, 76.22) --
	(159.22, 76.22) --
	(159.00, 76.22) --
	(158.78, 76.22) --
	(158.56, 76.22) --
	(158.33, 76.22) --
	(158.11, 76.22) --
	(157.89, 76.22) --
	(157.67, 76.22) --
	(157.44, 76.22) --
	(157.22, 76.22) --
	(157.00, 76.22) --
	(156.77, 76.22) --
	(156.55, 76.22) --
	(156.33, 76.22) --
	(156.11, 76.22) --
	(155.88, 76.22) --
	(155.66, 76.22) --
	(155.44, 76.22) --
	(155.22, 76.22) --
	(154.99, 76.22) --
	(154.77, 76.22) --
	(154.55, 76.22) --
	(154.32, 76.22) --
	(154.10, 76.22) --
	(153.88, 76.22) --
	(153.66, 76.22) --
	(153.43, 76.22) --
	(153.21, 76.22) --
	(152.99, 76.22) --
	(152.77, 76.22) --
	(152.54, 76.22) --
	(152.32, 76.22) --
	(152.10, 76.22) --
	(151.87, 76.22) --
	(151.65, 76.22) --
	(151.43, 76.22) --
	(151.21, 76.22) --
	(150.98, 76.22) --
	(150.76, 76.22) --
	(150.54, 76.22) --
	(150.31, 76.22) --
	(150.09, 76.22) --
	(149.87, 76.22) --
	(149.65, 76.22) --
	(149.42, 76.22) --
	(149.20, 76.22) --
	(148.98, 76.22) --
	(148.76, 76.22) --
	(148.53, 76.22) --
	(148.31, 76.22) --
	(148.09, 76.22) --
	(147.86, 76.22) --
	(147.64, 76.22) --
	(147.42, 76.22) --
	(147.20, 76.22) --
	(146.97, 76.22) --
	(146.75, 76.22) --
	(146.53, 76.22) --
	(146.31, 76.22) --
	(146.08, 76.22) --
	(145.86, 76.22) --
	(145.64, 76.22) --
	(145.41, 76.22) --
	(145.19, 76.22) --
	(144.97, 76.22) --
	(144.75, 76.22) --
	(144.52, 76.22) --
	(144.30, 76.22) --
	(144.08, 76.22) --
	(143.85, 76.22) --
	(143.63, 76.22) --
	(143.41, 76.22) --
	(143.19, 76.22) --
	(142.96, 76.22) --
	(142.74, 76.22) --
	(142.52, 76.22) --
	(142.30, 76.22) --
	(142.07, 76.22) --
	(141.85, 76.22) --
	(141.63, 76.22) --
	(141.40, 76.22) --
	(141.18, 76.22) --
	(140.96, 76.22) --
	(140.74, 76.22) --
	(140.51, 76.22) --
	(140.29, 76.22) --
	(140.07, 76.22) --
	(139.85, 76.22) --
	(139.62, 76.22) --
	(139.40, 76.22) --
	(139.18, 76.22) --
	(138.95, 76.22) --
	(138.73, 76.22) --
	(138.51, 76.22) --
	(138.29, 76.22) --
	(138.06, 76.22) --
	(137.84, 76.22) --
	(137.62, 76.22) --
	(137.39, 76.22) --
	(137.17, 76.22) --
	(136.95, 76.22) --
	(136.73, 76.22) --
	(136.50, 76.22) --
	(136.28, 76.22) --
	(136.06, 76.22) --
	(135.84, 76.22) --
	(135.61, 76.22) --
	(135.39, 76.22) --
	(135.17, 76.22) --
	(134.94, 76.22) --
	(134.72, 76.22) --
	(134.50, 76.22) --
	(134.28, 76.22) --
	(134.05, 76.22) --
	(133.83, 76.22) --
	(133.61, 76.22) --
	(133.39, 76.22) --
	(133.16, 76.22) --
	(132.94, 76.22) --
	(132.72, 76.22) --
	(132.49, 76.22) --
	(132.27, 76.22) --
	(132.05, 76.22) --
	(131.83, 76.22) --
	(131.60, 76.22) --
	(131.38, 76.22) --
	(131.16, 76.22) --
	(130.94, 76.22) --
	(130.71, 76.22) --
	(130.49, 76.22) --
	(130.27, 76.22) --
	(130.04, 76.22) --
	(129.82, 76.22) --
	(129.60, 76.22) --
	(129.38, 76.22) --
	(129.15, 76.22) --
	(128.93, 76.22) --
	(128.71, 76.22) --
	(128.48, 76.22) --
	(128.26, 76.22) --
	(128.04, 76.22) --
	(127.82, 76.22) --
	(127.59, 76.22) --
	(127.37, 76.22) --
	(127.15, 76.22) --
	(126.93, 76.22) --
	(126.70, 76.22) --
	(126.48, 76.22) --
	(126.26, 76.22) --
	(126.03, 76.22) --
	(125.81, 76.22) --
	(125.59, 76.22) --
	(125.37, 76.22) --
	(125.14, 76.22) --
	(124.92, 76.22) --
	(124.70, 76.22) --
	(124.48, 76.22) --
	(124.25, 76.22) --
	(124.03, 76.22) --
	(123.81, 76.22) --
	(123.58, 76.22) --
	(123.36, 76.22) --
	(123.14, 76.22) --
	(122.92, 76.22) --
	(122.69, 76.22) --
	(122.47, 76.22) --
	(122.25, 76.22) --
	(122.02, 76.22) --
	(121.80, 76.22) --
	(121.58, 76.22) --
	(121.36, 76.22) --
	(121.13, 76.22) --
	(120.91, 76.22) --
	(120.69, 76.22) --
	(120.47, 76.22) --
	(120.24, 76.22) --
	(120.02, 76.22) --
	(119.80, 76.22) --
	(119.57, 76.22) --
	(119.35, 76.22) --
	(119.13, 76.22) --
	(118.91, 76.22) --
	(118.68, 76.22) --
	(118.46, 76.22) --
	(118.24, 76.22) --
	(118.02, 76.22) --
	(117.79, 76.22) --
	(117.57, 76.22) --
	(117.35, 76.22) --
	(117.12, 76.22) --
	(116.90, 76.22) --
	(116.68, 76.22) --
	(116.46, 76.22) --
	(116.23, 76.22) --
	(116.01, 76.22) --
	(115.79, 76.22) --
	(115.56, 76.22) --
	(115.34, 76.22) --
	(115.12, 76.22) --
	(114.90, 76.22) --
	(114.67, 76.22) --
	(114.45, 76.22) --
	(114.23, 76.22) --
	(114.01, 76.22) --
	(113.78, 76.22) --
	(113.56, 76.22) --
	(113.34, 76.22) --
	(113.11, 76.22) --
	(112.89, 76.22) --
	(112.67, 76.22) --
	(112.45, 76.22) --
	(112.22, 76.22) --
	(112.00, 76.22) --
	(111.78, 76.22) --
	(111.56, 76.22) --
	(111.33, 76.22) --
	(111.11, 76.22) --
	(110.89, 76.22) --
	(110.66, 76.22) --
	(110.44, 76.22) --
	(110.22, 76.22) --
	(110.00, 76.22) --
	(109.77, 76.22) --
	(109.55, 76.22) --
	(109.33, 76.22) --
	(109.11, 76.22) --
	(108.88, 76.22) --
	(108.66, 76.22) --
	(108.44, 76.22) --
	(108.21, 76.22) --
	(107.99, 76.22) --
	(107.77, 76.22) --
	(107.55, 76.22) --
	(107.32, 76.22) --
	(107.10, 76.22) --
	(106.88, 76.22) --
	(106.65, 76.22) --
	(106.43, 76.22) --
	(106.21, 76.22) --
	(105.99, 76.22) --
	(105.76, 76.22) --
	(105.54, 76.22) --
	(105.32, 76.22) --
	(105.10, 76.22) --
	(104.87, 76.22) --
	(104.65, 76.22) --
	(104.43, 76.22) --
	(104.20, 76.22) --
	(103.98, 76.22) --
	(103.76, 76.22) --
	(103.54, 76.22) --
	(103.31, 76.22) --
	(103.09, 76.22) --
	(102.87, 76.22) --
	(102.65, 76.22) --
	(102.42, 76.22) --
	(102.20, 76.22) --
	(101.98, 76.22) --
	(101.75, 76.22) --
	(101.53, 76.22) --
	(101.31, 76.22) --
	(101.09, 76.22) --
	(100.86, 76.22) --
	(100.64, 76.22) --
	(100.42, 76.22) --
	(100.19, 76.22) --
	( 99.97, 76.22) --
	( 99.75, 76.22) --
	( 99.53, 76.22) --
	( 99.30, 76.22) --
	( 99.08, 76.22) --
	( 98.86, 76.22) --
	( 98.64, 76.22) --
	( 98.41, 76.22) --
	( 98.19, 76.22) --
	( 97.97, 76.22) --
	( 97.74, 76.22) --
	( 97.52, 76.22) --
	( 97.30, 76.22) --
	( 97.08, 76.22) --
	( 96.85, 76.22) --
	( 96.63, 76.22) --
	( 96.41, 76.22) --
	( 96.19, 76.22) --
	( 95.96, 76.22) --
	( 95.74, 76.22) --
	( 95.52, 76.22) --
	( 95.29, 76.22) --
	( 95.07, 76.22) --
	( 94.85, 76.22) --
	( 94.63, 76.22) --
	( 94.40, 76.22) --
	( 94.18, 76.22) --
	( 93.96, 76.22) --
	( 93.73, 76.22) --
	( 93.51, 76.22) --
	( 93.29, 76.22) --
	( 93.07, 76.22) --
	( 92.84, 76.22) --
	( 92.62, 76.22) --
	( 92.40, 76.22) --
	( 92.18, 76.22) --
	( 91.95, 76.22) --
	( 91.73, 76.22) --
	( 91.51, 76.22) --
	( 91.28, 76.22) --
	( 91.06, 76.22) --
	( 90.84, 76.22) --
	( 90.62, 76.22) --
	( 90.39, 76.22) --
	( 90.17, 76.22) --
	( 89.95, 76.22) --
	( 89.73, 76.22) --
	( 89.50, 76.22) --
	( 89.28, 76.22) --
	( 89.06, 76.22) --
	( 88.83, 76.22) --
	( 88.61, 76.22) --
	( 88.39, 76.22) --
	( 88.17, 76.22) --
	( 87.94, 76.22) --
	( 87.72, 76.22) --
	( 87.50, 76.22) --
	( 87.28, 76.22) --
	( 87.05, 76.22) --
	( 86.83, 76.22) --
	( 86.61, 76.22) --
	( 86.38, 76.22) --
	( 86.16, 76.22) --
	( 85.94, 76.22) --
	( 85.72, 76.22) --
	( 85.49, 76.22) --
	( 85.27, 76.22) --
	( 85.05, 76.22) --
	( 84.82, 76.22) --
	( 84.60, 76.22) --
	( 84.38, 76.22) --
	( 84.16, 76.22) --
	( 83.93, 76.22) --
	( 83.71, 76.22) --
	( 83.49, 76.22) --
	( 83.27, 76.22) --
	( 83.04, 76.22) --
	( 82.82, 76.22) --
	( 82.60, 76.22) --
	( 82.37, 76.22) --
	( 82.15, 76.22) --
	( 81.93, 76.22) --
	( 81.71, 76.22) --
	( 81.48, 76.22) --
	( 81.26, 76.22) --
	( 81.04, 76.22) --
	( 80.82, 76.22) --
	( 80.59, 76.22) --
	( 80.37, 76.22) --
	( 80.15, 76.22) --
	( 79.92, 76.22) --
	( 79.70, 76.22) --
	( 79.48, 76.22) --
	( 79.26, 76.22) --
	( 79.03, 76.22) --
	( 78.81, 76.22) --
	( 78.59, 76.22) --
	( 78.36, 76.22) --
	( 78.14, 76.22) --
	( 77.92, 76.22) --
	( 77.70, 76.22) --
	( 77.47, 76.22) --
	( 77.25, 76.22) --
	( 77.03, 76.22) --
	( 76.81, 76.22) --
	( 76.58, 76.22) --
	( 76.36, 76.22) --
	( 76.14, 76.22) --
	( 75.91, 76.22) --
	( 75.69, 76.22) --
	( 75.47, 76.22) --
	( 75.25, 76.22) --
	( 75.02, 76.22) --
	( 74.80, 76.22) --
	( 74.58, 76.22) --
	( 74.36, 76.22) --
	( 74.13, 76.22) --
	( 73.91, 76.22) --
	( 73.69, 76.22) --
	( 73.46, 76.22) --
	( 73.24, 76.22) --
	( 73.02, 76.22) --
	( 72.80, 76.22) --
	( 72.57, 76.22) --
	( 72.35, 76.22) --
	( 72.13, 76.22) --
	( 71.90, 76.22) --
	( 71.68, 76.22) --
	( 71.46, 76.22) --
	( 71.24, 76.22) --
	( 71.01, 76.22) --
	( 70.79, 76.22) --
	( 70.57, 76.22) --
	( 70.35, 76.22) --
	( 70.12, 76.22) --
	( 69.90, 76.22) --
	( 69.68, 76.22) --
	( 69.45, 76.22) --
	( 69.23, 76.22) --
	( 69.01, 76.22) --
	( 68.79, 76.22) --
	( 68.56, 76.22) --
	( 68.34, 76.22) --
	( 68.12, 76.22) --
	( 67.90, 76.22) --
	( 67.67, 76.22) --
	( 67.45, 76.22) --
	( 67.23, 76.22) --
	( 67.00, 76.22) --
	( 66.78, 76.22) --
	( 66.56, 76.22) --
	( 66.34, 76.22) --
	( 66.11, 76.22) --
	( 65.89, 76.22) --
	( 65.67, 76.22) --
	( 65.44, 76.22) --
	( 65.22, 76.22) --
	( 65.00, 76.22) --
	( 64.78, 76.22) --
	( 64.55, 76.22) --
	( 64.33, 76.22) --
	( 64.11, 76.22) --
	( 63.89, 76.22) --
	( 63.66, 76.22) --
	( 63.44, 76.22) --
	( 63.22, 76.22) --
	( 62.99, 76.22) --
	( 62.77, 76.22) --
	( 62.55, 76.22) --
	( 62.33, 76.22) --
	( 62.10, 76.22) --
	( 61.88, 76.22) --
	( 61.66, 76.22) --
	( 61.44, 76.22) --
	( 61.21, 76.22) --
	( 60.99, 76.22) --
	( 60.77, 76.22) --
	( 60.54, 76.22) --
	( 60.32, 76.22) --
	( 60.10, 76.22) --
	( 59.88, 76.22) --
	( 59.65, 76.22) --
	( 59.43, 76.22) --
	( 59.21, 76.22) --
	( 58.99, 76.22) --
	( 58.76, 76.22) --
	( 58.54, 76.22) --
	( 58.32, 76.22) --
	( 58.09, 76.22) --
	( 57.87, 76.22) --
	( 57.65, 76.22) --
	( 57.43, 76.22) --
	( 57.20, 76.22) --
	( 56.98, 76.22) --
	( 56.76, 76.22) --
	( 56.53, 76.22) --
	( 56.31, 76.22) --
	( 56.09, 76.22) --
	( 55.87, 76.22) --
	( 55.64, 76.22) --
	( 55.42, 76.22) --
	( 55.20, 76.22) --
	( 54.98, 76.22) --
	( 54.75, 76.22) --
	( 54.53, 76.22) --
	( 54.31, 76.22) --
	( 54.08, 76.22) --
	( 53.86, 76.22) --
	( 53.64, 76.22) --
	( 53.42, 76.22) --
	( 53.19, 76.22) --
	( 52.97, 76.22) --
	( 52.75, 76.22) --
	( 52.53, 76.22) --
	( 52.30, 76.22) --
	( 52.08, 76.22) --
	( 51.86, 76.22) --
	( 51.63, 76.22) --
	( 51.41, 76.22) --
	( 51.19, 76.22) --
	( 50.97, 76.22) --
	( 50.74, 76.22) --
	( 50.52, 76.22) --
	( 50.30, 76.22) --
	( 50.07, 76.22) --
	( 49.85, 76.22) --
	( 49.63, 76.22) --
	( 49.41, 76.22) --
	( 49.18, 76.22) --
	( 48.96, 76.22) --
	( 48.74, 76.22) --
	( 48.52, 76.22) --
	( 48.29, 76.22) --
	( 48.07, 76.22) --
	( 47.85, 76.22) --
	( 47.62, 76.22) --
	( 47.40, 76.22) --
	( 47.18, 76.22) --
	cycle;
\definecolor{drawColor}{RGB}{0,0,0}

\path[draw=drawColor,line width= 0.6pt,line join=round,line cap=round] ( 47.18, 76.25) --
	( 47.40, 76.26) --
	( 47.62, 76.28) --
	( 47.85, 76.30) --
	( 48.07, 76.33) --
	( 48.29, 76.37) --
	( 48.52, 76.42) --
	( 48.74, 76.49) --
	( 48.96, 76.58) --
	( 49.18, 76.69) --
	( 49.41, 76.84) --
	( 49.63, 77.04) --
	( 49.85, 77.29) --
	( 50.07, 77.60) --
	( 50.30, 77.99) --
	( 50.52, 78.46) --
	( 50.74, 79.04) --
	( 50.97, 79.76) --
	( 51.19, 80.64) --
	( 51.41, 81.69) --
	( 51.63, 82.93) --
	( 51.86, 84.39) --
	( 52.08, 86.09) --
	( 52.30, 88.06) --
	( 52.53, 90.38) --
	( 52.75, 93.03) --
	( 52.97, 96.03) --
	( 53.19, 99.38) --
	( 53.42,103.12) --
	( 53.64,107.25) --
	( 53.86,111.79) --
	( 54.08,116.80) --
	( 54.31,122.20) --
	( 54.53,127.99) --
	( 54.75,134.13) --
	( 54.98,140.62) --
	( 55.20,147.40) --
	( 55.42,154.47) --
	( 55.64,161.77) --
	( 55.87,169.21) --
	( 56.09,176.74) --
	( 56.31,184.30) --
	( 56.53,191.82) --
	( 56.76,199.25) --
	( 56.98,206.48) --
	( 57.20,213.43) --
	( 57.43,220.07) --
	( 57.65,226.32) --
	( 57.87,232.13) --
	( 58.09,237.46) --
	( 58.32,242.27) --
	( 58.54,246.43) --
	( 58.76,249.93) --
	( 58.99,252.81) --
	( 59.21,255.05) --
	( 59.43,256.64) --
	( 59.65,257.60) --
	( 59.88,257.92) --
	( 60.10,257.54) --
	( 60.32,256.56) --
	( 60.54,255.05) --
	( 60.77,253.04) --
	( 60.99,250.58) --
	( 61.21,247.71) --
	( 61.44,244.49) --
	( 61.66,240.91) --
	( 61.88,237.12) --
	( 62.10,233.15) --
	( 62.33,229.07) --
	( 62.55,224.92) --
	( 62.77,220.74) --
	( 62.99,216.58) --
	( 63.22,212.51) --
	( 63.44,208.54) --
	( 63.66,204.70) --
	( 63.89,201.02) --
	( 64.11,197.50) --
	( 64.33,194.16) --
	( 64.55,191.03) --
	( 64.78,188.10) --
	( 65.00,185.36) --
	( 65.22,182.79) --
	( 65.44,180.38) --
	( 65.67,178.11) --
	( 65.89,175.99) --
	( 66.11,173.99) --
	( 66.34,172.09) --
	( 66.56,170.27) --
	( 66.78,168.52) --
	( 67.00,166.81) --
	( 67.23,165.14) --
	( 67.45,163.49) --
	( 67.67,161.87) --
	( 67.90,160.25) --
	( 68.12,158.63) --
	( 68.34,157.02) --
	( 68.56,155.42) --
	( 68.79,153.82) --
	( 69.01,152.22) --
	( 69.23,150.65) --
	( 69.45,149.10) --
	( 69.68,147.58) --
	( 69.90,146.09) --
	( 70.12,144.65) --
	( 70.35,143.26) --
	( 70.57,141.92) --
	( 70.79,140.65) --
	( 71.01,139.44) --
	( 71.24,138.31) --
	( 71.46,137.24) --
	( 71.68,136.25) --
	( 71.90,135.32) --
	( 72.13,134.47) --
	( 72.35,133.69) --
	( 72.57,132.97) --
	( 72.80,132.31) --
	( 73.02,131.70) --
	( 73.24,131.14) --
	( 73.46,130.62) --
	( 73.69,130.13) --
	( 73.91,129.66) --
	( 74.13,129.21) --
	( 74.36,128.77) --
	( 74.58,128.33) --
	( 74.80,127.88) --
	( 75.02,127.42) --
	( 75.25,126.93) --
	( 75.47,126.42) --
	( 75.69,125.88) --
	( 75.91,125.31) --
	( 76.14,124.69) --
	( 76.36,124.05) --
	( 76.58,123.36) --
	( 76.81,122.63) --
	( 77.03,121.86) --
	( 77.25,121.07) --
	( 77.47,120.24) --
	( 77.70,119.39) --
	( 77.92,118.52) --
	( 78.14,117.63) --
	( 78.36,116.74) --
	( 78.59,115.85) --
	( 78.81,114.96) --
	( 79.03,114.08) --
	( 79.26,113.22) --
	( 79.48,112.39) --
	( 79.70,111.58) --
	( 79.92,110.81) --
	( 80.15,110.07) --
	( 80.37,109.38) --
	( 80.59,108.72) --
	( 80.82,108.10) --
	( 81.04,107.53) --
	( 81.26,107.00) --
	( 81.48,106.51) --
	( 81.71,106.06) --
	( 81.93,105.65) --
	( 82.15,105.27) --
	( 82.37,104.92) --
	( 82.60,104.60) --
	( 82.82,104.30) --
	( 83.04,104.02) --
	( 83.27,103.77) --
	( 83.49,103.52) --
	( 83.71,103.29) --
	( 83.93,103.06) --
	( 84.16,102.84) --
	( 84.38,102.62) --
	( 84.60,102.40) --
	( 84.82,102.17) --
	( 85.05,101.94) --
	( 85.27,101.71) --
	( 85.49,101.47) --
	( 85.72,101.21) --
	( 85.94,100.95) --
	( 86.16,100.68) --
	( 86.38,100.39) --
	( 86.61,100.09) --
	( 86.83, 99.78) --
	( 87.05, 99.46) --
	( 87.28, 99.13) --
	( 87.50, 98.78) --
	( 87.72, 98.43) --
	( 87.94, 98.07) --
	( 88.17, 97.70) --
	( 88.39, 97.32) --
	( 88.61, 96.95) --
	( 88.83, 96.57) --
	( 89.06, 96.19) --
	( 89.28, 95.82) --
	( 89.50, 95.45) --
	( 89.73, 95.10) --
	( 89.95, 94.76) --
	( 90.17, 94.43) --
	( 90.39, 94.11) --
	( 90.62, 93.82) --
	( 90.84, 93.54) --
	( 91.06, 93.29) --
	( 91.28, 93.06) --
	( 91.51, 92.84) --
	( 91.73, 92.65) --
	( 91.95, 92.47) --
	( 92.18, 92.32) --
	( 92.40, 92.18) --
	( 92.62, 92.05) --
	( 92.84, 91.93) --
	( 93.07, 91.83) --
	( 93.29, 91.73) --
	( 93.51, 91.64) --
	( 93.73, 91.54) --
	( 93.96, 91.45) --
	( 94.18, 91.36) --
	( 94.40, 91.26) --
	( 94.63, 91.16) --
	( 94.85, 91.06) --
	( 95.07, 90.95) --
	( 95.29, 90.83) --
	( 95.52, 90.71) --
	( 95.74, 90.59) --
	( 95.96, 90.46) --
	( 96.19, 90.33) --
	( 96.41, 90.20) --
	( 96.63, 90.07) --
	( 96.85, 89.94) --
	( 97.08, 89.82) --
	( 97.30, 89.69) --
	( 97.52, 89.57) --
	( 97.74, 89.45) --
	( 97.97, 89.34) --
	( 98.19, 89.23) --
	( 98.41, 89.13) --
	( 98.64, 89.03) --
	( 98.86, 88.94) --
	( 99.08, 88.85) --
	( 99.30, 88.76) --
	( 99.53, 88.68) --
	( 99.75, 88.61) --
	( 99.97, 88.54) --
	(100.19, 88.48) --
	(100.42, 88.42) --
	(100.64, 88.37) --
	(100.86, 88.33) --
	(101.09, 88.29) --
	(101.31, 88.26) --
	(101.53, 88.23) --
	(101.75, 88.21) --
	(101.98, 88.20) --
	(102.20, 88.18) --
	(102.42, 88.18) --
	(102.65, 88.18) --
	(102.87, 88.18) --
	(103.09, 88.18) --
	(103.31, 88.19) --
	(103.54, 88.20) --
	(103.76, 88.22) --
	(103.98, 88.23) --
	(104.20, 88.25) --
	(104.43, 88.28) --
	(104.65, 88.31) --
	(104.87, 88.34) --
	(105.10, 88.38) --
	(105.32, 88.43) --
	(105.54, 88.48) --
	(105.76, 88.53) --
	(105.99, 88.59) --
	(106.21, 88.65) --
	(106.43, 88.71) --
	(106.65, 88.78) --
	(106.88, 88.84) --
	(107.10, 88.89) --
	(107.32, 88.94) --
	(107.55, 88.97) --
	(107.77, 88.99) --
	(107.99, 88.99) --
	(108.21, 88.97) --
	(108.44, 88.93) --
	(108.66, 88.87) --
	(108.88, 88.78) --
	(109.11, 88.66) --
	(109.33, 88.52) --
	(109.55, 88.35) --
	(109.77, 88.16) --
	(110.00, 87.94) --
	(110.22, 87.71) --
	(110.44, 87.46) --
	(110.66, 87.19) --
	(110.89, 86.91) --
	(111.11, 86.62) --
	(111.33, 86.33) --
	(111.56, 86.04) --
	(111.78, 85.75) --
	(112.00, 85.46) --
	(112.22, 85.19) --
	(112.45, 84.92) --
	(112.67, 84.67) --
	(112.89, 84.42) --
	(113.11, 84.19) --
	(113.34, 83.98) --
	(113.56, 83.78) --
	(113.78, 83.60) --
	(114.01, 83.43) --
	(114.23, 83.28) --
	(114.45, 83.14) --
	(114.67, 83.01) --
	(114.90, 82.90) --
	(115.12, 82.80) --
	(115.34, 82.71) --
	(115.56, 82.63) --
	(115.79, 82.56) --
	(116.01, 82.50) --
	(116.23, 82.44) --
	(116.46, 82.41) --
	(116.68, 82.38) --
	(116.90, 82.36) --
	(117.12, 82.35) --
	(117.35, 82.35) --
	(117.57, 82.36) --
	(117.79, 82.39) --
	(118.02, 82.42) --
	(118.24, 82.46) --
	(118.46, 82.51) --
	(118.68, 82.56) --
	(118.91, 82.62) --
	(119.13, 82.68) --
	(119.35, 82.75) --
	(119.57, 82.81) --
	(119.80, 82.86) --
	(120.02, 82.91) --
	(120.24, 82.95) --
	(120.47, 82.98) --
	(120.69, 83.00) --
	(120.91, 83.01) --
	(121.13, 83.00) --
	(121.36, 82.97) --
	(121.58, 82.93) --
	(121.80, 82.88) --
	(122.02, 82.81) --
	(122.25, 82.72) --
	(122.47, 82.62) --
	(122.69, 82.51) --
	(122.92, 82.39) --
	(123.14, 82.26) --
	(123.36, 82.13) --
	(123.58, 81.98) --
	(123.81, 81.83) --
	(124.03, 81.68) --
	(124.25, 81.53) --
	(124.48, 81.38) --
	(124.70, 81.23) --
	(124.92, 81.08) --
	(125.14, 80.94) --
	(125.37, 80.80) --
	(125.59, 80.66) --
	(125.81, 80.53) --
	(126.03, 80.41) --
	(126.26, 80.30) --
	(126.48, 80.19) --
	(126.70, 80.08) --
	(126.93, 79.99) --
	(127.15, 79.90) --
	(127.37, 79.83) --
	(127.59, 79.76) --
	(127.82, 79.70) --
	(128.04, 79.65) --
	(128.26, 79.60) --
	(128.48, 79.57) --
	(128.71, 79.55) --
	(128.93, 79.53) --
	(129.15, 79.53) --
	(129.38, 79.53) --
	(129.60, 79.55) --
	(129.82, 79.57) --
	(130.04, 79.60) --
	(130.27, 79.63) --
	(130.49, 79.68) --
	(130.71, 79.72) --
	(130.94, 79.77) --
	(131.16, 79.82) --
	(131.38, 79.88) --
	(131.60, 79.93) --
	(131.83, 79.97) --
	(132.05, 80.02) --
	(132.27, 80.05) --
	(132.49, 80.08) --
	(132.72, 80.10) --
	(132.94, 80.10) --
	(133.16, 80.09) --
	(133.39, 80.07) --
	(133.61, 80.04) --
	(133.83, 79.99) --
	(134.05, 79.93) --
	(134.28, 79.85) --
	(134.50, 79.77) --
	(134.72, 79.67) --
	(134.94, 79.56) --
	(135.17, 79.44) --
	(135.39, 79.32) --
	(135.61, 79.19) --
	(135.84, 79.05) --
	(136.06, 78.92) --
	(136.28, 78.78) --
	(136.50, 78.65) --
	(136.73, 78.52) --
	(136.95, 78.39) --
	(137.17, 78.26) --
	(137.39, 78.14) --
	(137.62, 78.03) --
	(137.84, 77.92) --
	(138.06, 77.82) --
	(138.29, 77.73) --
	(138.51, 77.64) --
	(138.73, 77.56) --
	(138.95, 77.48) --
	(139.18, 77.41) --
	(139.40, 77.34) --
	(139.62, 77.28) --
	(139.85, 77.22) --
	(140.07, 77.16) --
	(140.29, 77.10) --
	(140.51, 77.05) --
	(140.74, 77.00) --
	(140.96, 76.95) --
	(141.18, 76.90) --
	(141.40, 76.85) --
	(141.63, 76.80) --
	(141.85, 76.76) --
	(142.07, 76.71) --
	(142.30, 76.67) --
	(142.52, 76.63) --
	(142.74, 76.59) --
	(142.96, 76.55) --
	(143.19, 76.52) --
	(143.41, 76.49) --
	(143.63, 76.45) --
	(143.85, 76.43) --
	(144.08, 76.40) --
	(144.30, 76.38) --
	(144.52, 76.35) --
	(144.75, 76.33) --
	(144.97, 76.32) --
	(145.19, 76.30) --
	(145.41, 76.29) --
	(145.64, 76.28) --
	(145.86, 76.27) --
	(146.08, 76.26) --
	(146.31, 76.25) --
	(146.53, 76.25) --
	(146.75, 76.24) --
	(146.97, 76.24) --
	(147.20, 76.24) --
	(147.42, 76.23) --
	(147.64, 76.23) --
	(147.86, 76.23) --
	(148.09, 76.23) --
	(148.31, 76.23) --
	(148.53, 76.23) --
	(148.76, 76.23) --
	(148.98, 76.23) --
	(149.20, 76.23) --
	(149.42, 76.23) --
	(149.65, 76.23) --
	(149.87, 76.23) --
	(150.09, 76.22) --
	(150.31, 76.22) --
	(150.54, 76.22) --
	(150.76, 76.22) --
	(150.98, 76.22) --
	(151.21, 76.22) --
	(151.43, 76.22) --
	(151.65, 76.22) --
	(151.87, 76.22) --
	(152.10, 76.22) --
	(152.32, 76.22) --
	(152.54, 76.22) --
	(152.77, 76.22) --
	(152.99, 76.22) --
	(153.21, 76.22) --
	(153.43, 76.22) --
	(153.66, 76.22) --
	(153.88, 76.22) --
	(154.10, 76.22) --
	(154.32, 76.22) --
	(154.55, 76.22) --
	(154.77, 76.22) --
	(154.99, 76.22) --
	(155.22, 76.22) --
	(155.44, 76.22) --
	(155.66, 76.22) --
	(155.88, 76.22) --
	(156.11, 76.22) --
	(156.33, 76.22) --
	(156.55, 76.22) --
	(156.77, 76.22) --
	(157.00, 76.22) --
	(157.22, 76.22) --
	(157.44, 76.22) --
	(157.67, 76.22) --
	(157.89, 76.22) --
	(158.11, 76.22) --
	(158.33, 76.22) --
	(158.56, 76.22) --
	(158.78, 76.22) --
	(159.00, 76.22) --
	(159.22, 76.22) --
	(159.45, 76.22) --
	(159.67, 76.22) --
	(159.89, 76.22) --
	(160.12, 76.22) --
	(160.34, 76.22) --
	(160.56, 76.22) --
	(160.78, 76.22) --
	(161.01, 76.22);
\definecolor{fillColor}{RGB}{55,126,184}

\path[fill=fillColor,fill opacity=0.50] ( 47.18, 76.22) --
	( 47.40, 76.22) --
	( 47.62, 76.22) --
	( 47.85, 76.22) --
	( 48.07, 76.22) --
	( 48.29, 76.22) --
	( 48.52, 76.22) --
	( 48.74, 76.23) --
	( 48.96, 76.23) --
	( 49.18, 76.23) --
	( 49.41, 76.23) --
	( 49.63, 76.23) --
	( 49.85, 76.23) --
	( 50.07, 76.24) --
	( 50.30, 76.25) --
	( 50.52, 76.26) --
	( 50.74, 76.29) --
	( 50.97, 76.32) --
	( 51.19, 76.38) --
	( 51.41, 76.46) --
	( 51.63, 76.59) --
	( 51.86, 76.76) --
	( 52.08, 77.02) --
	( 52.30, 77.36) --
	( 52.53, 77.83) --
	( 52.75, 78.45) --
	( 52.97, 79.25) --
	( 53.19, 80.26) --
	( 53.42, 81.53) --
	( 53.64, 83.07) --
	( 53.86, 84.97) --
	( 54.08, 87.20) --
	( 54.31, 89.77) --
	( 54.53, 92.66) --
	( 54.75, 95.87) --
	( 54.98, 99.36) --
	( 55.20,103.08) --
	( 55.42,106.97) --
	( 55.64,110.97) --
	( 55.87,114.99) --
	( 56.09,118.95) --
	( 56.31,122.76) --
	( 56.53,126.37) --
	( 56.76,129.73) --
	( 56.98,132.78) --
	( 57.20,135.52) --
	( 57.43,137.92) --
	( 57.65,139.98) --
	( 57.87,141.71) --
	( 58.09,143.14) --
	( 58.32,144.26) --
	( 58.54,145.14) --
	( 58.76,145.81) --
	( 58.99,146.30) --
	( 59.21,146.64) --
	( 59.43,146.85) --
	( 59.65,146.94) --
	( 59.88,146.95) --
	( 60.10,146.87) --
	( 60.32,146.72) --
	( 60.54,146.51) --
	( 60.77,146.24) --
	( 60.99,145.93) --
	( 61.21,145.57) --
	( 61.44,145.17) --
	( 61.66,144.75) --
	( 61.88,144.30) --
	( 62.10,143.83) --
	( 62.33,143.35) --
	( 62.55,142.87) --
	( 62.77,142.39) --
	( 62.99,141.93) --
	( 63.22,141.49) --
	( 63.44,141.07) --
	( 63.66,140.67) --
	( 63.89,140.30) --
	( 64.11,139.96) --
	( 64.33,139.64) --
	( 64.55,139.34) --
	( 64.78,139.06) --
	( 65.00,138.79) --
	( 65.22,138.53) --
	( 65.44,138.26) --
	( 65.67,137.98) --
	( 65.89,137.69) --
	( 66.11,137.38) --
	( 66.34,137.04) --
	( 66.56,136.69) --
	( 66.78,136.31) --
	( 67.00,135.90) --
	( 67.23,135.47) --
	( 67.45,135.04) --
	( 67.67,134.59) --
	( 67.90,134.14) --
	( 68.12,133.69) --
	( 68.34,133.25) --
	( 68.56,132.83) --
	( 68.79,132.42) --
	( 69.01,132.03) --
	( 69.23,131.66) --
	( 69.45,131.32) --
	( 69.68,131.00) --
	( 69.90,130.71) --
	( 70.12,130.44) --
	( 70.35,130.20) --
	( 70.57,129.98) --
	( 70.79,129.79) --
	( 71.01,129.62) --
	( 71.24,129.48) --
	( 71.46,129.35) --
	( 71.68,129.25) --
	( 71.90,129.17) --
	( 72.13,129.10) --
	( 72.35,129.04) --
	( 72.57,128.99) --
	( 72.80,128.96) --
	( 73.02,128.94) --
	( 73.24,128.92) --
	( 73.46,128.93) --
	( 73.69,128.94) --
	( 73.91,128.97) --
	( 74.13,129.02) --
	( 74.36,129.09) --
	( 74.58,129.17) --
	( 74.80,129.27) --
	( 75.02,129.39) --
	( 75.25,129.52) --
	( 75.47,129.66) --
	( 75.69,129.81) --
	( 75.91,129.95) --
	( 76.14,130.10) --
	( 76.36,130.23) --
	( 76.58,130.36) --
	( 76.81,130.47) --
	( 77.03,130.56) --
	( 77.25,130.64) --
	( 77.47,130.70) --
	( 77.70,130.73) --
	( 77.92,130.75) --
	( 78.14,130.75) --
	( 78.36,130.73) --
	( 78.59,130.70) --
	( 78.81,130.66) --
	( 79.03,130.60) --
	( 79.26,130.54) --
	( 79.48,130.47) --
	( 79.70,130.39) --
	( 79.92,130.30) --
	( 80.15,130.20) --
	( 80.37,130.10) --
	( 80.59,129.98) --
	( 80.82,129.86) --
	( 81.04,129.72) --
	( 81.26,129.57) --
	( 81.48,129.40) --
	( 81.71,129.22) --
	( 81.93,129.03) --
	( 82.15,128.82) --
	( 82.37,128.60) --
	( 82.60,128.37) --
	( 82.82,128.13) --
	( 83.04,127.88) --
	( 83.27,127.63) --
	( 83.49,127.37) --
	( 83.71,127.10) --
	( 83.93,126.84) --
	( 84.16,126.57) --
	( 84.38,126.31) --
	( 84.60,126.04) --
	( 84.82,125.79) --
	( 85.05,125.54) --
	( 85.27,125.29) --
	( 85.49,125.06) --
	( 85.72,124.83) --
	( 85.94,124.61) --
	( 86.16,124.40) --
	( 86.38,124.19) --
	( 86.61,124.00) --
	( 86.83,123.80) --
	( 87.05,123.61) --
	( 87.28,123.41) --
	( 87.50,123.21) --
	( 87.72,123.01) --
	( 87.94,122.79) --
	( 88.17,122.56) --
	( 88.39,122.32) --
	( 88.61,122.07) --
	( 88.83,121.80) --
	( 89.06,121.52) --
	( 89.28,121.23) --
	( 89.50,120.94) --
	( 89.73,120.64) --
	( 89.95,120.34) --
	( 90.17,120.05) --
	( 90.39,119.76) --
	( 90.62,119.48) --
	( 90.84,119.22) --
	( 91.06,118.97) --
	( 91.28,118.74) --
	( 91.51,118.53) --
	( 91.73,118.34) --
	( 91.95,118.16) --
	( 92.18,118.00) --
	( 92.40,117.86) --
	( 92.62,117.72) --
	( 92.84,117.60) --
	( 93.07,117.48) --
	( 93.29,117.38) --
	( 93.51,117.27) --
	( 93.73,117.17) --
	( 93.96,117.07) --
	( 94.18,116.97) --
	( 94.40,116.88) --
	( 94.63,116.78) --
	( 94.85,116.68) --
	( 95.07,116.57) --
	( 95.29,116.46) --
	( 95.52,116.35) --
	( 95.74,116.24) --
	( 95.96,116.13) --
	( 96.19,116.01) --
	( 96.41,115.89) --
	( 96.63,115.77) --
	( 96.85,115.65) --
	( 97.08,115.53) --
	( 97.30,115.42) --
	( 97.52,115.30) --
	( 97.74,115.18) --
	( 97.97,115.06) --
	( 98.19,114.95) --
	( 98.41,114.82) --
	( 98.64,114.70) --
	( 98.86,114.57) --
	( 99.08,114.43) --
	( 99.30,114.29) --
	( 99.53,114.14) --
	( 99.75,113.98) --
	( 99.97,113.82) --
	(100.19,113.65) --
	(100.42,113.47) --
	(100.64,113.29) --
	(100.86,113.10) --
	(101.09,112.92) --
	(101.31,112.73) --
	(101.53,112.54) --
	(101.75,112.34) --
	(101.98,112.15) --
	(102.20,111.95) --
	(102.42,111.76) --
	(102.65,111.55) --
	(102.87,111.35) --
	(103.09,111.14) --
	(103.31,110.92) --
	(103.54,110.70) --
	(103.76,110.47) --
	(103.98,110.24) --
	(104.20,110.00) --
	(104.43,109.75) --
	(104.65,109.51) --
	(104.87,109.26) --
	(105.10,109.01) --
	(105.32,108.77) --
	(105.54,108.52) --
	(105.76,108.29) --
	(105.99,108.07) --
	(106.21,107.86) --
	(106.43,107.66) --
	(106.65,107.47) --
	(106.88,107.30) --
	(107.10,107.15) --
	(107.32,107.01) --
	(107.55,106.88) --
	(107.77,106.77) --
	(107.99,106.66) --
	(108.21,106.57) --
	(108.44,106.48) --
	(108.66,106.40) --
	(108.88,106.32) --
	(109.11,106.24) --
	(109.33,106.15) --
	(109.55,106.06) --
	(109.77,105.96) --
	(110.00,105.84) --
	(110.22,105.72) --
	(110.44,105.59) --
	(110.66,105.44) --
	(110.89,105.28) --
	(111.11,105.10) --
	(111.33,104.92) --
	(111.56,104.72) --
	(111.78,104.51) --
	(112.00,104.30) --
	(112.22,104.08) --
	(112.45,103.85) --
	(112.67,103.62) --
	(112.89,103.39) --
	(113.11,103.15) --
	(113.34,102.91) --
	(113.56,102.68) --
	(113.78,102.44) --
	(114.01,102.20) --
	(114.23,101.95) --
	(114.45,101.71) --
	(114.67,101.47) --
	(114.90,101.22) --
	(115.12,100.97) --
	(115.34,100.72) --
	(115.56,100.47) --
	(115.79,100.22) --
	(116.01, 99.96) --
	(116.23, 99.71) --
	(116.46, 99.46) --
	(116.68, 99.21) --
	(116.90, 98.96) --
	(117.12, 98.71) --
	(117.35, 98.47) --
	(117.57, 98.23) --
	(117.79, 98.00) --
	(118.02, 97.77) --
	(118.24, 97.55) --
	(118.46, 97.33) --
	(118.68, 97.12) --
	(118.91, 96.91) --
	(119.13, 96.71) --
	(119.35, 96.51) --
	(119.57, 96.31) --
	(119.80, 96.11) --
	(120.02, 95.91) --
	(120.24, 95.71) --
	(120.47, 95.51) --
	(120.69, 95.30) --
	(120.91, 95.09) --
	(121.13, 94.86) --
	(121.36, 94.64) --
	(121.58, 94.40) --
	(121.80, 94.15) --
	(122.02, 93.90) --
	(122.25, 93.65) --
	(122.47, 93.39) --
	(122.69, 93.12) --
	(122.92, 92.86) --
	(123.14, 92.59) --
	(123.36, 92.33) --
	(123.58, 92.07) --
	(123.81, 91.81) --
	(124.03, 91.56) --
	(124.25, 91.32) --
	(124.48, 91.08) --
	(124.70, 90.85) --
	(124.92, 90.62) --
	(125.14, 90.41) --
	(125.37, 90.20) --
	(125.59, 89.99) --
	(125.81, 89.80) --
	(126.03, 89.61) --
	(126.26, 89.43) --
	(126.48, 89.25) --
	(126.70, 89.08) --
	(126.93, 88.91) --
	(127.15, 88.75) --
	(127.37, 88.58) --
	(127.59, 88.41) --
	(127.82, 88.24) --
	(128.04, 88.07) --
	(128.26, 87.89) --
	(128.48, 87.71) --
	(128.71, 87.52) --
	(128.93, 87.33) --
	(129.15, 87.14) --
	(129.38, 86.93) --
	(129.60, 86.73) --
	(129.82, 86.52) --
	(130.04, 86.32) --
	(130.27, 86.11) --
	(130.49, 85.90) --
	(130.71, 85.69) --
	(130.94, 85.49) --
	(131.16, 85.29) --
	(131.38, 85.09) --
	(131.60, 84.89) --
	(131.83, 84.69) --
	(132.05, 84.50) --
	(132.27, 84.31) --
	(132.49, 84.12) --
	(132.72, 83.94) --
	(132.94, 83.75) --
	(133.16, 83.57) --
	(133.39, 83.39) --
	(133.61, 83.22) --
	(133.83, 83.04) --
	(134.05, 82.87) --
	(134.28, 82.70) --
	(134.50, 82.53) --
	(134.72, 82.36) --
	(134.94, 82.20) --
	(135.17, 82.03) --
	(135.39, 81.87) --
	(135.61, 81.70) --
	(135.84, 81.54) --
	(136.06, 81.38) --
	(136.28, 81.22) --
	(136.50, 81.06) --
	(136.73, 80.90) --
	(136.95, 80.74) --
	(137.17, 80.59) --
	(137.39, 80.43) --
	(137.62, 80.28) --
	(137.84, 80.13) --
	(138.06, 79.98) --
	(138.29, 79.84) --
	(138.51, 79.70) --
	(138.73, 79.56) --
	(138.95, 79.42) --
	(139.18, 79.29) --
	(139.40, 79.17) --
	(139.62, 79.04) --
	(139.85, 78.92) --
	(140.07, 78.81) --
	(140.29, 78.69) --
	(140.51, 78.58) --
	(140.74, 78.48) --
	(140.96, 78.37) --
	(141.18, 78.27) --
	(141.40, 78.17) --
	(141.63, 78.06) --
	(141.85, 77.96) --
	(142.07, 77.86) --
	(142.30, 77.76) --
	(142.52, 77.67) --
	(142.74, 77.57) --
	(142.96, 77.48) --
	(143.19, 77.38) --
	(143.41, 77.30) --
	(143.63, 77.21) --
	(143.85, 77.13) --
	(144.08, 77.06) --
	(144.30, 76.99) --
	(144.52, 76.92) --
	(144.75, 76.86) --
	(144.97, 76.81) --
	(145.19, 76.76) --
	(145.41, 76.71) --
	(145.64, 76.67) --
	(145.86, 76.64) --
	(146.08, 76.61) --
	(146.31, 76.58) --
	(146.53, 76.56) --
	(146.75, 76.54) --
	(146.97, 76.52) --
	(147.20, 76.50) --
	(147.42, 76.48) --
	(147.64, 76.47) --
	(147.86, 76.45) --
	(148.09, 76.44) --
	(148.31, 76.42) --
	(148.53, 76.41) --
	(148.76, 76.39) --
	(148.98, 76.38) --
	(149.20, 76.36) --
	(149.42, 76.35) --
	(149.65, 76.33) --
	(149.87, 76.32) --
	(150.09, 76.31) --
	(150.31, 76.30) --
	(150.54, 76.29) --
	(150.76, 76.28) --
	(150.98, 76.27) --
	(151.21, 76.26) --
	(151.43, 76.26) --
	(151.65, 76.25) --
	(151.87, 76.25) --
	(152.10, 76.24) --
	(152.32, 76.24) --
	(152.54, 76.24) --
	(152.77, 76.23) --
	(152.99, 76.23) --
	(153.21, 76.23) --
	(153.43, 76.23) --
	(153.66, 76.23) --
	(153.88, 76.23) --
	(154.10, 76.23) --
	(154.32, 76.23) --
	(154.55, 76.23) --
	(154.77, 76.23) --
	(154.99, 76.23) --
	(155.22, 76.23) --
	(155.44, 76.23) --
	(155.66, 76.23) --
	(155.88, 76.22) --
	(156.11, 76.22) --
	(156.33, 76.22) --
	(156.55, 76.22) --
	(156.77, 76.22) --
	(157.00, 76.22) --
	(157.22, 76.22) --
	(157.44, 76.22) --
	(157.67, 76.22) --
	(157.89, 76.22) --
	(158.11, 76.22) --
	(158.33, 76.22) --
	(158.56, 76.22) --
	(158.78, 76.22) --
	(159.00, 76.22) --
	(159.22, 76.22) --
	(159.45, 76.22) --
	(159.67, 76.22) --
	(159.89, 76.22) --
	(160.12, 76.22) --
	(160.34, 76.22) --
	(160.56, 76.22) --
	(160.78, 76.22) --
	(161.01, 76.22) --
	(161.01, 76.22) --
	(160.78, 76.22) --
	(160.56, 76.22) --
	(160.34, 76.22) --
	(160.12, 76.22) --
	(159.89, 76.22) --
	(159.67, 76.22) --
	(159.45, 76.22) --
	(159.22, 76.22) --
	(159.00, 76.22) --
	(158.78, 76.22) --
	(158.56, 76.22) --
	(158.33, 76.22) --
	(158.11, 76.22) --
	(157.89, 76.22) --
	(157.67, 76.22) --
	(157.44, 76.22) --
	(157.22, 76.22) --
	(157.00, 76.22) --
	(156.77, 76.22) --
	(156.55, 76.22) --
	(156.33, 76.22) --
	(156.11, 76.22) --
	(155.88, 76.22) --
	(155.66, 76.22) --
	(155.44, 76.22) --
	(155.22, 76.22) --
	(154.99, 76.22) --
	(154.77, 76.22) --
	(154.55, 76.22) --
	(154.32, 76.22) --
	(154.10, 76.22) --
	(153.88, 76.22) --
	(153.66, 76.22) --
	(153.43, 76.22) --
	(153.21, 76.22) --
	(152.99, 76.22) --
	(152.77, 76.22) --
	(152.54, 76.22) --
	(152.32, 76.22) --
	(152.10, 76.22) --
	(151.87, 76.22) --
	(151.65, 76.22) --
	(151.43, 76.22) --
	(151.21, 76.22) --
	(150.98, 76.22) --
	(150.76, 76.22) --
	(150.54, 76.22) --
	(150.31, 76.22) --
	(150.09, 76.22) --
	(149.87, 76.22) --
	(149.65, 76.22) --
	(149.42, 76.22) --
	(149.20, 76.22) --
	(148.98, 76.22) --
	(148.76, 76.22) --
	(148.53, 76.22) --
	(148.31, 76.22) --
	(148.09, 76.22) --
	(147.86, 76.22) --
	(147.64, 76.22) --
	(147.42, 76.22) --
	(147.20, 76.22) --
	(146.97, 76.22) --
	(146.75, 76.22) --
	(146.53, 76.22) --
	(146.31, 76.22) --
	(146.08, 76.22) --
	(145.86, 76.22) --
	(145.64, 76.22) --
	(145.41, 76.22) --
	(145.19, 76.22) --
	(144.97, 76.22) --
	(144.75, 76.22) --
	(144.52, 76.22) --
	(144.30, 76.22) --
	(144.08, 76.22) --
	(143.85, 76.22) --
	(143.63, 76.22) --
	(143.41, 76.22) --
	(143.19, 76.22) --
	(142.96, 76.22) --
	(142.74, 76.22) --
	(142.52, 76.22) --
	(142.30, 76.22) --
	(142.07, 76.22) --
	(141.85, 76.22) --
	(141.63, 76.22) --
	(141.40, 76.22) --
	(141.18, 76.22) --
	(140.96, 76.22) --
	(140.74, 76.22) --
	(140.51, 76.22) --
	(140.29, 76.22) --
	(140.07, 76.22) --
	(139.85, 76.22) --
	(139.62, 76.22) --
	(139.40, 76.22) --
	(139.18, 76.22) --
	(138.95, 76.22) --
	(138.73, 76.22) --
	(138.51, 76.22) --
	(138.29, 76.22) --
	(138.06, 76.22) --
	(137.84, 76.22) --
	(137.62, 76.22) --
	(137.39, 76.22) --
	(137.17, 76.22) --
	(136.95, 76.22) --
	(136.73, 76.22) --
	(136.50, 76.22) --
	(136.28, 76.22) --
	(136.06, 76.22) --
	(135.84, 76.22) --
	(135.61, 76.22) --
	(135.39, 76.22) --
	(135.17, 76.22) --
	(134.94, 76.22) --
	(134.72, 76.22) --
	(134.50, 76.22) --
	(134.28, 76.22) --
	(134.05, 76.22) --
	(133.83, 76.22) --
	(133.61, 76.22) --
	(133.39, 76.22) --
	(133.16, 76.22) --
	(132.94, 76.22) --
	(132.72, 76.22) --
	(132.49, 76.22) --
	(132.27, 76.22) --
	(132.05, 76.22) --
	(131.83, 76.22) --
	(131.60, 76.22) --
	(131.38, 76.22) --
	(131.16, 76.22) --
	(130.94, 76.22) --
	(130.71, 76.22) --
	(130.49, 76.22) --
	(130.27, 76.22) --
	(130.04, 76.22) --
	(129.82, 76.22) --
	(129.60, 76.22) --
	(129.38, 76.22) --
	(129.15, 76.22) --
	(128.93, 76.22) --
	(128.71, 76.22) --
	(128.48, 76.22) --
	(128.26, 76.22) --
	(128.04, 76.22) --
	(127.82, 76.22) --
	(127.59, 76.22) --
	(127.37, 76.22) --
	(127.15, 76.22) --
	(126.93, 76.22) --
	(126.70, 76.22) --
	(126.48, 76.22) --
	(126.26, 76.22) --
	(126.03, 76.22) --
	(125.81, 76.22) --
	(125.59, 76.22) --
	(125.37, 76.22) --
	(125.14, 76.22) --
	(124.92, 76.22) --
	(124.70, 76.22) --
	(124.48, 76.22) --
	(124.25, 76.22) --
	(124.03, 76.22) --
	(123.81, 76.22) --
	(123.58, 76.22) --
	(123.36, 76.22) --
	(123.14, 76.22) --
	(122.92, 76.22) --
	(122.69, 76.22) --
	(122.47, 76.22) --
	(122.25, 76.22) --
	(122.02, 76.22) --
	(121.80, 76.22) --
	(121.58, 76.22) --
	(121.36, 76.22) --
	(121.13, 76.22) --
	(120.91, 76.22) --
	(120.69, 76.22) --
	(120.47, 76.22) --
	(120.24, 76.22) --
	(120.02, 76.22) --
	(119.80, 76.22) --
	(119.57, 76.22) --
	(119.35, 76.22) --
	(119.13, 76.22) --
	(118.91, 76.22) --
	(118.68, 76.22) --
	(118.46, 76.22) --
	(118.24, 76.22) --
	(118.02, 76.22) --
	(117.79, 76.22) --
	(117.57, 76.22) --
	(117.35, 76.22) --
	(117.12, 76.22) --
	(116.90, 76.22) --
	(116.68, 76.22) --
	(116.46, 76.22) --
	(116.23, 76.22) --
	(116.01, 76.22) --
	(115.79, 76.22) --
	(115.56, 76.22) --
	(115.34, 76.22) --
	(115.12, 76.22) --
	(114.90, 76.22) --
	(114.67, 76.22) --
	(114.45, 76.22) --
	(114.23, 76.22) --
	(114.01, 76.22) --
	(113.78, 76.22) --
	(113.56, 76.22) --
	(113.34, 76.22) --
	(113.11, 76.22) --
	(112.89, 76.22) --
	(112.67, 76.22) --
	(112.45, 76.22) --
	(112.22, 76.22) --
	(112.00, 76.22) --
	(111.78, 76.22) --
	(111.56, 76.22) --
	(111.33, 76.22) --
	(111.11, 76.22) --
	(110.89, 76.22) --
	(110.66, 76.22) --
	(110.44, 76.22) --
	(110.22, 76.22) --
	(110.00, 76.22) --
	(109.77, 76.22) --
	(109.55, 76.22) --
	(109.33, 76.22) --
	(109.11, 76.22) --
	(108.88, 76.22) --
	(108.66, 76.22) --
	(108.44, 76.22) --
	(108.21, 76.22) --
	(107.99, 76.22) --
	(107.77, 76.22) --
	(107.55, 76.22) --
	(107.32, 76.22) --
	(107.10, 76.22) --
	(106.88, 76.22) --
	(106.65, 76.22) --
	(106.43, 76.22) --
	(106.21, 76.22) --
	(105.99, 76.22) --
	(105.76, 76.22) --
	(105.54, 76.22) --
	(105.32, 76.22) --
	(105.10, 76.22) --
	(104.87, 76.22) --
	(104.65, 76.22) --
	(104.43, 76.22) --
	(104.20, 76.22) --
	(103.98, 76.22) --
	(103.76, 76.22) --
	(103.54, 76.22) --
	(103.31, 76.22) --
	(103.09, 76.22) --
	(102.87, 76.22) --
	(102.65, 76.22) --
	(102.42, 76.22) --
	(102.20, 76.22) --
	(101.98, 76.22) --
	(101.75, 76.22) --
	(101.53, 76.22) --
	(101.31, 76.22) --
	(101.09, 76.22) --
	(100.86, 76.22) --
	(100.64, 76.22) --
	(100.42, 76.22) --
	(100.19, 76.22) --
	( 99.97, 76.22) --
	( 99.75, 76.22) --
	( 99.53, 76.22) --
	( 99.30, 76.22) --
	( 99.08, 76.22) --
	( 98.86, 76.22) --
	( 98.64, 76.22) --
	( 98.41, 76.22) --
	( 98.19, 76.22) --
	( 97.97, 76.22) --
	( 97.74, 76.22) --
	( 97.52, 76.22) --
	( 97.30, 76.22) --
	( 97.08, 76.22) --
	( 96.85, 76.22) --
	( 96.63, 76.22) --
	( 96.41, 76.22) --
	( 96.19, 76.22) --
	( 95.96, 76.22) --
	( 95.74, 76.22) --
	( 95.52, 76.22) --
	( 95.29, 76.22) --
	( 95.07, 76.22) --
	( 94.85, 76.22) --
	( 94.63, 76.22) --
	( 94.40, 76.22) --
	( 94.18, 76.22) --
	( 93.96, 76.22) --
	( 93.73, 76.22) --
	( 93.51, 76.22) --
	( 93.29, 76.22) --
	( 93.07, 76.22) --
	( 92.84, 76.22) --
	( 92.62, 76.22) --
	( 92.40, 76.22) --
	( 92.18, 76.22) --
	( 91.95, 76.22) --
	( 91.73, 76.22) --
	( 91.51, 76.22) --
	( 91.28, 76.22) --
	( 91.06, 76.22) --
	( 90.84, 76.22) --
	( 90.62, 76.22) --
	( 90.39, 76.22) --
	( 90.17, 76.22) --
	( 89.95, 76.22) --
	( 89.73, 76.22) --
	( 89.50, 76.22) --
	( 89.28, 76.22) --
	( 89.06, 76.22) --
	( 88.83, 76.22) --
	( 88.61, 76.22) --
	( 88.39, 76.22) --
	( 88.17, 76.22) --
	( 87.94, 76.22) --
	( 87.72, 76.22) --
	( 87.50, 76.22) --
	( 87.28, 76.22) --
	( 87.05, 76.22) --
	( 86.83, 76.22) --
	( 86.61, 76.22) --
	( 86.38, 76.22) --
	( 86.16, 76.22) --
	( 85.94, 76.22) --
	( 85.72, 76.22) --
	( 85.49, 76.22) --
	( 85.27, 76.22) --
	( 85.05, 76.22) --
	( 84.82, 76.22) --
	( 84.60, 76.22) --
	( 84.38, 76.22) --
	( 84.16, 76.22) --
	( 83.93, 76.22) --
	( 83.71, 76.22) --
	( 83.49, 76.22) --
	( 83.27, 76.22) --
	( 83.04, 76.22) --
	( 82.82, 76.22) --
	( 82.60, 76.22) --
	( 82.37, 76.22) --
	( 82.15, 76.22) --
	( 81.93, 76.22) --
	( 81.71, 76.22) --
	( 81.48, 76.22) --
	( 81.26, 76.22) --
	( 81.04, 76.22) --
	( 80.82, 76.22) --
	( 80.59, 76.22) --
	( 80.37, 76.22) --
	( 80.15, 76.22) --
	( 79.92, 76.22) --
	( 79.70, 76.22) --
	( 79.48, 76.22) --
	( 79.26, 76.22) --
	( 79.03, 76.22) --
	( 78.81, 76.22) --
	( 78.59, 76.22) --
	( 78.36, 76.22) --
	( 78.14, 76.22) --
	( 77.92, 76.22) --
	( 77.70, 76.22) --
	( 77.47, 76.22) --
	( 77.25, 76.22) --
	( 77.03, 76.22) --
	( 76.81, 76.22) --
	( 76.58, 76.22) --
	( 76.36, 76.22) --
	( 76.14, 76.22) --
	( 75.91, 76.22) --
	( 75.69, 76.22) --
	( 75.47, 76.22) --
	( 75.25, 76.22) --
	( 75.02, 76.22) --
	( 74.80, 76.22) --
	( 74.58, 76.22) --
	( 74.36, 76.22) --
	( 74.13, 76.22) --
	( 73.91, 76.22) --
	( 73.69, 76.22) --
	( 73.46, 76.22) --
	( 73.24, 76.22) --
	( 73.02, 76.22) --
	( 72.80, 76.22) --
	( 72.57, 76.22) --
	( 72.35, 76.22) --
	( 72.13, 76.22) --
	( 71.90, 76.22) --
	( 71.68, 76.22) --
	( 71.46, 76.22) --
	( 71.24, 76.22) --
	( 71.01, 76.22) --
	( 70.79, 76.22) --
	( 70.57, 76.22) --
	( 70.35, 76.22) --
	( 70.12, 76.22) --
	( 69.90, 76.22) --
	( 69.68, 76.22) --
	( 69.45, 76.22) --
	( 69.23, 76.22) --
	( 69.01, 76.22) --
	( 68.79, 76.22) --
	( 68.56, 76.22) --
	( 68.34, 76.22) --
	( 68.12, 76.22) --
	( 67.90, 76.22) --
	( 67.67, 76.22) --
	( 67.45, 76.22) --
	( 67.23, 76.22) --
	( 67.00, 76.22) --
	( 66.78, 76.22) --
	( 66.56, 76.22) --
	( 66.34, 76.22) --
	( 66.11, 76.22) --
	( 65.89, 76.22) --
	( 65.67, 76.22) --
	( 65.44, 76.22) --
	( 65.22, 76.22) --
	( 65.00, 76.22) --
	( 64.78, 76.22) --
	( 64.55, 76.22) --
	( 64.33, 76.22) --
	( 64.11, 76.22) --
	( 63.89, 76.22) --
	( 63.66, 76.22) --
	( 63.44, 76.22) --
	( 63.22, 76.22) --
	( 62.99, 76.22) --
	( 62.77, 76.22) --
	( 62.55, 76.22) --
	( 62.33, 76.22) --
	( 62.10, 76.22) --
	( 61.88, 76.22) --
	( 61.66, 76.22) --
	( 61.44, 76.22) --
	( 61.21, 76.22) --
	( 60.99, 76.22) --
	( 60.77, 76.22) --
	( 60.54, 76.22) --
	( 60.32, 76.22) --
	( 60.10, 76.22) --
	( 59.88, 76.22) --
	( 59.65, 76.22) --
	( 59.43, 76.22) --
	( 59.21, 76.22) --
	( 58.99, 76.22) --
	( 58.76, 76.22) --
	( 58.54, 76.22) --
	( 58.32, 76.22) --
	( 58.09, 76.22) --
	( 57.87, 76.22) --
	( 57.65, 76.22) --
	( 57.43, 76.22) --
	( 57.20, 76.22) --
	( 56.98, 76.22) --
	( 56.76, 76.22) --
	( 56.53, 76.22) --
	( 56.31, 76.22) --
	( 56.09, 76.22) --
	( 55.87, 76.22) --
	( 55.64, 76.22) --
	( 55.42, 76.22) --
	( 55.20, 76.22) --
	( 54.98, 76.22) --
	( 54.75, 76.22) --
	( 54.53, 76.22) --
	( 54.31, 76.22) --
	( 54.08, 76.22) --
	( 53.86, 76.22) --
	( 53.64, 76.22) --
	( 53.42, 76.22) --
	( 53.19, 76.22) --
	( 52.97, 76.22) --
	( 52.75, 76.22) --
	( 52.53, 76.22) --
	( 52.30, 76.22) --
	( 52.08, 76.22) --
	( 51.86, 76.22) --
	( 51.63, 76.22) --
	( 51.41, 76.22) --
	( 51.19, 76.22) --
	( 50.97, 76.22) --
	( 50.74, 76.22) --
	( 50.52, 76.22) --
	( 50.30, 76.22) --
	( 50.07, 76.22) --
	( 49.85, 76.22) --
	( 49.63, 76.22) --
	( 49.41, 76.22) --
	( 49.18, 76.22) --
	( 48.96, 76.22) --
	( 48.74, 76.22) --
	( 48.52, 76.22) --
	( 48.29, 76.22) --
	( 48.07, 76.22) --
	( 47.85, 76.22) --
	( 47.62, 76.22) --
	( 47.40, 76.22) --
	( 47.18, 76.22) --
	cycle;

\path[draw=drawColor,line width= 0.6pt,line join=round,line cap=round] ( 47.18, 76.22) --
	( 47.40, 76.22) --
	( 47.62, 76.22) --
	( 47.85, 76.22) --
	( 48.07, 76.22) --
	( 48.29, 76.22) --
	( 48.52, 76.22) --
	( 48.74, 76.23) --
	( 48.96, 76.23) --
	( 49.18, 76.23) --
	( 49.41, 76.23) --
	( 49.63, 76.23) --
	( 49.85, 76.23) --
	( 50.07, 76.24) --
	( 50.30, 76.25) --
	( 50.52, 76.26) --
	( 50.74, 76.29) --
	( 50.97, 76.32) --
	( 51.19, 76.38) --
	( 51.41, 76.46) --
	( 51.63, 76.59) --
	( 51.86, 76.76) --
	( 52.08, 77.02) --
	( 52.30, 77.36) --
	( 52.53, 77.83) --
	( 52.75, 78.45) --
	( 52.97, 79.25) --
	( 53.19, 80.26) --
	( 53.42, 81.53) --
	( 53.64, 83.07) --
	( 53.86, 84.97) --
	( 54.08, 87.20) --
	( 54.31, 89.77) --
	( 54.53, 92.66) --
	( 54.75, 95.87) --
	( 54.98, 99.36) --
	( 55.20,103.08) --
	( 55.42,106.97) --
	( 55.64,110.97) --
	( 55.87,114.99) --
	( 56.09,118.95) --
	( 56.31,122.76) --
	( 56.53,126.37) --
	( 56.76,129.73) --
	( 56.98,132.78) --
	( 57.20,135.52) --
	( 57.43,137.92) --
	( 57.65,139.98) --
	( 57.87,141.71) --
	( 58.09,143.14) --
	( 58.32,144.26) --
	( 58.54,145.14) --
	( 58.76,145.81) --
	( 58.99,146.30) --
	( 59.21,146.64) --
	( 59.43,146.85) --
	( 59.65,146.94) --
	( 59.88,146.95) --
	( 60.10,146.87) --
	( 60.32,146.72) --
	( 60.54,146.51) --
	( 60.77,146.24) --
	( 60.99,145.93) --
	( 61.21,145.57) --
	( 61.44,145.17) --
	( 61.66,144.75) --
	( 61.88,144.30) --
	( 62.10,143.83) --
	( 62.33,143.35) --
	( 62.55,142.87) --
	( 62.77,142.39) --
	( 62.99,141.93) --
	( 63.22,141.49) --
	( 63.44,141.07) --
	( 63.66,140.67) --
	( 63.89,140.30) --
	( 64.11,139.96) --
	( 64.33,139.64) --
	( 64.55,139.34) --
	( 64.78,139.06) --
	( 65.00,138.79) --
	( 65.22,138.53) --
	( 65.44,138.26) --
	( 65.67,137.98) --
	( 65.89,137.69) --
	( 66.11,137.38) --
	( 66.34,137.04) --
	( 66.56,136.69) --
	( 66.78,136.31) --
	( 67.00,135.90) --
	( 67.23,135.47) --
	( 67.45,135.04) --
	( 67.67,134.59) --
	( 67.90,134.14) --
	( 68.12,133.69) --
	( 68.34,133.25) --
	( 68.56,132.83) --
	( 68.79,132.42) --
	( 69.01,132.03) --
	( 69.23,131.66) --
	( 69.45,131.32) --
	( 69.68,131.00) --
	( 69.90,130.71) --
	( 70.12,130.44) --
	( 70.35,130.20) --
	( 70.57,129.98) --
	( 70.79,129.79) --
	( 71.01,129.62) --
	( 71.24,129.48) --
	( 71.46,129.35) --
	( 71.68,129.25) --
	( 71.90,129.17) --
	( 72.13,129.10) --
	( 72.35,129.04) --
	( 72.57,128.99) --
	( 72.80,128.96) --
	( 73.02,128.94) --
	( 73.24,128.92) --
	( 73.46,128.93) --
	( 73.69,128.94) --
	( 73.91,128.97) --
	( 74.13,129.02) --
	( 74.36,129.09) --
	( 74.58,129.17) --
	( 74.80,129.27) --
	( 75.02,129.39) --
	( 75.25,129.52) --
	( 75.47,129.66) --
	( 75.69,129.81) --
	( 75.91,129.95) --
	( 76.14,130.10) --
	( 76.36,130.23) --
	( 76.58,130.36) --
	( 76.81,130.47) --
	( 77.03,130.56) --
	( 77.25,130.64) --
	( 77.47,130.70) --
	( 77.70,130.73) --
	( 77.92,130.75) --
	( 78.14,130.75) --
	( 78.36,130.73) --
	( 78.59,130.70) --
	( 78.81,130.66) --
	( 79.03,130.60) --
	( 79.26,130.54) --
	( 79.48,130.47) --
	( 79.70,130.39) --
	( 79.92,130.30) --
	( 80.15,130.20) --
	( 80.37,130.10) --
	( 80.59,129.98) --
	( 80.82,129.86) --
	( 81.04,129.72) --
	( 81.26,129.57) --
	( 81.48,129.40) --
	( 81.71,129.22) --
	( 81.93,129.03) --
	( 82.15,128.82) --
	( 82.37,128.60) --
	( 82.60,128.37) --
	( 82.82,128.13) --
	( 83.04,127.88) --
	( 83.27,127.63) --
	( 83.49,127.37) --
	( 83.71,127.10) --
	( 83.93,126.84) --
	( 84.16,126.57) --
	( 84.38,126.31) --
	( 84.60,126.04) --
	( 84.82,125.79) --
	( 85.05,125.54) --
	( 85.27,125.29) --
	( 85.49,125.06) --
	( 85.72,124.83) --
	( 85.94,124.61) --
	( 86.16,124.40) --
	( 86.38,124.19) --
	( 86.61,124.00) --
	( 86.83,123.80) --
	( 87.05,123.61) --
	( 87.28,123.41) --
	( 87.50,123.21) --
	( 87.72,123.01) --
	( 87.94,122.79) --
	( 88.17,122.56) --
	( 88.39,122.32) --
	( 88.61,122.07) --
	( 88.83,121.80) --
	( 89.06,121.52) --
	( 89.28,121.23) --
	( 89.50,120.94) --
	( 89.73,120.64) --
	( 89.95,120.34) --
	( 90.17,120.05) --
	( 90.39,119.76) --
	( 90.62,119.48) --
	( 90.84,119.22) --
	( 91.06,118.97) --
	( 91.28,118.74) --
	( 91.51,118.53) --
	( 91.73,118.34) --
	( 91.95,118.16) --
	( 92.18,118.00) --
	( 92.40,117.86) --
	( 92.62,117.72) --
	( 92.84,117.60) --
	( 93.07,117.48) --
	( 93.29,117.38) --
	( 93.51,117.27) --
	( 93.73,117.17) --
	( 93.96,117.07) --
	( 94.18,116.97) --
	( 94.40,116.88) --
	( 94.63,116.78) --
	( 94.85,116.68) --
	( 95.07,116.57) --
	( 95.29,116.46) --
	( 95.52,116.35) --
	( 95.74,116.24) --
	( 95.96,116.13) --
	( 96.19,116.01) --
	( 96.41,115.89) --
	( 96.63,115.77) --
	( 96.85,115.65) --
	( 97.08,115.53) --
	( 97.30,115.42) --
	( 97.52,115.30) --
	( 97.74,115.18) --
	( 97.97,115.06) --
	( 98.19,114.95) --
	( 98.41,114.82) --
	( 98.64,114.70) --
	( 98.86,114.57) --
	( 99.08,114.43) --
	( 99.30,114.29) --
	( 99.53,114.14) --
	( 99.75,113.98) --
	( 99.97,113.82) --
	(100.19,113.65) --
	(100.42,113.47) --
	(100.64,113.29) --
	(100.86,113.10) --
	(101.09,112.92) --
	(101.31,112.73) --
	(101.53,112.54) --
	(101.75,112.34) --
	(101.98,112.15) --
	(102.20,111.95) --
	(102.42,111.76) --
	(102.65,111.55) --
	(102.87,111.35) --
	(103.09,111.14) --
	(103.31,110.92) --
	(103.54,110.70) --
	(103.76,110.47) --
	(103.98,110.24) --
	(104.20,110.00) --
	(104.43,109.75) --
	(104.65,109.51) --
	(104.87,109.26) --
	(105.10,109.01) --
	(105.32,108.77) --
	(105.54,108.52) --
	(105.76,108.29) --
	(105.99,108.07) --
	(106.21,107.86) --
	(106.43,107.66) --
	(106.65,107.47) --
	(106.88,107.30) --
	(107.10,107.15) --
	(107.32,107.01) --
	(107.55,106.88) --
	(107.77,106.77) --
	(107.99,106.66) --
	(108.21,106.57) --
	(108.44,106.48) --
	(108.66,106.40) --
	(108.88,106.32) --
	(109.11,106.24) --
	(109.33,106.15) --
	(109.55,106.06) --
	(109.77,105.96) --
	(110.00,105.84) --
	(110.22,105.72) --
	(110.44,105.59) --
	(110.66,105.44) --
	(110.89,105.28) --
	(111.11,105.10) --
	(111.33,104.92) --
	(111.56,104.72) --
	(111.78,104.51) --
	(112.00,104.30) --
	(112.22,104.08) --
	(112.45,103.85) --
	(112.67,103.62) --
	(112.89,103.39) --
	(113.11,103.15) --
	(113.34,102.91) --
	(113.56,102.68) --
	(113.78,102.44) --
	(114.01,102.20) --
	(114.23,101.95) --
	(114.45,101.71) --
	(114.67,101.47) --
	(114.90,101.22) --
	(115.12,100.97) --
	(115.34,100.72) --
	(115.56,100.47) --
	(115.79,100.22) --
	(116.01, 99.96) --
	(116.23, 99.71) --
	(116.46, 99.46) --
	(116.68, 99.21) --
	(116.90, 98.96) --
	(117.12, 98.71) --
	(117.35, 98.47) --
	(117.57, 98.23) --
	(117.79, 98.00) --
	(118.02, 97.77) --
	(118.24, 97.55) --
	(118.46, 97.33) --
	(118.68, 97.12) --
	(118.91, 96.91) --
	(119.13, 96.71) --
	(119.35, 96.51) --
	(119.57, 96.31) --
	(119.80, 96.11) --
	(120.02, 95.91) --
	(120.24, 95.71) --
	(120.47, 95.51) --
	(120.69, 95.30) --
	(120.91, 95.09) --
	(121.13, 94.86) --
	(121.36, 94.64) --
	(121.58, 94.40) --
	(121.80, 94.15) --
	(122.02, 93.90) --
	(122.25, 93.65) --
	(122.47, 93.39) --
	(122.69, 93.12) --
	(122.92, 92.86) --
	(123.14, 92.59) --
	(123.36, 92.33) --
	(123.58, 92.07) --
	(123.81, 91.81) --
	(124.03, 91.56) --
	(124.25, 91.32) --
	(124.48, 91.08) --
	(124.70, 90.85) --
	(124.92, 90.62) --
	(125.14, 90.41) --
	(125.37, 90.20) --
	(125.59, 89.99) --
	(125.81, 89.80) --
	(126.03, 89.61) --
	(126.26, 89.43) --
	(126.48, 89.25) --
	(126.70, 89.08) --
	(126.93, 88.91) --
	(127.15, 88.75) --
	(127.37, 88.58) --
	(127.59, 88.41) --
	(127.82, 88.24) --
	(128.04, 88.07) --
	(128.26, 87.89) --
	(128.48, 87.71) --
	(128.71, 87.52) --
	(128.93, 87.33) --
	(129.15, 87.14) --
	(129.38, 86.93) --
	(129.60, 86.73) --
	(129.82, 86.52) --
	(130.04, 86.32) --
	(130.27, 86.11) --
	(130.49, 85.90) --
	(130.71, 85.69) --
	(130.94, 85.49) --
	(131.16, 85.29) --
	(131.38, 85.09) --
	(131.60, 84.89) --
	(131.83, 84.69) --
	(132.05, 84.50) --
	(132.27, 84.31) --
	(132.49, 84.12) --
	(132.72, 83.94) --
	(132.94, 83.75) --
	(133.16, 83.57) --
	(133.39, 83.39) --
	(133.61, 83.22) --
	(133.83, 83.04) --
	(134.05, 82.87) --
	(134.28, 82.70) --
	(134.50, 82.53) --
	(134.72, 82.36) --
	(134.94, 82.20) --
	(135.17, 82.03) --
	(135.39, 81.87) --
	(135.61, 81.70) --
	(135.84, 81.54) --
	(136.06, 81.38) --
	(136.28, 81.22) --
	(136.50, 81.06) --
	(136.73, 80.90) --
	(136.95, 80.74) --
	(137.17, 80.59) --
	(137.39, 80.43) --
	(137.62, 80.28) --
	(137.84, 80.13) --
	(138.06, 79.98) --
	(138.29, 79.84) --
	(138.51, 79.70) --
	(138.73, 79.56) --
	(138.95, 79.42) --
	(139.18, 79.29) --
	(139.40, 79.17) --
	(139.62, 79.04) --
	(139.85, 78.92) --
	(140.07, 78.81) --
	(140.29, 78.69) --
	(140.51, 78.58) --
	(140.74, 78.48) --
	(140.96, 78.37) --
	(141.18, 78.27) --
	(141.40, 78.17) --
	(141.63, 78.06) --
	(141.85, 77.96) --
	(142.07, 77.86) --
	(142.30, 77.76) --
	(142.52, 77.67) --
	(142.74, 77.57) --
	(142.96, 77.48) --
	(143.19, 77.38) --
	(143.41, 77.30) --
	(143.63, 77.21) --
	(143.85, 77.13) --
	(144.08, 77.06) --
	(144.30, 76.99) --
	(144.52, 76.92) --
	(144.75, 76.86) --
	(144.97, 76.81) --
	(145.19, 76.76) --
	(145.41, 76.71) --
	(145.64, 76.67) --
	(145.86, 76.64) --
	(146.08, 76.61) --
	(146.31, 76.58) --
	(146.53, 76.56) --
	(146.75, 76.54) --
	(146.97, 76.52) --
	(147.20, 76.50) --
	(147.42, 76.48) --
	(147.64, 76.47) --
	(147.86, 76.45) --
	(148.09, 76.44) --
	(148.31, 76.42) --
	(148.53, 76.41) --
	(148.76, 76.39) --
	(148.98, 76.38) --
	(149.20, 76.36) --
	(149.42, 76.35) --
	(149.65, 76.33) --
	(149.87, 76.32) --
	(150.09, 76.31) --
	(150.31, 76.30) --
	(150.54, 76.29) --
	(150.76, 76.28) --
	(150.98, 76.27) --
	(151.21, 76.26) --
	(151.43, 76.26) --
	(151.65, 76.25) --
	(151.87, 76.25) --
	(152.10, 76.24) --
	(152.32, 76.24) --
	(152.54, 76.24) --
	(152.77, 76.23) --
	(152.99, 76.23) --
	(153.21, 76.23) --
	(153.43, 76.23) --
	(153.66, 76.23) --
	(153.88, 76.23) --
	(154.10, 76.23) --
	(154.32, 76.23) --
	(154.55, 76.23) --
	(154.77, 76.23) --
	(154.99, 76.23) --
	(155.22, 76.23) --
	(155.44, 76.23) --
	(155.66, 76.23) --
	(155.88, 76.22) --
	(156.11, 76.22) --
	(156.33, 76.22) --
	(156.55, 76.22) --
	(156.77, 76.22) --
	(157.00, 76.22) --
	(157.22, 76.22) --
	(157.44, 76.22) --
	(157.67, 76.22) --
	(157.89, 76.22) --
	(158.11, 76.22) --
	(158.33, 76.22) --
	(158.56, 76.22) --
	(158.78, 76.22) --
	(159.00, 76.22) --
	(159.22, 76.22) --
	(159.45, 76.22) --
	(159.67, 76.22) --
	(159.89, 76.22) --
	(160.12, 76.22) --
	(160.34, 76.22) --
	(160.56, 76.22) --
	(160.78, 76.22) --
	(161.01, 76.22);
\end{scope}
\begin{scope}
\path[clip] (172.20, 67.14) rectangle (297.41,267.01);
\definecolor{drawColor}{RGB}{255,255,255}

\path[draw=drawColor,line width= 0.3pt,line join=round] (172.20,104.00) --
	(297.41,104.00);

\path[draw=drawColor,line width= 0.3pt,line join=round] (172.20,159.56) --
	(297.41,159.56);

\path[draw=drawColor,line width= 0.3pt,line join=round] (172.20,215.11) --
	(297.41,215.11);

\path[draw=drawColor,line width= 0.3pt,line join=round] (183.58, 67.14) --
	(183.58,267.01);

\path[draw=drawColor,line width= 0.3pt,line join=round] (217.73, 67.14) --
	(217.73,267.01);

\path[draw=drawColor,line width= 0.3pt,line join=round] (251.88, 67.14) --
	(251.88,267.01);

\path[draw=drawColor,line width= 0.3pt,line join=round] (286.03, 67.14) --
	(286.03,267.01);

\path[draw=drawColor,line width= 0.6pt,line join=round] (172.20, 76.22) --
	(297.41, 76.22);

\path[draw=drawColor,line width= 0.6pt,line join=round] (172.20,131.78) --
	(297.41,131.78);

\path[draw=drawColor,line width= 0.6pt,line join=round] (172.20,187.33) --
	(297.41,187.33);

\path[draw=drawColor,line width= 0.6pt,line join=round] (172.20,242.89) --
	(297.41,242.89);

\path[draw=drawColor,line width= 0.6pt,line join=round] (200.66, 67.14) --
	(200.66,267.01);

\path[draw=drawColor,line width= 0.6pt,line join=round] (234.80, 67.14) --
	(234.80,267.01);

\path[draw=drawColor,line width= 0.6pt,line join=round] (268.95, 67.14) --
	(268.95,267.01);
\definecolor{fillColor}{RGB}{228,26,28}

\path[fill=fillColor,fill opacity=0.50] (177.89, 76.25) --
	(178.11, 76.27) --
	(178.34, 76.28) --
	(178.56, 76.30) --
	(178.78, 76.33) --
	(179.00, 76.36) --
	(179.23, 76.41) --
	(179.45, 76.47) --
	(179.67, 76.55) --
	(179.89, 76.65) --
	(180.12, 76.77) --
	(180.34, 76.92) --
	(180.56, 77.12) --
	(180.79, 77.36) --
	(181.01, 77.66) --
	(181.23, 78.02) --
	(181.45, 78.46) --
	(181.68, 78.99) --
	(181.90, 79.61) --
	(182.12, 80.37) --
	(182.34, 81.26) --
	(182.57, 82.30) --
	(182.79, 83.50) --
	(183.01, 84.89) --
	(183.24, 86.46) --
	(183.46, 88.25) --
	(183.68, 90.30) --
	(183.90, 92.59) --
	(184.13, 95.13) --
	(184.35, 97.93) --
	(184.57,100.99) --
	(184.80,104.31) --
	(185.02,107.92) --
	(185.24,111.82) --
	(185.46,115.95) --
	(185.69,120.31) --
	(185.91,124.87) --
	(186.13,129.62) --
	(186.35,134.52) --
	(186.58,139.57) --
	(186.80,144.69) --
	(187.02,149.84) --
	(187.25,154.98) --
	(187.47,160.08) --
	(187.69,165.10) --
	(187.91,169.96) --
	(188.14,174.61) --
	(188.36,179.03) --
	(188.58,183.19) --
	(188.80,187.04) --
	(189.03,190.56) --
	(189.25,193.73) --
	(189.47,196.45) --
	(189.70,198.76) --
	(189.92,200.67) --
	(190.14,202.17) --
	(190.36,203.27) --
	(190.59,203.98) --
	(190.81,204.28) --
	(191.03,204.16) --
	(191.26,203.70) --
	(191.48,202.94) --
	(191.70,201.89) --
	(191.92,200.59) --
	(192.15,199.06) --
	(192.37,197.33) --
	(192.59,195.43) --
	(192.81,193.43) --
	(193.04,191.34) --
	(193.26,189.19) --
	(193.48,187.01) --
	(193.71,184.83) --
	(193.93,182.67) --
	(194.15,180.56) --
	(194.37,178.50) --
	(194.60,176.53) --
	(194.82,174.63) --
	(195.04,172.83) --
	(195.26,171.14) --
	(195.49,169.56) --
	(195.71,168.09) --
	(195.93,166.74) --
	(196.16,165.48) --
	(196.38,164.34) --
	(196.60,163.30) --
	(196.82,162.38) --
	(197.05,161.55) --
	(197.27,160.82) --
	(197.49,160.18) --
	(197.72,159.61) --
	(197.94,159.13) --
	(198.16,158.73) --
	(198.38,158.40) --
	(198.61,158.14) --
	(198.83,157.93) --
	(199.05,157.77) --
	(199.27,157.66) --
	(199.50,157.59) --
	(199.72,157.56) --
	(199.94,157.56) --
	(200.17,157.58) --
	(200.39,157.61) --
	(200.61,157.66) --
	(200.83,157.70) --
	(201.06,157.75) --
	(201.28,157.79) --
	(201.50,157.82) --
	(201.72,157.83) --
	(201.95,157.82) --
	(202.17,157.80) --
	(202.39,157.75) --
	(202.62,157.67) --
	(202.84,157.56) --
	(203.06,157.42) --
	(203.28,157.25) --
	(203.51,157.05) --
	(203.73,156.82) --
	(203.95,156.55) --
	(204.17,156.24) --
	(204.40,155.90) --
	(204.62,155.52) --
	(204.84,155.10) --
	(205.07,154.65) --
	(205.29,154.15) --
	(205.51,153.62) --
	(205.73,153.04) --
	(205.96,152.43) --
	(206.18,151.78) --
	(206.40,151.09) --
	(206.63,150.38) --
	(206.85,149.63) --
	(207.07,148.85) --
	(207.29,148.04) --
	(207.52,147.22) --
	(207.74,146.38) --
	(207.96,145.52) --
	(208.18,144.66) --
	(208.41,143.79) --
	(208.63,142.92) --
	(208.85,142.05) --
	(209.08,141.19) --
	(209.30,140.32) --
	(209.52,139.46) --
	(209.74,138.61) --
	(209.97,137.76) --
	(210.19,136.91) --
	(210.41,136.07) --
	(210.63,135.23) --
	(210.86,134.38) --
	(211.08,133.54) --
	(211.30,132.69) --
	(211.53,131.84) --
	(211.75,130.98) --
	(211.97,130.12) --
	(212.19,129.25) --
	(212.42,128.37) --
	(212.64,127.49) --
	(212.86,126.60) --
	(213.09,125.71) --
	(213.31,124.82) --
	(213.53,123.94) --
	(213.75,123.06) --
	(213.98,122.18) --
	(214.20,121.31) --
	(214.42,120.46) --
	(214.64,119.63) --
	(214.87,118.81) --
	(215.09,118.01) --
	(215.31,117.23) --
	(215.54,116.48) --
	(215.76,115.75) --
	(215.98,115.05) --
	(216.20,114.37) --
	(216.43,113.72) --
	(216.65,113.09) --
	(216.87,112.49) --
	(217.09,111.91) --
	(217.32,111.35) --
	(217.54,110.82) --
	(217.76,110.31) --
	(217.99,109.81) --
	(218.21,109.33) --
	(218.43,108.86) --
	(218.65,108.41) --
	(218.88,107.97) --
	(219.10,107.53) --
	(219.32,107.10) --
	(219.55,106.68) --
	(219.77,106.26) --
	(219.99,105.83) --
	(220.21,105.41) --
	(220.44,104.98) --
	(220.66,104.54) --
	(220.88,104.10) --
	(221.10,103.65) --
	(221.33,103.19) --
	(221.55,102.72) --
	(221.77,102.24) --
	(222.00,101.76) --
	(222.22,101.26) --
	(222.44,100.76) --
	(222.66,100.25) --
	(222.89, 99.73) --
	(223.11, 99.21) --
	(223.33, 98.70) --
	(223.55, 98.19) --
	(223.78, 97.69) --
	(224.00, 97.19) --
	(224.22, 96.72) --
	(224.45, 96.26) --
	(224.67, 95.82) --
	(224.89, 95.41) --
	(225.11, 95.03) --
	(225.34, 94.67) --
	(225.56, 94.35) --
	(225.78, 94.06) --
	(226.00, 93.81) --
	(226.23, 93.59) --
	(226.45, 93.41) --
	(226.67, 93.26) --
	(226.90, 93.14) --
	(227.12, 93.04) --
	(227.34, 92.98) --
	(227.56, 92.93) --
	(227.79, 92.91) --
	(228.01, 92.89) --
	(228.23, 92.89) --
	(228.46, 92.90) --
	(228.68, 92.90) --
	(228.90, 92.91) --
	(229.12, 92.91) --
	(229.35, 92.91) --
	(229.57, 92.89) --
	(229.79, 92.87) --
	(230.01, 92.83) --
	(230.24, 92.78) --
	(230.46, 92.71) --
	(230.68, 92.63) --
	(230.91, 92.53) --
	(231.13, 92.42) --
	(231.35, 92.29) --
	(231.57, 92.16) --
	(231.80, 92.01) --
	(232.02, 91.85) --
	(232.24, 91.69) --
	(232.46, 91.52) --
	(232.69, 91.34) --
	(232.91, 91.16) --
	(233.13, 90.98) --
	(233.36, 90.80) --
	(233.58, 90.62) --
	(233.80, 90.44) --
	(234.02, 90.27) --
	(234.25, 90.10) --
	(234.47, 89.94) --
	(234.69, 89.78) --
	(234.92, 89.63) --
	(235.14, 89.48) --
	(235.36, 89.35) --
	(235.58, 89.22) --
	(235.81, 89.10) --
	(236.03, 88.99) --
	(236.25, 88.89) --
	(236.47, 88.79) --
	(236.70, 88.70) --
	(236.92, 88.62) --
	(237.14, 88.54) --
	(237.37, 88.47) --
	(237.59, 88.40) --
	(237.81, 88.33) --
	(238.03, 88.27) --
	(238.26, 88.20) --
	(238.48, 88.14) --
	(238.70, 88.07) --
	(238.92, 88.01) --
	(239.15, 87.93) --
	(239.37, 87.86) --
	(239.59, 87.78) --
	(239.82, 87.70) --
	(240.04, 87.61) --
	(240.26, 87.52) --
	(240.48, 87.42) --
	(240.71, 87.32) --
	(240.93, 87.22) --
	(241.15, 87.11) --
	(241.38, 87.00) --
	(241.60, 86.89) --
	(241.82, 86.78) --
	(242.04, 86.67) --
	(242.27, 86.56) --
	(242.49, 86.45) --
	(242.71, 86.34) --
	(242.93, 86.23) --
	(243.16, 86.13) --
	(243.38, 86.03) --
	(243.60, 85.92) --
	(243.83, 85.82) --
	(244.05, 85.73) --
	(244.27, 85.63) --
	(244.49, 85.53) --
	(244.72, 85.43) --
	(244.94, 85.34) --
	(245.16, 85.24) --
	(245.38, 85.13) --
	(245.61, 85.03) --
	(245.83, 84.91) --
	(246.05, 84.80) --
	(246.28, 84.68) --
	(246.50, 84.55) --
	(246.72, 84.42) --
	(246.94, 84.29) --
	(247.17, 84.15) --
	(247.39, 84.00) --
	(247.61, 83.85) --
	(247.84, 83.69) --
	(248.06, 83.53) --
	(248.28, 83.37) --
	(248.50, 83.21) --
	(248.73, 83.04) --
	(248.95, 82.87) --
	(249.17, 82.71) --
	(249.39, 82.54) --
	(249.62, 82.38) --
	(249.84, 82.22) --
	(250.06, 82.07) --
	(250.29, 81.92) --
	(250.51, 81.78) --
	(250.73, 81.64) --
	(250.95, 81.51) --
	(251.18, 81.39) --
	(251.40, 81.28) --
	(251.62, 81.17) --
	(251.84, 81.07) --
	(252.07, 80.99) --
	(252.29, 80.91) --
	(252.51, 80.84) --
	(252.74, 80.77) --
	(252.96, 80.72) --
	(253.18, 80.66) --
	(253.40, 80.62) --
	(253.63, 80.58) --
	(253.85, 80.54) --
	(254.07, 80.50) --
	(254.29, 80.46) --
	(254.52, 80.42) --
	(254.74, 80.37) --
	(254.96, 80.33) --
	(255.19, 80.28) --
	(255.41, 80.22) --
	(255.63, 80.16) --
	(255.85, 80.09) --
	(256.08, 80.01) --
	(256.30, 79.93) --
	(256.52, 79.85) --
	(256.75, 79.76) --
	(256.97, 79.66) --
	(257.19, 79.57) --
	(257.41, 79.47) --
	(257.64, 79.37) --
	(257.86, 79.27) --
	(258.08, 79.17) --
	(258.30, 79.08) --
	(258.53, 78.99) --
	(258.75, 78.91) --
	(258.97, 78.83) --
	(259.20, 78.76) --
	(259.42, 78.70) --
	(259.64, 78.64) --
	(259.86, 78.59) --
	(260.09, 78.55) --
	(260.31, 78.52) --
	(260.53, 78.50) --
	(260.75, 78.48) --
	(260.98, 78.47) --
	(261.20, 78.46) --
	(261.42, 78.46) --
	(261.65, 78.46) --
	(261.87, 78.46) --
	(262.09, 78.46) --
	(262.31, 78.46) --
	(262.54, 78.47) --
	(262.76, 78.47) --
	(262.98, 78.47) --
	(263.21, 78.47) --
	(263.43, 78.46) --
	(263.65, 78.46) --
	(263.87, 78.45) --
	(264.10, 78.44) --
	(264.32, 78.43) --
	(264.54, 78.42) --
	(264.76, 78.40) --
	(264.99, 78.39) --
	(265.21, 78.37) --
	(265.43, 78.36) --
	(265.66, 78.35) --
	(265.88, 78.33) --
	(266.10, 78.32) --
	(266.32, 78.30) --
	(266.55, 78.29) --
	(266.77, 78.28) --
	(266.99, 78.26) --
	(267.21, 78.25) --
	(267.44, 78.24) --
	(267.66, 78.22) --
	(267.88, 78.20) --
	(268.11, 78.18) --
	(268.33, 78.16) --
	(268.55, 78.14) --
	(268.77, 78.11) --
	(269.00, 78.08) --
	(269.22, 78.05) --
	(269.44, 78.01) --
	(269.67, 77.97) --
	(269.89, 77.93) --
	(270.11, 77.88) --
	(270.33, 77.82) --
	(270.56, 77.77) --
	(270.78, 77.71) --
	(271.00, 77.64) --
	(271.22, 77.57) --
	(271.45, 77.51) --
	(271.67, 77.44) --
	(271.89, 77.36) --
	(272.12, 77.29) --
	(272.34, 77.22) --
	(272.56, 77.15) --
	(272.78, 77.08) --
	(273.01, 77.01) --
	(273.23, 76.95) --
	(273.45, 76.88) --
	(273.67, 76.82) --
	(273.90, 76.77) --
	(274.12, 76.72) --
	(274.34, 76.67) --
	(274.57, 76.63) --
	(274.79, 76.59) --
	(275.01, 76.56) --
	(275.23, 76.53) --
	(275.46, 76.50) --
	(275.68, 76.48) --
	(275.90, 76.46) --
	(276.12, 76.45) --
	(276.35, 76.44) --
	(276.57, 76.43) --
	(276.79, 76.42) --
	(277.02, 76.41) --
	(277.24, 76.40) --
	(277.46, 76.40) --
	(277.68, 76.39) --
	(277.91, 76.39) --
	(278.13, 76.38) --
	(278.35, 76.38) --
	(278.58, 76.37) --
	(278.80, 76.37) --
	(279.02, 76.36) --
	(279.24, 76.36) --
	(279.47, 76.35) --
	(279.69, 76.34) --
	(279.91, 76.34) --
	(280.13, 76.33) --
	(280.36, 76.32) --
	(280.58, 76.31) --
	(280.80, 76.31) --
	(281.03, 76.30) --
	(281.25, 76.29) --
	(281.47, 76.28) --
	(281.69, 76.28) --
	(281.92, 76.27) --
	(282.14, 76.26) --
	(282.36, 76.26) --
	(282.58, 76.25) --
	(282.81, 76.25) --
	(283.03, 76.25) --
	(283.25, 76.24) --
	(283.48, 76.24) --
	(283.70, 76.24) --
	(283.92, 76.23) --
	(284.14, 76.23) --
	(284.37, 76.23) --
	(284.59, 76.23) --
	(284.81, 76.23) --
	(285.04, 76.23) --
	(285.26, 76.23) --
	(285.48, 76.23) --
	(285.70, 76.23) --
	(285.93, 76.23) --
	(286.15, 76.23) --
	(286.37, 76.23) --
	(286.59, 76.23) --
	(286.82, 76.23) --
	(287.04, 76.23) --
	(287.26, 76.22) --
	(287.49, 76.22) --
	(287.71, 76.22) --
	(287.93, 76.22) --
	(288.15, 76.22) --
	(288.38, 76.22) --
	(288.60, 76.22) --
	(288.82, 76.22) --
	(289.04, 76.22) --
	(289.27, 76.22) --
	(289.49, 76.22) --
	(289.71, 76.22) --
	(289.94, 76.22) --
	(290.16, 76.22) --
	(290.38, 76.22) --
	(290.60, 76.22) --
	(290.83, 76.22) --
	(291.05, 76.22) --
	(291.27, 76.22) --
	(291.50, 76.22) --
	(291.72, 76.22) --
	(291.72, 76.22) --
	(291.50, 76.22) --
	(291.27, 76.22) --
	(291.05, 76.22) --
	(290.83, 76.22) --
	(290.60, 76.22) --
	(290.38, 76.22) --
	(290.16, 76.22) --
	(289.94, 76.22) --
	(289.71, 76.22) --
	(289.49, 76.22) --
	(289.27, 76.22) --
	(289.04, 76.22) --
	(288.82, 76.22) --
	(288.60, 76.22) --
	(288.38, 76.22) --
	(288.15, 76.22) --
	(287.93, 76.22) --
	(287.71, 76.22) --
	(287.49, 76.22) --
	(287.26, 76.22) --
	(287.04, 76.22) --
	(286.82, 76.22) --
	(286.59, 76.22) --
	(286.37, 76.22) --
	(286.15, 76.22) --
	(285.93, 76.22) --
	(285.70, 76.22) --
	(285.48, 76.22) --
	(285.26, 76.22) --
	(285.04, 76.22) --
	(284.81, 76.22) --
	(284.59, 76.22) --
	(284.37, 76.22) --
	(284.14, 76.22) --
	(283.92, 76.22) --
	(283.70, 76.22) --
	(283.48, 76.22) --
	(283.25, 76.22) --
	(283.03, 76.22) --
	(282.81, 76.22) --
	(282.58, 76.22) --
	(282.36, 76.22) --
	(282.14, 76.22) --
	(281.92, 76.22) --
	(281.69, 76.22) --
	(281.47, 76.22) --
	(281.25, 76.22) --
	(281.03, 76.22) --
	(280.80, 76.22) --
	(280.58, 76.22) --
	(280.36, 76.22) --
	(280.13, 76.22) --
	(279.91, 76.22) --
	(279.69, 76.22) --
	(279.47, 76.22) --
	(279.24, 76.22) --
	(279.02, 76.22) --
	(278.80, 76.22) --
	(278.58, 76.22) --
	(278.35, 76.22) --
	(278.13, 76.22) --
	(277.91, 76.22) --
	(277.68, 76.22) --
	(277.46, 76.22) --
	(277.24, 76.22) --
	(277.02, 76.22) --
	(276.79, 76.22) --
	(276.57, 76.22) --
	(276.35, 76.22) --
	(276.12, 76.22) --
	(275.90, 76.22) --
	(275.68, 76.22) --
	(275.46, 76.22) --
	(275.23, 76.22) --
	(275.01, 76.22) --
	(274.79, 76.22) --
	(274.57, 76.22) --
	(274.34, 76.22) --
	(274.12, 76.22) --
	(273.90, 76.22) --
	(273.67, 76.22) --
	(273.45, 76.22) --
	(273.23, 76.22) --
	(273.01, 76.22) --
	(272.78, 76.22) --
	(272.56, 76.22) --
	(272.34, 76.22) --
	(272.12, 76.22) --
	(271.89, 76.22) --
	(271.67, 76.22) --
	(271.45, 76.22) --
	(271.22, 76.22) --
	(271.00, 76.22) --
	(270.78, 76.22) --
	(270.56, 76.22) --
	(270.33, 76.22) --
	(270.11, 76.22) --
	(269.89, 76.22) --
	(269.67, 76.22) --
	(269.44, 76.22) --
	(269.22, 76.22) --
	(269.00, 76.22) --
	(268.77, 76.22) --
	(268.55, 76.22) --
	(268.33, 76.22) --
	(268.11, 76.22) --
	(267.88, 76.22) --
	(267.66, 76.22) --
	(267.44, 76.22) --
	(267.21, 76.22) --
	(266.99, 76.22) --
	(266.77, 76.22) --
	(266.55, 76.22) --
	(266.32, 76.22) --
	(266.10, 76.22) --
	(265.88, 76.22) --
	(265.66, 76.22) --
	(265.43, 76.22) --
	(265.21, 76.22) --
	(264.99, 76.22) --
	(264.76, 76.22) --
	(264.54, 76.22) --
	(264.32, 76.22) --
	(264.10, 76.22) --
	(263.87, 76.22) --
	(263.65, 76.22) --
	(263.43, 76.22) --
	(263.21, 76.22) --
	(262.98, 76.22) --
	(262.76, 76.22) --
	(262.54, 76.22) --
	(262.31, 76.22) --
	(262.09, 76.22) --
	(261.87, 76.22) --
	(261.65, 76.22) --
	(261.42, 76.22) --
	(261.20, 76.22) --
	(260.98, 76.22) --
	(260.75, 76.22) --
	(260.53, 76.22) --
	(260.31, 76.22) --
	(260.09, 76.22) --
	(259.86, 76.22) --
	(259.64, 76.22) --
	(259.42, 76.22) --
	(259.20, 76.22) --
	(258.97, 76.22) --
	(258.75, 76.22) --
	(258.53, 76.22) --
	(258.30, 76.22) --
	(258.08, 76.22) --
	(257.86, 76.22) --
	(257.64, 76.22) --
	(257.41, 76.22) --
	(257.19, 76.22) --
	(256.97, 76.22) --
	(256.75, 76.22) --
	(256.52, 76.22) --
	(256.30, 76.22) --
	(256.08, 76.22) --
	(255.85, 76.22) --
	(255.63, 76.22) --
	(255.41, 76.22) --
	(255.19, 76.22) --
	(254.96, 76.22) --
	(254.74, 76.22) --
	(254.52, 76.22) --
	(254.29, 76.22) --
	(254.07, 76.22) --
	(253.85, 76.22) --
	(253.63, 76.22) --
	(253.40, 76.22) --
	(253.18, 76.22) --
	(252.96, 76.22) --
	(252.74, 76.22) --
	(252.51, 76.22) --
	(252.29, 76.22) --
	(252.07, 76.22) --
	(251.84, 76.22) --
	(251.62, 76.22) --
	(251.40, 76.22) --
	(251.18, 76.22) --
	(250.95, 76.22) --
	(250.73, 76.22) --
	(250.51, 76.22) --
	(250.29, 76.22) --
	(250.06, 76.22) --
	(249.84, 76.22) --
	(249.62, 76.22) --
	(249.39, 76.22) --
	(249.17, 76.22) --
	(248.95, 76.22) --
	(248.73, 76.22) --
	(248.50, 76.22) --
	(248.28, 76.22) --
	(248.06, 76.22) --
	(247.84, 76.22) --
	(247.61, 76.22) --
	(247.39, 76.22) --
	(247.17, 76.22) --
	(246.94, 76.22) --
	(246.72, 76.22) --
	(246.50, 76.22) --
	(246.28, 76.22) --
	(246.05, 76.22) --
	(245.83, 76.22) --
	(245.61, 76.22) --
	(245.38, 76.22) --
	(245.16, 76.22) --
	(244.94, 76.22) --
	(244.72, 76.22) --
	(244.49, 76.22) --
	(244.27, 76.22) --
	(244.05, 76.22) --
	(243.83, 76.22) --
	(243.60, 76.22) --
	(243.38, 76.22) --
	(243.16, 76.22) --
	(242.93, 76.22) --
	(242.71, 76.22) --
	(242.49, 76.22) --
	(242.27, 76.22) --
	(242.04, 76.22) --
	(241.82, 76.22) --
	(241.60, 76.22) --
	(241.38, 76.22) --
	(241.15, 76.22) --
	(240.93, 76.22) --
	(240.71, 76.22) --
	(240.48, 76.22) --
	(240.26, 76.22) --
	(240.04, 76.22) --
	(239.82, 76.22) --
	(239.59, 76.22) --
	(239.37, 76.22) --
	(239.15, 76.22) --
	(238.92, 76.22) --
	(238.70, 76.22) --
	(238.48, 76.22) --
	(238.26, 76.22) --
	(238.03, 76.22) --
	(237.81, 76.22) --
	(237.59, 76.22) --
	(237.37, 76.22) --
	(237.14, 76.22) --
	(236.92, 76.22) --
	(236.70, 76.22) --
	(236.47, 76.22) --
	(236.25, 76.22) --
	(236.03, 76.22) --
	(235.81, 76.22) --
	(235.58, 76.22) --
	(235.36, 76.22) --
	(235.14, 76.22) --
	(234.92, 76.22) --
	(234.69, 76.22) --
	(234.47, 76.22) --
	(234.25, 76.22) --
	(234.02, 76.22) --
	(233.80, 76.22) --
	(233.58, 76.22) --
	(233.36, 76.22) --
	(233.13, 76.22) --
	(232.91, 76.22) --
	(232.69, 76.22) --
	(232.46, 76.22) --
	(232.24, 76.22) --
	(232.02, 76.22) --
	(231.80, 76.22) --
	(231.57, 76.22) --
	(231.35, 76.22) --
	(231.13, 76.22) --
	(230.91, 76.22) --
	(230.68, 76.22) --
	(230.46, 76.22) --
	(230.24, 76.22) --
	(230.01, 76.22) --
	(229.79, 76.22) --
	(229.57, 76.22) --
	(229.35, 76.22) --
	(229.12, 76.22) --
	(228.90, 76.22) --
	(228.68, 76.22) --
	(228.46, 76.22) --
	(228.23, 76.22) --
	(228.01, 76.22) --
	(227.79, 76.22) --
	(227.56, 76.22) --
	(227.34, 76.22) --
	(227.12, 76.22) --
	(226.90, 76.22) --
	(226.67, 76.22) --
	(226.45, 76.22) --
	(226.23, 76.22) --
	(226.00, 76.22) --
	(225.78, 76.22) --
	(225.56, 76.22) --
	(225.34, 76.22) --
	(225.11, 76.22) --
	(224.89, 76.22) --
	(224.67, 76.22) --
	(224.45, 76.22) --
	(224.22, 76.22) --
	(224.00, 76.22) --
	(223.78, 76.22) --
	(223.55, 76.22) --
	(223.33, 76.22) --
	(223.11, 76.22) --
	(222.89, 76.22) --
	(222.66, 76.22) --
	(222.44, 76.22) --
	(222.22, 76.22) --
	(222.00, 76.22) --
	(221.77, 76.22) --
	(221.55, 76.22) --
	(221.33, 76.22) --
	(221.10, 76.22) --
	(220.88, 76.22) --
	(220.66, 76.22) --
	(220.44, 76.22) --
	(220.21, 76.22) --
	(219.99, 76.22) --
	(219.77, 76.22) --
	(219.55, 76.22) --
	(219.32, 76.22) --
	(219.10, 76.22) --
	(218.88, 76.22) --
	(218.65, 76.22) --
	(218.43, 76.22) --
	(218.21, 76.22) --
	(217.99, 76.22) --
	(217.76, 76.22) --
	(217.54, 76.22) --
	(217.32, 76.22) --
	(217.09, 76.22) --
	(216.87, 76.22) --
	(216.65, 76.22) --
	(216.43, 76.22) --
	(216.20, 76.22) --
	(215.98, 76.22) --
	(215.76, 76.22) --
	(215.54, 76.22) --
	(215.31, 76.22) --
	(215.09, 76.22) --
	(214.87, 76.22) --
	(214.64, 76.22) --
	(214.42, 76.22) --
	(214.20, 76.22) --
	(213.98, 76.22) --
	(213.75, 76.22) --
	(213.53, 76.22) --
	(213.31, 76.22) --
	(213.09, 76.22) --
	(212.86, 76.22) --
	(212.64, 76.22) --
	(212.42, 76.22) --
	(212.19, 76.22) --
	(211.97, 76.22) --
	(211.75, 76.22) --
	(211.53, 76.22) --
	(211.30, 76.22) --
	(211.08, 76.22) --
	(210.86, 76.22) --
	(210.63, 76.22) --
	(210.41, 76.22) --
	(210.19, 76.22) --
	(209.97, 76.22) --
	(209.74, 76.22) --
	(209.52, 76.22) --
	(209.30, 76.22) --
	(209.08, 76.22) --
	(208.85, 76.22) --
	(208.63, 76.22) --
	(208.41, 76.22) --
	(208.18, 76.22) --
	(207.96, 76.22) --
	(207.74, 76.22) --
	(207.52, 76.22) --
	(207.29, 76.22) --
	(207.07, 76.22) --
	(206.85, 76.22) --
	(206.63, 76.22) --
	(206.40, 76.22) --
	(206.18, 76.22) --
	(205.96, 76.22) --
	(205.73, 76.22) --
	(205.51, 76.22) --
	(205.29, 76.22) --
	(205.07, 76.22) --
	(204.84, 76.22) --
	(204.62, 76.22) --
	(204.40, 76.22) --
	(204.17, 76.22) --
	(203.95, 76.22) --
	(203.73, 76.22) --
	(203.51, 76.22) --
	(203.28, 76.22) --
	(203.06, 76.22) --
	(202.84, 76.22) --
	(202.62, 76.22) --
	(202.39, 76.22) --
	(202.17, 76.22) --
	(201.95, 76.22) --
	(201.72, 76.22) --
	(201.50, 76.22) --
	(201.28, 76.22) --
	(201.06, 76.22) --
	(200.83, 76.22) --
	(200.61, 76.22) --
	(200.39, 76.22) --
	(200.17, 76.22) --
	(199.94, 76.22) --
	(199.72, 76.22) --
	(199.50, 76.22) --
	(199.27, 76.22) --
	(199.05, 76.22) --
	(198.83, 76.22) --
	(198.61, 76.22) --
	(198.38, 76.22) --
	(198.16, 76.22) --
	(197.94, 76.22) --
	(197.72, 76.22) --
	(197.49, 76.22) --
	(197.27, 76.22) --
	(197.05, 76.22) --
	(196.82, 76.22) --
	(196.60, 76.22) --
	(196.38, 76.22) --
	(196.16, 76.22) --
	(195.93, 76.22) --
	(195.71, 76.22) --
	(195.49, 76.22) --
	(195.26, 76.22) --
	(195.04, 76.22) --
	(194.82, 76.22) --
	(194.60, 76.22) --
	(194.37, 76.22) --
	(194.15, 76.22) --
	(193.93, 76.22) --
	(193.71, 76.22) --
	(193.48, 76.22) --
	(193.26, 76.22) --
	(193.04, 76.22) --
	(192.81, 76.22) --
	(192.59, 76.22) --
	(192.37, 76.22) --
	(192.15, 76.22) --
	(191.92, 76.22) --
	(191.70, 76.22) --
	(191.48, 76.22) --
	(191.26, 76.22) --
	(191.03, 76.22) --
	(190.81, 76.22) --
	(190.59, 76.22) --
	(190.36, 76.22) --
	(190.14, 76.22) --
	(189.92, 76.22) --
	(189.70, 76.22) --
	(189.47, 76.22) --
	(189.25, 76.22) --
	(189.03, 76.22) --
	(188.80, 76.22) --
	(188.58, 76.22) --
	(188.36, 76.22) --
	(188.14, 76.22) --
	(187.91, 76.22) --
	(187.69, 76.22) --
	(187.47, 76.22) --
	(187.25, 76.22) --
	(187.02, 76.22) --
	(186.80, 76.22) --
	(186.58, 76.22) --
	(186.35, 76.22) --
	(186.13, 76.22) --
	(185.91, 76.22) --
	(185.69, 76.22) --
	(185.46, 76.22) --
	(185.24, 76.22) --
	(185.02, 76.22) --
	(184.80, 76.22) --
	(184.57, 76.22) --
	(184.35, 76.22) --
	(184.13, 76.22) --
	(183.90, 76.22) --
	(183.68, 76.22) --
	(183.46, 76.22) --
	(183.24, 76.22) --
	(183.01, 76.22) --
	(182.79, 76.22) --
	(182.57, 76.22) --
	(182.34, 76.22) --
	(182.12, 76.22) --
	(181.90, 76.22) --
	(181.68, 76.22) --
	(181.45, 76.22) --
	(181.23, 76.22) --
	(181.01, 76.22) --
	(180.79, 76.22) --
	(180.56, 76.22) --
	(180.34, 76.22) --
	(180.12, 76.22) --
	(179.89, 76.22) --
	(179.67, 76.22) --
	(179.45, 76.22) --
	(179.23, 76.22) --
	(179.00, 76.22) --
	(178.78, 76.22) --
	(178.56, 76.22) --
	(178.34, 76.22) --
	(178.11, 76.22) --
	(177.89, 76.22) --
	cycle;
\definecolor{drawColor}{RGB}{0,0,0}

\path[draw=drawColor,line width= 0.6pt,line join=round,line cap=round] (177.89, 76.25) --
	(178.11, 76.27) --
	(178.34, 76.28) --
	(178.56, 76.30) --
	(178.78, 76.33) --
	(179.00, 76.36) --
	(179.23, 76.41) --
	(179.45, 76.47) --
	(179.67, 76.55) --
	(179.89, 76.65) --
	(180.12, 76.77) --
	(180.34, 76.92) --
	(180.56, 77.12) --
	(180.79, 77.36) --
	(181.01, 77.66) --
	(181.23, 78.02) --
	(181.45, 78.46) --
	(181.68, 78.99) --
	(181.90, 79.61) --
	(182.12, 80.37) --
	(182.34, 81.26) --
	(182.57, 82.30) --
	(182.79, 83.50) --
	(183.01, 84.89) --
	(183.24, 86.46) --
	(183.46, 88.25) --
	(183.68, 90.30) --
	(183.90, 92.59) --
	(184.13, 95.13) --
	(184.35, 97.93) --
	(184.57,100.99) --
	(184.80,104.31) --
	(185.02,107.92) --
	(185.24,111.82) --
	(185.46,115.95) --
	(185.69,120.31) --
	(185.91,124.87) --
	(186.13,129.62) --
	(186.35,134.52) --
	(186.58,139.57) --
	(186.80,144.69) --
	(187.02,149.84) --
	(187.25,154.98) --
	(187.47,160.08) --
	(187.69,165.10) --
	(187.91,169.96) --
	(188.14,174.61) --
	(188.36,179.03) --
	(188.58,183.19) --
	(188.80,187.04) --
	(189.03,190.56) --
	(189.25,193.73) --
	(189.47,196.45) --
	(189.70,198.76) --
	(189.92,200.67) --
	(190.14,202.17) --
	(190.36,203.27) --
	(190.59,203.98) --
	(190.81,204.28) --
	(191.03,204.16) --
	(191.26,203.70) --
	(191.48,202.94) --
	(191.70,201.89) --
	(191.92,200.59) --
	(192.15,199.06) --
	(192.37,197.33) --
	(192.59,195.43) --
	(192.81,193.43) --
	(193.04,191.34) --
	(193.26,189.19) --
	(193.48,187.01) --
	(193.71,184.83) --
	(193.93,182.67) --
	(194.15,180.56) --
	(194.37,178.50) --
	(194.60,176.53) --
	(194.82,174.63) --
	(195.04,172.83) --
	(195.26,171.14) --
	(195.49,169.56) --
	(195.71,168.09) --
	(195.93,166.74) --
	(196.16,165.48) --
	(196.38,164.34) --
	(196.60,163.30) --
	(196.82,162.38) --
	(197.05,161.55) --
	(197.27,160.82) --
	(197.49,160.18) --
	(197.72,159.61) --
	(197.94,159.13) --
	(198.16,158.73) --
	(198.38,158.40) --
	(198.61,158.14) --
	(198.83,157.93) --
	(199.05,157.77) --
	(199.27,157.66) --
	(199.50,157.59) --
	(199.72,157.56) --
	(199.94,157.56) --
	(200.17,157.58) --
	(200.39,157.61) --
	(200.61,157.66) --
	(200.83,157.70) --
	(201.06,157.75) --
	(201.28,157.79) --
	(201.50,157.82) --
	(201.72,157.83) --
	(201.95,157.82) --
	(202.17,157.80) --
	(202.39,157.75) --
	(202.62,157.67) --
	(202.84,157.56) --
	(203.06,157.42) --
	(203.28,157.25) --
	(203.51,157.05) --
	(203.73,156.82) --
	(203.95,156.55) --
	(204.17,156.24) --
	(204.40,155.90) --
	(204.62,155.52) --
	(204.84,155.10) --
	(205.07,154.65) --
	(205.29,154.15) --
	(205.51,153.62) --
	(205.73,153.04) --
	(205.96,152.43) --
	(206.18,151.78) --
	(206.40,151.09) --
	(206.63,150.38) --
	(206.85,149.63) --
	(207.07,148.85) --
	(207.29,148.04) --
	(207.52,147.22) --
	(207.74,146.38) --
	(207.96,145.52) --
	(208.18,144.66) --
	(208.41,143.79) --
	(208.63,142.92) --
	(208.85,142.05) --
	(209.08,141.19) --
	(209.30,140.32) --
	(209.52,139.46) --
	(209.74,138.61) --
	(209.97,137.76) --
	(210.19,136.91) --
	(210.41,136.07) --
	(210.63,135.23) --
	(210.86,134.38) --
	(211.08,133.54) --
	(211.30,132.69) --
	(211.53,131.84) --
	(211.75,130.98) --
	(211.97,130.12) --
	(212.19,129.25) --
	(212.42,128.37) --
	(212.64,127.49) --
	(212.86,126.60) --
	(213.09,125.71) --
	(213.31,124.82) --
	(213.53,123.94) --
	(213.75,123.06) --
	(213.98,122.18) --
	(214.20,121.31) --
	(214.42,120.46) --
	(214.64,119.63) --
	(214.87,118.81) --
	(215.09,118.01) --
	(215.31,117.23) --
	(215.54,116.48) --
	(215.76,115.75) --
	(215.98,115.05) --
	(216.20,114.37) --
	(216.43,113.72) --
	(216.65,113.09) --
	(216.87,112.49) --
	(217.09,111.91) --
	(217.32,111.35) --
	(217.54,110.82) --
	(217.76,110.31) --
	(217.99,109.81) --
	(218.21,109.33) --
	(218.43,108.86) --
	(218.65,108.41) --
	(218.88,107.97) --
	(219.10,107.53) --
	(219.32,107.10) --
	(219.55,106.68) --
	(219.77,106.26) --
	(219.99,105.83) --
	(220.21,105.41) --
	(220.44,104.98) --
	(220.66,104.54) --
	(220.88,104.10) --
	(221.10,103.65) --
	(221.33,103.19) --
	(221.55,102.72) --
	(221.77,102.24) --
	(222.00,101.76) --
	(222.22,101.26) --
	(222.44,100.76) --
	(222.66,100.25) --
	(222.89, 99.73) --
	(223.11, 99.21) --
	(223.33, 98.70) --
	(223.55, 98.19) --
	(223.78, 97.69) --
	(224.00, 97.19) --
	(224.22, 96.72) --
	(224.45, 96.26) --
	(224.67, 95.82) --
	(224.89, 95.41) --
	(225.11, 95.03) --
	(225.34, 94.67) --
	(225.56, 94.35) --
	(225.78, 94.06) --
	(226.00, 93.81) --
	(226.23, 93.59) --
	(226.45, 93.41) --
	(226.67, 93.26) --
	(226.90, 93.14) --
	(227.12, 93.04) --
	(227.34, 92.98) --
	(227.56, 92.93) --
	(227.79, 92.91) --
	(228.01, 92.89) --
	(228.23, 92.89) --
	(228.46, 92.90) --
	(228.68, 92.90) --
	(228.90, 92.91) --
	(229.12, 92.91) --
	(229.35, 92.91) --
	(229.57, 92.89) --
	(229.79, 92.87) --
	(230.01, 92.83) --
	(230.24, 92.78) --
	(230.46, 92.71) --
	(230.68, 92.63) --
	(230.91, 92.53) --
	(231.13, 92.42) --
	(231.35, 92.29) --
	(231.57, 92.16) --
	(231.80, 92.01) --
	(232.02, 91.85) --
	(232.24, 91.69) --
	(232.46, 91.52) --
	(232.69, 91.34) --
	(232.91, 91.16) --
	(233.13, 90.98) --
	(233.36, 90.80) --
	(233.58, 90.62) --
	(233.80, 90.44) --
	(234.02, 90.27) --
	(234.25, 90.10) --
	(234.47, 89.94) --
	(234.69, 89.78) --
	(234.92, 89.63) --
	(235.14, 89.48) --
	(235.36, 89.35) --
	(235.58, 89.22) --
	(235.81, 89.10) --
	(236.03, 88.99) --
	(236.25, 88.89) --
	(236.47, 88.79) --
	(236.70, 88.70) --
	(236.92, 88.62) --
	(237.14, 88.54) --
	(237.37, 88.47) --
	(237.59, 88.40) --
	(237.81, 88.33) --
	(238.03, 88.27) --
	(238.26, 88.20) --
	(238.48, 88.14) --
	(238.70, 88.07) --
	(238.92, 88.01) --
	(239.15, 87.93) --
	(239.37, 87.86) --
	(239.59, 87.78) --
	(239.82, 87.70) --
	(240.04, 87.61) --
	(240.26, 87.52) --
	(240.48, 87.42) --
	(240.71, 87.32) --
	(240.93, 87.22) --
	(241.15, 87.11) --
	(241.38, 87.00) --
	(241.60, 86.89) --
	(241.82, 86.78) --
	(242.04, 86.67) --
	(242.27, 86.56) --
	(242.49, 86.45) --
	(242.71, 86.34) --
	(242.93, 86.23) --
	(243.16, 86.13) --
	(243.38, 86.03) --
	(243.60, 85.92) --
	(243.83, 85.82) --
	(244.05, 85.73) --
	(244.27, 85.63) --
	(244.49, 85.53) --
	(244.72, 85.43) --
	(244.94, 85.34) --
	(245.16, 85.24) --
	(245.38, 85.13) --
	(245.61, 85.03) --
	(245.83, 84.91) --
	(246.05, 84.80) --
	(246.28, 84.68) --
	(246.50, 84.55) --
	(246.72, 84.42) --
	(246.94, 84.29) --
	(247.17, 84.15) --
	(247.39, 84.00) --
	(247.61, 83.85) --
	(247.84, 83.69) --
	(248.06, 83.53) --
	(248.28, 83.37) --
	(248.50, 83.21) --
	(248.73, 83.04) --
	(248.95, 82.87) --
	(249.17, 82.71) --
	(249.39, 82.54) --
	(249.62, 82.38) --
	(249.84, 82.22) --
	(250.06, 82.07) --
	(250.29, 81.92) --
	(250.51, 81.78) --
	(250.73, 81.64) --
	(250.95, 81.51) --
	(251.18, 81.39) --
	(251.40, 81.28) --
	(251.62, 81.17) --
	(251.84, 81.07) --
	(252.07, 80.99) --
	(252.29, 80.91) --
	(252.51, 80.84) --
	(252.74, 80.77) --
	(252.96, 80.72) --
	(253.18, 80.66) --
	(253.40, 80.62) --
	(253.63, 80.58) --
	(253.85, 80.54) --
	(254.07, 80.50) --
	(254.29, 80.46) --
	(254.52, 80.42) --
	(254.74, 80.37) --
	(254.96, 80.33) --
	(255.19, 80.28) --
	(255.41, 80.22) --
	(255.63, 80.16) --
	(255.85, 80.09) --
	(256.08, 80.01) --
	(256.30, 79.93) --
	(256.52, 79.85) --
	(256.75, 79.76) --
	(256.97, 79.66) --
	(257.19, 79.57) --
	(257.41, 79.47) --
	(257.64, 79.37) --
	(257.86, 79.27) --
	(258.08, 79.17) --
	(258.30, 79.08) --
	(258.53, 78.99) --
	(258.75, 78.91) --
	(258.97, 78.83) --
	(259.20, 78.76) --
	(259.42, 78.70) --
	(259.64, 78.64) --
	(259.86, 78.59) --
	(260.09, 78.55) --
	(260.31, 78.52) --
	(260.53, 78.50) --
	(260.75, 78.48) --
	(260.98, 78.47) --
	(261.20, 78.46) --
	(261.42, 78.46) --
	(261.65, 78.46) --
	(261.87, 78.46) --
	(262.09, 78.46) --
	(262.31, 78.46) --
	(262.54, 78.47) --
	(262.76, 78.47) --
	(262.98, 78.47) --
	(263.21, 78.47) --
	(263.43, 78.46) --
	(263.65, 78.46) --
	(263.87, 78.45) --
	(264.10, 78.44) --
	(264.32, 78.43) --
	(264.54, 78.42) --
	(264.76, 78.40) --
	(264.99, 78.39) --
	(265.21, 78.37) --
	(265.43, 78.36) --
	(265.66, 78.35) --
	(265.88, 78.33) --
	(266.10, 78.32) --
	(266.32, 78.30) --
	(266.55, 78.29) --
	(266.77, 78.28) --
	(266.99, 78.26) --
	(267.21, 78.25) --
	(267.44, 78.24) --
	(267.66, 78.22) --
	(267.88, 78.20) --
	(268.11, 78.18) --
	(268.33, 78.16) --
	(268.55, 78.14) --
	(268.77, 78.11) --
	(269.00, 78.08) --
	(269.22, 78.05) --
	(269.44, 78.01) --
	(269.67, 77.97) --
	(269.89, 77.93) --
	(270.11, 77.88) --
	(270.33, 77.82) --
	(270.56, 77.77) --
	(270.78, 77.71) --
	(271.00, 77.64) --
	(271.22, 77.57) --
	(271.45, 77.51) --
	(271.67, 77.44) --
	(271.89, 77.36) --
	(272.12, 77.29) --
	(272.34, 77.22) --
	(272.56, 77.15) --
	(272.78, 77.08) --
	(273.01, 77.01) --
	(273.23, 76.95) --
	(273.45, 76.88) --
	(273.67, 76.82) --
	(273.90, 76.77) --
	(274.12, 76.72) --
	(274.34, 76.67) --
	(274.57, 76.63) --
	(274.79, 76.59) --
	(275.01, 76.56) --
	(275.23, 76.53) --
	(275.46, 76.50) --
	(275.68, 76.48) --
	(275.90, 76.46) --
	(276.12, 76.45) --
	(276.35, 76.44) --
	(276.57, 76.43) --
	(276.79, 76.42) --
	(277.02, 76.41) --
	(277.24, 76.40) --
	(277.46, 76.40) --
	(277.68, 76.39) --
	(277.91, 76.39) --
	(278.13, 76.38) --
	(278.35, 76.38) --
	(278.58, 76.37) --
	(278.80, 76.37) --
	(279.02, 76.36) --
	(279.24, 76.36) --
	(279.47, 76.35) --
	(279.69, 76.34) --
	(279.91, 76.34) --
	(280.13, 76.33) --
	(280.36, 76.32) --
	(280.58, 76.31) --
	(280.80, 76.31) --
	(281.03, 76.30) --
	(281.25, 76.29) --
	(281.47, 76.28) --
	(281.69, 76.28) --
	(281.92, 76.27) --
	(282.14, 76.26) --
	(282.36, 76.26) --
	(282.58, 76.25) --
	(282.81, 76.25) --
	(283.03, 76.25) --
	(283.25, 76.24) --
	(283.48, 76.24) --
	(283.70, 76.24) --
	(283.92, 76.23) --
	(284.14, 76.23) --
	(284.37, 76.23) --
	(284.59, 76.23) --
	(284.81, 76.23) --
	(285.04, 76.23) --
	(285.26, 76.23) --
	(285.48, 76.23) --
	(285.70, 76.23) --
	(285.93, 76.23) --
	(286.15, 76.23) --
	(286.37, 76.23) --
	(286.59, 76.23) --
	(286.82, 76.23) --
	(287.04, 76.23) --
	(287.26, 76.22) --
	(287.49, 76.22) --
	(287.71, 76.22) --
	(287.93, 76.22) --
	(288.15, 76.22) --
	(288.38, 76.22) --
	(288.60, 76.22) --
	(288.82, 76.22) --
	(289.04, 76.22) --
	(289.27, 76.22) --
	(289.49, 76.22) --
	(289.71, 76.22) --
	(289.94, 76.22) --
	(290.16, 76.22) --
	(290.38, 76.22) --
	(290.60, 76.22) --
	(290.83, 76.22) --
	(291.05, 76.22) --
	(291.27, 76.22) --
	(291.50, 76.22) --
	(291.72, 76.22);
\definecolor{fillColor}{RGB}{55,126,184}

\path[fill=fillColor,fill opacity=0.50] (177.89, 76.22) --
	(178.11, 76.22) --
	(178.34, 76.22) --
	(178.56, 76.22) --
	(178.78, 76.22) --
	(179.00, 76.22) --
	(179.23, 76.22) --
	(179.45, 76.22) --
	(179.67, 76.22) --
	(179.89, 76.23) --
	(180.12, 76.23) --
	(180.34, 76.23) --
	(180.56, 76.23) --
	(180.79, 76.23) --
	(181.01, 76.24) --
	(181.23, 76.24) --
	(181.45, 76.26) --
	(181.68, 76.28) --
	(181.90, 76.32) --
	(182.12, 76.38) --
	(182.34, 76.47) --
	(182.57, 76.60) --
	(182.79, 76.80) --
	(183.01, 77.09) --
	(183.24, 77.48) --
	(183.46, 78.03) --
	(183.68, 78.76) --
	(183.90, 79.75) --
	(184.13, 81.03) --
	(184.35, 82.64) --
	(184.57, 84.62) --
	(184.80, 86.99) --
	(185.02, 89.78) --
	(185.24, 92.99) --
	(185.46, 96.59) --
	(185.69,100.55) --
	(185.91,104.82) --
	(186.13,109.31) --
	(186.35,113.93) --
	(186.58,118.56) --
	(186.80,123.10) --
	(187.02,127.44) --
	(187.25,131.51) --
	(187.47,135.24) --
	(187.69,138.57) --
	(187.91,141.48) --
	(188.14,143.96) --
	(188.36,146.02) --
	(188.58,147.65) --
	(188.80,148.93) --
	(189.03,149.90) --
	(189.25,150.63) --
	(189.47,151.14) --
	(189.70,151.48) --
	(189.92,151.69) --
	(190.14,151.80) --
	(190.36,151.84) --
	(190.59,151.81) --
	(190.81,151.73) --
	(191.03,151.61) --
	(191.26,151.46) --
	(191.48,151.28) --
	(191.70,151.07) --
	(191.92,150.83) --
	(192.15,150.57) --
	(192.37,150.28) --
	(192.59,149.97) --
	(192.81,149.65) --
	(193.04,149.30) --
	(193.26,148.95) --
	(193.48,148.57) --
	(193.71,148.19) --
	(193.93,147.78) --
	(194.15,147.37) --
	(194.37,146.94) --
	(194.60,146.49) --
	(194.82,146.01) --
	(195.04,145.52) --
	(195.26,145.00) --
	(195.49,144.46) --
	(195.71,143.90) --
	(195.93,143.32) --
	(196.16,142.71) --
	(196.38,142.10) --
	(196.60,141.47) --
	(196.82,140.84) --
	(197.05,140.20) --
	(197.27,139.56) --
	(197.49,138.93) --
	(197.72,138.30) --
	(197.94,137.69) --
	(198.16,137.10) --
	(198.38,136.54) --
	(198.61,136.00) --
	(198.83,135.50) --
	(199.05,135.03) --
	(199.27,134.60) --
	(199.50,134.21) --
	(199.72,133.85) --
	(199.94,133.52) --
	(200.17,133.23) --
	(200.39,132.95) --
	(200.61,132.70) --
	(200.83,132.45) --
	(201.06,132.22) --
	(201.28,132.00) --
	(201.50,131.78) --
	(201.72,131.57) --
	(201.95,131.36) --
	(202.17,131.16) --
	(202.39,130.97) --
	(202.62,130.77) --
	(202.84,130.58) --
	(203.06,130.39) --
	(203.28,130.21) --
	(203.51,130.03) --
	(203.73,129.85) --
	(203.95,129.68) --
	(204.17,129.53) --
	(204.40,129.39) --
	(204.62,129.27) --
	(204.84,129.18) --
	(205.07,129.11) --
	(205.29,129.08) --
	(205.51,129.07) --
	(205.73,129.10) --
	(205.96,129.15) --
	(206.18,129.23) --
	(206.40,129.33) --
	(206.63,129.45) --
	(206.85,129.59) --
	(207.07,129.74) --
	(207.29,129.90) --
	(207.52,130.07) --
	(207.74,130.25) --
	(207.96,130.43) --
	(208.18,130.61) --
	(208.41,130.78) --
	(208.63,130.95) --
	(208.85,131.10) --
	(209.08,131.24) --
	(209.30,131.35) --
	(209.52,131.43) --
	(209.74,131.47) --
	(209.97,131.48) --
	(210.19,131.45) --
	(210.41,131.38) --
	(210.63,131.27) --
	(210.86,131.12) --
	(211.08,130.93) --
	(211.30,130.71) --
	(211.53,130.45) --
	(211.75,130.16) --
	(211.97,129.85) --
	(212.19,129.51) --
	(212.42,129.17) --
	(212.64,128.81) --
	(212.86,128.45) --
	(213.09,128.08) --
	(213.31,127.72) --
	(213.53,127.35) --
	(213.75,126.99) --
	(213.98,126.62) --
	(214.20,126.25) --
	(214.42,125.87) --
	(214.64,125.49) --
	(214.87,125.09) --
	(215.09,124.68) --
	(215.31,124.26) --
	(215.54,123.83) --
	(215.76,123.41) --
	(215.98,122.98) --
	(216.20,122.56) --
	(216.43,122.15) --
	(216.65,121.76) --
	(216.87,121.39) --
	(217.09,121.03) --
	(217.32,120.69) --
	(217.54,120.37) --
	(217.76,120.07) --
	(217.99,119.78) --
	(218.21,119.50) --
	(218.43,119.23) --
	(218.65,118.97) --
	(218.88,118.71) --
	(219.10,118.47) --
	(219.32,118.24) --
	(219.55,118.01) --
	(219.77,117.80) --
	(219.99,117.61) --
	(220.21,117.43) --
	(220.44,117.26) --
	(220.66,117.11) --
	(220.88,116.98) --
	(221.10,116.87) --
	(221.33,116.77) --
	(221.55,116.67) --
	(221.77,116.59) --
	(222.00,116.50) --
	(222.22,116.42) --
	(222.44,116.33) --
	(222.66,116.23) --
	(222.89,116.13) --
	(223.11,116.02) --
	(223.33,115.90) --
	(223.55,115.77) --
	(223.78,115.64) --
	(224.00,115.51) --
	(224.22,115.39) --
	(224.45,115.28) --
	(224.67,115.19) --
	(224.89,115.11) --
	(225.11,115.05) --
	(225.34,115.01) --
	(225.56,114.99) --
	(225.78,114.98) --
	(226.00,114.98) --
	(226.23,114.99) --
	(226.45,114.99) --
	(226.67,114.99) --
	(226.90,114.97) --
	(227.12,114.95) --
	(227.34,114.91) --
	(227.56,114.86) --
	(227.79,114.78) --
	(228.01,114.69) --
	(228.23,114.59) --
	(228.46,114.47) --
	(228.68,114.35) --
	(228.90,114.21) --
	(229.12,114.06) --
	(229.35,113.90) --
	(229.57,113.74) --
	(229.79,113.57) --
	(230.01,113.39) --
	(230.24,113.21) --
	(230.46,113.03) --
	(230.68,112.85) --
	(230.91,112.68) --
	(231.13,112.50) --
	(231.35,112.34) --
	(231.57,112.18) --
	(231.80,112.03) --
	(232.02,111.89) --
	(232.24,111.77) --
	(232.46,111.65) --
	(232.69,111.56) --
	(232.91,111.47) --
	(233.13,111.39) --
	(233.36,111.32) --
	(233.58,111.25) --
	(233.80,111.19) --
	(234.02,111.12) --
	(234.25,111.06) --
	(234.47,110.99) --
	(234.69,110.91) --
	(234.92,110.82) --
	(235.14,110.73) --
	(235.36,110.63) --
	(235.58,110.52) --
	(235.81,110.40) --
	(236.03,110.28) --
	(236.25,110.16) --
	(236.47,110.03) --
	(236.70,109.90) --
	(236.92,109.77) --
	(237.14,109.64) --
	(237.37,109.50) --
	(237.59,109.36) --
	(237.81,109.22) --
	(238.03,109.08) --
	(238.26,108.93) --
	(238.48,108.77) --
	(238.70,108.61) --
	(238.92,108.45) --
	(239.15,108.28) --
	(239.37,108.10) --
	(239.59,107.92) --
	(239.82,107.73) --
	(240.04,107.52) --
	(240.26,107.31) --
	(240.48,107.09) --
	(240.71,106.85) --
	(240.93,106.61) --
	(241.15,106.35) --
	(241.38,106.08) --
	(241.60,105.81) --
	(241.82,105.53) --
	(242.04,105.24) --
	(242.27,104.95) --
	(242.49,104.66) --
	(242.71,104.36) --
	(242.93,104.07) --
	(243.16,103.78) --
	(243.38,103.49) --
	(243.60,103.20) --
	(243.83,102.92) --
	(244.05,102.63) --
	(244.27,102.34) --
	(244.49,102.06) --
	(244.72,101.77) --
	(244.94,101.48) --
	(245.16,101.19) --
	(245.38,100.89) --
	(245.61,100.60) --
	(245.83,100.30) --
	(246.05, 99.99) --
	(246.28, 99.69) --
	(246.50, 99.39) --
	(246.72, 99.08) --
	(246.94, 98.78) --
	(247.17, 98.48) --
	(247.39, 98.19) --
	(247.61, 97.90) --
	(247.84, 97.62) --
	(248.06, 97.33) --
	(248.28, 97.05) --
	(248.50, 96.77) --
	(248.73, 96.49) --
	(248.95, 96.21) --
	(249.17, 95.92) --
	(249.39, 95.63) --
	(249.62, 95.33) --
	(249.84, 95.03) --
	(250.06, 94.72) --
	(250.29, 94.41) --
	(250.51, 94.09) --
	(250.73, 93.78) --
	(250.95, 93.46) --
	(251.18, 93.14) --
	(251.40, 92.83) --
	(251.62, 92.51) --
	(251.84, 92.21) --
	(252.07, 91.90) --
	(252.29, 91.61) --
	(252.51, 91.32) --
	(252.74, 91.04) --
	(252.96, 90.76) --
	(253.18, 90.50) --
	(253.40, 90.24) --
	(253.63, 90.00) --
	(253.85, 89.76) --
	(254.07, 89.53) --
	(254.29, 89.32) --
	(254.52, 89.11) --
	(254.74, 88.91) --
	(254.96, 88.72) --
	(255.19, 88.54) --
	(255.41, 88.37) --
	(255.63, 88.20) --
	(255.85, 88.04) --
	(256.08, 87.88) --
	(256.30, 87.72) --
	(256.52, 87.56) --
	(256.75, 87.41) --
	(256.97, 87.25) --
	(257.19, 87.09) --
	(257.41, 86.92) --
	(257.64, 86.76) --
	(257.86, 86.59) --
	(258.08, 86.41) --
	(258.30, 86.24) --
	(258.53, 86.06) --
	(258.75, 85.89) --
	(258.97, 85.71) --
	(259.20, 85.54) --
	(259.42, 85.37) --
	(259.64, 85.20) --
	(259.86, 85.04) --
	(260.09, 84.87) --
	(260.31, 84.71) --
	(260.53, 84.55) --
	(260.75, 84.39) --
	(260.98, 84.24) --
	(261.20, 84.09) --
	(261.42, 83.94) --
	(261.65, 83.80) --
	(261.87, 83.66) --
	(262.09, 83.52) --
	(262.31, 83.38) --
	(262.54, 83.25) --
	(262.76, 83.12) --
	(262.98, 82.98) --
	(263.21, 82.85) --
	(263.43, 82.72) --
	(263.65, 82.58) --
	(263.87, 82.44) --
	(264.10, 82.29) --
	(264.32, 82.14) --
	(264.54, 81.99) --
	(264.76, 81.82) --
	(264.99, 81.65) --
	(265.21, 81.48) --
	(265.43, 81.30) --
	(265.66, 81.12) --
	(265.88, 80.93) --
	(266.10, 80.74) --
	(266.32, 80.56) --
	(266.55, 80.37) --
	(266.77, 80.19) --
	(266.99, 80.02) --
	(267.21, 79.85) --
	(267.44, 79.69) --
	(267.66, 79.54) --
	(267.88, 79.39) --
	(268.11, 79.25) --
	(268.33, 79.12) --
	(268.55, 79.00) --
	(268.77, 78.88) --
	(269.00, 78.76) --
	(269.22, 78.65) --
	(269.44, 78.55) --
	(269.67, 78.44) --
	(269.89, 78.35) --
	(270.11, 78.25) --
	(270.33, 78.16) --
	(270.56, 78.07) --
	(270.78, 77.98) --
	(271.00, 77.90) --
	(271.22, 77.82) --
	(271.45, 77.75) --
	(271.67, 77.68) --
	(271.89, 77.61) --
	(272.12, 77.55) --
	(272.34, 77.49) --
	(272.56, 77.43) --
	(272.78, 77.38) --
	(273.01, 77.33) --
	(273.23, 77.28) --
	(273.45, 77.23) --
	(273.67, 77.18) --
	(273.90, 77.14) --
	(274.12, 77.09) --
	(274.34, 77.04) --
	(274.57, 77.00) --
	(274.79, 76.96) --
	(275.01, 76.91) --
	(275.23, 76.87) --
	(275.46, 76.83) --
	(275.68, 76.78) --
	(275.90, 76.74) --
	(276.12, 76.70) --
	(276.35, 76.67) --
	(276.57, 76.63) --
	(276.79, 76.60) --
	(277.02, 76.57) --
	(277.24, 76.54) --
	(277.46, 76.51) --
	(277.68, 76.49) --
	(277.91, 76.47) --
	(278.13, 76.44) --
	(278.35, 76.42) --
	(278.58, 76.41) --
	(278.80, 76.39) --
	(279.02, 76.37) --
	(279.24, 76.36) --
	(279.47, 76.35) --
	(279.69, 76.34) --
	(279.91, 76.33) --
	(280.13, 76.32) --
	(280.36, 76.31) --
	(280.58, 76.30) --
	(280.80, 76.29) --
	(281.03, 76.29) --
	(281.25, 76.28) --
	(281.47, 76.28) --
	(281.69, 76.27) --
	(281.92, 76.27) --
	(282.14, 76.26) --
	(282.36, 76.26) --
	(282.58, 76.26) --
	(282.81, 76.26) --
	(283.03, 76.25) --
	(283.25, 76.25) --
	(283.48, 76.25) --
	(283.70, 76.25) --
	(283.92, 76.24) --
	(284.14, 76.24) --
	(284.37, 76.24) --
	(284.59, 76.24) --
	(284.81, 76.24) --
	(285.04, 76.24) --
	(285.26, 76.23) --
	(285.48, 76.23) --
	(285.70, 76.23) --
	(285.93, 76.23) --
	(286.15, 76.23) --
	(286.37, 76.23) --
	(286.59, 76.23) --
	(286.82, 76.23) --
	(287.04, 76.23) --
	(287.26, 76.23) --
	(287.49, 76.23) --
	(287.71, 76.23) --
	(287.93, 76.23) --
	(288.15, 76.23) --
	(288.38, 76.23) --
	(288.60, 76.23) --
	(288.82, 76.22) --
	(289.04, 76.22) --
	(289.27, 76.22) --
	(289.49, 76.22) --
	(289.71, 76.22) --
	(289.94, 76.22) --
	(290.16, 76.22) --
	(290.38, 76.22) --
	(290.60, 76.22) --
	(290.83, 76.22) --
	(291.05, 76.22) --
	(291.27, 76.22) --
	(291.50, 76.22) --
	(291.72, 76.22) --
	(291.72, 76.22) --
	(291.50, 76.22) --
	(291.27, 76.22) --
	(291.05, 76.22) --
	(290.83, 76.22) --
	(290.60, 76.22) --
	(290.38, 76.22) --
	(290.16, 76.22) --
	(289.94, 76.22) --
	(289.71, 76.22) --
	(289.49, 76.22) --
	(289.27, 76.22) --
	(289.04, 76.22) --
	(288.82, 76.22) --
	(288.60, 76.22) --
	(288.38, 76.22) --
	(288.15, 76.22) --
	(287.93, 76.22) --
	(287.71, 76.22) --
	(287.49, 76.22) --
	(287.26, 76.22) --
	(287.04, 76.22) --
	(286.82, 76.22) --
	(286.59, 76.22) --
	(286.37, 76.22) --
	(286.15, 76.22) --
	(285.93, 76.22) --
	(285.70, 76.22) --
	(285.48, 76.22) --
	(285.26, 76.22) --
	(285.04, 76.22) --
	(284.81, 76.22) --
	(284.59, 76.22) --
	(284.37, 76.22) --
	(284.14, 76.22) --
	(283.92, 76.22) --
	(283.70, 76.22) --
	(283.48, 76.22) --
	(283.25, 76.22) --
	(283.03, 76.22) --
	(282.81, 76.22) --
	(282.58, 76.22) --
	(282.36, 76.22) --
	(282.14, 76.22) --
	(281.92, 76.22) --
	(281.69, 76.22) --
	(281.47, 76.22) --
	(281.25, 76.22) --
	(281.03, 76.22) --
	(280.80, 76.22) --
	(280.58, 76.22) --
	(280.36, 76.22) --
	(280.13, 76.22) --
	(279.91, 76.22) --
	(279.69, 76.22) --
	(279.47, 76.22) --
	(279.24, 76.22) --
	(279.02, 76.22) --
	(278.80, 76.22) --
	(278.58, 76.22) --
	(278.35, 76.22) --
	(278.13, 76.22) --
	(277.91, 76.22) --
	(277.68, 76.22) --
	(277.46, 76.22) --
	(277.24, 76.22) --
	(277.02, 76.22) --
	(276.79, 76.22) --
	(276.57, 76.22) --
	(276.35, 76.22) --
	(276.12, 76.22) --
	(275.90, 76.22) --
	(275.68, 76.22) --
	(275.46, 76.22) --
	(275.23, 76.22) --
	(275.01, 76.22) --
	(274.79, 76.22) --
	(274.57, 76.22) --
	(274.34, 76.22) --
	(274.12, 76.22) --
	(273.90, 76.22) --
	(273.67, 76.22) --
	(273.45, 76.22) --
	(273.23, 76.22) --
	(273.01, 76.22) --
	(272.78, 76.22) --
	(272.56, 76.22) --
	(272.34, 76.22) --
	(272.12, 76.22) --
	(271.89, 76.22) --
	(271.67, 76.22) --
	(271.45, 76.22) --
	(271.22, 76.22) --
	(271.00, 76.22) --
	(270.78, 76.22) --
	(270.56, 76.22) --
	(270.33, 76.22) --
	(270.11, 76.22) --
	(269.89, 76.22) --
	(269.67, 76.22) --
	(269.44, 76.22) --
	(269.22, 76.22) --
	(269.00, 76.22) --
	(268.77, 76.22) --
	(268.55, 76.22) --
	(268.33, 76.22) --
	(268.11, 76.22) --
	(267.88, 76.22) --
	(267.66, 76.22) --
	(267.44, 76.22) --
	(267.21, 76.22) --
	(266.99, 76.22) --
	(266.77, 76.22) --
	(266.55, 76.22) --
	(266.32, 76.22) --
	(266.10, 76.22) --
	(265.88, 76.22) --
	(265.66, 76.22) --
	(265.43, 76.22) --
	(265.21, 76.22) --
	(264.99, 76.22) --
	(264.76, 76.22) --
	(264.54, 76.22) --
	(264.32, 76.22) --
	(264.10, 76.22) --
	(263.87, 76.22) --
	(263.65, 76.22) --
	(263.43, 76.22) --
	(263.21, 76.22) --
	(262.98, 76.22) --
	(262.76, 76.22) --
	(262.54, 76.22) --
	(262.31, 76.22) --
	(262.09, 76.22) --
	(261.87, 76.22) --
	(261.65, 76.22) --
	(261.42, 76.22) --
	(261.20, 76.22) --
	(260.98, 76.22) --
	(260.75, 76.22) --
	(260.53, 76.22) --
	(260.31, 76.22) --
	(260.09, 76.22) --
	(259.86, 76.22) --
	(259.64, 76.22) --
	(259.42, 76.22) --
	(259.20, 76.22) --
	(258.97, 76.22) --
	(258.75, 76.22) --
	(258.53, 76.22) --
	(258.30, 76.22) --
	(258.08, 76.22) --
	(257.86, 76.22) --
	(257.64, 76.22) --
	(257.41, 76.22) --
	(257.19, 76.22) --
	(256.97, 76.22) --
	(256.75, 76.22) --
	(256.52, 76.22) --
	(256.30, 76.22) --
	(256.08, 76.22) --
	(255.85, 76.22) --
	(255.63, 76.22) --
	(255.41, 76.22) --
	(255.19, 76.22) --
	(254.96, 76.22) --
	(254.74, 76.22) --
	(254.52, 76.22) --
	(254.29, 76.22) --
	(254.07, 76.22) --
	(253.85, 76.22) --
	(253.63, 76.22) --
	(253.40, 76.22) --
	(253.18, 76.22) --
	(252.96, 76.22) --
	(252.74, 76.22) --
	(252.51, 76.22) --
	(252.29, 76.22) --
	(252.07, 76.22) --
	(251.84, 76.22) --
	(251.62, 76.22) --
	(251.40, 76.22) --
	(251.18, 76.22) --
	(250.95, 76.22) --
	(250.73, 76.22) --
	(250.51, 76.22) --
	(250.29, 76.22) --
	(250.06, 76.22) --
	(249.84, 76.22) --
	(249.62, 76.22) --
	(249.39, 76.22) --
	(249.17, 76.22) --
	(248.95, 76.22) --
	(248.73, 76.22) --
	(248.50, 76.22) --
	(248.28, 76.22) --
	(248.06, 76.22) --
	(247.84, 76.22) --
	(247.61, 76.22) --
	(247.39, 76.22) --
	(247.17, 76.22) --
	(246.94, 76.22) --
	(246.72, 76.22) --
	(246.50, 76.22) --
	(246.28, 76.22) --
	(246.05, 76.22) --
	(245.83, 76.22) --
	(245.61, 76.22) --
	(245.38, 76.22) --
	(245.16, 76.22) --
	(244.94, 76.22) --
	(244.72, 76.22) --
	(244.49, 76.22) --
	(244.27, 76.22) --
	(244.05, 76.22) --
	(243.83, 76.22) --
	(243.60, 76.22) --
	(243.38, 76.22) --
	(243.16, 76.22) --
	(242.93, 76.22) --
	(242.71, 76.22) --
	(242.49, 76.22) --
	(242.27, 76.22) --
	(242.04, 76.22) --
	(241.82, 76.22) --
	(241.60, 76.22) --
	(241.38, 76.22) --
	(241.15, 76.22) --
	(240.93, 76.22) --
	(240.71, 76.22) --
	(240.48, 76.22) --
	(240.26, 76.22) --
	(240.04, 76.22) --
	(239.82, 76.22) --
	(239.59, 76.22) --
	(239.37, 76.22) --
	(239.15, 76.22) --
	(238.92, 76.22) --
	(238.70, 76.22) --
	(238.48, 76.22) --
	(238.26, 76.22) --
	(238.03, 76.22) --
	(237.81, 76.22) --
	(237.59, 76.22) --
	(237.37, 76.22) --
	(237.14, 76.22) --
	(236.92, 76.22) --
	(236.70, 76.22) --
	(236.47, 76.22) --
	(236.25, 76.22) --
	(236.03, 76.22) --
	(235.81, 76.22) --
	(235.58, 76.22) --
	(235.36, 76.22) --
	(235.14, 76.22) --
	(234.92, 76.22) --
	(234.69, 76.22) --
	(234.47, 76.22) --
	(234.25, 76.22) --
	(234.02, 76.22) --
	(233.80, 76.22) --
	(233.58, 76.22) --
	(233.36, 76.22) --
	(233.13, 76.22) --
	(232.91, 76.22) --
	(232.69, 76.22) --
	(232.46, 76.22) --
	(232.24, 76.22) --
	(232.02, 76.22) --
	(231.80, 76.22) --
	(231.57, 76.22) --
	(231.35, 76.22) --
	(231.13, 76.22) --
	(230.91, 76.22) --
	(230.68, 76.22) --
	(230.46, 76.22) --
	(230.24, 76.22) --
	(230.01, 76.22) --
	(229.79, 76.22) --
	(229.57, 76.22) --
	(229.35, 76.22) --
	(229.12, 76.22) --
	(228.90, 76.22) --
	(228.68, 76.22) --
	(228.46, 76.22) --
	(228.23, 76.22) --
	(228.01, 76.22) --
	(227.79, 76.22) --
	(227.56, 76.22) --
	(227.34, 76.22) --
	(227.12, 76.22) --
	(226.90, 76.22) --
	(226.67, 76.22) --
	(226.45, 76.22) --
	(226.23, 76.22) --
	(226.00, 76.22) --
	(225.78, 76.22) --
	(225.56, 76.22) --
	(225.34, 76.22) --
	(225.11, 76.22) --
	(224.89, 76.22) --
	(224.67, 76.22) --
	(224.45, 76.22) --
	(224.22, 76.22) --
	(224.00, 76.22) --
	(223.78, 76.22) --
	(223.55, 76.22) --
	(223.33, 76.22) --
	(223.11, 76.22) --
	(222.89, 76.22) --
	(222.66, 76.22) --
	(222.44, 76.22) --
	(222.22, 76.22) --
	(222.00, 76.22) --
	(221.77, 76.22) --
	(221.55, 76.22) --
	(221.33, 76.22) --
	(221.10, 76.22) --
	(220.88, 76.22) --
	(220.66, 76.22) --
	(220.44, 76.22) --
	(220.21, 76.22) --
	(219.99, 76.22) --
	(219.77, 76.22) --
	(219.55, 76.22) --
	(219.32, 76.22) --
	(219.10, 76.22) --
	(218.88, 76.22) --
	(218.65, 76.22) --
	(218.43, 76.22) --
	(218.21, 76.22) --
	(217.99, 76.22) --
	(217.76, 76.22) --
	(217.54, 76.22) --
	(217.32, 76.22) --
	(217.09, 76.22) --
	(216.87, 76.22) --
	(216.65, 76.22) --
	(216.43, 76.22) --
	(216.20, 76.22) --
	(215.98, 76.22) --
	(215.76, 76.22) --
	(215.54, 76.22) --
	(215.31, 76.22) --
	(215.09, 76.22) --
	(214.87, 76.22) --
	(214.64, 76.22) --
	(214.42, 76.22) --
	(214.20, 76.22) --
	(213.98, 76.22) --
	(213.75, 76.22) --
	(213.53, 76.22) --
	(213.31, 76.22) --
	(213.09, 76.22) --
	(212.86, 76.22) --
	(212.64, 76.22) --
	(212.42, 76.22) --
	(212.19, 76.22) --
	(211.97, 76.22) --
	(211.75, 76.22) --
	(211.53, 76.22) --
	(211.30, 76.22) --
	(211.08, 76.22) --
	(210.86, 76.22) --
	(210.63, 76.22) --
	(210.41, 76.22) --
	(210.19, 76.22) --
	(209.97, 76.22) --
	(209.74, 76.22) --
	(209.52, 76.22) --
	(209.30, 76.22) --
	(209.08, 76.22) --
	(208.85, 76.22) --
	(208.63, 76.22) --
	(208.41, 76.22) --
	(208.18, 76.22) --
	(207.96, 76.22) --
	(207.74, 76.22) --
	(207.52, 76.22) --
	(207.29, 76.22) --
	(207.07, 76.22) --
	(206.85, 76.22) --
	(206.63, 76.22) --
	(206.40, 76.22) --
	(206.18, 76.22) --
	(205.96, 76.22) --
	(205.73, 76.22) --
	(205.51, 76.22) --
	(205.29, 76.22) --
	(205.07, 76.22) --
	(204.84, 76.22) --
	(204.62, 76.22) --
	(204.40, 76.22) --
	(204.17, 76.22) --
	(203.95, 76.22) --
	(203.73, 76.22) --
	(203.51, 76.22) --
	(203.28, 76.22) --
	(203.06, 76.22) --
	(202.84, 76.22) --
	(202.62, 76.22) --
	(202.39, 76.22) --
	(202.17, 76.22) --
	(201.95, 76.22) --
	(201.72, 76.22) --
	(201.50, 76.22) --
	(201.28, 76.22) --
	(201.06, 76.22) --
	(200.83, 76.22) --
	(200.61, 76.22) --
	(200.39, 76.22) --
	(200.17, 76.22) --
	(199.94, 76.22) --
	(199.72, 76.22) --
	(199.50, 76.22) --
	(199.27, 76.22) --
	(199.05, 76.22) --
	(198.83, 76.22) --
	(198.61, 76.22) --
	(198.38, 76.22) --
	(198.16, 76.22) --
	(197.94, 76.22) --
	(197.72, 76.22) --
	(197.49, 76.22) --
	(197.27, 76.22) --
	(197.05, 76.22) --
	(196.82, 76.22) --
	(196.60, 76.22) --
	(196.38, 76.22) --
	(196.16, 76.22) --
	(195.93, 76.22) --
	(195.71, 76.22) --
	(195.49, 76.22) --
	(195.26, 76.22) --
	(195.04, 76.22) --
	(194.82, 76.22) --
	(194.60, 76.22) --
	(194.37, 76.22) --
	(194.15, 76.22) --
	(193.93, 76.22) --
	(193.71, 76.22) --
	(193.48, 76.22) --
	(193.26, 76.22) --
	(193.04, 76.22) --
	(192.81, 76.22) --
	(192.59, 76.22) --
	(192.37, 76.22) --
	(192.15, 76.22) --
	(191.92, 76.22) --
	(191.70, 76.22) --
	(191.48, 76.22) --
	(191.26, 76.22) --
	(191.03, 76.22) --
	(190.81, 76.22) --
	(190.59, 76.22) --
	(190.36, 76.22) --
	(190.14, 76.22) --
	(189.92, 76.22) --
	(189.70, 76.22) --
	(189.47, 76.22) --
	(189.25, 76.22) --
	(189.03, 76.22) --
	(188.80, 76.22) --
	(188.58, 76.22) --
	(188.36, 76.22) --
	(188.14, 76.22) --
	(187.91, 76.22) --
	(187.69, 76.22) --
	(187.47, 76.22) --
	(187.25, 76.22) --
	(187.02, 76.22) --
	(186.80, 76.22) --
	(186.58, 76.22) --
	(186.35, 76.22) --
	(186.13, 76.22) --
	(185.91, 76.22) --
	(185.69, 76.22) --
	(185.46, 76.22) --
	(185.24, 76.22) --
	(185.02, 76.22) --
	(184.80, 76.22) --
	(184.57, 76.22) --
	(184.35, 76.22) --
	(184.13, 76.22) --
	(183.90, 76.22) --
	(183.68, 76.22) --
	(183.46, 76.22) --
	(183.24, 76.22) --
	(183.01, 76.22) --
	(182.79, 76.22) --
	(182.57, 76.22) --
	(182.34, 76.22) --
	(182.12, 76.22) --
	(181.90, 76.22) --
	(181.68, 76.22) --
	(181.45, 76.22) --
	(181.23, 76.22) --
	(181.01, 76.22) --
	(180.79, 76.22) --
	(180.56, 76.22) --
	(180.34, 76.22) --
	(180.12, 76.22) --
	(179.89, 76.22) --
	(179.67, 76.22) --
	(179.45, 76.22) --
	(179.23, 76.22) --
	(179.00, 76.22) --
	(178.78, 76.22) --
	(178.56, 76.22) --
	(178.34, 76.22) --
	(178.11, 76.22) --
	(177.89, 76.22) --
	cycle;

\path[draw=drawColor,line width= 0.6pt,line join=round,line cap=round] (177.89, 76.22) --
	(178.11, 76.22) --
	(178.34, 76.22) --
	(178.56, 76.22) --
	(178.78, 76.22) --
	(179.00, 76.22) --
	(179.23, 76.22) --
	(179.45, 76.22) --
	(179.67, 76.22) --
	(179.89, 76.23) --
	(180.12, 76.23) --
	(180.34, 76.23) --
	(180.56, 76.23) --
	(180.79, 76.23) --
	(181.01, 76.24) --
	(181.23, 76.24) --
	(181.45, 76.26) --
	(181.68, 76.28) --
	(181.90, 76.32) --
	(182.12, 76.38) --
	(182.34, 76.47) --
	(182.57, 76.60) --
	(182.79, 76.80) --
	(183.01, 77.09) --
	(183.24, 77.48) --
	(183.46, 78.03) --
	(183.68, 78.76) --
	(183.90, 79.75) --
	(184.13, 81.03) --
	(184.35, 82.64) --
	(184.57, 84.62) --
	(184.80, 86.99) --
	(185.02, 89.78) --
	(185.24, 92.99) --
	(185.46, 96.59) --
	(185.69,100.55) --
	(185.91,104.82) --
	(186.13,109.31) --
	(186.35,113.93) --
	(186.58,118.56) --
	(186.80,123.10) --
	(187.02,127.44) --
	(187.25,131.51) --
	(187.47,135.24) --
	(187.69,138.57) --
	(187.91,141.48) --
	(188.14,143.96) --
	(188.36,146.02) --
	(188.58,147.65) --
	(188.80,148.93) --
	(189.03,149.90) --
	(189.25,150.63) --
	(189.47,151.14) --
	(189.70,151.48) --
	(189.92,151.69) --
	(190.14,151.80) --
	(190.36,151.84) --
	(190.59,151.81) --
	(190.81,151.73) --
	(191.03,151.61) --
	(191.26,151.46) --
	(191.48,151.28) --
	(191.70,151.07) --
	(191.92,150.83) --
	(192.15,150.57) --
	(192.37,150.28) --
	(192.59,149.97) --
	(192.81,149.65) --
	(193.04,149.30) --
	(193.26,148.95) --
	(193.48,148.57) --
	(193.71,148.19) --
	(193.93,147.78) --
	(194.15,147.37) --
	(194.37,146.94) --
	(194.60,146.49) --
	(194.82,146.01) --
	(195.04,145.52) --
	(195.26,145.00) --
	(195.49,144.46) --
	(195.71,143.90) --
	(195.93,143.32) --
	(196.16,142.71) --
	(196.38,142.10) --
	(196.60,141.47) --
	(196.82,140.84) --
	(197.05,140.20) --
	(197.27,139.56) --
	(197.49,138.93) --
	(197.72,138.30) --
	(197.94,137.69) --
	(198.16,137.10) --
	(198.38,136.54) --
	(198.61,136.00) --
	(198.83,135.50) --
	(199.05,135.03) --
	(199.27,134.60) --
	(199.50,134.21) --
	(199.72,133.85) --
	(199.94,133.52) --
	(200.17,133.23) --
	(200.39,132.95) --
	(200.61,132.70) --
	(200.83,132.45) --
	(201.06,132.22) --
	(201.28,132.00) --
	(201.50,131.78) --
	(201.72,131.57) --
	(201.95,131.36) --
	(202.17,131.16) --
	(202.39,130.97) --
	(202.62,130.77) --
	(202.84,130.58) --
	(203.06,130.39) --
	(203.28,130.21) --
	(203.51,130.03) --
	(203.73,129.85) --
	(203.95,129.68) --
	(204.17,129.53) --
	(204.40,129.39) --
	(204.62,129.27) --
	(204.84,129.18) --
	(205.07,129.11) --
	(205.29,129.08) --
	(205.51,129.07) --
	(205.73,129.10) --
	(205.96,129.15) --
	(206.18,129.23) --
	(206.40,129.33) --
	(206.63,129.45) --
	(206.85,129.59) --
	(207.07,129.74) --
	(207.29,129.90) --
	(207.52,130.07) --
	(207.74,130.25) --
	(207.96,130.43) --
	(208.18,130.61) --
	(208.41,130.78) --
	(208.63,130.95) --
	(208.85,131.10) --
	(209.08,131.24) --
	(209.30,131.35) --
	(209.52,131.43) --
	(209.74,131.47) --
	(209.97,131.48) --
	(210.19,131.45) --
	(210.41,131.38) --
	(210.63,131.27) --
	(210.86,131.12) --
	(211.08,130.93) --
	(211.30,130.71) --
	(211.53,130.45) --
	(211.75,130.16) --
	(211.97,129.85) --
	(212.19,129.51) --
	(212.42,129.17) --
	(212.64,128.81) --
	(212.86,128.45) --
	(213.09,128.08) --
	(213.31,127.72) --
	(213.53,127.35) --
	(213.75,126.99) --
	(213.98,126.62) --
	(214.20,126.25) --
	(214.42,125.87) --
	(214.64,125.49) --
	(214.87,125.09) --
	(215.09,124.68) --
	(215.31,124.26) --
	(215.54,123.83) --
	(215.76,123.41) --
	(215.98,122.98) --
	(216.20,122.56) --
	(216.43,122.15) --
	(216.65,121.76) --
	(216.87,121.39) --
	(217.09,121.03) --
	(217.32,120.69) --
	(217.54,120.37) --
	(217.76,120.07) --
	(217.99,119.78) --
	(218.21,119.50) --
	(218.43,119.23) --
	(218.65,118.97) --
	(218.88,118.71) --
	(219.10,118.47) --
	(219.32,118.24) --
	(219.55,118.01) --
	(219.77,117.80) --
	(219.99,117.61) --
	(220.21,117.43) --
	(220.44,117.26) --
	(220.66,117.11) --
	(220.88,116.98) --
	(221.10,116.87) --
	(221.33,116.77) --
	(221.55,116.67) --
	(221.77,116.59) --
	(222.00,116.50) --
	(222.22,116.42) --
	(222.44,116.33) --
	(222.66,116.23) --
	(222.89,116.13) --
	(223.11,116.02) --
	(223.33,115.90) --
	(223.55,115.77) --
	(223.78,115.64) --
	(224.00,115.51) --
	(224.22,115.39) --
	(224.45,115.28) --
	(224.67,115.19) --
	(224.89,115.11) --
	(225.11,115.05) --
	(225.34,115.01) --
	(225.56,114.99) --
	(225.78,114.98) --
	(226.00,114.98) --
	(226.23,114.99) --
	(226.45,114.99) --
	(226.67,114.99) --
	(226.90,114.97) --
	(227.12,114.95) --
	(227.34,114.91) --
	(227.56,114.86) --
	(227.79,114.78) --
	(228.01,114.69) --
	(228.23,114.59) --
	(228.46,114.47) --
	(228.68,114.35) --
	(228.90,114.21) --
	(229.12,114.06) --
	(229.35,113.90) --
	(229.57,113.74) --
	(229.79,113.57) --
	(230.01,113.39) --
	(230.24,113.21) --
	(230.46,113.03) --
	(230.68,112.85) --
	(230.91,112.68) --
	(231.13,112.50) --
	(231.35,112.34) --
	(231.57,112.18) --
	(231.80,112.03) --
	(232.02,111.89) --
	(232.24,111.77) --
	(232.46,111.65) --
	(232.69,111.56) --
	(232.91,111.47) --
	(233.13,111.39) --
	(233.36,111.32) --
	(233.58,111.25) --
	(233.80,111.19) --
	(234.02,111.12) --
	(234.25,111.06) --
	(234.47,110.99) --
	(234.69,110.91) --
	(234.92,110.82) --
	(235.14,110.73) --
	(235.36,110.63) --
	(235.58,110.52) --
	(235.81,110.40) --
	(236.03,110.28) --
	(236.25,110.16) --
	(236.47,110.03) --
	(236.70,109.90) --
	(236.92,109.77) --
	(237.14,109.64) --
	(237.37,109.50) --
	(237.59,109.36) --
	(237.81,109.22) --
	(238.03,109.08) --
	(238.26,108.93) --
	(238.48,108.77) --
	(238.70,108.61) --
	(238.92,108.45) --
	(239.15,108.28) --
	(239.37,108.10) --
	(239.59,107.92) --
	(239.82,107.73) --
	(240.04,107.52) --
	(240.26,107.31) --
	(240.48,107.09) --
	(240.71,106.85) --
	(240.93,106.61) --
	(241.15,106.35) --
	(241.38,106.08) --
	(241.60,105.81) --
	(241.82,105.53) --
	(242.04,105.24) --
	(242.27,104.95) --
	(242.49,104.66) --
	(242.71,104.36) --
	(242.93,104.07) --
	(243.16,103.78) --
	(243.38,103.49) --
	(243.60,103.20) --
	(243.83,102.92) --
	(244.05,102.63) --
	(244.27,102.34) --
	(244.49,102.06) --
	(244.72,101.77) --
	(244.94,101.48) --
	(245.16,101.19) --
	(245.38,100.89) --
	(245.61,100.60) --
	(245.83,100.30) --
	(246.05, 99.99) --
	(246.28, 99.69) --
	(246.50, 99.39) --
	(246.72, 99.08) --
	(246.94, 98.78) --
	(247.17, 98.48) --
	(247.39, 98.19) --
	(247.61, 97.90) --
	(247.84, 97.62) --
	(248.06, 97.33) --
	(248.28, 97.05) --
	(248.50, 96.77) --
	(248.73, 96.49) --
	(248.95, 96.21) --
	(249.17, 95.92) --
	(249.39, 95.63) --
	(249.62, 95.33) --
	(249.84, 95.03) --
	(250.06, 94.72) --
	(250.29, 94.41) --
	(250.51, 94.09) --
	(250.73, 93.78) --
	(250.95, 93.46) --
	(251.18, 93.14) --
	(251.40, 92.83) --
	(251.62, 92.51) --
	(251.84, 92.21) --
	(252.07, 91.90) --
	(252.29, 91.61) --
	(252.51, 91.32) --
	(252.74, 91.04) --
	(252.96, 90.76) --
	(253.18, 90.50) --
	(253.40, 90.24) --
	(253.63, 90.00) --
	(253.85, 89.76) --
	(254.07, 89.53) --
	(254.29, 89.32) --
	(254.52, 89.11) --
	(254.74, 88.91) --
	(254.96, 88.72) --
	(255.19, 88.54) --
	(255.41, 88.37) --
	(255.63, 88.20) --
	(255.85, 88.04) --
	(256.08, 87.88) --
	(256.30, 87.72) --
	(256.52, 87.56) --
	(256.75, 87.41) --
	(256.97, 87.25) --
	(257.19, 87.09) --
	(257.41, 86.92) --
	(257.64, 86.76) --
	(257.86, 86.59) --
	(258.08, 86.41) --
	(258.30, 86.24) --
	(258.53, 86.06) --
	(258.75, 85.89) --
	(258.97, 85.71) --
	(259.20, 85.54) --
	(259.42, 85.37) --
	(259.64, 85.20) --
	(259.86, 85.04) --
	(260.09, 84.87) --
	(260.31, 84.71) --
	(260.53, 84.55) --
	(260.75, 84.39) --
	(260.98, 84.24) --
	(261.20, 84.09) --
	(261.42, 83.94) --
	(261.65, 83.80) --
	(261.87, 83.66) --
	(262.09, 83.52) --
	(262.31, 83.38) --
	(262.54, 83.25) --
	(262.76, 83.12) --
	(262.98, 82.98) --
	(263.21, 82.85) --
	(263.43, 82.72) --
	(263.65, 82.58) --
	(263.87, 82.44) --
	(264.10, 82.29) --
	(264.32, 82.14) --
	(264.54, 81.99) --
	(264.76, 81.82) --
	(264.99, 81.65) --
	(265.21, 81.48) --
	(265.43, 81.30) --
	(265.66, 81.12) --
	(265.88, 80.93) --
	(266.10, 80.74) --
	(266.32, 80.56) --
	(266.55, 80.37) --
	(266.77, 80.19) --
	(266.99, 80.02) --
	(267.21, 79.85) --
	(267.44, 79.69) --
	(267.66, 79.54) --
	(267.88, 79.39) --
	(268.11, 79.25) --
	(268.33, 79.12) --
	(268.55, 79.00) --
	(268.77, 78.88) --
	(269.00, 78.76) --
	(269.22, 78.65) --
	(269.44, 78.55) --
	(269.67, 78.44) --
	(269.89, 78.35) --
	(270.11, 78.25) --
	(270.33, 78.16) --
	(270.56, 78.07) --
	(270.78, 77.98) --
	(271.00, 77.90) --
	(271.22, 77.82) --
	(271.45, 77.75) --
	(271.67, 77.68) --
	(271.89, 77.61) --
	(272.12, 77.55) --
	(272.34, 77.49) --
	(272.56, 77.43) --
	(272.78, 77.38) --
	(273.01, 77.33) --
	(273.23, 77.28) --
	(273.45, 77.23) --
	(273.67, 77.18) --
	(273.90, 77.14) --
	(274.12, 77.09) --
	(274.34, 77.04) --
	(274.57, 77.00) --
	(274.79, 76.96) --
	(275.01, 76.91) --
	(275.23, 76.87) --
	(275.46, 76.83) --
	(275.68, 76.78) --
	(275.90, 76.74) --
	(276.12, 76.70) --
	(276.35, 76.67) --
	(276.57, 76.63) --
	(276.79, 76.60) --
	(277.02, 76.57) --
	(277.24, 76.54) --
	(277.46, 76.51) --
	(277.68, 76.49) --
	(277.91, 76.47) --
	(278.13, 76.44) --
	(278.35, 76.42) --
	(278.58, 76.41) --
	(278.80, 76.39) --
	(279.02, 76.37) --
	(279.24, 76.36) --
	(279.47, 76.35) --
	(279.69, 76.34) --
	(279.91, 76.33) --
	(280.13, 76.32) --
	(280.36, 76.31) --
	(280.58, 76.30) --
	(280.80, 76.29) --
	(281.03, 76.29) --
	(281.25, 76.28) --
	(281.47, 76.28) --
	(281.69, 76.27) --
	(281.92, 76.27) --
	(282.14, 76.26) --
	(282.36, 76.26) --
	(282.58, 76.26) --
	(282.81, 76.26) --
	(283.03, 76.25) --
	(283.25, 76.25) --
	(283.48, 76.25) --
	(283.70, 76.25) --
	(283.92, 76.24) --
	(284.14, 76.24) --
	(284.37, 76.24) --
	(284.59, 76.24) --
	(284.81, 76.24) --
	(285.04, 76.24) --
	(285.26, 76.23) --
	(285.48, 76.23) --
	(285.70, 76.23) --
	(285.93, 76.23) --
	(286.15, 76.23) --
	(286.37, 76.23) --
	(286.59, 76.23) --
	(286.82, 76.23) --
	(287.04, 76.23) --
	(287.26, 76.23) --
	(287.49, 76.23) --
	(287.71, 76.23) --
	(287.93, 76.23) --
	(288.15, 76.23) --
	(288.38, 76.23) --
	(288.60, 76.23) --
	(288.82, 76.22) --
	(289.04, 76.22) --
	(289.27, 76.22) --
	(289.49, 76.22) --
	(289.71, 76.22) --
	(289.94, 76.22) --
	(290.16, 76.22) --
	(290.38, 76.22) --
	(290.60, 76.22) --
	(290.83, 76.22) --
	(291.05, 76.22) --
	(291.27, 76.22) --
	(291.50, 76.22) --
	(291.72, 76.22);
\end{scope}
\begin{scope}
\path[clip] (302.91, 67.14) rectangle (428.12,267.01);
\definecolor{drawColor}{RGB}{255,255,255}

\path[draw=drawColor,line width= 0.3pt,line join=round] (302.91,104.00) --
	(428.12,104.00);

\path[draw=drawColor,line width= 0.3pt,line join=round] (302.91,159.56) --
	(428.12,159.56);

\path[draw=drawColor,line width= 0.3pt,line join=round] (302.91,215.11) --
	(428.12,215.11);

\path[draw=drawColor,line width= 0.3pt,line join=round] (314.29, 67.14) --
	(314.29,267.01);

\path[draw=drawColor,line width= 0.3pt,line join=round] (348.44, 67.14) --
	(348.44,267.01);

\path[draw=drawColor,line width= 0.3pt,line join=round] (382.59, 67.14) --
	(382.59,267.01);

\path[draw=drawColor,line width= 0.3pt,line join=round] (416.74, 67.14) --
	(416.74,267.01);

\path[draw=drawColor,line width= 0.6pt,line join=round] (302.91, 76.22) --
	(428.12, 76.22);

\path[draw=drawColor,line width= 0.6pt,line join=round] (302.91,131.78) --
	(428.12,131.78);

\path[draw=drawColor,line width= 0.6pt,line join=round] (302.91,187.33) --
	(428.12,187.33);

\path[draw=drawColor,line width= 0.6pt,line join=round] (302.91,242.89) --
	(428.12,242.89);

\path[draw=drawColor,line width= 0.6pt,line join=round] (331.37, 67.14) --
	(331.37,267.01);

\path[draw=drawColor,line width= 0.6pt,line join=round] (365.51, 67.14) --
	(365.51,267.01);

\path[draw=drawColor,line width= 0.6pt,line join=round] (399.66, 67.14) --
	(399.66,267.01);
\definecolor{fillColor}{RGB}{228,26,28}

\path[fill=fillColor,fill opacity=0.50] (308.60, 76.23) --
	(308.82, 76.23) --
	(309.05, 76.23) --
	(309.27, 76.23) --
	(309.49, 76.24) --
	(309.71, 76.24) --
	(309.94, 76.25) --
	(310.16, 76.27) --
	(310.38, 76.28) --
	(310.61, 76.30) --
	(310.83, 76.33) --
	(311.05, 76.38) --
	(311.27, 76.43) --
	(311.50, 76.50) --
	(311.72, 76.60) --
	(311.94, 76.72) --
	(312.16, 76.87) --
	(312.39, 77.06) --
	(312.61, 77.30) --
	(312.83, 77.60) --
	(313.06, 77.97) --
	(313.28, 78.41) --
	(313.50, 78.94) --
	(313.72, 79.57) --
	(313.95, 80.30) --
	(314.17, 81.15) --
	(314.39, 82.16) --
	(314.62, 83.31) --
	(314.84, 84.60) --
	(315.06, 86.05) --
	(315.28, 87.66) --
	(315.51, 89.43) --
	(315.73, 91.36) --
	(315.95, 93.46) --
	(316.17, 95.73) --
	(316.40, 98.13) --
	(316.62,100.67) --
	(316.84,103.33) --
	(317.07,106.09) --
	(317.29,108.94) --
	(317.51,111.86) --
	(317.73,114.84) --
	(317.96,117.86) --
	(318.18,120.89) --
	(318.40,123.93) --
	(318.62,126.95) --
	(318.85,129.94) --
	(319.07,132.89) --
	(319.29,135.78) --
	(319.52,138.59) --
	(319.74,141.34) --
	(319.96,144.01) --
	(320.18,146.58) --
	(320.41,149.07) --
	(320.63,151.47) --
	(320.85,153.77) --
	(321.07,155.96) --
	(321.30,158.05) --
	(321.52,160.06) --
	(321.74,161.97) --
	(321.97,163.81) --
	(322.19,165.56) --
	(322.41,167.24) --
	(322.63,168.84) --
	(322.86,170.37) --
	(323.08,171.86) --
	(323.30,173.29) --
	(323.53,174.68) --
	(323.75,176.04) --
	(323.97,177.36) --
	(324.19,178.65) --
	(324.42,179.91) --
	(324.64,181.14) --
	(324.86,182.35) --
	(325.08,183.54) --
	(325.31,184.71) --
	(325.53,185.85) --
	(325.75,186.97) --
	(325.98,188.06) --
	(326.20,189.12) --
	(326.42,190.15) --
	(326.64,191.15) --
	(326.87,192.12) --
	(327.09,193.05) --
	(327.31,193.95) --
	(327.53,194.81) --
	(327.76,195.62) --
	(327.98,196.38) --
	(328.20,197.10) --
	(328.43,197.76) --
	(328.65,198.36) --
	(328.87,198.89) --
	(329.09,199.36) --
	(329.32,199.73) --
	(329.54,200.01) --
	(329.76,200.20) --
	(329.99,200.29) --
	(330.21,200.27) --
	(330.43,200.14) --
	(330.65,199.90) --
	(330.88,199.52) --
	(331.10,199.00) --
	(331.32,198.36) --
	(331.54,197.60) --
	(331.77,196.72) --
	(331.99,195.71) --
	(332.21,194.59) --
	(332.44,193.36) --
	(332.66,192.02) --
	(332.88,190.59) --
	(333.10,189.08) --
	(333.33,187.50) --
	(333.55,185.86) --
	(333.77,184.18) --
	(333.99,182.47) --
	(334.22,180.73) --
	(334.44,178.98) --
	(334.66,177.24) --
	(334.89,175.51) --
	(335.11,173.80) --
	(335.33,172.13) --
	(335.55,170.48) --
	(335.78,168.88) --
	(336.00,167.34) --
	(336.22,165.84) --
	(336.45,164.39) --
	(336.67,162.99) --
	(336.89,161.63) --
	(337.11,160.33) --
	(337.34,159.07) --
	(337.56,157.86) --
	(337.78,156.70) --
	(338.00,155.57) --
	(338.23,154.48) --
	(338.45,153.43) --
	(338.67,152.42) --
	(338.90,151.44) --
	(339.12,150.50) --
	(339.34,149.60) --
	(339.56,148.74) --
	(339.79,147.93) --
	(340.01,147.15) --
	(340.23,146.42) --
	(340.45,145.73) --
	(340.68,145.08) --
	(340.90,144.48) --
	(341.12,143.92) --
	(341.35,143.39) --
	(341.57,142.89) --
	(341.79,142.41) --
	(342.01,141.96) --
	(342.24,141.52) --
	(342.46,141.08) --
	(342.68,140.64) --
	(342.90,140.19) --
	(343.13,139.72) --
	(343.35,139.22) --
	(343.57,138.70) --
	(343.80,138.13) --
	(344.02,137.53) --
	(344.24,136.87) --
	(344.46,136.16) --
	(344.69,135.39) --
	(344.91,134.57) --
	(345.13,133.69) --
	(345.36,132.77) --
	(345.58,131.78) --
	(345.80,130.74) --
	(346.02,129.66) --
	(346.25,128.53) --
	(346.47,127.37) --
	(346.69,126.18) --
	(346.91,124.97) --
	(347.14,123.74) --
	(347.36,122.50) --
	(347.58,121.27) --
	(347.81,120.04) --
	(348.03,118.83) --
	(348.25,117.64) --
	(348.47,116.47) --
	(348.70,115.33) --
	(348.92,114.23) --
	(349.14,113.17) --
	(349.36,112.16) --
	(349.59,111.18) --
	(349.81,110.25) --
	(350.03,109.36) --
	(350.26,108.51) --
	(350.48,107.70) --
	(350.70,106.94) --
	(350.92,106.22) --
	(351.15,105.53) --
	(351.37,104.88) --
	(351.59,104.26) --
	(351.82,103.67) --
	(352.04,103.11) --
	(352.26,102.58) --
	(352.48,102.07) --
	(352.71,101.59) --
	(352.93,101.13) --
	(353.15,100.69) --
	(353.37,100.27) --
	(353.60, 99.85) --
	(353.82, 99.46) --
	(354.04, 99.07) --
	(354.27, 98.69) --
	(354.49, 98.31) --
	(354.71, 97.94) --
	(354.93, 97.57) --
	(355.16, 97.19) --
	(355.38, 96.81) --
	(355.60, 96.43) --
	(355.82, 96.04) --
	(356.05, 95.64) --
	(356.27, 95.24) --
	(356.49, 94.82) --
	(356.72, 94.40) --
	(356.94, 93.96) --
	(357.16, 93.52) --
	(357.38, 93.07) --
	(357.61, 92.62) --
	(357.83, 92.16) --
	(358.05, 91.70) --
	(358.28, 91.24) --
	(358.50, 90.78) --
	(358.72, 90.32) --
	(358.94, 89.88) --
	(359.17, 89.44) --
	(359.39, 89.01) --
	(359.61, 88.60) --
	(359.83, 88.20) --
	(360.06, 87.82) --
	(360.28, 87.45) --
	(360.50, 87.11) --
	(360.73, 86.79) --
	(360.95, 86.49) --
	(361.17, 86.21) --
	(361.39, 85.95) --
	(361.62, 85.72) --
	(361.84, 85.51) --
	(362.06, 85.31) --
	(362.28, 85.14) --
	(362.51, 84.99) --
	(362.73, 84.85) --
	(362.95, 84.73) --
	(363.18, 84.62) --
	(363.40, 84.52) --
	(363.62, 84.42) --
	(363.84, 84.34) --
	(364.07, 84.26) --
	(364.29, 84.18) --
	(364.51, 84.10) --
	(364.73, 84.02) --
	(364.96, 83.95) --
	(365.18, 83.88) --
	(365.40, 83.81) --
	(365.63, 83.74) --
	(365.85, 83.67) --
	(366.07, 83.62) --
	(366.29, 83.57) --
	(366.52, 83.52) --
	(366.74, 83.49) --
	(366.96, 83.47) --
	(367.19, 83.46) --
	(367.41, 83.47) --
	(367.63, 83.48) --
	(367.85, 83.50) --
	(368.08, 83.53) --
	(368.30, 83.57) --
	(368.52, 83.61) --
	(368.74, 83.65) --
	(368.97, 83.69) --
	(369.19, 83.72) --
	(369.41, 83.74) --
	(369.64, 83.75) --
	(369.86, 83.75) --
	(370.08, 83.73) --
	(370.30, 83.70) --
	(370.53, 83.65) --
	(370.75, 83.58) --
	(370.97, 83.51) --
	(371.19, 83.42) --
	(371.42, 83.32) --
	(371.64, 83.22) --
	(371.86, 83.12) --
	(372.09, 83.01) --
	(372.31, 82.91) --
	(372.53, 82.81) --
	(372.75, 82.72) --
	(372.98, 82.64) --
	(373.20, 82.57) --
	(373.42, 82.51) --
	(373.65, 82.46) --
	(373.87, 82.42) --
	(374.09, 82.39) --
	(374.31, 82.36) --
	(374.54, 82.35) --
	(374.76, 82.33) --
	(374.98, 82.31) --
	(375.20, 82.30) --
	(375.43, 82.28) --
	(375.65, 82.25) --
	(375.87, 82.22) --
	(376.10, 82.18) --
	(376.32, 82.13) --
	(376.54, 82.07) --
	(376.76, 82.00) --
	(376.99, 81.92) --
	(377.21, 81.83) --
	(377.43, 81.73) --
	(377.65, 81.61) --
	(377.88, 81.48) --
	(378.10, 81.34) --
	(378.32, 81.19) --
	(378.55, 81.03) --
	(378.77, 80.86) --
	(378.99, 80.69) --
	(379.21, 80.50) --
	(379.44, 80.31) --
	(379.66, 80.11) --
	(379.88, 79.91) --
	(380.11, 79.71) --
	(380.33, 79.51) --
	(380.55, 79.31) --
	(380.77, 79.11) --
	(381.00, 78.92) --
	(381.22, 78.74) --
	(381.44, 78.57) --
	(381.66, 78.40) --
	(381.89, 78.25) --
	(382.11, 78.11) --
	(382.33, 77.99) --
	(382.56, 77.88) --
	(382.78, 77.78) --
	(383.00, 77.69) --
	(383.22, 77.62) --
	(383.45, 77.57) --
	(383.67, 77.52) --
	(383.89, 77.49) --
	(384.11, 77.47) --
	(384.34, 77.47) --
	(384.56, 77.47) --
	(384.78, 77.48) --
	(385.01, 77.51) --
	(385.23, 77.54) --
	(385.45, 77.58) --
	(385.67, 77.63) --
	(385.90, 77.69) --
	(386.12, 77.75) --
	(386.34, 77.82) --
	(386.56, 77.90) --
	(386.79, 77.98) --
	(387.01, 78.06) --
	(387.23, 78.14) --
	(387.46, 78.22) --
	(387.68, 78.30) --
	(387.90, 78.37) --
	(388.12, 78.44) --
	(388.35, 78.51) --
	(388.57, 78.56) --
	(388.79, 78.61) --
	(389.02, 78.65) --
	(389.24, 78.68) --
	(389.46, 78.70) --
	(389.68, 78.70) --
	(389.91, 78.70) --
	(390.13, 78.69) --
	(390.35, 78.67) --
	(390.57, 78.64) --
	(390.80, 78.60) --
	(391.02, 78.55) --
	(391.24, 78.50) --
	(391.47, 78.44) --
	(391.69, 78.38) --
	(391.91, 78.32) --
	(392.13, 78.25) --
	(392.36, 78.19) --
	(392.58, 78.13) --
	(392.80, 78.06) --
	(393.02, 78.00) --
	(393.25, 77.95) --
	(393.47, 77.89) --
	(393.69, 77.84) --
	(393.92, 77.80) --
	(394.14, 77.76) --
	(394.36, 77.72) --
	(394.58, 77.69) --
	(394.81, 77.66) --
	(395.03, 77.63) --
	(395.25, 77.60) --
	(395.48, 77.58) --
	(395.70, 77.55) --
	(395.92, 77.53) --
	(396.14, 77.50) --
	(396.37, 77.47) --
	(396.59, 77.44) --
	(396.81, 77.41) --
	(397.03, 77.38) --
	(397.26, 77.34) --
	(397.48, 77.30) --
	(397.70, 77.25) --
	(397.93, 77.20) --
	(398.15, 77.15) --
	(398.37, 77.10) --
	(398.59, 77.05) --
	(398.82, 76.99) --
	(399.04, 76.93) --
	(399.26, 76.88) --
	(399.48, 76.82) --
	(399.71, 76.76) --
	(399.93, 76.71) --
	(400.15, 76.66) --
	(400.38, 76.61) --
	(400.60, 76.56) --
	(400.82, 76.52) --
	(401.04, 76.48) --
	(401.27, 76.44) --
	(401.49, 76.41) --
	(401.71, 76.38) --
	(401.94, 76.36) --
	(402.16, 76.34) --
	(402.38, 76.32) --
	(402.60, 76.31) --
	(402.83, 76.30) --
	(403.05, 76.30) --
	(403.27, 76.29) --
	(403.49, 76.29) --
	(403.72, 76.29) --
	(403.94, 76.30) --
	(404.16, 76.30) --
	(404.39, 76.31) --
	(404.61, 76.32) --
	(404.83, 76.33) --
	(405.05, 76.34) --
	(405.28, 76.35) --
	(405.50, 76.36) --
	(405.72, 76.37) --
	(405.94, 76.39) --
	(406.17, 76.40) --
	(406.39, 76.41) --
	(406.61, 76.42) --
	(406.84, 76.43) --
	(407.06, 76.43) --
	(407.28, 76.44) --
	(407.50, 76.44) --
	(407.73, 76.44) --
	(407.95, 76.44) --
	(408.17, 76.43) --
	(408.40, 76.43) --
	(408.62, 76.42) --
	(408.84, 76.41) --
	(409.06, 76.40) --
	(409.29, 76.39) --
	(409.51, 76.37) --
	(409.73, 76.36) --
	(409.95, 76.35) --
	(410.18, 76.34) --
	(410.40, 76.32) --
	(410.62, 76.31) --
	(410.85, 76.30) --
	(411.07, 76.29) --
	(411.29, 76.28) --
	(411.51, 76.27) --
	(411.74, 76.26) --
	(411.96, 76.26) --
	(412.18, 76.25) --
	(412.40, 76.25) --
	(412.63, 76.24) --
	(412.85, 76.24) --
	(413.07, 76.24) --
	(413.30, 76.23) --
	(413.52, 76.23) --
	(413.74, 76.23) --
	(413.96, 76.23) --
	(414.19, 76.23) --
	(414.41, 76.23) --
	(414.63, 76.23) --
	(414.85, 76.23) --
	(415.08, 76.23) --
	(415.30, 76.23) --
	(415.52, 76.23) --
	(415.75, 76.23) --
	(415.97, 76.22) --
	(416.19, 76.22) --
	(416.41, 76.22) --
	(416.64, 76.22) --
	(416.86, 76.22) --
	(417.08, 76.22) --
	(417.31, 76.22) --
	(417.53, 76.22) --
	(417.75, 76.22) --
	(417.97, 76.22) --
	(418.20, 76.22) --
	(418.42, 76.22) --
	(418.64, 76.22) --
	(418.86, 76.22) --
	(419.09, 76.22) --
	(419.31, 76.22) --
	(419.53, 76.22) --
	(419.76, 76.22) --
	(419.98, 76.22) --
	(420.20, 76.22) --
	(420.42, 76.22) --
	(420.65, 76.22) --
	(420.87, 76.22) --
	(421.09, 76.22) --
	(421.31, 76.22) --
	(421.54, 76.22) --
	(421.76, 76.22) --
	(421.98, 76.22) --
	(422.21, 76.22) --
	(422.43, 76.22) --
	(422.43, 76.22) --
	(422.21, 76.22) --
	(421.98, 76.22) --
	(421.76, 76.22) --
	(421.54, 76.22) --
	(421.31, 76.22) --
	(421.09, 76.22) --
	(420.87, 76.22) --
	(420.65, 76.22) --
	(420.42, 76.22) --
	(420.20, 76.22) --
	(419.98, 76.22) --
	(419.76, 76.22) --
	(419.53, 76.22) --
	(419.31, 76.22) --
	(419.09, 76.22) --
	(418.86, 76.22) --
	(418.64, 76.22) --
	(418.42, 76.22) --
	(418.20, 76.22) --
	(417.97, 76.22) --
	(417.75, 76.22) --
	(417.53, 76.22) --
	(417.31, 76.22) --
	(417.08, 76.22) --
	(416.86, 76.22) --
	(416.64, 76.22) --
	(416.41, 76.22) --
	(416.19, 76.22) --
	(415.97, 76.22) --
	(415.75, 76.22) --
	(415.52, 76.22) --
	(415.30, 76.22) --
	(415.08, 76.22) --
	(414.85, 76.22) --
	(414.63, 76.22) --
	(414.41, 76.22) --
	(414.19, 76.22) --
	(413.96, 76.22) --
	(413.74, 76.22) --
	(413.52, 76.22) --
	(413.30, 76.22) --
	(413.07, 76.22) --
	(412.85, 76.22) --
	(412.63, 76.22) --
	(412.40, 76.22) --
	(412.18, 76.22) --
	(411.96, 76.22) --
	(411.74, 76.22) --
	(411.51, 76.22) --
	(411.29, 76.22) --
	(411.07, 76.22) --
	(410.85, 76.22) --
	(410.62, 76.22) --
	(410.40, 76.22) --
	(410.18, 76.22) --
	(409.95, 76.22) --
	(409.73, 76.22) --
	(409.51, 76.22) --
	(409.29, 76.22) --
	(409.06, 76.22) --
	(408.84, 76.22) --
	(408.62, 76.22) --
	(408.40, 76.22) --
	(408.17, 76.22) --
	(407.95, 76.22) --
	(407.73, 76.22) --
	(407.50, 76.22) --
	(407.28, 76.22) --
	(407.06, 76.22) --
	(406.84, 76.22) --
	(406.61, 76.22) --
	(406.39, 76.22) --
	(406.17, 76.22) --
	(405.94, 76.22) --
	(405.72, 76.22) --
	(405.50, 76.22) --
	(405.28, 76.22) --
	(405.05, 76.22) --
	(404.83, 76.22) --
	(404.61, 76.22) --
	(404.39, 76.22) --
	(404.16, 76.22) --
	(403.94, 76.22) --
	(403.72, 76.22) --
	(403.49, 76.22) --
	(403.27, 76.22) --
	(403.05, 76.22) --
	(402.83, 76.22) --
	(402.60, 76.22) --
	(402.38, 76.22) --
	(402.16, 76.22) --
	(401.94, 76.22) --
	(401.71, 76.22) --
	(401.49, 76.22) --
	(401.27, 76.22) --
	(401.04, 76.22) --
	(400.82, 76.22) --
	(400.60, 76.22) --
	(400.38, 76.22) --
	(400.15, 76.22) --
	(399.93, 76.22) --
	(399.71, 76.22) --
	(399.48, 76.22) --
	(399.26, 76.22) --
	(399.04, 76.22) --
	(398.82, 76.22) --
	(398.59, 76.22) --
	(398.37, 76.22) --
	(398.15, 76.22) --
	(397.93, 76.22) --
	(397.70, 76.22) --
	(397.48, 76.22) --
	(397.26, 76.22) --
	(397.03, 76.22) --
	(396.81, 76.22) --
	(396.59, 76.22) --
	(396.37, 76.22) --
	(396.14, 76.22) --
	(395.92, 76.22) --
	(395.70, 76.22) --
	(395.48, 76.22) --
	(395.25, 76.22) --
	(395.03, 76.22) --
	(394.81, 76.22) --
	(394.58, 76.22) --
	(394.36, 76.22) --
	(394.14, 76.22) --
	(393.92, 76.22) --
	(393.69, 76.22) --
	(393.47, 76.22) --
	(393.25, 76.22) --
	(393.02, 76.22) --
	(392.80, 76.22) --
	(392.58, 76.22) --
	(392.36, 76.22) --
	(392.13, 76.22) --
	(391.91, 76.22) --
	(391.69, 76.22) --
	(391.47, 76.22) --
	(391.24, 76.22) --
	(391.02, 76.22) --
	(390.80, 76.22) --
	(390.57, 76.22) --
	(390.35, 76.22) --
	(390.13, 76.22) --
	(389.91, 76.22) --
	(389.68, 76.22) --
	(389.46, 76.22) --
	(389.24, 76.22) --
	(389.02, 76.22) --
	(388.79, 76.22) --
	(388.57, 76.22) --
	(388.35, 76.22) --
	(388.12, 76.22) --
	(387.90, 76.22) --
	(387.68, 76.22) --
	(387.46, 76.22) --
	(387.23, 76.22) --
	(387.01, 76.22) --
	(386.79, 76.22) --
	(386.56, 76.22) --
	(386.34, 76.22) --
	(386.12, 76.22) --
	(385.90, 76.22) --
	(385.67, 76.22) --
	(385.45, 76.22) --
	(385.23, 76.22) --
	(385.01, 76.22) --
	(384.78, 76.22) --
	(384.56, 76.22) --
	(384.34, 76.22) --
	(384.11, 76.22) --
	(383.89, 76.22) --
	(383.67, 76.22) --
	(383.45, 76.22) --
	(383.22, 76.22) --
	(383.00, 76.22) --
	(382.78, 76.22) --
	(382.56, 76.22) --
	(382.33, 76.22) --
	(382.11, 76.22) --
	(381.89, 76.22) --
	(381.66, 76.22) --
	(381.44, 76.22) --
	(381.22, 76.22) --
	(381.00, 76.22) --
	(380.77, 76.22) --
	(380.55, 76.22) --
	(380.33, 76.22) --
	(380.11, 76.22) --
	(379.88, 76.22) --
	(379.66, 76.22) --
	(379.44, 76.22) --
	(379.21, 76.22) --
	(378.99, 76.22) --
	(378.77, 76.22) --
	(378.55, 76.22) --
	(378.32, 76.22) --
	(378.10, 76.22) --
	(377.88, 76.22) --
	(377.65, 76.22) --
	(377.43, 76.22) --
	(377.21, 76.22) --
	(376.99, 76.22) --
	(376.76, 76.22) --
	(376.54, 76.22) --
	(376.32, 76.22) --
	(376.10, 76.22) --
	(375.87, 76.22) --
	(375.65, 76.22) --
	(375.43, 76.22) --
	(375.20, 76.22) --
	(374.98, 76.22) --
	(374.76, 76.22) --
	(374.54, 76.22) --
	(374.31, 76.22) --
	(374.09, 76.22) --
	(373.87, 76.22) --
	(373.65, 76.22) --
	(373.42, 76.22) --
	(373.20, 76.22) --
	(372.98, 76.22) --
	(372.75, 76.22) --
	(372.53, 76.22) --
	(372.31, 76.22) --
	(372.09, 76.22) --
	(371.86, 76.22) --
	(371.64, 76.22) --
	(371.42, 76.22) --
	(371.19, 76.22) --
	(370.97, 76.22) --
	(370.75, 76.22) --
	(370.53, 76.22) --
	(370.30, 76.22) --
	(370.08, 76.22) --
	(369.86, 76.22) --
	(369.64, 76.22) --
	(369.41, 76.22) --
	(369.19, 76.22) --
	(368.97, 76.22) --
	(368.74, 76.22) --
	(368.52, 76.22) --
	(368.30, 76.22) --
	(368.08, 76.22) --
	(367.85, 76.22) --
	(367.63, 76.22) --
	(367.41, 76.22) --
	(367.19, 76.22) --
	(366.96, 76.22) --
	(366.74, 76.22) --
	(366.52, 76.22) --
	(366.29, 76.22) --
	(366.07, 76.22) --
	(365.85, 76.22) --
	(365.63, 76.22) --
	(365.40, 76.22) --
	(365.18, 76.22) --
	(364.96, 76.22) --
	(364.73, 76.22) --
	(364.51, 76.22) --
	(364.29, 76.22) --
	(364.07, 76.22) --
	(363.84, 76.22) --
	(363.62, 76.22) --
	(363.40, 76.22) --
	(363.18, 76.22) --
	(362.95, 76.22) --
	(362.73, 76.22) --
	(362.51, 76.22) --
	(362.28, 76.22) --
	(362.06, 76.22) --
	(361.84, 76.22) --
	(361.62, 76.22) --
	(361.39, 76.22) --
	(361.17, 76.22) --
	(360.95, 76.22) --
	(360.73, 76.22) --
	(360.50, 76.22) --
	(360.28, 76.22) --
	(360.06, 76.22) --
	(359.83, 76.22) --
	(359.61, 76.22) --
	(359.39, 76.22) --
	(359.17, 76.22) --
	(358.94, 76.22) --
	(358.72, 76.22) --
	(358.50, 76.22) --
	(358.28, 76.22) --
	(358.05, 76.22) --
	(357.83, 76.22) --
	(357.61, 76.22) --
	(357.38, 76.22) --
	(357.16, 76.22) --
	(356.94, 76.22) --
	(356.72, 76.22) --
	(356.49, 76.22) --
	(356.27, 76.22) --
	(356.05, 76.22) --
	(355.82, 76.22) --
	(355.60, 76.22) --
	(355.38, 76.22) --
	(355.16, 76.22) --
	(354.93, 76.22) --
	(354.71, 76.22) --
	(354.49, 76.22) --
	(354.27, 76.22) --
	(354.04, 76.22) --
	(353.82, 76.22) --
	(353.60, 76.22) --
	(353.37, 76.22) --
	(353.15, 76.22) --
	(352.93, 76.22) --
	(352.71, 76.22) --
	(352.48, 76.22) --
	(352.26, 76.22) --
	(352.04, 76.22) --
	(351.82, 76.22) --
	(351.59, 76.22) --
	(351.37, 76.22) --
	(351.15, 76.22) --
	(350.92, 76.22) --
	(350.70, 76.22) --
	(350.48, 76.22) --
	(350.26, 76.22) --
	(350.03, 76.22) --
	(349.81, 76.22) --
	(349.59, 76.22) --
	(349.36, 76.22) --
	(349.14, 76.22) --
	(348.92, 76.22) --
	(348.70, 76.22) --
	(348.47, 76.22) --
	(348.25, 76.22) --
	(348.03, 76.22) --
	(347.81, 76.22) --
	(347.58, 76.22) --
	(347.36, 76.22) --
	(347.14, 76.22) --
	(346.91, 76.22) --
	(346.69, 76.22) --
	(346.47, 76.22) --
	(346.25, 76.22) --
	(346.02, 76.22) --
	(345.80, 76.22) --
	(345.58, 76.22) --
	(345.36, 76.22) --
	(345.13, 76.22) --
	(344.91, 76.22) --
	(344.69, 76.22) --
	(344.46, 76.22) --
	(344.24, 76.22) --
	(344.02, 76.22) --
	(343.80, 76.22) --
	(343.57, 76.22) --
	(343.35, 76.22) --
	(343.13, 76.22) --
	(342.90, 76.22) --
	(342.68, 76.22) --
	(342.46, 76.22) --
	(342.24, 76.22) --
	(342.01, 76.22) --
	(341.79, 76.22) --
	(341.57, 76.22) --
	(341.35, 76.22) --
	(341.12, 76.22) --
	(340.90, 76.22) --
	(340.68, 76.22) --
	(340.45, 76.22) --
	(340.23, 76.22) --
	(340.01, 76.22) --
	(339.79, 76.22) --
	(339.56, 76.22) --
	(339.34, 76.22) --
	(339.12, 76.22) --
	(338.90, 76.22) --
	(338.67, 76.22) --
	(338.45, 76.22) --
	(338.23, 76.22) --
	(338.00, 76.22) --
	(337.78, 76.22) --
	(337.56, 76.22) --
	(337.34, 76.22) --
	(337.11, 76.22) --
	(336.89, 76.22) --
	(336.67, 76.22) --
	(336.45, 76.22) --
	(336.22, 76.22) --
	(336.00, 76.22) --
	(335.78, 76.22) --
	(335.55, 76.22) --
	(335.33, 76.22) --
	(335.11, 76.22) --
	(334.89, 76.22) --
	(334.66, 76.22) --
	(334.44, 76.22) --
	(334.22, 76.22) --
	(333.99, 76.22) --
	(333.77, 76.22) --
	(333.55, 76.22) --
	(333.33, 76.22) --
	(333.10, 76.22) --
	(332.88, 76.22) --
	(332.66, 76.22) --
	(332.44, 76.22) --
	(332.21, 76.22) --
	(331.99, 76.22) --
	(331.77, 76.22) --
	(331.54, 76.22) --
	(331.32, 76.22) --
	(331.10, 76.22) --
	(330.88, 76.22) --
	(330.65, 76.22) --
	(330.43, 76.22) --
	(330.21, 76.22) --
	(329.99, 76.22) --
	(329.76, 76.22) --
	(329.54, 76.22) --
	(329.32, 76.22) --
	(329.09, 76.22) --
	(328.87, 76.22) --
	(328.65, 76.22) --
	(328.43, 76.22) --
	(328.20, 76.22) --
	(327.98, 76.22) --
	(327.76, 76.22) --
	(327.53, 76.22) --
	(327.31, 76.22) --
	(327.09, 76.22) --
	(326.87, 76.22) --
	(326.64, 76.22) --
	(326.42, 76.22) --
	(326.20, 76.22) --
	(325.98, 76.22) --
	(325.75, 76.22) --
	(325.53, 76.22) --
	(325.31, 76.22) --
	(325.08, 76.22) --
	(324.86, 76.22) --
	(324.64, 76.22) --
	(324.42, 76.22) --
	(324.19, 76.22) --
	(323.97, 76.22) --
	(323.75, 76.22) --
	(323.53, 76.22) --
	(323.30, 76.22) --
	(323.08, 76.22) --
	(322.86, 76.22) --
	(322.63, 76.22) --
	(322.41, 76.22) --
	(322.19, 76.22) --
	(321.97, 76.22) --
	(321.74, 76.22) --
	(321.52, 76.22) --
	(321.30, 76.22) --
	(321.07, 76.22) --
	(320.85, 76.22) --
	(320.63, 76.22) --
	(320.41, 76.22) --
	(320.18, 76.22) --
	(319.96, 76.22) --
	(319.74, 76.22) --
	(319.52, 76.22) --
	(319.29, 76.22) --
	(319.07, 76.22) --
	(318.85, 76.22) --
	(318.62, 76.22) --
	(318.40, 76.22) --
	(318.18, 76.22) --
	(317.96, 76.22) --
	(317.73, 76.22) --
	(317.51, 76.22) --
	(317.29, 76.22) --
	(317.07, 76.22) --
	(316.84, 76.22) --
	(316.62, 76.22) --
	(316.40, 76.22) --
	(316.17, 76.22) --
	(315.95, 76.22) --
	(315.73, 76.22) --
	(315.51, 76.22) --
	(315.28, 76.22) --
	(315.06, 76.22) --
	(314.84, 76.22) --
	(314.62, 76.22) --
	(314.39, 76.22) --
	(314.17, 76.22) --
	(313.95, 76.22) --
	(313.72, 76.22) --
	(313.50, 76.22) --
	(313.28, 76.22) --
	(313.06, 76.22) --
	(312.83, 76.22) --
	(312.61, 76.22) --
	(312.39, 76.22) --
	(312.16, 76.22) --
	(311.94, 76.22) --
	(311.72, 76.22) --
	(311.50, 76.22) --
	(311.27, 76.22) --
	(311.05, 76.22) --
	(310.83, 76.22) --
	(310.61, 76.22) --
	(310.38, 76.22) --
	(310.16, 76.22) --
	(309.94, 76.22) --
	(309.71, 76.22) --
	(309.49, 76.22) --
	(309.27, 76.22) --
	(309.05, 76.22) --
	(308.82, 76.22) --
	(308.60, 76.22) --
	cycle;
\definecolor{drawColor}{RGB}{0,0,0}

\path[draw=drawColor,line width= 0.6pt,line join=round,line cap=round] (308.60, 76.23) --
	(308.82, 76.23) --
	(309.05, 76.23) --
	(309.27, 76.23) --
	(309.49, 76.24) --
	(309.71, 76.24) --
	(309.94, 76.25) --
	(310.16, 76.27) --
	(310.38, 76.28) --
	(310.61, 76.30) --
	(310.83, 76.33) --
	(311.05, 76.38) --
	(311.27, 76.43) --
	(311.50, 76.50) --
	(311.72, 76.60) --
	(311.94, 76.72) --
	(312.16, 76.87) --
	(312.39, 77.06) --
	(312.61, 77.30) --
	(312.83, 77.60) --
	(313.06, 77.97) --
	(313.28, 78.41) --
	(313.50, 78.94) --
	(313.72, 79.57) --
	(313.95, 80.30) --
	(314.17, 81.15) --
	(314.39, 82.16) --
	(314.62, 83.31) --
	(314.84, 84.60) --
	(315.06, 86.05) --
	(315.28, 87.66) --
	(315.51, 89.43) --
	(315.73, 91.36) --
	(315.95, 93.46) --
	(316.17, 95.73) --
	(316.40, 98.13) --
	(316.62,100.67) --
	(316.84,103.33) --
	(317.07,106.09) --
	(317.29,108.94) --
	(317.51,111.86) --
	(317.73,114.84) --
	(317.96,117.86) --
	(318.18,120.89) --
	(318.40,123.93) --
	(318.62,126.95) --
	(318.85,129.94) --
	(319.07,132.89) --
	(319.29,135.78) --
	(319.52,138.59) --
	(319.74,141.34) --
	(319.96,144.01) --
	(320.18,146.58) --
	(320.41,149.07) --
	(320.63,151.47) --
	(320.85,153.77) --
	(321.07,155.96) --
	(321.30,158.05) --
	(321.52,160.06) --
	(321.74,161.97) --
	(321.97,163.81) --
	(322.19,165.56) --
	(322.41,167.24) --
	(322.63,168.84) --
	(322.86,170.37) --
	(323.08,171.86) --
	(323.30,173.29) --
	(323.53,174.68) --
	(323.75,176.04) --
	(323.97,177.36) --
	(324.19,178.65) --
	(324.42,179.91) --
	(324.64,181.14) --
	(324.86,182.35) --
	(325.08,183.54) --
	(325.31,184.71) --
	(325.53,185.85) --
	(325.75,186.97) --
	(325.98,188.06) --
	(326.20,189.12) --
	(326.42,190.15) --
	(326.64,191.15) --
	(326.87,192.12) --
	(327.09,193.05) --
	(327.31,193.95) --
	(327.53,194.81) --
	(327.76,195.62) --
	(327.98,196.38) --
	(328.20,197.10) --
	(328.43,197.76) --
	(328.65,198.36) --
	(328.87,198.89) --
	(329.09,199.36) --
	(329.32,199.73) --
	(329.54,200.01) --
	(329.76,200.20) --
	(329.99,200.29) --
	(330.21,200.27) --
	(330.43,200.14) --
	(330.65,199.90) --
	(330.88,199.52) --
	(331.10,199.00) --
	(331.32,198.36) --
	(331.54,197.60) --
	(331.77,196.72) --
	(331.99,195.71) --
	(332.21,194.59) --
	(332.44,193.36) --
	(332.66,192.02) --
	(332.88,190.59) --
	(333.10,189.08) --
	(333.33,187.50) --
	(333.55,185.86) --
	(333.77,184.18) --
	(333.99,182.47) --
	(334.22,180.73) --
	(334.44,178.98) --
	(334.66,177.24) --
	(334.89,175.51) --
	(335.11,173.80) --
	(335.33,172.13) --
	(335.55,170.48) --
	(335.78,168.88) --
	(336.00,167.34) --
	(336.22,165.84) --
	(336.45,164.39) --
	(336.67,162.99) --
	(336.89,161.63) --
	(337.11,160.33) --
	(337.34,159.07) --
	(337.56,157.86) --
	(337.78,156.70) --
	(338.00,155.57) --
	(338.23,154.48) --
	(338.45,153.43) --
	(338.67,152.42) --
	(338.90,151.44) --
	(339.12,150.50) --
	(339.34,149.60) --
	(339.56,148.74) --
	(339.79,147.93) --
	(340.01,147.15) --
	(340.23,146.42) --
	(340.45,145.73) --
	(340.68,145.08) --
	(340.90,144.48) --
	(341.12,143.92) --
	(341.35,143.39) --
	(341.57,142.89) --
	(341.79,142.41) --
	(342.01,141.96) --
	(342.24,141.52) --
	(342.46,141.08) --
	(342.68,140.64) --
	(342.90,140.19) --
	(343.13,139.72) --
	(343.35,139.22) --
	(343.57,138.70) --
	(343.80,138.13) --
	(344.02,137.53) --
	(344.24,136.87) --
	(344.46,136.16) --
	(344.69,135.39) --
	(344.91,134.57) --
	(345.13,133.69) --
	(345.36,132.77) --
	(345.58,131.78) --
	(345.80,130.74) --
	(346.02,129.66) --
	(346.25,128.53) --
	(346.47,127.37) --
	(346.69,126.18) --
	(346.91,124.97) --
	(347.14,123.74) --
	(347.36,122.50) --
	(347.58,121.27) --
	(347.81,120.04) --
	(348.03,118.83) --
	(348.25,117.64) --
	(348.47,116.47) --
	(348.70,115.33) --
	(348.92,114.23) --
	(349.14,113.17) --
	(349.36,112.16) --
	(349.59,111.18) --
	(349.81,110.25) --
	(350.03,109.36) --
	(350.26,108.51) --
	(350.48,107.70) --
	(350.70,106.94) --
	(350.92,106.22) --
	(351.15,105.53) --
	(351.37,104.88) --
	(351.59,104.26) --
	(351.82,103.67) --
	(352.04,103.11) --
	(352.26,102.58) --
	(352.48,102.07) --
	(352.71,101.59) --
	(352.93,101.13) --
	(353.15,100.69) --
	(353.37,100.27) --
	(353.60, 99.85) --
	(353.82, 99.46) --
	(354.04, 99.07) --
	(354.27, 98.69) --
	(354.49, 98.31) --
	(354.71, 97.94) --
	(354.93, 97.57) --
	(355.16, 97.19) --
	(355.38, 96.81) --
	(355.60, 96.43) --
	(355.82, 96.04) --
	(356.05, 95.64) --
	(356.27, 95.24) --
	(356.49, 94.82) --
	(356.72, 94.40) --
	(356.94, 93.96) --
	(357.16, 93.52) --
	(357.38, 93.07) --
	(357.61, 92.62) --
	(357.83, 92.16) --
	(358.05, 91.70) --
	(358.28, 91.24) --
	(358.50, 90.78) --
	(358.72, 90.32) --
	(358.94, 89.88) --
	(359.17, 89.44) --
	(359.39, 89.01) --
	(359.61, 88.60) --
	(359.83, 88.20) --
	(360.06, 87.82) --
	(360.28, 87.45) --
	(360.50, 87.11) --
	(360.73, 86.79) --
	(360.95, 86.49) --
	(361.17, 86.21) --
	(361.39, 85.95) --
	(361.62, 85.72) --
	(361.84, 85.51) --
	(362.06, 85.31) --
	(362.28, 85.14) --
	(362.51, 84.99) --
	(362.73, 84.85) --
	(362.95, 84.73) --
	(363.18, 84.62) --
	(363.40, 84.52) --
	(363.62, 84.42) --
	(363.84, 84.34) --
	(364.07, 84.26) --
	(364.29, 84.18) --
	(364.51, 84.10) --
	(364.73, 84.02) --
	(364.96, 83.95) --
	(365.18, 83.88) --
	(365.40, 83.81) --
	(365.63, 83.74) --
	(365.85, 83.67) --
	(366.07, 83.62) --
	(366.29, 83.57) --
	(366.52, 83.52) --
	(366.74, 83.49) --
	(366.96, 83.47) --
	(367.19, 83.46) --
	(367.41, 83.47) --
	(367.63, 83.48) --
	(367.85, 83.50) --
	(368.08, 83.53) --
	(368.30, 83.57) --
	(368.52, 83.61) --
	(368.74, 83.65) --
	(368.97, 83.69) --
	(369.19, 83.72) --
	(369.41, 83.74) --
	(369.64, 83.75) --
	(369.86, 83.75) --
	(370.08, 83.73) --
	(370.30, 83.70) --
	(370.53, 83.65) --
	(370.75, 83.58) --
	(370.97, 83.51) --
	(371.19, 83.42) --
	(371.42, 83.32) --
	(371.64, 83.22) --
	(371.86, 83.12) --
	(372.09, 83.01) --
	(372.31, 82.91) --
	(372.53, 82.81) --
	(372.75, 82.72) --
	(372.98, 82.64) --
	(373.20, 82.57) --
	(373.42, 82.51) --
	(373.65, 82.46) --
	(373.87, 82.42) --
	(374.09, 82.39) --
	(374.31, 82.36) --
	(374.54, 82.35) --
	(374.76, 82.33) --
	(374.98, 82.31) --
	(375.20, 82.30) --
	(375.43, 82.28) --
	(375.65, 82.25) --
	(375.87, 82.22) --
	(376.10, 82.18) --
	(376.32, 82.13) --
	(376.54, 82.07) --
	(376.76, 82.00) --
	(376.99, 81.92) --
	(377.21, 81.83) --
	(377.43, 81.73) --
	(377.65, 81.61) --
	(377.88, 81.48) --
	(378.10, 81.34) --
	(378.32, 81.19) --
	(378.55, 81.03) --
	(378.77, 80.86) --
	(378.99, 80.69) --
	(379.21, 80.50) --
	(379.44, 80.31) --
	(379.66, 80.11) --
	(379.88, 79.91) --
	(380.11, 79.71) --
	(380.33, 79.51) --
	(380.55, 79.31) --
	(380.77, 79.11) --
	(381.00, 78.92) --
	(381.22, 78.74) --
	(381.44, 78.57) --
	(381.66, 78.40) --
	(381.89, 78.25) --
	(382.11, 78.11) --
	(382.33, 77.99) --
	(382.56, 77.88) --
	(382.78, 77.78) --
	(383.00, 77.69) --
	(383.22, 77.62) --
	(383.45, 77.57) --
	(383.67, 77.52) --
	(383.89, 77.49) --
	(384.11, 77.47) --
	(384.34, 77.47) --
	(384.56, 77.47) --
	(384.78, 77.48) --
	(385.01, 77.51) --
	(385.23, 77.54) --
	(385.45, 77.58) --
	(385.67, 77.63) --
	(385.90, 77.69) --
	(386.12, 77.75) --
	(386.34, 77.82) --
	(386.56, 77.90) --
	(386.79, 77.98) --
	(387.01, 78.06) --
	(387.23, 78.14) --
	(387.46, 78.22) --
	(387.68, 78.30) --
	(387.90, 78.37) --
	(388.12, 78.44) --
	(388.35, 78.51) --
	(388.57, 78.56) --
	(388.79, 78.61) --
	(389.02, 78.65) --
	(389.24, 78.68) --
	(389.46, 78.70) --
	(389.68, 78.70) --
	(389.91, 78.70) --
	(390.13, 78.69) --
	(390.35, 78.67) --
	(390.57, 78.64) --
	(390.80, 78.60) --
	(391.02, 78.55) --
	(391.24, 78.50) --
	(391.47, 78.44) --
	(391.69, 78.38) --
	(391.91, 78.32) --
	(392.13, 78.25) --
	(392.36, 78.19) --
	(392.58, 78.13) --
	(392.80, 78.06) --
	(393.02, 78.00) --
	(393.25, 77.95) --
	(393.47, 77.89) --
	(393.69, 77.84) --
	(393.92, 77.80) --
	(394.14, 77.76) --
	(394.36, 77.72) --
	(394.58, 77.69) --
	(394.81, 77.66) --
	(395.03, 77.63) --
	(395.25, 77.60) --
	(395.48, 77.58) --
	(395.70, 77.55) --
	(395.92, 77.53) --
	(396.14, 77.50) --
	(396.37, 77.47) --
	(396.59, 77.44) --
	(396.81, 77.41) --
	(397.03, 77.38) --
	(397.26, 77.34) --
	(397.48, 77.30) --
	(397.70, 77.25) --
	(397.93, 77.20) --
	(398.15, 77.15) --
	(398.37, 77.10) --
	(398.59, 77.05) --
	(398.82, 76.99) --
	(399.04, 76.93) --
	(399.26, 76.88) --
	(399.48, 76.82) --
	(399.71, 76.76) --
	(399.93, 76.71) --
	(400.15, 76.66) --
	(400.38, 76.61) --
	(400.60, 76.56) --
	(400.82, 76.52) --
	(401.04, 76.48) --
	(401.27, 76.44) --
	(401.49, 76.41) --
	(401.71, 76.38) --
	(401.94, 76.36) --
	(402.16, 76.34) --
	(402.38, 76.32) --
	(402.60, 76.31) --
	(402.83, 76.30) --
	(403.05, 76.30) --
	(403.27, 76.29) --
	(403.49, 76.29) --
	(403.72, 76.29) --
	(403.94, 76.30) --
	(404.16, 76.30) --
	(404.39, 76.31) --
	(404.61, 76.32) --
	(404.83, 76.33) --
	(405.05, 76.34) --
	(405.28, 76.35) --
	(405.50, 76.36) --
	(405.72, 76.37) --
	(405.94, 76.39) --
	(406.17, 76.40) --
	(406.39, 76.41) --
	(406.61, 76.42) --
	(406.84, 76.43) --
	(407.06, 76.43) --
	(407.28, 76.44) --
	(407.50, 76.44) --
	(407.73, 76.44) --
	(407.95, 76.44) --
	(408.17, 76.43) --
	(408.40, 76.43) --
	(408.62, 76.42) --
	(408.84, 76.41) --
	(409.06, 76.40) --
	(409.29, 76.39) --
	(409.51, 76.37) --
	(409.73, 76.36) --
	(409.95, 76.35) --
	(410.18, 76.34) --
	(410.40, 76.32) --
	(410.62, 76.31) --
	(410.85, 76.30) --
	(411.07, 76.29) --
	(411.29, 76.28) --
	(411.51, 76.27) --
	(411.74, 76.26) --
	(411.96, 76.26) --
	(412.18, 76.25) --
	(412.40, 76.25) --
	(412.63, 76.24) --
	(412.85, 76.24) --
	(413.07, 76.24) --
	(413.30, 76.23) --
	(413.52, 76.23) --
	(413.74, 76.23) --
	(413.96, 76.23) --
	(414.19, 76.23) --
	(414.41, 76.23) --
	(414.63, 76.23) --
	(414.85, 76.23) --
	(415.08, 76.23) --
	(415.30, 76.23) --
	(415.52, 76.23) --
	(415.75, 76.23) --
	(415.97, 76.22) --
	(416.19, 76.22) --
	(416.41, 76.22) --
	(416.64, 76.22) --
	(416.86, 76.22) --
	(417.08, 76.22) --
	(417.31, 76.22) --
	(417.53, 76.22) --
	(417.75, 76.22) --
	(417.97, 76.22) --
	(418.20, 76.22) --
	(418.42, 76.22) --
	(418.64, 76.22) --
	(418.86, 76.22) --
	(419.09, 76.22) --
	(419.31, 76.22) --
	(419.53, 76.22) --
	(419.76, 76.22) --
	(419.98, 76.22) --
	(420.20, 76.22) --
	(420.42, 76.22) --
	(420.65, 76.22) --
	(420.87, 76.22) --
	(421.09, 76.22) --
	(421.31, 76.22) --
	(421.54, 76.22) --
	(421.76, 76.22) --
	(421.98, 76.22) --
	(422.21, 76.22) --
	(422.43, 76.22);
\definecolor{fillColor}{RGB}{55,126,184}

\path[fill=fillColor,fill opacity=0.50] (308.60, 76.23) --
	(308.82, 76.23) --
	(309.05, 76.23) --
	(309.27, 76.23) --
	(309.49, 76.23) --
	(309.71, 76.23) --
	(309.94, 76.24) --
	(310.16, 76.24) --
	(310.38, 76.25) --
	(310.61, 76.27) --
	(310.83, 76.29) --
	(311.05, 76.31) --
	(311.27, 76.35) --
	(311.50, 76.41) --
	(311.72, 76.49) --
	(311.94, 76.60) --
	(312.16, 76.74) --
	(312.39, 76.93) --
	(312.61, 77.17) --
	(312.83, 77.49) --
	(313.06, 77.89) --
	(313.28, 78.40) --
	(313.50, 79.05) --
	(313.72, 79.83) --
	(313.95, 80.79) --
	(314.17, 81.93) --
	(314.39, 83.27) --
	(314.62, 84.83) --
	(314.84, 86.63) --
	(315.06, 88.70) --
	(315.28, 91.04) --
	(315.51, 93.63) --
	(315.73, 96.47) --
	(315.95, 99.53) --
	(316.17,102.80) --
	(316.40,106.26) --
	(316.62,109.87) --
	(316.84,113.60) --
	(317.07,117.39) --
	(317.29,121.18) --
	(317.51,124.95) --
	(317.73,128.64) --
	(317.96,132.20) --
	(318.18,135.61) --
	(318.40,138.83) --
	(318.62,141.81) --
	(318.85,144.53) --
	(319.07,146.98) --
	(319.29,149.18) --
	(319.52,151.11) --
	(319.74,152.78) --
	(319.96,154.19) --
	(320.18,155.37) --
	(320.41,156.31) --
	(320.63,157.03) --
	(320.85,157.56) --
	(321.07,157.92) --
	(321.30,158.13) --
	(321.52,158.21) --
	(321.74,158.17) --
	(321.97,158.03) --
	(322.19,157.79) --
	(322.41,157.47) --
	(322.63,157.08) --
	(322.86,156.64) --
	(323.08,156.15) --
	(323.30,155.62) --
	(323.53,155.06) --
	(323.75,154.47) --
	(323.97,153.87) --
	(324.19,153.25) --
	(324.42,152.63) --
	(324.64,152.01) --
	(324.86,151.38) --
	(325.08,150.77) --
	(325.31,150.16) --
	(325.53,149.56) --
	(325.75,148.97) --
	(325.98,148.40) --
	(326.20,147.84) --
	(326.42,147.30) --
	(326.64,146.78) --
	(326.87,146.26) --
	(327.09,145.77) --
	(327.31,145.28) --
	(327.53,144.81) --
	(327.76,144.36) --
	(327.98,143.92) --
	(328.20,143.50) --
	(328.43,143.10) --
	(328.65,142.72) --
	(328.87,142.35) --
	(329.09,142.01) --
	(329.32,141.70) --
	(329.54,141.41) --
	(329.76,141.15) --
	(329.99,140.91) --
	(330.21,140.70) --
	(330.43,140.50) --
	(330.65,140.33) --
	(330.88,140.17) --
	(331.10,140.03) --
	(331.32,139.89) --
	(331.54,139.75) --
	(331.77,139.60) --
	(331.99,139.44) --
	(332.21,139.26) --
	(332.44,139.05) --
	(332.66,138.82) --
	(332.88,138.57) --
	(333.10,138.27) --
	(333.33,137.95) --
	(333.55,137.59) --
	(333.77,137.21) --
	(333.99,136.80) --
	(334.22,136.38) --
	(334.44,135.94) --
	(334.66,135.49) --
	(334.89,135.04) --
	(335.11,134.59) --
	(335.33,134.15) --
	(335.55,133.73) --
	(335.78,133.32) --
	(336.00,132.94) --
	(336.22,132.57) --
	(336.45,132.23) --
	(336.67,131.92) --
	(336.89,131.64) --
	(337.11,131.37) --
	(337.34,131.14) --
	(337.56,130.92) --
	(337.78,130.73) --
	(338.00,130.55) --
	(338.23,130.40) --
	(338.45,130.26) --
	(338.67,130.15) --
	(338.90,130.06) --
	(339.12,129.98) --
	(339.34,129.93) --
	(339.56,129.89) --
	(339.79,129.88) --
	(340.01,129.88) --
	(340.23,129.91) --
	(340.45,129.95) --
	(340.68,130.01) --
	(340.90,130.08) --
	(341.12,130.16) --
	(341.35,130.25) --
	(341.57,130.34) --
	(341.79,130.43) --
	(342.01,130.52) --
	(342.24,130.60) --
	(342.46,130.67) --
	(342.68,130.72) --
	(342.90,130.77) --
	(343.13,130.79) --
	(343.35,130.80) --
	(343.57,130.78) --
	(343.80,130.75) --
	(344.02,130.69) --
	(344.24,130.61) --
	(344.46,130.51) --
	(344.69,130.39) --
	(344.91,130.24) --
	(345.13,130.07) --
	(345.36,129.88) --
	(345.58,129.66) --
	(345.80,129.42) --
	(346.02,129.16) --
	(346.25,128.88) --
	(346.47,128.57) --
	(346.69,128.25) --
	(346.91,127.91) --
	(347.14,127.55) --
	(347.36,127.17) --
	(347.58,126.78) --
	(347.81,126.38) --
	(348.03,125.96) --
	(348.25,125.54) --
	(348.47,125.11) --
	(348.70,124.67) --
	(348.92,124.22) --
	(349.14,123.77) --
	(349.36,123.31) --
	(349.59,122.86) --
	(349.81,122.40) --
	(350.03,121.95) --
	(350.26,121.49) --
	(350.48,121.05) --
	(350.70,120.61) --
	(350.92,120.17) --
	(351.15,119.75) --
	(351.37,119.35) --
	(351.59,118.95) --
	(351.82,118.57) --
	(352.04,118.21) --
	(352.26,117.86) --
	(352.48,117.53) --
	(352.71,117.22) --
	(352.93,116.92) --
	(353.15,116.64) --
	(353.37,116.37) --
	(353.60,116.12) --
	(353.82,115.87) --
	(354.04,115.64) --
	(354.27,115.42) --
	(354.49,115.20) --
	(354.71,114.99) --
	(354.93,114.79) --
	(355.16,114.60) --
	(355.38,114.41) --
	(355.60,114.22) --
	(355.82,114.04) --
	(356.05,113.87) --
	(356.27,113.70) --
	(356.49,113.53) --
	(356.72,113.37) --
	(356.94,113.21) --
	(357.16,113.06) --
	(357.38,112.90) --
	(357.61,112.75) --
	(357.83,112.60) --
	(358.05,112.44) --
	(358.28,112.28) --
	(358.50,112.12) --
	(358.72,111.95) --
	(358.94,111.76) --
	(359.17,111.57) --
	(359.39,111.35) --
	(359.61,111.13) --
	(359.83,110.88) --
	(360.06,110.62) --
	(360.28,110.33) --
	(360.50,110.02) --
	(360.73,109.69) --
	(360.95,109.34) --
	(361.17,108.98) --
	(361.39,108.61) --
	(361.62,108.23) --
	(361.84,107.84) --
	(362.06,107.46) --
	(362.28,107.08) --
	(362.51,106.72) --
	(362.73,106.37) --
	(362.95,106.05) --
	(363.18,105.74) --
	(363.40,105.46) --
	(363.62,105.21) --
	(363.84,105.00) --
	(364.07,104.81) --
	(364.29,104.65) --
	(364.51,104.51) --
	(364.73,104.40) --
	(364.96,104.30) --
	(365.18,104.22) --
	(365.40,104.16) --
	(365.63,104.10) --
	(365.85,104.04) --
	(366.07,103.99) --
	(366.29,103.93) --
	(366.52,103.86) --
	(366.74,103.79) --
	(366.96,103.70) --
	(367.19,103.61) --
	(367.41,103.50) --
	(367.63,103.38) --
	(367.85,103.25) --
	(368.08,103.11) --
	(368.30,102.96) --
	(368.52,102.80) --
	(368.74,102.63) --
	(368.97,102.45) --
	(369.19,102.27) --
	(369.41,102.08) --
	(369.64,101.88) --
	(369.86,101.69) --
	(370.08,101.49) --
	(370.30,101.29) --
	(370.53,101.08) --
	(370.75,100.88) --
	(370.97,100.68) --
	(371.19,100.47) --
	(371.42,100.27) --
	(371.64,100.06) --
	(371.86, 99.86) --
	(372.09, 99.67) --
	(372.31, 99.47) --
	(372.53, 99.28) --
	(372.75, 99.10) --
	(372.98, 98.93) --
	(373.20, 98.76) --
	(373.42, 98.60) --
	(373.65, 98.44) --
	(373.87, 98.30) --
	(374.09, 98.15) --
	(374.31, 98.02) --
	(374.54, 97.88) --
	(374.76, 97.75) --
	(374.98, 97.61) --
	(375.20, 97.48) --
	(375.43, 97.33) --
	(375.65, 97.18) --
	(375.87, 97.02) --
	(376.10, 96.85) --
	(376.32, 96.67) --
	(376.54, 96.48) --
	(376.76, 96.28) --
	(376.99, 96.07) --
	(377.21, 95.85) --
	(377.43, 95.62) --
	(377.65, 95.38) --
	(377.88, 95.14) --
	(378.10, 94.90) --
	(378.32, 94.66) --
	(378.55, 94.42) --
	(378.77, 94.18) --
	(378.99, 93.95) --
	(379.21, 93.72) --
	(379.44, 93.49) --
	(379.66, 93.26) --
	(379.88, 93.03) --
	(380.11, 92.80) --
	(380.33, 92.57) --
	(380.55, 92.34) --
	(380.77, 92.10) --
	(381.00, 91.86) --
	(381.22, 91.61) --
	(381.44, 91.36) --
	(381.66, 91.10) --
	(381.89, 90.84) --
	(382.11, 90.57) --
	(382.33, 90.30) --
	(382.56, 90.02) --
	(382.78, 89.75) --
	(383.00, 89.48) --
	(383.22, 89.21) --
	(383.45, 88.95) --
	(383.67, 88.70) --
	(383.89, 88.46) --
	(384.11, 88.22) --
	(384.34, 88.00) --
	(384.56, 87.78) --
	(384.78, 87.57) --
	(385.01, 87.38) --
	(385.23, 87.18) --
	(385.45, 87.00) --
	(385.67, 86.82) --
	(385.90, 86.65) --
	(386.12, 86.48) --
	(386.34, 86.31) --
	(386.56, 86.15) --
	(386.79, 85.98) --
	(387.01, 85.81) --
	(387.23, 85.64) --
	(387.46, 85.48) --
	(387.68, 85.31) --
	(387.90, 85.13) --
	(388.12, 84.96) --
	(388.35, 84.79) --
	(388.57, 84.62) --
	(388.79, 84.45) --
	(389.02, 84.27) --
	(389.24, 84.10) --
	(389.46, 83.93) --
	(389.68, 83.76) --
	(389.91, 83.60) --
	(390.13, 83.43) --
	(390.35, 83.27) --
	(390.57, 83.11) --
	(390.80, 82.95) --
	(391.02, 82.79) --
	(391.24, 82.64) --
	(391.47, 82.49) --
	(391.69, 82.34) --
	(391.91, 82.20) --
	(392.13, 82.06) --
	(392.36, 81.93) --
	(392.58, 81.80) --
	(392.80, 81.68) --
	(393.02, 81.56) --
	(393.25, 81.44) --
	(393.47, 81.33) --
	(393.69, 81.23) --
	(393.92, 81.12) --
	(394.14, 81.03) --
	(394.36, 80.93) --
	(394.58, 80.84) --
	(394.81, 80.74) --
	(395.03, 80.65) --
	(395.25, 80.56) --
	(395.48, 80.47) --
	(395.70, 80.38) --
	(395.92, 80.28) --
	(396.14, 80.19) --
	(396.37, 80.09) --
	(396.59, 79.99) --
	(396.81, 79.89) --
	(397.03, 79.79) --
	(397.26, 79.69) --
	(397.48, 79.58) --
	(397.70, 79.48) --
	(397.93, 79.37) --
	(398.15, 79.26) --
	(398.37, 79.16) --
	(398.59, 79.05) --
	(398.82, 78.94) --
	(399.04, 78.84) --
	(399.26, 78.73) --
	(399.48, 78.63) --
	(399.71, 78.53) --
	(399.93, 78.44) --
	(400.15, 78.35) --
	(400.38, 78.26) --
	(400.60, 78.17) --
	(400.82, 78.09) --
	(401.04, 78.01) --
	(401.27, 77.94) --
	(401.49, 77.87) --
	(401.71, 77.81) --
	(401.94, 77.75) --
	(402.16, 77.69) --
	(402.38, 77.64) --
	(402.60, 77.58) --
	(402.83, 77.53) --
	(403.05, 77.49) --
	(403.27, 77.44) --
	(403.49, 77.40) --
	(403.72, 77.35) --
	(403.94, 77.31) --
	(404.16, 77.27) --
	(404.39, 77.22) --
	(404.61, 77.18) --
	(404.83, 77.14) --
	(405.05, 77.09) --
	(405.28, 77.05) --
	(405.50, 77.01) --
	(405.72, 76.97) --
	(405.94, 76.93) --
	(406.17, 76.89) --
	(406.39, 76.85) --
	(406.61, 76.82) --
	(406.84, 76.78) --
	(407.06, 76.75) --
	(407.28, 76.72) --
	(407.50, 76.69) --
	(407.73, 76.66) --
	(407.95, 76.64) --
	(408.17, 76.61) --
	(408.40, 76.59) --
	(408.62, 76.57) --
	(408.84, 76.55) --
	(409.06, 76.52) --
	(409.29, 76.50) --
	(409.51, 76.48) --
	(409.73, 76.47) --
	(409.95, 76.45) --
	(410.18, 76.43) --
	(410.40, 76.41) --
	(410.62, 76.39) --
	(410.85, 76.38) --
	(411.07, 76.36) --
	(411.29, 76.34) --
	(411.51, 76.33) --
	(411.74, 76.32) --
	(411.96, 76.30) --
	(412.18, 76.29) --
	(412.40, 76.28) --
	(412.63, 76.27) --
	(412.85, 76.26) --
	(413.07, 76.26) --
	(413.30, 76.25) --
	(413.52, 76.25) --
	(413.74, 76.24) --
	(413.96, 76.24) --
	(414.19, 76.23) --
	(414.41, 76.23) --
	(414.63, 76.23) --
	(414.85, 76.23) --
	(415.08, 76.23) --
	(415.30, 76.23) --
	(415.52, 76.23) --
	(415.75, 76.23) --
	(415.97, 76.23) --
	(416.19, 76.23) --
	(416.41, 76.23) --
	(416.64, 76.23) --
	(416.86, 76.23) --
	(417.08, 76.23) --
	(417.31, 76.23) --
	(417.53, 76.23) --
	(417.75, 76.23) --
	(417.97, 76.23) --
	(418.20, 76.23) --
	(418.42, 76.23) --
	(418.64, 76.23) --
	(418.86, 76.23) --
	(419.09, 76.23) --
	(419.31, 76.23) --
	(419.53, 76.23) --
	(419.76, 76.23) --
	(419.98, 76.24) --
	(420.20, 76.24) --
	(420.42, 76.24) --
	(420.65, 76.24) --
	(420.87, 76.24) --
	(421.09, 76.24) --
	(421.31, 76.25) --
	(421.54, 76.25) --
	(421.76, 76.25) --
	(421.98, 76.25) --
	(422.21, 76.25) --
	(422.43, 76.25) --
	(422.43, 76.22) --
	(422.21, 76.22) --
	(421.98, 76.22) --
	(421.76, 76.22) --
	(421.54, 76.22) --
	(421.31, 76.22) --
	(421.09, 76.22) --
	(420.87, 76.22) --
	(420.65, 76.22) --
	(420.42, 76.22) --
	(420.20, 76.22) --
	(419.98, 76.22) --
	(419.76, 76.22) --
	(419.53, 76.22) --
	(419.31, 76.22) --
	(419.09, 76.22) --
	(418.86, 76.22) --
	(418.64, 76.22) --
	(418.42, 76.22) --
	(418.20, 76.22) --
	(417.97, 76.22) --
	(417.75, 76.22) --
	(417.53, 76.22) --
	(417.31, 76.22) --
	(417.08, 76.22) --
	(416.86, 76.22) --
	(416.64, 76.22) --
	(416.41, 76.22) --
	(416.19, 76.22) --
	(415.97, 76.22) --
	(415.75, 76.22) --
	(415.52, 76.22) --
	(415.30, 76.22) --
	(415.08, 76.22) --
	(414.85, 76.22) --
	(414.63, 76.22) --
	(414.41, 76.22) --
	(414.19, 76.22) --
	(413.96, 76.22) --
	(413.74, 76.22) --
	(413.52, 76.22) --
	(413.30, 76.22) --
	(413.07, 76.22) --
	(412.85, 76.22) --
	(412.63, 76.22) --
	(412.40, 76.22) --
	(412.18, 76.22) --
	(411.96, 76.22) --
	(411.74, 76.22) --
	(411.51, 76.22) --
	(411.29, 76.22) --
	(411.07, 76.22) --
	(410.85, 76.22) --
	(410.62, 76.22) --
	(410.40, 76.22) --
	(410.18, 76.22) --
	(409.95, 76.22) --
	(409.73, 76.22) --
	(409.51, 76.22) --
	(409.29, 76.22) --
	(409.06, 76.22) --
	(408.84, 76.22) --
	(408.62, 76.22) --
	(408.40, 76.22) --
	(408.17, 76.22) --
	(407.95, 76.22) --
	(407.73, 76.22) --
	(407.50, 76.22) --
	(407.28, 76.22) --
	(407.06, 76.22) --
	(406.84, 76.22) --
	(406.61, 76.22) --
	(406.39, 76.22) --
	(406.17, 76.22) --
	(405.94, 76.22) --
	(405.72, 76.22) --
	(405.50, 76.22) --
	(405.28, 76.22) --
	(405.05, 76.22) --
	(404.83, 76.22) --
	(404.61, 76.22) --
	(404.39, 76.22) --
	(404.16, 76.22) --
	(403.94, 76.22) --
	(403.72, 76.22) --
	(403.49, 76.22) --
	(403.27, 76.22) --
	(403.05, 76.22) --
	(402.83, 76.22) --
	(402.60, 76.22) --
	(402.38, 76.22) --
	(402.16, 76.22) --
	(401.94, 76.22) --
	(401.71, 76.22) --
	(401.49, 76.22) --
	(401.27, 76.22) --
	(401.04, 76.22) --
	(400.82, 76.22) --
	(400.60, 76.22) --
	(400.38, 76.22) --
	(400.15, 76.22) --
	(399.93, 76.22) --
	(399.71, 76.22) --
	(399.48, 76.22) --
	(399.26, 76.22) --
	(399.04, 76.22) --
	(398.82, 76.22) --
	(398.59, 76.22) --
	(398.37, 76.22) --
	(398.15, 76.22) --
	(397.93, 76.22) --
	(397.70, 76.22) --
	(397.48, 76.22) --
	(397.26, 76.22) --
	(397.03, 76.22) --
	(396.81, 76.22) --
	(396.59, 76.22) --
	(396.37, 76.22) --
	(396.14, 76.22) --
	(395.92, 76.22) --
	(395.70, 76.22) --
	(395.48, 76.22) --
	(395.25, 76.22) --
	(395.03, 76.22) --
	(394.81, 76.22) --
	(394.58, 76.22) --
	(394.36, 76.22) --
	(394.14, 76.22) --
	(393.92, 76.22) --
	(393.69, 76.22) --
	(393.47, 76.22) --
	(393.25, 76.22) --
	(393.02, 76.22) --
	(392.80, 76.22) --
	(392.58, 76.22) --
	(392.36, 76.22) --
	(392.13, 76.22) --
	(391.91, 76.22) --
	(391.69, 76.22) --
	(391.47, 76.22) --
	(391.24, 76.22) --
	(391.02, 76.22) --
	(390.80, 76.22) --
	(390.57, 76.22) --
	(390.35, 76.22) --
	(390.13, 76.22) --
	(389.91, 76.22) --
	(389.68, 76.22) --
	(389.46, 76.22) --
	(389.24, 76.22) --
	(389.02, 76.22) --
	(388.79, 76.22) --
	(388.57, 76.22) --
	(388.35, 76.22) --
	(388.12, 76.22) --
	(387.90, 76.22) --
	(387.68, 76.22) --
	(387.46, 76.22) --
	(387.23, 76.22) --
	(387.01, 76.22) --
	(386.79, 76.22) --
	(386.56, 76.22) --
	(386.34, 76.22) --
	(386.12, 76.22) --
	(385.90, 76.22) --
	(385.67, 76.22) --
	(385.45, 76.22) --
	(385.23, 76.22) --
	(385.01, 76.22) --
	(384.78, 76.22) --
	(384.56, 76.22) --
	(384.34, 76.22) --
	(384.11, 76.22) --
	(383.89, 76.22) --
	(383.67, 76.22) --
	(383.45, 76.22) --
	(383.22, 76.22) --
	(383.00, 76.22) --
	(382.78, 76.22) --
	(382.56, 76.22) --
	(382.33, 76.22) --
	(382.11, 76.22) --
	(381.89, 76.22) --
	(381.66, 76.22) --
	(381.44, 76.22) --
	(381.22, 76.22) --
	(381.00, 76.22) --
	(380.77, 76.22) --
	(380.55, 76.22) --
	(380.33, 76.22) --
	(380.11, 76.22) --
	(379.88, 76.22) --
	(379.66, 76.22) --
	(379.44, 76.22) --
	(379.21, 76.22) --
	(378.99, 76.22) --
	(378.77, 76.22) --
	(378.55, 76.22) --
	(378.32, 76.22) --
	(378.10, 76.22) --
	(377.88, 76.22) --
	(377.65, 76.22) --
	(377.43, 76.22) --
	(377.21, 76.22) --
	(376.99, 76.22) --
	(376.76, 76.22) --
	(376.54, 76.22) --
	(376.32, 76.22) --
	(376.10, 76.22) --
	(375.87, 76.22) --
	(375.65, 76.22) --
	(375.43, 76.22) --
	(375.20, 76.22) --
	(374.98, 76.22) --
	(374.76, 76.22) --
	(374.54, 76.22) --
	(374.31, 76.22) --
	(374.09, 76.22) --
	(373.87, 76.22) --
	(373.65, 76.22) --
	(373.42, 76.22) --
	(373.20, 76.22) --
	(372.98, 76.22) --
	(372.75, 76.22) --
	(372.53, 76.22) --
	(372.31, 76.22) --
	(372.09, 76.22) --
	(371.86, 76.22) --
	(371.64, 76.22) --
	(371.42, 76.22) --
	(371.19, 76.22) --
	(370.97, 76.22) --
	(370.75, 76.22) --
	(370.53, 76.22) --
	(370.30, 76.22) --
	(370.08, 76.22) --
	(369.86, 76.22) --
	(369.64, 76.22) --
	(369.41, 76.22) --
	(369.19, 76.22) --
	(368.97, 76.22) --
	(368.74, 76.22) --
	(368.52, 76.22) --
	(368.30, 76.22) --
	(368.08, 76.22) --
	(367.85, 76.22) --
	(367.63, 76.22) --
	(367.41, 76.22) --
	(367.19, 76.22) --
	(366.96, 76.22) --
	(366.74, 76.22) --
	(366.52, 76.22) --
	(366.29, 76.22) --
	(366.07, 76.22) --
	(365.85, 76.22) --
	(365.63, 76.22) --
	(365.40, 76.22) --
	(365.18, 76.22) --
	(364.96, 76.22) --
	(364.73, 76.22) --
	(364.51, 76.22) --
	(364.29, 76.22) --
	(364.07, 76.22) --
	(363.84, 76.22) --
	(363.62, 76.22) --
	(363.40, 76.22) --
	(363.18, 76.22) --
	(362.95, 76.22) --
	(362.73, 76.22) --
	(362.51, 76.22) --
	(362.28, 76.22) --
	(362.06, 76.22) --
	(361.84, 76.22) --
	(361.62, 76.22) --
	(361.39, 76.22) --
	(361.17, 76.22) --
	(360.95, 76.22) --
	(360.73, 76.22) --
	(360.50, 76.22) --
	(360.28, 76.22) --
	(360.06, 76.22) --
	(359.83, 76.22) --
	(359.61, 76.22) --
	(359.39, 76.22) --
	(359.17, 76.22) --
	(358.94, 76.22) --
	(358.72, 76.22) --
	(358.50, 76.22) --
	(358.28, 76.22) --
	(358.05, 76.22) --
	(357.83, 76.22) --
	(357.61, 76.22) --
	(357.38, 76.22) --
	(357.16, 76.22) --
	(356.94, 76.22) --
	(356.72, 76.22) --
	(356.49, 76.22) --
	(356.27, 76.22) --
	(356.05, 76.22) --
	(355.82, 76.22) --
	(355.60, 76.22) --
	(355.38, 76.22) --
	(355.16, 76.22) --
	(354.93, 76.22) --
	(354.71, 76.22) --
	(354.49, 76.22) --
	(354.27, 76.22) --
	(354.04, 76.22) --
	(353.82, 76.22) --
	(353.60, 76.22) --
	(353.37, 76.22) --
	(353.15, 76.22) --
	(352.93, 76.22) --
	(352.71, 76.22) --
	(352.48, 76.22) --
	(352.26, 76.22) --
	(352.04, 76.22) --
	(351.82, 76.22) --
	(351.59, 76.22) --
	(351.37, 76.22) --
	(351.15, 76.22) --
	(350.92, 76.22) --
	(350.70, 76.22) --
	(350.48, 76.22) --
	(350.26, 76.22) --
	(350.03, 76.22) --
	(349.81, 76.22) --
	(349.59, 76.22) --
	(349.36, 76.22) --
	(349.14, 76.22) --
	(348.92, 76.22) --
	(348.70, 76.22) --
	(348.47, 76.22) --
	(348.25, 76.22) --
	(348.03, 76.22) --
	(347.81, 76.22) --
	(347.58, 76.22) --
	(347.36, 76.22) --
	(347.14, 76.22) --
	(346.91, 76.22) --
	(346.69, 76.22) --
	(346.47, 76.22) --
	(346.25, 76.22) --
	(346.02, 76.22) --
	(345.80, 76.22) --
	(345.58, 76.22) --
	(345.36, 76.22) --
	(345.13, 76.22) --
	(344.91, 76.22) --
	(344.69, 76.22) --
	(344.46, 76.22) --
	(344.24, 76.22) --
	(344.02, 76.22) --
	(343.80, 76.22) --
	(343.57, 76.22) --
	(343.35, 76.22) --
	(343.13, 76.22) --
	(342.90, 76.22) --
	(342.68, 76.22) --
	(342.46, 76.22) --
	(342.24, 76.22) --
	(342.01, 76.22) --
	(341.79, 76.22) --
	(341.57, 76.22) --
	(341.35, 76.22) --
	(341.12, 76.22) --
	(340.90, 76.22) --
	(340.68, 76.22) --
	(340.45, 76.22) --
	(340.23, 76.22) --
	(340.01, 76.22) --
	(339.79, 76.22) --
	(339.56, 76.22) --
	(339.34, 76.22) --
	(339.12, 76.22) --
	(338.90, 76.22) --
	(338.67, 76.22) --
	(338.45, 76.22) --
	(338.23, 76.22) --
	(338.00, 76.22) --
	(337.78, 76.22) --
	(337.56, 76.22) --
	(337.34, 76.22) --
	(337.11, 76.22) --
	(336.89, 76.22) --
	(336.67, 76.22) --
	(336.45, 76.22) --
	(336.22, 76.22) --
	(336.00, 76.22) --
	(335.78, 76.22) --
	(335.55, 76.22) --
	(335.33, 76.22) --
	(335.11, 76.22) --
	(334.89, 76.22) --
	(334.66, 76.22) --
	(334.44, 76.22) --
	(334.22, 76.22) --
	(333.99, 76.22) --
	(333.77, 76.22) --
	(333.55, 76.22) --
	(333.33, 76.22) --
	(333.10, 76.22) --
	(332.88, 76.22) --
	(332.66, 76.22) --
	(332.44, 76.22) --
	(332.21, 76.22) --
	(331.99, 76.22) --
	(331.77, 76.22) --
	(331.54, 76.22) --
	(331.32, 76.22) --
	(331.10, 76.22) --
	(330.88, 76.22) --
	(330.65, 76.22) --
	(330.43, 76.22) --
	(330.21, 76.22) --
	(329.99, 76.22) --
	(329.76, 76.22) --
	(329.54, 76.22) --
	(329.32, 76.22) --
	(329.09, 76.22) --
	(328.87, 76.22) --
	(328.65, 76.22) --
	(328.43, 76.22) --
	(328.20, 76.22) --
	(327.98, 76.22) --
	(327.76, 76.22) --
	(327.53, 76.22) --
	(327.31, 76.22) --
	(327.09, 76.22) --
	(326.87, 76.22) --
	(326.64, 76.22) --
	(326.42, 76.22) --
	(326.20, 76.22) --
	(325.98, 76.22) --
	(325.75, 76.22) --
	(325.53, 76.22) --
	(325.31, 76.22) --
	(325.08, 76.22) --
	(324.86, 76.22) --
	(324.64, 76.22) --
	(324.42, 76.22) --
	(324.19, 76.22) --
	(323.97, 76.22) --
	(323.75, 76.22) --
	(323.53, 76.22) --
	(323.30, 76.22) --
	(323.08, 76.22) --
	(322.86, 76.22) --
	(322.63, 76.22) --
	(322.41, 76.22) --
	(322.19, 76.22) --
	(321.97, 76.22) --
	(321.74, 76.22) --
	(321.52, 76.22) --
	(321.30, 76.22) --
	(321.07, 76.22) --
	(320.85, 76.22) --
	(320.63, 76.22) --
	(320.41, 76.22) --
	(320.18, 76.22) --
	(319.96, 76.22) --
	(319.74, 76.22) --
	(319.52, 76.22) --
	(319.29, 76.22) --
	(319.07, 76.22) --
	(318.85, 76.22) --
	(318.62, 76.22) --
	(318.40, 76.22) --
	(318.18, 76.22) --
	(317.96, 76.22) --
	(317.73, 76.22) --
	(317.51, 76.22) --
	(317.29, 76.22) --
	(317.07, 76.22) --
	(316.84, 76.22) --
	(316.62, 76.22) --
	(316.40, 76.22) --
	(316.17, 76.22) --
	(315.95, 76.22) --
	(315.73, 76.22) --
	(315.51, 76.22) --
	(315.28, 76.22) --
	(315.06, 76.22) --
	(314.84, 76.22) --
	(314.62, 76.22) --
	(314.39, 76.22) --
	(314.17, 76.22) --
	(313.95, 76.22) --
	(313.72, 76.22) --
	(313.50, 76.22) --
	(313.28, 76.22) --
	(313.06, 76.22) --
	(312.83, 76.22) --
	(312.61, 76.22) --
	(312.39, 76.22) --
	(312.16, 76.22) --
	(311.94, 76.22) --
	(311.72, 76.22) --
	(311.50, 76.22) --
	(311.27, 76.22) --
	(311.05, 76.22) --
	(310.83, 76.22) --
	(310.61, 76.22) --
	(310.38, 76.22) --
	(310.16, 76.22) --
	(309.94, 76.22) --
	(309.71, 76.22) --
	(309.49, 76.22) --
	(309.27, 76.22) --
	(309.05, 76.22) --
	(308.82, 76.22) --
	(308.60, 76.22) --
	cycle;

\path[draw=drawColor,line width= 0.6pt,line join=round,line cap=round] (308.60, 76.23) --
	(308.82, 76.23) --
	(309.05, 76.23) --
	(309.27, 76.23) --
	(309.49, 76.23) --
	(309.71, 76.23) --
	(309.94, 76.24) --
	(310.16, 76.24) --
	(310.38, 76.25) --
	(310.61, 76.27) --
	(310.83, 76.29) --
	(311.05, 76.31) --
	(311.27, 76.35) --
	(311.50, 76.41) --
	(311.72, 76.49) --
	(311.94, 76.60) --
	(312.16, 76.74) --
	(312.39, 76.93) --
	(312.61, 77.17) --
	(312.83, 77.49) --
	(313.06, 77.89) --
	(313.28, 78.40) --
	(313.50, 79.05) --
	(313.72, 79.83) --
	(313.95, 80.79) --
	(314.17, 81.93) --
	(314.39, 83.27) --
	(314.62, 84.83) --
	(314.84, 86.63) --
	(315.06, 88.70) --
	(315.28, 91.04) --
	(315.51, 93.63) --
	(315.73, 96.47) --
	(315.95, 99.53) --
	(316.17,102.80) --
	(316.40,106.26) --
	(316.62,109.87) --
	(316.84,113.60) --
	(317.07,117.39) --
	(317.29,121.18) --
	(317.51,124.95) --
	(317.73,128.64) --
	(317.96,132.20) --
	(318.18,135.61) --
	(318.40,138.83) --
	(318.62,141.81) --
	(318.85,144.53) --
	(319.07,146.98) --
	(319.29,149.18) --
	(319.52,151.11) --
	(319.74,152.78) --
	(319.96,154.19) --
	(320.18,155.37) --
	(320.41,156.31) --
	(320.63,157.03) --
	(320.85,157.56) --
	(321.07,157.92) --
	(321.30,158.13) --
	(321.52,158.21) --
	(321.74,158.17) --
	(321.97,158.03) --
	(322.19,157.79) --
	(322.41,157.47) --
	(322.63,157.08) --
	(322.86,156.64) --
	(323.08,156.15) --
	(323.30,155.62) --
	(323.53,155.06) --
	(323.75,154.47) --
	(323.97,153.87) --
	(324.19,153.25) --
	(324.42,152.63) --
	(324.64,152.01) --
	(324.86,151.38) --
	(325.08,150.77) --
	(325.31,150.16) --
	(325.53,149.56) --
	(325.75,148.97) --
	(325.98,148.40) --
	(326.20,147.84) --
	(326.42,147.30) --
	(326.64,146.78) --
	(326.87,146.26) --
	(327.09,145.77) --
	(327.31,145.28) --
	(327.53,144.81) --
	(327.76,144.36) --
	(327.98,143.92) --
	(328.20,143.50) --
	(328.43,143.10) --
	(328.65,142.72) --
	(328.87,142.35) --
	(329.09,142.01) --
	(329.32,141.70) --
	(329.54,141.41) --
	(329.76,141.15) --
	(329.99,140.91) --
	(330.21,140.70) --
	(330.43,140.50) --
	(330.65,140.33) --
	(330.88,140.17) --
	(331.10,140.03) --
	(331.32,139.89) --
	(331.54,139.75) --
	(331.77,139.60) --
	(331.99,139.44) --
	(332.21,139.26) --
	(332.44,139.05) --
	(332.66,138.82) --
	(332.88,138.57) --
	(333.10,138.27) --
	(333.33,137.95) --
	(333.55,137.59) --
	(333.77,137.21) --
	(333.99,136.80) --
	(334.22,136.38) --
	(334.44,135.94) --
	(334.66,135.49) --
	(334.89,135.04) --
	(335.11,134.59) --
	(335.33,134.15) --
	(335.55,133.73) --
	(335.78,133.32) --
	(336.00,132.94) --
	(336.22,132.57) --
	(336.45,132.23) --
	(336.67,131.92) --
	(336.89,131.64) --
	(337.11,131.37) --
	(337.34,131.14) --
	(337.56,130.92) --
	(337.78,130.73) --
	(338.00,130.55) --
	(338.23,130.40) --
	(338.45,130.26) --
	(338.67,130.15) --
	(338.90,130.06) --
	(339.12,129.98) --
	(339.34,129.93) --
	(339.56,129.89) --
	(339.79,129.88) --
	(340.01,129.88) --
	(340.23,129.91) --
	(340.45,129.95) --
	(340.68,130.01) --
	(340.90,130.08) --
	(341.12,130.16) --
	(341.35,130.25) --
	(341.57,130.34) --
	(341.79,130.43) --
	(342.01,130.52) --
	(342.24,130.60) --
	(342.46,130.67) --
	(342.68,130.72) --
	(342.90,130.77) --
	(343.13,130.79) --
	(343.35,130.80) --
	(343.57,130.78) --
	(343.80,130.75) --
	(344.02,130.69) --
	(344.24,130.61) --
	(344.46,130.51) --
	(344.69,130.39) --
	(344.91,130.24) --
	(345.13,130.07) --
	(345.36,129.88) --
	(345.58,129.66) --
	(345.80,129.42) --
	(346.02,129.16) --
	(346.25,128.88) --
	(346.47,128.57) --
	(346.69,128.25) --
	(346.91,127.91) --
	(347.14,127.55) --
	(347.36,127.17) --
	(347.58,126.78) --
	(347.81,126.38) --
	(348.03,125.96) --
	(348.25,125.54) --
	(348.47,125.11) --
	(348.70,124.67) --
	(348.92,124.22) --
	(349.14,123.77) --
	(349.36,123.31) --
	(349.59,122.86) --
	(349.81,122.40) --
	(350.03,121.95) --
	(350.26,121.49) --
	(350.48,121.05) --
	(350.70,120.61) --
	(350.92,120.17) --
	(351.15,119.75) --
	(351.37,119.35) --
	(351.59,118.95) --
	(351.82,118.57) --
	(352.04,118.21) --
	(352.26,117.86) --
	(352.48,117.53) --
	(352.71,117.22) --
	(352.93,116.92) --
	(353.15,116.64) --
	(353.37,116.37) --
	(353.60,116.12) --
	(353.82,115.87) --
	(354.04,115.64) --
	(354.27,115.42) --
	(354.49,115.20) --
	(354.71,114.99) --
	(354.93,114.79) --
	(355.16,114.60) --
	(355.38,114.41) --
	(355.60,114.22) --
	(355.82,114.04) --
	(356.05,113.87) --
	(356.27,113.70) --
	(356.49,113.53) --
	(356.72,113.37) --
	(356.94,113.21) --
	(357.16,113.06) --
	(357.38,112.90) --
	(357.61,112.75) --
	(357.83,112.60) --
	(358.05,112.44) --
	(358.28,112.28) --
	(358.50,112.12) --
	(358.72,111.95) --
	(358.94,111.76) --
	(359.17,111.57) --
	(359.39,111.35) --
	(359.61,111.13) --
	(359.83,110.88) --
	(360.06,110.62) --
	(360.28,110.33) --
	(360.50,110.02) --
	(360.73,109.69) --
	(360.95,109.34) --
	(361.17,108.98) --
	(361.39,108.61) --
	(361.62,108.23) --
	(361.84,107.84) --
	(362.06,107.46) --
	(362.28,107.08) --
	(362.51,106.72) --
	(362.73,106.37) --
	(362.95,106.05) --
	(363.18,105.74) --
	(363.40,105.46) --
	(363.62,105.21) --
	(363.84,105.00) --
	(364.07,104.81) --
	(364.29,104.65) --
	(364.51,104.51) --
	(364.73,104.40) --
	(364.96,104.30) --
	(365.18,104.22) --
	(365.40,104.16) --
	(365.63,104.10) --
	(365.85,104.04) --
	(366.07,103.99) --
	(366.29,103.93) --
	(366.52,103.86) --
	(366.74,103.79) --
	(366.96,103.70) --
	(367.19,103.61) --
	(367.41,103.50) --
	(367.63,103.38) --
	(367.85,103.25) --
	(368.08,103.11) --
	(368.30,102.96) --
	(368.52,102.80) --
	(368.74,102.63) --
	(368.97,102.45) --
	(369.19,102.27) --
	(369.41,102.08) --
	(369.64,101.88) --
	(369.86,101.69) --
	(370.08,101.49) --
	(370.30,101.29) --
	(370.53,101.08) --
	(370.75,100.88) --
	(370.97,100.68) --
	(371.19,100.47) --
	(371.42,100.27) --
	(371.64,100.06) --
	(371.86, 99.86) --
	(372.09, 99.67) --
	(372.31, 99.47) --
	(372.53, 99.28) --
	(372.75, 99.10) --
	(372.98, 98.93) --
	(373.20, 98.76) --
	(373.42, 98.60) --
	(373.65, 98.44) --
	(373.87, 98.30) --
	(374.09, 98.15) --
	(374.31, 98.02) --
	(374.54, 97.88) --
	(374.76, 97.75) --
	(374.98, 97.61) --
	(375.20, 97.48) --
	(375.43, 97.33) --
	(375.65, 97.18) --
	(375.87, 97.02) --
	(376.10, 96.85) --
	(376.32, 96.67) --
	(376.54, 96.48) --
	(376.76, 96.28) --
	(376.99, 96.07) --
	(377.21, 95.85) --
	(377.43, 95.62) --
	(377.65, 95.38) --
	(377.88, 95.14) --
	(378.10, 94.90) --
	(378.32, 94.66) --
	(378.55, 94.42) --
	(378.77, 94.18) --
	(378.99, 93.95) --
	(379.21, 93.72) --
	(379.44, 93.49) --
	(379.66, 93.26) --
	(379.88, 93.03) --
	(380.11, 92.80) --
	(380.33, 92.57) --
	(380.55, 92.34) --
	(380.77, 92.10) --
	(381.00, 91.86) --
	(381.22, 91.61) --
	(381.44, 91.36) --
	(381.66, 91.10) --
	(381.89, 90.84) --
	(382.11, 90.57) --
	(382.33, 90.30) --
	(382.56, 90.02) --
	(382.78, 89.75) --
	(383.00, 89.48) --
	(383.22, 89.21) --
	(383.45, 88.95) --
	(383.67, 88.70) --
	(383.89, 88.46) --
	(384.11, 88.22) --
	(384.34, 88.00) --
	(384.56, 87.78) --
	(384.78, 87.57) --
	(385.01, 87.38) --
	(385.23, 87.18) --
	(385.45, 87.00) --
	(385.67, 86.82) --
	(385.90, 86.65) --
	(386.12, 86.48) --
	(386.34, 86.31) --
	(386.56, 86.15) --
	(386.79, 85.98) --
	(387.01, 85.81) --
	(387.23, 85.64) --
	(387.46, 85.48) --
	(387.68, 85.31) --
	(387.90, 85.13) --
	(388.12, 84.96) --
	(388.35, 84.79) --
	(388.57, 84.62) --
	(388.79, 84.45) --
	(389.02, 84.27) --
	(389.24, 84.10) --
	(389.46, 83.93) --
	(389.68, 83.76) --
	(389.91, 83.60) --
	(390.13, 83.43) --
	(390.35, 83.27) --
	(390.57, 83.11) --
	(390.80, 82.95) --
	(391.02, 82.79) --
	(391.24, 82.64) --
	(391.47, 82.49) --
	(391.69, 82.34) --
	(391.91, 82.20) --
	(392.13, 82.06) --
	(392.36, 81.93) --
	(392.58, 81.80) --
	(392.80, 81.68) --
	(393.02, 81.56) --
	(393.25, 81.44) --
	(393.47, 81.33) --
	(393.69, 81.23) --
	(393.92, 81.12) --
	(394.14, 81.03) --
	(394.36, 80.93) --
	(394.58, 80.84) --
	(394.81, 80.74) --
	(395.03, 80.65) --
	(395.25, 80.56) --
	(395.48, 80.47) --
	(395.70, 80.38) --
	(395.92, 80.28) --
	(396.14, 80.19) --
	(396.37, 80.09) --
	(396.59, 79.99) --
	(396.81, 79.89) --
	(397.03, 79.79) --
	(397.26, 79.69) --
	(397.48, 79.58) --
	(397.70, 79.48) --
	(397.93, 79.37) --
	(398.15, 79.26) --
	(398.37, 79.16) --
	(398.59, 79.05) --
	(398.82, 78.94) --
	(399.04, 78.84) --
	(399.26, 78.73) --
	(399.48, 78.63) --
	(399.71, 78.53) --
	(399.93, 78.44) --
	(400.15, 78.35) --
	(400.38, 78.26) --
	(400.60, 78.17) --
	(400.82, 78.09) --
	(401.04, 78.01) --
	(401.27, 77.94) --
	(401.49, 77.87) --
	(401.71, 77.81) --
	(401.94, 77.75) --
	(402.16, 77.69) --
	(402.38, 77.64) --
	(402.60, 77.58) --
	(402.83, 77.53) --
	(403.05, 77.49) --
	(403.27, 77.44) --
	(403.49, 77.40) --
	(403.72, 77.35) --
	(403.94, 77.31) --
	(404.16, 77.27) --
	(404.39, 77.22) --
	(404.61, 77.18) --
	(404.83, 77.14) --
	(405.05, 77.09) --
	(405.28, 77.05) --
	(405.50, 77.01) --
	(405.72, 76.97) --
	(405.94, 76.93) --
	(406.17, 76.89) --
	(406.39, 76.85) --
	(406.61, 76.82) --
	(406.84, 76.78) --
	(407.06, 76.75) --
	(407.28, 76.72) --
	(407.50, 76.69) --
	(407.73, 76.66) --
	(407.95, 76.64) --
	(408.17, 76.61) --
	(408.40, 76.59) --
	(408.62, 76.57) --
	(408.84, 76.55) --
	(409.06, 76.52) --
	(409.29, 76.50) --
	(409.51, 76.48) --
	(409.73, 76.47) --
	(409.95, 76.45) --
	(410.18, 76.43) --
	(410.40, 76.41) --
	(410.62, 76.39) --
	(410.85, 76.38) --
	(411.07, 76.36) --
	(411.29, 76.34) --
	(411.51, 76.33) --
	(411.74, 76.32) --
	(411.96, 76.30) --
	(412.18, 76.29) --
	(412.40, 76.28) --
	(412.63, 76.27) --
	(412.85, 76.26) --
	(413.07, 76.26) --
	(413.30, 76.25) --
	(413.52, 76.25) --
	(413.74, 76.24) --
	(413.96, 76.24) --
	(414.19, 76.23) --
	(414.41, 76.23) --
	(414.63, 76.23) --
	(414.85, 76.23) --
	(415.08, 76.23) --
	(415.30, 76.23) --
	(415.52, 76.23) --
	(415.75, 76.23) --
	(415.97, 76.23) --
	(416.19, 76.23) --
	(416.41, 76.23) --
	(416.64, 76.23) --
	(416.86, 76.23) --
	(417.08, 76.23) --
	(417.31, 76.23) --
	(417.53, 76.23) --
	(417.75, 76.23) --
	(417.97, 76.23) --
	(418.20, 76.23) --
	(418.42, 76.23) --
	(418.64, 76.23) --
	(418.86, 76.23) --
	(419.09, 76.23) --
	(419.31, 76.23) --
	(419.53, 76.23) --
	(419.76, 76.23) --
	(419.98, 76.24) --
	(420.20, 76.24) --
	(420.42, 76.24) --
	(420.65, 76.24) --
	(420.87, 76.24) --
	(421.09, 76.24) --
	(421.31, 76.25) --
	(421.54, 76.25) --
	(421.76, 76.25) --
	(421.98, 76.25) --
	(422.21, 76.25) --
	(422.43, 76.25);
\end{scope}
\begin{scope}
\path[clip] ( 41.49,267.01) rectangle (166.70,283.58);
\definecolor{drawColor}{gray}{0.10}

\node[text=drawColor,anchor=base,inner sep=0pt, outer sep=0pt, scale=  0.70] at (104.09,272.26) {\textbf{Región Centro}};
\end{scope}
\begin{scope}
\path[clip] (172.20,267.01) rectangle (297.41,283.58);
\definecolor{drawColor}{gray}{0.10}

\node[text=drawColor,anchor=base,inner sep=0pt, outer sep=0pt, scale=  0.70] at (234.80,272.26) {\textbf{Región Norte}};
\end{scope}
\begin{scope}
\path[clip] (302.91,267.01) rectangle (428.12,283.58);
\definecolor{drawColor}{gray}{0.10}

\node[text=drawColor,anchor=base,inner sep=0pt, outer sep=0pt, scale=  0.70] at (365.51,272.26) {\textbf{Región Sur}};
\end{scope}
\begin{scope}
\path[clip] (  0.00,  0.00) rectangle (433.62,289.08);
\definecolor{drawColor}{RGB}{0,0,0}

\path[draw=drawColor,line width= 0.6pt,line join=round] ( 41.49, 67.14) --
	(166.70, 67.14);
\end{scope}
\begin{scope}
\path[clip] (  0.00,  0.00) rectangle (433.62,289.08);
\definecolor{drawColor}{gray}{0.20}

\path[draw=drawColor,line width= 0.6pt,line join=round] ( 69.94, 64.39) --
	( 69.94, 67.14);

\path[draw=drawColor,line width= 0.6pt,line join=round] (104.09, 64.39) --
	(104.09, 67.14);

\path[draw=drawColor,line width= 0.6pt,line join=round] (138.24, 64.39) --
	(138.24, 67.14);
\end{scope}
\begin{scope}
\path[clip] (  0.00,  0.00) rectangle (433.62,289.08);
\definecolor{drawColor}{RGB}{0,0,0}

\node[text=drawColor,anchor=base,inner sep=0pt, outer sep=0pt, scale=  0.88] at ( 69.94, 56.13) {30};

\node[text=drawColor,anchor=base,inner sep=0pt, outer sep=0pt, scale=  0.88] at (104.09, 56.13) {60};

\node[text=drawColor,anchor=base,inner sep=0pt, outer sep=0pt, scale=  0.88] at (138.24, 56.13) {90};
\end{scope}
\begin{scope}
\path[clip] (  0.00,  0.00) rectangle (433.62,289.08);
\definecolor{drawColor}{RGB}{0,0,0}

\path[draw=drawColor,line width= 0.6pt,line join=round] (172.20, 67.14) --
	(297.41, 67.14);
\end{scope}
\begin{scope}
\path[clip] (  0.00,  0.00) rectangle (433.62,289.08);
\definecolor{drawColor}{gray}{0.20}

\path[draw=drawColor,line width= 0.6pt,line join=round] (200.66, 64.39) --
	(200.66, 67.14);

\path[draw=drawColor,line width= 0.6pt,line join=round] (234.80, 64.39) --
	(234.80, 67.14);

\path[draw=drawColor,line width= 0.6pt,line join=round] (268.95, 64.39) --
	(268.95, 67.14);
\end{scope}
\begin{scope}
\path[clip] (  0.00,  0.00) rectangle (433.62,289.08);
\definecolor{drawColor}{RGB}{0,0,0}

\node[text=drawColor,anchor=base,inner sep=0pt, outer sep=0pt, scale=  0.88] at (200.66, 56.13) {30};

\node[text=drawColor,anchor=base,inner sep=0pt, outer sep=0pt, scale=  0.88] at (234.80, 56.13) {60};

\node[text=drawColor,anchor=base,inner sep=0pt, outer sep=0pt, scale=  0.88] at (268.95, 56.13) {90};
\end{scope}
\begin{scope}
\path[clip] (  0.00,  0.00) rectangle (433.62,289.08);
\definecolor{drawColor}{RGB}{0,0,0}

\path[draw=drawColor,line width= 0.6pt,line join=round] (302.91, 67.14) --
	(428.12, 67.14);
\end{scope}
\begin{scope}
\path[clip] (  0.00,  0.00) rectangle (433.62,289.08);
\definecolor{drawColor}{gray}{0.20}

\path[draw=drawColor,line width= 0.6pt,line join=round] (331.37, 64.39) --
	(331.37, 67.14);

\path[draw=drawColor,line width= 0.6pt,line join=round] (365.51, 64.39) --
	(365.51, 67.14);

\path[draw=drawColor,line width= 0.6pt,line join=round] (399.66, 64.39) --
	(399.66, 67.14);
\end{scope}
\begin{scope}
\path[clip] (  0.00,  0.00) rectangle (433.62,289.08);
\definecolor{drawColor}{RGB}{0,0,0}

\node[text=drawColor,anchor=base,inner sep=0pt, outer sep=0pt, scale=  0.88] at (331.37, 56.13) {30};

\node[text=drawColor,anchor=base,inner sep=0pt, outer sep=0pt, scale=  0.88] at (365.51, 56.13) {60};

\node[text=drawColor,anchor=base,inner sep=0pt, outer sep=0pt, scale=  0.88] at (399.66, 56.13) {90};
\end{scope}
\begin{scope}
\path[clip] (  0.00,  0.00) rectangle (433.62,289.08);
\definecolor{drawColor}{RGB}{0,0,0}

\path[draw=drawColor,line width= 0.6pt,line join=round] ( 41.49, 67.14) --
	( 41.49,267.01);
\end{scope}
\begin{scope}
\path[clip] (  0.00,  0.00) rectangle (433.62,289.08);
\definecolor{drawColor}{RGB}{0,0,0}

\node[text=drawColor,anchor=base east,inner sep=0pt, outer sep=0pt, scale=  0.88] at ( 36.54, 73.19) {0.0{\%}};

\node[text=drawColor,anchor=base east,inner sep=0pt, outer sep=0pt, scale=  0.88] at ( 36.54,128.75) {2.0{\%}};

\node[text=drawColor,anchor=base east,inner sep=0pt, outer sep=0pt, scale=  0.88] at ( 36.54,184.30) {4.0{\%}};

\node[text=drawColor,anchor=base east,inner sep=0pt, outer sep=0pt, scale=  0.88] at ( 36.54,239.86) {6.0{\%}};
\end{scope}
\begin{scope}
\path[clip] (  0.00,  0.00) rectangle (433.62,289.08);
\definecolor{drawColor}{gray}{0.20}

\path[draw=drawColor,line width= 0.6pt,line join=round] ( 38.74, 76.22) --
	( 41.49, 76.22);

\path[draw=drawColor,line width= 0.6pt,line join=round] ( 38.74,131.78) --
	( 41.49,131.78);

\path[draw=drawColor,line width= 0.6pt,line join=round] ( 38.74,187.33) --
	( 41.49,187.33);

\path[draw=drawColor,line width= 0.6pt,line join=round] ( 38.74,242.89) --
	( 41.49,242.89);
\end{scope}
\begin{scope}
\path[clip] (  0.00,  0.00) rectangle (433.62,289.08);
\definecolor{drawColor}{RGB}{0,0,0}

\node[text=drawColor,anchor=base,inner sep=0pt, outer sep=0pt, scale=  0.60] at (234.80, 44.09) {Edad};
\end{scope}
\begin{scope}
\path[clip] (  0.00,  0.00) rectangle (433.62,289.08);
\definecolor{drawColor}{RGB}{0,0,0}

\node[text=drawColor,rotate= 90.00,anchor=base,inner sep=0pt, outer sep=0pt, scale=  0.60] at ( 13.08,167.07) {Densidad};
\end{scope}
\begin{scope}
\path[clip] (  0.00,  0.00) rectangle (433.62,289.08);
\definecolor{fillColor}{RGB}{255,255,255}

\path[fill=fillColor] (170.19,  5.50) rectangle (299.42, 30.95);
\end{scope}
\begin{scope}
\path[clip] (  0.00,  0.00) rectangle (433.62,289.08);
\definecolor{fillColor}{gray}{0.95}

\path[fill=fillColor] (181.19, 11.00) rectangle (195.64, 25.45);
\end{scope}
\begin{scope}
\path[clip] (  0.00,  0.00) rectangle (433.62,289.08);
\definecolor{drawColor}{RGB}{0,0,0}
\definecolor{fillColor}{RGB}{228,26,28}

\path[draw=drawColor,line width= 0.6pt,line cap=rect,fill=fillColor,fill opacity=0.50] (181.90, 11.71) rectangle (194.93, 24.74);
\end{scope}
\begin{scope}
\path[clip] (  0.00,  0.00) rectangle (433.62,289.08);
\definecolor{fillColor}{gray}{0.95}

\path[fill=fillColor] (244.84, 11.00) rectangle (259.29, 25.45);
\end{scope}
\begin{scope}
\path[clip] (  0.00,  0.00) rectangle (433.62,289.08);
\definecolor{drawColor}{RGB}{0,0,0}
\definecolor{fillColor}{RGB}{55,126,184}

\path[draw=drawColor,line width= 0.6pt,line cap=rect,fill=fillColor,fill opacity=0.50] (245.55, 11.71) rectangle (258.58, 24.74);
\end{scope}
\begin{scope}
\path[clip] (  0.00,  0.00) rectangle (433.62,289.08);
\definecolor{drawColor}{RGB}{0,0,0}

\node[text=drawColor,anchor=base west,inner sep=0pt, outer sep=0pt, scale=  0.70] at (201.14, 15.20) {Migrantes};
\end{scope}
\begin{scope}
\path[clip] (  0.00,  0.00) rectangle (433.62,289.08);
\definecolor{drawColor}{RGB}{0,0,0}

\node[text=drawColor,anchor=base west,inner sep=0pt, outer sep=0pt, scale=  0.70] at (264.79, 15.20) {Nativos};
\end{scope}
\end{tikzpicture}
 
\label{figure:edad_mig}
\begin{flushleft}
\begin{scriptsize}
Fuente: Elaboración propia en base a EPH.\\
Nota: Solamente se tienen en cuenta nativos y migrantes mayores a 18 años. Los migrantes están definidos como personas que vivían hace cinco años en otra provincia. Los nativos están definidos como personas que nacieron y viven en la misma provincia. Las estimaciones corresponden al período desde el segundo trimestre de 2016 hasta el cuarto trimestre de 2019.
\end{scriptsize}
\end{flushleft}
\end{center}
\end{figure}

\newpage
Como se puede observar en el Cuadro \ref{cuadro:edad_mig}, la mitad de los migrantes en las regiones Norte y Sur tienen menos de 30 o 31 años. Si se observa a la Región Centro la mitad de los migrantes tienen 26 años o menos, siendo considerablemente menor a la edad mediana de la población nativa, la cual gira en torno a los 39 a 42 años.

La diferencia de edad entre los nativos y migrantes no presenta, en este caso, una dimensión distinta dependiendo del género de la persona. Para los hombres y las mujeres de todas las regiones la edad mediana de los migrantes es considerablemente inferior que la de los nativos. 

Los indicadores etarios de los migrantes internos presentados anteriormente refuerzan la teoría de que la edad tiene una relación negativa con la decisión del éxodo, particularmente en donde más marcada se encuentra esta relación es en la Región Centro.

\begin{table}[ht!]
\caption{Edad y género de los migrantes por regiones} 
\centering
\footnotesize
\begin{tabular}{lllc}
  \hline
  \hline
Región & Condición & Género & Mediana de Edad \\ 
  \hline
  \hline
 Centro & Migrantes & Hombres & 26 \\ 
 Centro & Migrantes & Mujeres & 26 \\ 
 Centro & Nativos & Hombres & 39 \\ 
 Centro & Nativos & Mujeres & 42 \\ 
 Norte & Migrantes & Hombres & 31\\ 
 Norte & Migrantes & Mujeres & 30 \\ 
 Norte & Nativos & Hombres & 38 \\ 
 Norte & Nativos & Mujeres & 41 \\ 
 Sur & Migrantes & Hombres & 31 \\ 
 Sur & Migrantes & Mujeres & 31 \\ 
 Sur & Nativos & Hombres & 36\\ 
 Sur & Nativos & Mujeres & 39 \\ 
   \hline
\end{tabular}
\label{cuadro:edad_mig}
\begin{flushleft}
\begin{scriptsize}
Fuente: Elaboración propia en base a EPH.\\
Nota: Solamente se tienen en cuenta nativos y migrantes mayores a 18 años. Los migrantes están definidos como personas que vivían hace cinco años en otra provincia. Los nativos están definidos como personas que nacieron y viven en la misma provincia. Las estimaciones corresponden al período desde el segundo trimestre de 2016 hasta el cuarto trimestre de 2019.
\end{scriptsize}
\end{flushleft}
\end{table}

\newpage
\subsection{Pobreza y patrimonio}

Uno de los principales factores económicos de la migración es la busqueda de un estandar de vida más elevado del que se podría costear en la localidad de origen. Este factor económico de expulsión  puede verse reflejado en los ingresos que percibe una persona, laborales y no laborales(subsidios y transferencias), como en la posibilidad de acumulación de activos patrimoniales, como por ejemplo el acceso a la vivienda propia. 

En la Figura \ref{figure:pobre_mig} se puede observar como la incidencia en la pobreza de los migrantes es considerablemente menor que la de los nativos para las tres regiones analizadas. Esta diferencia brinda un indicio de un mejor pasar económico, en promedio, de las personas que decidieron migrar con respecto de la población nativa para cada una de las regiones analizadas.

La percepción de subsidios y la contribución a las arcas públicas por parte de la población migrante posee mucha relevancia en la literatura de migraciones. Esto puede ser analizado al nivel de migraciones regionales considerando la recepción de ingresos no laborales (subsidios) de los migrantes y nativos para cada región. 


El acceso a transferencias o subsidios puede actuar como un determinante para la migración de las personas. Se puede observar en el Cuadro \ref{cuadro:subsidio_mig} que es más probable que un nativo reciba algun tipo de subsidio que un migrante, siendo estos los que menor participación en la percepción de subsidios poseen en promedio. En la región Norte es en donde la brecha entre los nativos y migrantes que reciben subsidios es más amplia.

\newpage
\begin{figure}[ht!]
\begin{center}
\caption{\\Incidencia en la pobreza de los nativos y migrantes por regiones}
\label{figure:pobre_mig}
% Created by tikzDevice version 0.12.3.1 on 2021-07-01 12:50:32
% !TEX encoding = UTF-8 Unicode
\begin{tikzpicture}[x=1pt,y=1pt]
\definecolor{fillColor}{RGB}{255,255,255}
\path[use as bounding box,fill=fillColor,fill opacity=0.00] (0,0) rectangle (433.62,289.08);
\begin{scope}
\path[clip] (  0.00,  0.00) rectangle (433.62,289.08);
\definecolor{drawColor}{RGB}{255,255,255}
\definecolor{fillColor}{RGB}{255,255,255}

\path[draw=drawColor,line width= 0.6pt,line join=round,line cap=round,fill=fillColor] (  0.00, -0.00) rectangle (433.62,289.08);
\end{scope}
\begin{scope}
\path[clip] ( 43.44, 67.14) rectangle (168.00,267.01);
\definecolor{drawColor}{RGB}{255,255,255}

\path[draw=drawColor,line width= 0.3pt,line join=round] ( 43.44, 98.94) --
	(168.00, 98.94);

\path[draw=drawColor,line width= 0.3pt,line join=round] ( 43.44,144.36) --
	(168.00,144.36);

\path[draw=drawColor,line width= 0.3pt,line join=round] ( 43.44,189.79) --
	(168.00,189.79);

\path[draw=drawColor,line width= 0.3pt,line join=round] ( 43.44,235.21) --
	(168.00,235.21);

\path[draw=drawColor,line width= 0.6pt,line join=round] ( 43.44, 76.22) --
	(168.00, 76.22);

\path[draw=drawColor,line width= 0.6pt,line join=round] ( 43.44,121.65) --
	(168.00,121.65);

\path[draw=drawColor,line width= 0.6pt,line join=round] ( 43.44,167.07) --
	(168.00,167.07);

\path[draw=drawColor,line width= 0.6pt,line join=round] ( 43.44,212.50) --
	(168.00,212.50);

\path[draw=drawColor,line width= 0.6pt,line join=round] ( 43.44,257.92) --
	(168.00,257.92);

\path[draw=drawColor,line width= 0.6pt,line join=round] ( 77.41, 67.14) --
	( 77.41,267.01);

\path[draw=drawColor,line width= 0.6pt,line join=round] (134.03, 67.14) --
	(134.03,267.01);
\definecolor{fillColor}{RGB}{228,26,28}

\path[fill=fillColor] ( 54.77,149.74) rectangle (100.06,257.92);
\definecolor{fillColor}{RGB}{55,126,184}

\path[fill=fillColor] ( 54.77, 76.22) rectangle (100.06,149.74);
\definecolor{fillColor}{RGB}{228,26,28}

\path[fill=fillColor] (111.38,176.68) rectangle (156.68,257.92);
\definecolor{fillColor}{RGB}{55,126,184}

\path[fill=fillColor] (111.38, 76.22) rectangle (156.68,176.68);
\definecolor{drawColor}{RGB}{0,0,0}

\node[text=drawColor,anchor=base,inner sep=0pt, outer sep=0pt, scale=  0.85] at ( 77.41,190.07) {59.54{\%}};

\node[text=drawColor,anchor=base,inner sep=0pt, outer sep=0pt, scale=  0.85] at ( 77.41,102.69) {40.46{\%}};

\node[text=drawColor,anchor=base,inner sep=0pt, outer sep=0pt, scale=  0.85] at (134.03,206.24) {44.72{\%}};

\node[text=drawColor,anchor=base,inner sep=0pt, outer sep=0pt, scale=  0.85] at (134.03,113.47) {55.28{\%}};
\end{scope}
\begin{scope}
\path[clip] (173.50, 67.14) rectangle (298.06,267.01);
\definecolor{drawColor}{RGB}{255,255,255}

\path[draw=drawColor,line width= 0.3pt,line join=round] (173.50, 98.94) --
	(298.06, 98.94);

\path[draw=drawColor,line width= 0.3pt,line join=round] (173.50,144.36) --
	(298.06,144.36);

\path[draw=drawColor,line width= 0.3pt,line join=round] (173.50,189.79) --
	(298.06,189.79);

\path[draw=drawColor,line width= 0.3pt,line join=round] (173.50,235.21) --
	(298.06,235.21);

\path[draw=drawColor,line width= 0.6pt,line join=round] (173.50, 76.22) --
	(298.06, 76.22);

\path[draw=drawColor,line width= 0.6pt,line join=round] (173.50,121.65) --
	(298.06,121.65);

\path[draw=drawColor,line width= 0.6pt,line join=round] (173.50,167.07) --
	(298.06,167.07);

\path[draw=drawColor,line width= 0.6pt,line join=round] (173.50,212.50) --
	(298.06,212.50);

\path[draw=drawColor,line width= 0.6pt,line join=round] (173.50,257.92) --
	(298.06,257.92);

\path[draw=drawColor,line width= 0.6pt,line join=round] (207.47, 67.14) --
	(207.47,267.01);

\path[draw=drawColor,line width= 0.6pt,line join=round] (264.09, 67.14) --
	(264.09,267.01);
\definecolor{fillColor}{RGB}{228,26,28}

\path[fill=fillColor] (184.83,137.47) rectangle (230.12,257.92);
\definecolor{fillColor}{RGB}{55,126,184}

\path[fill=fillColor] (184.83, 76.22) rectangle (230.12,137.47);
\definecolor{fillColor}{RGB}{228,26,28}

\path[fill=fillColor] (241.44,153.36) rectangle (286.74,257.92);
\definecolor{fillColor}{RGB}{55,126,184}

\path[fill=fillColor] (241.44, 76.22) rectangle (286.74,153.36);
\definecolor{drawColor}{RGB}{0,0,0}

\node[text=drawColor,anchor=base,inner sep=0pt, outer sep=0pt, scale=  0.85] at (207.47,182.71) {66.29{\%}};

\node[text=drawColor,anchor=base,inner sep=0pt, outer sep=0pt, scale=  0.85] at (207.47, 97.78) {33.71{\%}};

\node[text=drawColor,anchor=base,inner sep=0pt, outer sep=0pt, scale=  0.85] at (264.09,192.25) {57.55{\%}};

\node[text=drawColor,anchor=base,inner sep=0pt, outer sep=0pt, scale=  0.85] at (264.09,104.14) {42.45{\%}};
\end{scope}
\begin{scope}
\path[clip] (303.56, 67.14) rectangle (428.12,267.01);
\definecolor{drawColor}{RGB}{255,255,255}

\path[draw=drawColor,line width= 0.3pt,line join=round] (303.56, 98.94) --
	(428.12, 98.94);

\path[draw=drawColor,line width= 0.3pt,line join=round] (303.56,144.36) --
	(428.12,144.36);

\path[draw=drawColor,line width= 0.3pt,line join=round] (303.56,189.79) --
	(428.12,189.79);

\path[draw=drawColor,line width= 0.3pt,line join=round] (303.56,235.21) --
	(428.12,235.21);

\path[draw=drawColor,line width= 0.6pt,line join=round] (303.56, 76.22) --
	(428.12, 76.22);

\path[draw=drawColor,line width= 0.6pt,line join=round] (303.56,121.65) --
	(428.12,121.65);

\path[draw=drawColor,line width= 0.6pt,line join=round] (303.56,167.07) --
	(428.12,167.07);

\path[draw=drawColor,line width= 0.6pt,line join=round] (303.56,212.50) --
	(428.12,212.50);

\path[draw=drawColor,line width= 0.6pt,line join=round] (303.56,257.92) --
	(428.12,257.92);

\path[draw=drawColor,line width= 0.6pt,line join=round] (337.53, 67.14) --
	(337.53,267.01);

\path[draw=drawColor,line width= 0.6pt,line join=round] (394.15, 67.14) --
	(394.15,267.01);
\definecolor{fillColor}{RGB}{228,26,28}

\path[fill=fillColor] (314.88,139.29) rectangle (360.18,257.92);
\definecolor{fillColor}{RGB}{55,126,184}

\path[fill=fillColor] (314.88, 76.22) rectangle (360.18,139.29);
\definecolor{fillColor}{RGB}{228,26,28}

\path[fill=fillColor] (371.50,154.69) rectangle (416.80,257.92);
\definecolor{fillColor}{RGB}{55,126,184}

\path[fill=fillColor] (371.50, 76.22) rectangle (416.80,154.69);
\definecolor{drawColor}{RGB}{0,0,0}

\node[text=drawColor,anchor=base,inner sep=0pt, outer sep=0pt, scale=  0.85] at (337.53,183.80) {65.29{\%}};

\node[text=drawColor,anchor=base,inner sep=0pt, outer sep=0pt, scale=  0.85] at (337.53, 98.51) {34.71{\%}};

\node[text=drawColor,anchor=base,inner sep=0pt, outer sep=0pt, scale=  0.85] at (394.15,193.05) {56.81{\%}};

\node[text=drawColor,anchor=base,inner sep=0pt, outer sep=0pt, scale=  0.85] at (394.15,104.67) {43.19{\%}};
\end{scope}
\begin{scope}
\path[clip] ( 43.44,267.01) rectangle (168.00,283.58);
\definecolor{drawColor}{gray}{0.10}

\node[text=drawColor,anchor=base,inner sep=0pt, outer sep=0pt, scale=  0.88] at (105.72,272.26) {\textbf{Región Centro}};
\end{scope}
\begin{scope}
\path[clip] (173.50,267.01) rectangle (298.06,283.58);
\definecolor{drawColor}{gray}{0.10}

\node[text=drawColor,anchor=base,inner sep=0pt, outer sep=0pt, scale=  0.88] at (235.78,272.26) {\textbf{Región Norte}};
\end{scope}
\begin{scope}
\path[clip] (303.56,267.01) rectangle (428.12,283.58);
\definecolor{drawColor}{gray}{0.10}

\node[text=drawColor,anchor=base,inner sep=0pt, outer sep=0pt, scale=  0.88] at (365.84,272.26) {\textbf{Región Sur}};
\end{scope}
\begin{scope}
\path[clip] (  0.00,  0.00) rectangle (433.62,289.08);
\definecolor{drawColor}{gray}{0.20}

\path[draw=drawColor,line width= 0.6pt,line join=round] ( 77.41, 64.39) --
	( 77.41, 67.14);

\path[draw=drawColor,line width= 0.6pt,line join=round] (134.03, 64.39) --
	(134.03, 67.14);
\end{scope}
\begin{scope}
\path[clip] (  0.00,  0.00) rectangle (433.62,289.08);
\definecolor{drawColor}{RGB}{0,0,0}

\node[text=drawColor,anchor=base,inner sep=0pt, outer sep=0pt, scale=  0.88] at ( 77.41, 56.13) {Migrantes};

\node[text=drawColor,anchor=base,inner sep=0pt, outer sep=0pt, scale=  0.88] at (134.03, 56.13) {Nativos};
\end{scope}
\begin{scope}
\path[clip] (  0.00,  0.00) rectangle (433.62,289.08);
\definecolor{drawColor}{gray}{0.20}

\path[draw=drawColor,line width= 0.6pt,line join=round] (207.47, 64.39) --
	(207.47, 67.14);

\path[draw=drawColor,line width= 0.6pt,line join=round] (264.09, 64.39) --
	(264.09, 67.14);
\end{scope}
\begin{scope}
\path[clip] (  0.00,  0.00) rectangle (433.62,289.08);
\definecolor{drawColor}{RGB}{0,0,0}

\node[text=drawColor,anchor=base,inner sep=0pt, outer sep=0pt, scale=  0.88] at (207.47, 56.13) {Migrantes};

\node[text=drawColor,anchor=base,inner sep=0pt, outer sep=0pt, scale=  0.88] at (264.09, 56.13) {Nativos};
\end{scope}
\begin{scope}
\path[clip] (  0.00,  0.00) rectangle (433.62,289.08);
\definecolor{drawColor}{gray}{0.20}

\path[draw=drawColor,line width= 0.6pt,line join=round] (337.53, 64.39) --
	(337.53, 67.14);

\path[draw=drawColor,line width= 0.6pt,line join=round] (394.15, 64.39) --
	(394.15, 67.14);
\end{scope}
\begin{scope}
\path[clip] (  0.00,  0.00) rectangle (433.62,289.08);
\definecolor{drawColor}{RGB}{0,0,0}

\node[text=drawColor,anchor=base,inner sep=0pt, outer sep=0pt, scale=  0.88] at (337.53, 56.13) {Migrantes};

\node[text=drawColor,anchor=base,inner sep=0pt, outer sep=0pt, scale=  0.88] at (394.15, 56.13) {Nativos};
\end{scope}
\begin{scope}
\path[clip] (  0.00,  0.00) rectangle (433.62,289.08);
\definecolor{drawColor}{RGB}{0,0,0}

\node[text=drawColor,anchor=base east,inner sep=0pt, outer sep=0pt, scale=  0.88] at ( 38.49, 73.19) {0{\%}};

\node[text=drawColor,anchor=base east,inner sep=0pt, outer sep=0pt, scale=  0.88] at ( 38.49,118.62) {25{\%}};

\node[text=drawColor,anchor=base east,inner sep=0pt, outer sep=0pt, scale=  0.88] at ( 38.49,164.04) {50{\%}};

\node[text=drawColor,anchor=base east,inner sep=0pt, outer sep=0pt, scale=  0.88] at ( 38.49,209.47) {75{\%}};

\node[text=drawColor,anchor=base east,inner sep=0pt, outer sep=0pt, scale=  0.88] at ( 38.49,254.89) {100{\%}};
\end{scope}
\begin{scope}
\path[clip] (  0.00,  0.00) rectangle (433.62,289.08);
\definecolor{drawColor}{gray}{0.20}

\path[draw=drawColor,line width= 0.6pt,line join=round] ( 40.69, 76.22) --
	( 43.44, 76.22);

\path[draw=drawColor,line width= 0.6pt,line join=round] ( 40.69,121.65) --
	( 43.44,121.65);

\path[draw=drawColor,line width= 0.6pt,line join=round] ( 40.69,167.07) --
	( 43.44,167.07);

\path[draw=drawColor,line width= 0.6pt,line join=round] ( 40.69,212.50) --
	( 43.44,212.50);

\path[draw=drawColor,line width= 0.6pt,line join=round] ( 40.69,257.92) --
	( 43.44,257.92);
\end{scope}
\begin{scope}
\path[clip] (  0.00,  0.00) rectangle (433.62,289.08);
\definecolor{fillColor}{RGB}{255,255,255}

\path[fill=fillColor] (175.78,  5.50) rectangle (295.78, 30.95);
\end{scope}
\begin{scope}
\path[clip] (  0.00,  0.00) rectangle (433.62,289.08);
\definecolor{fillColor}{gray}{0.95}

\path[fill=fillColor] (186.78, 11.00) rectangle (201.24, 25.45);
\end{scope}
\begin{scope}
\path[clip] (  0.00,  0.00) rectangle (433.62,289.08);
\definecolor{fillColor}{RGB}{228,26,28}

\path[fill=fillColor] (187.49, 11.71) rectangle (200.52, 24.74);
\end{scope}
\begin{scope}
\path[clip] (  0.00,  0.00) rectangle (433.62,289.08);
\definecolor{fillColor}{gray}{0.95}

\path[fill=fillColor] (247.94, 11.00) rectangle (262.39, 25.45);
\end{scope}
\begin{scope}
\path[clip] (  0.00,  0.00) rectangle (433.62,289.08);
\definecolor{fillColor}{RGB}{55,126,184}

\path[fill=fillColor] (248.65, 11.71) rectangle (261.68, 24.74);
\end{scope}
\begin{scope}
\path[clip] (  0.00,  0.00) rectangle (433.62,289.08);
\definecolor{drawColor}{RGB}{0,0,0}

\node[text=drawColor,anchor=base west,inner sep=0pt, outer sep=0pt, scale=  0.70] at (206.74, 15.20) {No pobre};
\end{scope}
\begin{scope}
\path[clip] (  0.00,  0.00) rectangle (433.62,289.08);
\definecolor{drawColor}{RGB}{0,0,0}

\node[text=drawColor,anchor=base west,inner sep=0pt, outer sep=0pt, scale=  0.70] at (267.89, 15.20) {Pobre};
\end{scope}
\end{tikzpicture}
 
\end{center}
\begin{flushleft}
\begin{scriptsize}
Fuente: Elaboración propia en base a EPH.\\
Nota: Se consideran pobres las personas que viven en un hogar en donde el ingreso total familiar es menor a la canasta básica total. Los migrantes están definidos como personas que vivían hace cinco años en otra provincia. Los nativos están definidos como personas que nacieron y viven en la misma provincia. Las estimaciones corresponden al período desde el segundo trimestre de 2016 hasta el cuarto trimestre de 2019.
\end{scriptsize}
\end{flushleft}
\end{figure}

\begin{table}[htbp!]
\caption{\\Percepción de subsidios de los nativos y migrantes por regiones} 
\label{cuadro:subsidio_mig}
\centering
\begin{tabular}{lcccc}
\hline
\multicolumn{1}{c}{Región} & \multicolumn{2}{c}{Migrante} & \multicolumn{2}{c}{Nativo} \\ \cline{2-5} 
                           & No recibe      & Recibe      & No recibe     & Recibe     \\ \hline
Centro                     & 91.78\%        & 8.22\%      & 87.32\%       & 12.68\%    \\ 
Norte                      & 85.42\%        & 14.58\%     & 77.51\%       & 22.49\%    \\ 
Sur                        & 89.12\%        & 10.88\%     & 88.37\%       & 11.63\%    \\ \hline
\end{tabular}
\begin{flushleft}
\begin{scriptsize}
Fuente: Elaboración propia en base a EPH.\\
Nota: Se considera que las personas reciben subsidios si en los últimos tres meses desde el día del relevamiento el hogar recibio subsidios o ayuda social en dinero. Los migrantes están definidos como personas que vivían hace cinco años en otra provincia. Los nativos están definidos como personas que nacieron y viven en la misma provincia. Las estimaciones corresponden al período desde el segundo trimestre de 2016 hasta el cuarto trimestre de 2019.
\end{scriptsize}
\end{flushleft}
\end{table}

En la Figura \ref{figure:vivienda_mig} se encuentran las diferencias en el acceso a la vivienda propia de los nativos y migrantes  en las distintas regiones. La posibilidad de ser propietario de una vivienda brinda mayores niveles de estabilidad en la planificación individual y familiar de la persona, lo cual puede desencadenar en un mayor costo de migrar a la hora de planificar la decisión del éxodo. En este aspecto para las regiones Norte, Centro y Sur los nativos tienen un mayor acceso a la vivienda propia que los migrantes, esta diferencia es más marcada en la región Sur, en donde solo el 9.5\% de los migrantes tienen acceso a una vivienda propia.

\begin{figure}[htbp!]
\begin{center}
\caption{\\Propiedad de la vivienda de los nativos y migrantes por regiones}
\label{figure:vivienda_mig}
% Created by tikzDevice version 0.12.3.1 on 2021-05-23 22:13:50
% !TEX encoding = UTF-8 Unicode
\begin{tikzpicture}[x=1pt,y=1pt]
\definecolor{fillColor}{RGB}{255,255,255}
\path[use as bounding box,fill=fillColor,fill opacity=0.00] (0,0) rectangle (433.62,289.08);
\begin{scope}
\path[clip] (144.60,191.21) rectangle (234.70,281.31);
\definecolor{fillColor}{RGB}{228,26,28}

\path[fill=fillColor] (189.65,236.26) --
	(189.33,235.06) --
	(189.01,233.86) --
	(188.69,232.65) --
	(188.37,231.45) --
	(188.05,230.25) --
	(187.73,229.05) --
	(187.41,227.85) --
	(187.09,226.65) --
	(186.76,225.45) --
	(186.44,224.25) --
	(186.12,223.05) --
	(185.80,221.85) --
	(185.48,220.65) --
	(185.16,219.45) --
	(184.84,218.25) --
	(184.52,217.05) --
	(184.20,215.85) --
	(183.88,214.64) --
	(183.56,213.44) --
	(183.24,212.24) --
	(182.92,211.04) --
	(182.60,209.84) --
	(182.27,208.64) --
	(181.95,207.44) --
	(181.63,206.24) --
	(181.31,205.04) --
	(180.99,203.84) --
	(180.67,202.64) --
	(180.35,201.44) --
	(181.54,201.14) --
	(182.74,200.88) --
	(183.95,200.67) --
	(185.16,200.50) --
	(186.38,200.36) --
	(187.60,200.27) --
	(188.83,200.23) --
	(190.05,200.22) --
	(191.28,200.25) --
	(192.50,200.33) --
	(193.73,200.45) --
	(194.94,200.61) --
	(196.15,200.81) --
	(197.35,201.05) --
	(198.55,201.33) --
	(199.73,201.65) --
	(200.90,202.02) --
	(202.06,202.42) --
	(203.20,202.86) --
	(204.33,203.34) --
	(205.44,203.86) --
	(206.54,204.42) --
	(207.61,205.01) --
	(208.66,205.64) --
	(209.69,206.30) --
	(210.70,207.00) --
	(211.68,207.73) --
	(212.64,208.50) --
	(213.57,209.30) --
	(214.47,210.13) --
	(215.35,210.99) --
	(216.19,211.88) --
	(217.01,212.79) --
	(217.79,213.74) --
	(218.54,214.71) --
	(219.26,215.70) --
	(219.94,216.72) --
	(220.59,217.76) --
	(221.20,218.83) --
	(221.77,219.91) --
	(222.31,221.01) --
	(222.81,222.13) --
	(223.27,223.27) --
	(223.69,224.42) --
	(224.08,225.58) --
	(224.42,226.76) --
	(224.72,227.95) --
	(224.98,229.15) --
	(225.21,230.35) --
	(225.39,231.57) --
	(225.52,232.79) --
	(225.62,234.01) --
	(225.68,235.23) --
	(225.69,236.46) --
	(225.66,237.68) --
	(225.59,238.91) --
	(225.48,240.13) --
	(225.33,241.35) --
	(225.14,242.56) --
	(224.90,243.76) --
	(224.63,244.96) --
	(224.31,246.14) --
	(223.95,247.31) --
	(223.56,248.48) --
	(223.12,249.62) --
	(222.65,250.75) --
	(222.14,251.87) --
	(221.59,252.96) --
	(221.00,254.04) --
	(220.38,255.10) --
	(219.72,256.13) --
	(219.02,257.14) --
	(218.30,258.13) --
	(217.54,259.09) --
	(216.74,260.03) --
	(215.92,260.93) --
	(215.06,261.81) --
	(214.18,262.66) --
	(213.27,263.48) --
	(212.33,264.27) --
	(211.36,265.03) --
	(210.37,265.75) --
	(209.36,266.43) --
	(208.32,267.09) --
	(207.26,267.70) --
	(206.18,268.29) --
	(205.08,268.83) --
	(203.96,269.34) --
	(202.83,269.80) --
	(201.68,270.23) --
	(200.52,270.62) --
	(199.34,270.97) --
	(198.15,271.28) --
	(196.96,271.55) --
	(195.75,271.78) --
	(194.54,271.96) --
	(193.32,272.11) --
	(192.10,272.21) --
	(190.88,272.28) --
	(189.65,272.30) --
	(189.65,271.06) --
	(189.65,269.81) --
	(189.65,268.57) --
	(189.65,267.33) --
	(189.65,266.08) --
	(189.65,264.84) --
	(189.65,263.60) --
	(189.65,262.36) --
	(189.65,261.11) --
	(189.65,259.87) --
	(189.65,258.63) --
	(189.65,257.38) --
	(189.65,256.14) --
	(189.65,254.90) --
	(189.65,253.66) --
	(189.65,252.41) --
	(189.65,251.17) --
	(189.65,249.93) --
	(189.65,248.69) --
	(189.65,247.44) --
	(189.65,246.20) --
	(189.65,244.96) --
	(189.65,243.71) --
	(189.65,242.47) --
	(189.65,241.23) --
	(189.65,239.99) --
	(189.65,238.74) --
	(189.65,237.50) --
	(189.65,236.26) --
	(189.65,236.26) --
	cycle;
\definecolor{fillColor}{RGB}{55,126,184}

\path[fill=fillColor] (189.65,236.26) --
	(189.65,237.50) --
	(189.65,238.74) --
	(189.65,239.99) --
	(189.65,241.23) --
	(189.65,242.47) --
	(189.65,243.71) --
	(189.65,244.96) --
	(189.65,246.20) --
	(189.65,247.44) --
	(189.65,248.69) --
	(189.65,249.93) --
	(189.65,251.17) --
	(189.65,252.41) --
	(189.65,253.66) --
	(189.65,254.90) --
	(189.65,256.14) --
	(189.65,257.38) --
	(189.65,258.63) --
	(189.65,259.87) --
	(189.65,261.11) --
	(189.65,262.36) --
	(189.65,263.60) --
	(189.65,264.84) --
	(189.65,266.08) --
	(189.65,267.33) --
	(189.65,268.57) --
	(189.65,269.81) --
	(189.65,271.06) --
	(189.65,272.30) --
	(188.43,272.28) --
	(187.21,272.22) --
	(185.99,272.11) --
	(184.78,271.97) --
	(183.57,271.78) --
	(182.37,271.56) --
	(181.18,271.29) --
	(180.00,270.98) --
	(178.83,270.63) --
	(177.67,270.25) --
	(176.52,269.82) --
	(175.40,269.36) --
	(174.28,268.86) --
	(173.19,268.32) --
	(172.11,267.74) --
	(171.05,267.13) --
	(170.02,266.48) --
	(169.00,265.80) --
	(168.02,265.08) --
	(167.05,264.33) --
	(166.11,263.55) --
	(165.20,262.74) --
	(164.32,261.89) --
	(163.46,261.02) --
	(162.64,260.12) --
	(161.85,259.19) --
	(161.09,258.23) --
	(160.36,257.25) --
	(159.66,256.25) --
	(159.00,255.22) --
	(158.38,254.17) --
	(157.79,253.10) --
	(157.24,252.01) --
	(156.72,250.91) --
	(156.24,249.78) --
	(155.80,248.64) --
	(155.40,247.49) --
	(155.04,246.32) --
	(154.72,245.14) --
	(154.44,243.95) --
	(154.20,242.76) --
	(154.00,241.55) --
	(153.84,240.34) --
	(153.72,239.13) --
	(153.65,237.91) --
	(153.61,236.69) --
	(153.62,235.46) --
	(153.67,234.24) --
	(153.76,233.03) --
	(153.89,231.81) --
	(154.06,230.60) --
	(154.27,229.40) --
	(154.52,228.20) --
	(154.81,227.02) --
	(155.15,225.84) --
	(155.52,224.68) --
	(155.93,223.53) --
	(156.38,222.40) --
	(156.87,221.28) --
	(157.40,220.17) --
	(157.96,219.09) --
	(158.56,218.03) --
	(159.20,216.98) --
	(159.87,215.96) --
	(160.57,214.97) --
	(161.31,213.99) --
	(162.08,213.04) --
	(162.88,212.12) --
	(163.72,211.23) --
	(164.58,210.37) --
	(165.47,209.53) --
	(166.39,208.73) --
	(167.34,207.96) --
	(168.31,207.22) --
	(169.30,206.51) --
	(170.32,205.84) --
	(171.37,205.20) --
	(172.43,204.60) --
	(173.51,204.03) --
	(174.61,203.50) --
	(175.73,203.01) --
	(176.86,202.56) --
	(178.01,202.15) --
	(179.18,201.77) --
	(180.35,201.44) --
	(180.67,202.64) --
	(180.99,203.84) --
	(181.31,205.04) --
	(181.63,206.24) --
	(181.95,207.44) --
	(182.27,208.64) --
	(182.60,209.84) --
	(182.92,211.04) --
	(183.24,212.24) --
	(183.56,213.44) --
	(183.88,214.64) --
	(184.20,215.85) --
	(184.52,217.05) --
	(184.84,218.25) --
	(185.16,219.45) --
	(185.48,220.65) --
	(185.80,221.85) --
	(186.12,223.05) --
	(186.44,224.25) --
	(186.76,225.45) --
	(187.09,226.65) --
	(187.41,227.85) --
	(187.73,229.05) --
	(188.05,230.25) --
	(188.37,231.45) --
	(188.69,232.65) --
	(189.01,233.86) --
	(189.33,235.06) --
	(189.65,236.26) --
	(189.65,236.26) --
	cycle;
\definecolor{drawColor}{RGB}{0,0,0}

\node[text=drawColor,anchor=base,inner sep=0pt, outer sep=0pt, scale=  0.71] at (207.52,231.46) {54.15{\%}};

\node[text=drawColor,anchor=base,inner sep=0pt, outer sep=0pt, scale=  0.71] at (171.78,236.15) {45.85{\%}};
\end{scope}
\begin{scope}
\path[clip] (144.60, 95.60) rectangle (234.70,185.71);
\definecolor{fillColor}{RGB}{228,26,28}

\path[fill=fillColor] (189.65,140.65) --
	(190.17,139.52) --
	(190.68,138.39) --
	(191.20,137.26) --
	(191.72,136.13) --
	(192.24,135.00) --
	(192.75,133.87) --
	(193.27,132.74) --
	(193.79,131.61) --
	(194.30,130.48) --
	(194.82,129.35) --
	(195.34,128.22) --
	(195.85,127.09) --
	(196.37,125.96) --
	(196.89,124.83) --
	(197.40,123.70) --
	(197.92,122.57) --
	(198.44,121.44) --
	(198.95,120.31) --
	(199.47,119.18) --
	(199.99,118.05) --
	(200.50,116.92) --
	(201.02,115.79) --
	(201.54,114.66) --
	(202.06,113.53) --
	(202.57,112.40) --
	(203.09,111.27) --
	(203.61,110.14) --
	(204.12,109.01) --
	(204.64,107.88) --
	(205.74,108.40) --
	(206.83,108.97) --
	(207.89,109.57) --
	(208.93,110.21) --
	(209.95,110.88) --
	(210.95,111.58) --
	(211.93,112.32) --
	(212.87,113.09) --
	(213.80,113.90) --
	(214.69,114.73) --
	(215.55,115.59) --
	(216.39,116.49) --
	(217.19,117.41) --
	(217.96,118.35) --
	(218.70,119.33) --
	(219.41,120.32) --
	(220.08,121.34) --
	(220.72,122.39) --
	(221.32,123.45) --
	(221.89,124.53) --
	(222.41,125.64) --
	(222.90,126.75) --
	(223.36,127.89) --
	(223.77,129.04) --
	(224.14,130.20) --
	(224.48,131.38) --
	(224.77,132.57) --
	(225.03,133.76) --
	(225.24,134.96) --
	(225.41,136.17) --
	(225.54,137.39) --
	(225.63,138.61) --
	(225.68,139.83) --
	(225.69,141.05) --
	(225.66,142.27) --
	(225.58,143.49) --
	(225.46,144.71) --
	(225.31,145.92) --
	(225.11,147.13) --
	(224.87,148.32) --
	(224.59,149.51) --
	(224.27,150.69) --
	(223.91,151.86) --
	(223.51,153.02) --
	(223.07,154.16) --
	(222.59,155.28) --
	(222.08,156.39) --
	(221.52,157.48) --
	(220.94,158.55) --
	(220.31,159.60) --
	(219.65,160.63) --
	(218.96,161.64) --
	(218.23,162.62) --
	(217.47,163.57) --
	(216.67,164.50) --
	(215.85,165.40) --
	(215.00,166.28) --
	(214.11,167.12) --
	(213.20,167.94) --
	(212.26,168.72) --
	(211.30,169.47) --
	(210.31,170.19) --
	(209.30,170.87) --
	(208.26,171.52) --
	(207.20,172.13) --
	(206.13,172.71) --
	(205.03,173.25) --
	(203.92,173.75) --
	(202.78,174.22) --
	(201.64,174.64) --
	(200.48,175.03) --
	(199.31,175.38) --
	(198.13,175.68) --
	(196.93,175.95) --
	(195.73,176.18) --
	(194.52,176.36) --
	(193.31,176.51) --
	(192.09,176.61) --
	(190.87,176.67) --
	(189.65,176.70) --
	(189.65,175.45) --
	(189.65,174.21) --
	(189.65,172.97) --
	(189.65,171.72) --
	(189.65,170.48) --
	(189.65,169.24) --
	(189.65,168.00) --
	(189.65,166.75) --
	(189.65,165.51) --
	(189.65,164.27) --
	(189.65,163.02) --
	(189.65,161.78) --
	(189.65,160.54) --
	(189.65,159.30) --
	(189.65,158.05) --
	(189.65,156.81) --
	(189.65,155.57) --
	(189.65,154.33) --
	(189.65,153.08) --
	(189.65,151.84) --
	(189.65,150.60) --
	(189.65,149.35) --
	(189.65,148.11) --
	(189.65,146.87) --
	(189.65,145.63) --
	(189.65,144.38) --
	(189.65,143.14) --
	(189.65,141.90) --
	(189.65,140.65) --
	(189.65,140.65) --
	cycle;
\definecolor{fillColor}{RGB}{55,126,184}

\path[fill=fillColor] (189.65,140.65) --
	(189.65,141.90) --
	(189.65,143.14) --
	(189.65,144.38) --
	(189.65,145.63) --
	(189.65,146.87) --
	(189.65,148.11) --
	(189.65,149.35) --
	(189.65,150.60) --
	(189.65,151.84) --
	(189.65,153.08) --
	(189.65,154.33) --
	(189.65,155.57) --
	(189.65,156.81) --
	(189.65,158.05) --
	(189.65,159.30) --
	(189.65,160.54) --
	(189.65,161.78) --
	(189.65,163.02) --
	(189.65,164.27) --
	(189.65,165.51) --
	(189.65,166.75) --
	(189.65,168.00) --
	(189.65,169.24) --
	(189.65,170.48) --
	(189.65,171.72) --
	(189.65,172.97) --
	(189.65,174.21) --
	(189.65,175.45) --
	(189.65,176.70) --
	(188.43,176.67) --
	(187.20,176.61) --
	(185.98,176.51) --
	(184.76,176.36) --
	(183.55,176.18) --
	(182.35,175.95) --
	(181.15,175.68) --
	(179.97,175.37) --
	(178.79,175.02) --
	(177.63,174.63) --
	(176.48,174.20) --
	(175.35,173.74) --
	(174.23,173.23) --
	(173.13,172.69) --
	(172.05,172.11) --
	(171.00,171.49) --
	(169.96,170.84) --
	(168.94,170.15) --
	(167.95,169.43) --
	(166.99,168.68) --
	(166.05,167.89) --
	(165.13,167.07) --
	(164.25,166.22) --
	(163.40,165.34) --
	(162.57,164.44) --
	(161.78,163.50) --
	(161.02,162.54) --
	(160.29,161.56) --
	(159.60,160.55) --
	(158.94,159.51) --
	(158.31,158.46) --
	(157.73,157.38) --
	(157.18,156.29) --
	(156.66,155.17) --
	(156.19,154.04) --
	(155.75,152.90) --
	(155.36,151.74) --
	(155.00,150.57) --
	(154.68,149.38) --
	(154.41,148.19) --
	(154.17,146.99) --
	(153.98,145.78) --
	(153.82,144.56) --
	(153.71,143.34) --
	(153.64,142.12) --
	(153.61,140.89) --
	(153.62,139.67) --
	(153.68,138.44) --
	(153.77,137.22) --
	(153.91,136.00) --
	(154.09,134.79) --
	(154.31,133.58) --
	(154.57,132.39) --
	(154.87,131.20) --
	(155.21,130.02) --
	(155.60,128.86) --
	(156.02,127.71) --
	(156.48,126.57) --
	(156.97,125.45) --
	(157.51,124.35) --
	(158.08,123.27) --
	(158.69,122.20) --
	(159.34,121.16) --
	(160.02,120.14) --
	(160.73,119.15) --
	(161.48,118.17) --
	(162.26,117.23) --
	(163.07,116.31) --
	(163.92,115.42) --
	(164.79,114.56) --
	(165.69,113.73) --
	(166.62,112.93) --
	(167.57,112.17) --
	(168.56,111.43) --
	(169.56,110.73) --
	(170.59,110.07) --
	(171.64,109.44) --
	(172.71,108.84) --
	(173.81,108.28) --
	(174.91,107.76) --
	(176.04,107.28) --
	(177.18,106.84) --
	(178.34,106.43) --
	(179.51,106.07) --
	(180.69,105.74) --
	(181.88,105.46) --
	(183.09,105.22) --
	(184.29,105.01) --
	(185.51,104.85) --
	(186.73,104.73) --
	(187.95,104.65) --
	(189.18,104.62) --
	(190.40,104.62) --
	(191.63,104.67) --
	(192.85,104.76) --
	(194.07,104.88) --
	(195.28,105.06) --
	(196.49,105.27) --
	(197.69,105.52) --
	(198.88,105.81) --
	(200.06,106.15) --
	(201.22,106.52) --
	(202.38,106.93) --
	(203.52,107.39) --
	(204.64,107.88) --
	(204.12,109.01) --
	(203.61,110.14) --
	(203.09,111.27) --
	(202.57,112.40) --
	(202.06,113.53) --
	(201.54,114.66) --
	(201.02,115.79) --
	(200.50,116.92) --
	(199.99,118.05) --
	(199.47,119.18) --
	(198.95,120.31) --
	(198.44,121.44) --
	(197.92,122.57) --
	(197.40,123.70) --
	(196.89,124.83) --
	(196.37,125.96) --
	(195.85,127.09) --
	(195.34,128.22) --
	(194.82,129.35) --
	(194.30,130.48) --
	(193.79,131.61) --
	(193.27,132.74) --
	(192.75,133.87) --
	(192.24,135.00) --
	(191.72,136.13) --
	(191.20,137.26) --
	(190.68,138.39) --
	(190.17,139.52) --
	(189.65,140.65) --
	(189.65,140.65) --
	cycle;
\definecolor{drawColor}{RGB}{0,0,0}

\node[text=drawColor,anchor=base,inner sep=0pt, outer sep=0pt, scale=  0.71] at (207.26,142.04) {43.17{\%}};

\node[text=drawColor,anchor=base,inner sep=0pt, outer sep=0pt, scale=  0.71] at (172.04,134.37) {56.83{\%}};
\end{scope}
\begin{scope}
\path[clip] (144.60,  0.00) rectangle (234.70, 90.10);
\definecolor{fillColor}{RGB}{228,26,28}

\path[fill=fillColor] (189.65, 45.05) --
	(189.39, 43.84) --
	(189.13, 42.62) --
	(188.87, 41.41) --
	(188.60, 40.19) --
	(188.34, 38.98) --
	(188.08, 37.76) --
	(187.82, 36.55) --
	(187.56, 35.33) --
	(187.30, 34.12) --
	(187.03, 32.90) --
	(186.77, 31.69) --
	(186.51, 30.47) --
	(186.25, 29.26) --
	(185.99, 28.04) --
	(185.72, 26.83) --
	(185.46, 25.61) --
	(185.20, 24.40) --
	(184.94, 23.18) --
	(184.68, 21.97) --
	(184.42, 20.75) --
	(184.15, 19.54) --
	(183.89, 18.32) --
	(183.63, 17.11) --
	(183.37, 15.89) --
	(183.11, 14.68) --
	(182.84, 13.46) --
	(182.58, 12.25) --
	(182.32, 11.03) --
	(182.06,  9.82) --
	(183.26,  9.58) --
	(184.46,  9.39) --
	(185.67,  9.23) --
	(186.89,  9.12) --
	(188.11,  9.04) --
	(189.33,  9.01) --
	(190.55,  9.02) --
	(191.77,  9.07) --
	(192.99,  9.16) --
	(194.20,  9.30) --
	(195.41,  9.47) --
	(196.61,  9.69) --
	(197.80,  9.94) --
	(198.99, 10.24) --
	(200.16, 10.58) --
	(201.32, 10.95) --
	(202.47, 11.37) --
	(203.61, 11.82) --
	(204.72, 12.31) --
	(205.82, 12.84) --
	(206.90, 13.41) --
	(207.97, 14.01) --
	(209.01, 14.65) --
	(210.03, 15.32) --
	(211.02, 16.03) --
	(211.99, 16.77) --
	(212.94, 17.54) --
	(213.86, 18.35) --
	(214.75, 19.18) --
	(215.61, 20.05) --
	(216.44, 20.94) --
	(217.24, 21.86) --
	(218.01, 22.81) --
	(218.75, 23.78) --
	(219.45, 24.78) --
	(220.12, 25.80) --
	(220.76, 26.84) --
	(221.35, 27.91) --
	(221.92, 28.99) --
	(222.44, 30.09) --
	(222.93, 31.21) --
	(223.38, 32.35) --
	(223.79, 33.50) --
	(224.16, 34.66) --
	(224.49, 35.84) --
	(224.79, 37.02) --
	(225.04, 38.22) --
	(225.25, 39.42) --
	(225.42, 40.63) --
	(225.55, 41.84) --
	(225.64, 43.06) --
	(225.68, 44.28) --
	(225.69, 45.50) --
	(225.65, 46.72) --
	(225.58, 47.94) --
	(225.46, 49.15) --
	(225.30, 50.36) --
	(225.10, 51.57) --
	(224.86, 52.77) --
	(224.58, 53.95) --
	(224.25, 55.13) --
	(223.89, 56.30) --
	(223.49, 57.45) --
	(223.05, 58.59) --
	(222.57, 59.71) --
	(222.06, 60.82) --
	(221.51, 61.91) --
	(220.92, 62.98) --
	(220.29, 64.03) --
	(219.63, 65.05) --
	(218.94, 66.06) --
	(218.21, 67.04) --
	(217.45, 67.99) --
	(216.65, 68.92) --
	(215.83, 69.82) --
	(214.98, 70.69) --
	(214.09, 71.54) --
	(213.18, 72.35) --
	(212.24, 73.13) --
	(211.28, 73.88) --
	(210.29, 74.60) --
	(209.28, 75.28) --
	(208.24, 75.93) --
	(207.19, 76.54) --
	(206.11, 77.11) --
	(205.02, 77.65) --
	(203.90, 78.16) --
	(202.77, 78.62) --
	(201.63, 79.04) --
	(200.47, 79.43) --
	(199.30, 79.78) --
	(198.12, 80.08) --
	(196.93, 80.35) --
	(195.73, 80.58) --
	(194.52, 80.76) --
	(193.31, 80.91) --
	(192.09, 81.01) --
	(190.87, 81.07) --
	(189.65, 81.09) --
	(189.65, 79.85) --
	(189.65, 78.61) --
	(189.65, 77.36) --
	(189.65, 76.12) --
	(189.65, 74.88) --
	(189.65, 73.64) --
	(189.65, 72.39) --
	(189.65, 71.15) --
	(189.65, 69.91) --
	(189.65, 68.66) --
	(189.65, 67.42) --
	(189.65, 66.18) --
	(189.65, 64.94) --
	(189.65, 63.69) --
	(189.65, 62.45) --
	(189.65, 61.21) --
	(189.65, 59.96) --
	(189.65, 58.72) --
	(189.65, 57.48) --
	(189.65, 56.24) --
	(189.65, 54.99) --
	(189.65, 53.75) --
	(189.65, 52.51) --
	(189.65, 51.27) --
	(189.65, 50.02) --
	(189.65, 48.78) --
	(189.65, 47.54) --
	(189.65, 46.29) --
	(189.65, 45.05) --
	(189.65, 45.05) --
	cycle;
\definecolor{fillColor}{RGB}{55,126,184}

\path[fill=fillColor] (189.65, 45.05) --
	(189.65, 46.29) --
	(189.65, 47.54) --
	(189.65, 48.78) --
	(189.65, 50.02) --
	(189.65, 51.27) --
	(189.65, 52.51) --
	(189.65, 53.75) --
	(189.65, 54.99) --
	(189.65, 56.24) --
	(189.65, 57.48) --
	(189.65, 58.72) --
	(189.65, 59.96) --
	(189.65, 61.21) --
	(189.65, 62.45) --
	(189.65, 63.69) --
	(189.65, 64.94) --
	(189.65, 66.18) --
	(189.65, 67.42) --
	(189.65, 68.66) --
	(189.65, 69.91) --
	(189.65, 71.15) --
	(189.65, 72.39) --
	(189.65, 73.64) --
	(189.65, 74.88) --
	(189.65, 76.12) --
	(189.65, 77.36) --
	(189.65, 78.61) --
	(189.65, 79.85) --
	(189.65, 81.09) --
	(188.42, 81.07) --
	(187.20, 81.01) --
	(185.97, 80.90) --
	(184.76, 80.76) --
	(183.54, 80.57) --
	(182.34, 80.34) --
	(181.14, 80.07) --
	(179.95, 79.76) --
	(178.77, 79.41) --
	(177.61, 79.02) --
	(176.46, 78.59) --
	(175.33, 78.12) --
	(174.21, 77.62) --
	(173.11, 77.07) --
	(172.03, 76.49) --
	(170.97, 75.87) --
	(169.93, 75.22) --
	(168.91, 74.53) --
	(167.92, 73.80) --
	(166.95, 73.05) --
	(166.01, 72.26) --
	(165.10, 71.44) --
	(164.22, 70.59) --
	(163.36, 69.71) --
	(162.54, 68.80) --
	(161.74, 67.86) --
	(160.98, 66.89) --
	(160.26, 65.91) --
	(159.56, 64.89) --
	(158.91, 63.86) --
	(158.28, 62.80) --
	(157.70, 61.72) --
	(157.15, 60.62) --
	(156.64, 59.51) --
	(156.16, 58.37) --
	(155.73, 57.23) --
	(155.33, 56.06) --
	(154.98, 54.89) --
	(154.66, 53.70) --
	(154.39, 52.50) --
	(154.16, 51.30) --
	(153.96, 50.09) --
	(153.81, 48.87) --
	(153.70, 47.65) --
	(153.64, 46.42) --
	(153.61, 45.19) --
	(153.63, 43.97) --
	(153.68, 42.74) --
	(153.78, 41.52) --
	(153.93, 40.30) --
	(154.11, 39.08) --
	(154.33, 37.88) --
	(154.60, 36.68) --
	(154.90, 35.49) --
	(155.25, 34.31) --
	(155.63, 33.14) --
	(156.06, 31.99) --
	(156.52, 30.86) --
	(157.03, 29.74) --
	(157.57, 28.63) --
	(158.14, 27.55) --
	(158.76, 26.49) --
	(159.41, 25.45) --
	(160.09, 24.43) --
	(160.81, 23.43) --
	(161.57, 22.46) --
	(162.35, 21.52) --
	(163.17, 20.60) --
	(164.02, 19.72) --
	(164.89, 18.86) --
	(165.80, 18.03) --
	(166.73, 17.23) --
	(167.70, 16.47) --
	(168.68, 15.74) --
	(169.69, 15.04) --
	(170.73, 14.38) --
	(171.78, 13.75) --
	(172.86, 13.16) --
	(173.95, 12.61) --
	(175.07, 12.09) --
	(176.20, 11.62) --
	(177.34, 11.18) --
	(178.51, 10.78) --
	(179.68, 10.42) --
	(180.86, 10.10) --
	(182.06,  9.82) --
	(182.32, 11.03) --
	(182.58, 12.25) --
	(182.84, 13.46) --
	(183.11, 14.68) --
	(183.37, 15.89) --
	(183.63, 17.11) --
	(183.89, 18.32) --
	(184.15, 19.54) --
	(184.42, 20.75) --
	(184.68, 21.97) --
	(184.94, 23.18) --
	(185.20, 24.40) --
	(185.46, 25.61) --
	(185.72, 26.83) --
	(185.99, 28.04) --
	(186.25, 29.26) --
	(186.51, 30.47) --
	(186.77, 31.69) --
	(187.03, 32.90) --
	(187.30, 34.12) --
	(187.56, 35.33) --
	(187.82, 36.55) --
	(188.08, 37.76) --
	(188.34, 38.98) --
	(188.60, 40.19) --
	(188.87, 41.41) --
	(189.13, 42.62) --
	(189.39, 43.84) --
	(189.65, 45.05) --
	(189.65, 45.05) --
	cycle;
\definecolor{drawColor}{RGB}{0,0,0}

\node[text=drawColor,anchor=base,inner sep=0pt, outer sep=0pt, scale=  0.71] at (207.57, 40.69) {53.38{\%}};

\node[text=drawColor,anchor=base,inner sep=0pt, outer sep=0pt, scale=  0.71] at (171.73, 44.51) {46.62{\%}};
\end{scope}
\begin{scope}
\path[clip] (240.20,191.21) rectangle (330.31,281.31);
\definecolor{fillColor}{RGB}{228,26,28}

\path[fill=fillColor] (285.25,236.26) --
	(285.97,235.24) --
	(286.68,234.22) --
	(287.39,233.20) --
	(288.10,232.18) --
	(288.81,231.16) --
	(289.53,230.15) --
	(290.24,229.13) --
	(290.95,228.11) --
	(291.66,227.09) --
	(292.38,226.07) --
	(293.09,225.05) --
	(293.80,224.03) --
	(294.51,223.02) --
	(295.22,222.00) --
	(295.94,220.98) --
	(296.65,219.96) --
	(297.36,218.94) --
	(298.07,217.92) --
	(298.78,216.90) --
	(299.50,215.89) --
	(300.21,214.87) --
	(300.92,213.85) --
	(301.63,212.83) --
	(302.34,211.81) --
	(303.06,210.79) --
	(303.77,209.77) --
	(304.48,208.76) --
	(305.19,207.74) --
	(305.91,206.72) --
	(306.90,207.44) --
	(307.88,208.20) --
	(308.82,208.99) --
	(309.74,209.81) --
	(310.63,210.67) --
	(311.49,211.55) --
	(312.32,212.46) --
	(313.12,213.40) --
	(313.89,214.37) --
	(314.62,215.36) --
	(315.31,216.37) --
	(315.98,217.41) --
	(316.60,218.48) --
	(317.19,219.56) --
	(317.75,220.66) --
	(318.26,221.78) --
	(318.74,222.92) --
	(319.17,224.07) --
	(319.57,225.24) --
	(319.93,226.42) --
	(320.24,227.61) --
	(320.52,228.81) --
	(320.75,230.02) --
	(320.94,231.24) --
	(321.10,232.46) --
	(321.20,233.69) --
	(321.27,234.92) --
	(321.30,236.16) --
	(321.28,237.39) --
	(321.22,238.62) --
	(321.12,239.85) --
	(320.97,241.07) --
	(320.79,242.29) --
	(320.56,243.50) --
	(320.29,244.71) --
	(319.98,245.90) --
	(319.63,247.08) --
	(319.24,248.25) --
	(318.81,249.41) --
	(318.34,250.55) --
	(317.83,251.67) --
	(317.29,252.78) --
	(316.70,253.86) --
	(316.08,254.93) --
	(315.43,255.97) --
	(314.73,256.99) --
	(314.01,257.99) --
	(313.25,258.96) --
	(312.45,259.90) --
	(311.63,260.82) --
	(310.77,261.71) --
	(309.89,262.56) --
	(308.98,263.39) --
	(308.03,264.19) --
	(307.06,264.95) --
	(306.07,265.68) --
	(305.05,266.37) --
	(304.01,267.03) --
	(302.95,267.66) --
	(301.86,268.24) --
	(300.76,268.79) --
	(299.64,269.30) --
	(298.50,269.78) --
	(297.34,270.21) --
	(296.18,270.60) --
	(294.99,270.96) --
	(293.80,271.27) --
	(292.60,271.54) --
	(291.39,271.77) --
	(290.17,271.96) --
	(288.95,272.11) --
	(287.72,272.21) --
	(286.49,272.28) --
	(285.25,272.30) --
	(285.25,271.06) --
	(285.25,269.81) --
	(285.25,268.57) --
	(285.25,267.33) --
	(285.25,266.08) --
	(285.25,264.84) --
	(285.25,263.60) --
	(285.25,262.36) --
	(285.25,261.11) --
	(285.25,259.87) --
	(285.25,258.63) --
	(285.25,257.38) --
	(285.25,256.14) --
	(285.25,254.90) --
	(285.25,253.66) --
	(285.25,252.41) --
	(285.25,251.17) --
	(285.25,249.93) --
	(285.25,248.69) --
	(285.25,247.44) --
	(285.25,246.20) --
	(285.25,244.96) --
	(285.25,243.71) --
	(285.25,242.47) --
	(285.25,241.23) --
	(285.25,239.99) --
	(285.25,238.74) --
	(285.25,237.50) --
	(285.25,236.26) --
	(285.25,236.26) --
	cycle;
\definecolor{fillColor}{RGB}{55,126,184}

\path[fill=fillColor] (285.25,236.26) --
	(285.25,237.50) --
	(285.25,238.74) --
	(285.25,239.99) --
	(285.25,241.23) --
	(285.25,242.47) --
	(285.25,243.71) --
	(285.25,244.96) --
	(285.25,246.20) --
	(285.25,247.44) --
	(285.25,248.69) --
	(285.25,249.93) --
	(285.25,251.17) --
	(285.25,252.41) --
	(285.25,253.66) --
	(285.25,254.90) --
	(285.25,256.14) --
	(285.25,257.38) --
	(285.25,258.63) --
	(285.25,259.87) --
	(285.25,261.11) --
	(285.25,262.36) --
	(285.25,263.60) --
	(285.25,264.84) --
	(285.25,266.08) --
	(285.25,267.33) --
	(285.25,268.57) --
	(285.25,269.81) --
	(285.25,271.06) --
	(285.25,272.30) --
	(284.03,272.28) --
	(282.80,272.21) --
	(281.57,272.11) --
	(280.35,271.96) --
	(279.14,271.78) --
	(277.93,271.55) --
	(276.73,271.28) --
	(275.54,270.96) --
	(274.36,270.61) --
	(273.20,270.22) --
	(272.05,269.79) --
	(270.91,269.32) --
	(269.79,268.81) --
	(268.69,268.27) --
	(267.61,267.68) --
	(266.55,267.06) --
	(265.51,266.41) --
	(264.49,265.72) --
	(263.50,264.99) --
	(262.53,264.23) --
	(261.59,263.44) --
	(260.68,262.62) --
	(259.79,261.77) --
	(258.94,260.88) --
	(258.11,259.97) --
	(257.32,259.03) --
	(256.56,258.07) --
	(255.83,257.08) --
	(255.14,256.06) --
	(254.48,255.02) --
	(253.86,253.96) --
	(253.28,252.88) --
	(252.73,251.78) --
	(252.22,250.66) --
	(251.74,249.53) --
	(251.31,248.38) --
	(250.92,247.21) --
	(250.56,246.03) --
	(250.25,244.85) --
	(249.98,243.65) --
	(249.75,242.44) --
	(249.56,241.23) --
	(249.41,240.01) --
	(249.30,238.78) --
	(249.24,237.55) --
	(249.21,236.32) --
	(249.23,235.10) --
	(249.29,233.87) --
	(249.39,232.64) --
	(249.54,231.42) --
	(249.72,230.21) --
	(249.95,229.00) --
	(250.22,227.80) --
	(250.53,226.61) --
	(250.88,225.43) --
	(251.27,224.27) --
	(251.69,223.11) --
	(252.16,221.98) --
	(252.67,220.86) --
	(253.21,219.75) --
	(253.79,218.67) --
	(254.41,217.61) --
	(255.07,216.57) --
	(255.76,215.55) --
	(256.48,214.56) --
	(257.24,213.59) --
	(258.02,212.65) --
	(258.85,211.73) --
	(259.70,210.84) --
	(260.58,209.99) --
	(261.49,209.16) --
	(262.43,208.37) --
	(263.39,207.60) --
	(264.38,206.88) --
	(265.39,206.18) --
	(266.43,205.52) --
	(267.49,204.90) --
	(268.57,204.31) --
	(269.67,203.76) --
	(270.79,203.25) --
	(271.92,202.77) --
	(273.07,202.34) --
	(274.23,201.94) --
	(275.41,201.59) --
	(276.60,201.27) --
	(277.80,201.00) --
	(279.00,200.76) --
	(280.22,200.57) --
	(281.44,200.42) --
	(282.66,200.31) --
	(283.89,200.24) --
	(285.12,200.22) --
	(286.35,200.23) --
	(287.58,200.29) --
	(288.80,200.39) --
	(290.02,200.53) --
	(291.24,200.72) --
	(292.45,200.94) --
	(293.65,201.21) --
	(294.84,201.51) --
	(296.01,201.86) --
	(297.18,202.25) --
	(298.33,202.67) --
	(299.47,203.14) --
	(300.59,203.64) --
	(301.70,204.19) --
	(302.78,204.76) --
	(303.84,205.38) --
	(304.89,206.03) --
	(305.91,206.72) --
	(305.19,207.74) --
	(304.48,208.76) --
	(303.77,209.77) --
	(303.06,210.79) --
	(302.34,211.81) --
	(301.63,212.83) --
	(300.92,213.85) --
	(300.21,214.87) --
	(299.50,215.89) --
	(298.78,216.90) --
	(298.07,217.92) --
	(297.36,218.94) --
	(296.65,219.96) --
	(295.94,220.98) --
	(295.22,222.00) --
	(294.51,223.02) --
	(293.80,224.03) --
	(293.09,225.05) --
	(292.38,226.07) --
	(291.66,227.09) --
	(290.95,228.11) --
	(290.24,229.13) --
	(289.53,230.15) --
	(288.81,231.16) --
	(288.10,232.18) --
	(287.39,233.20) --
	(286.68,234.22) --
	(285.97,235.24) --
	(285.25,236.26) --
	(285.25,236.26) --
	cycle;
\definecolor{drawColor}{RGB}{0,0,0}

\node[text=drawColor,anchor=base,inner sep=0pt, outer sep=0pt, scale=  0.71] at (302.44,239.22) {40.29{\%}};

\node[text=drawColor,anchor=base,inner sep=0pt, outer sep=0pt, scale=  0.71] at (268.07,228.39) {59.71{\%}};
\end{scope}
\begin{scope}
\path[clip] (240.20, 95.60) rectangle (330.31,185.71);
\definecolor{fillColor}{RGB}{228,26,28}

\path[fill=fillColor] (285.25,140.65) --
	(286.24,139.89) --
	(287.22,139.13) --
	(288.20,138.37) --
	(289.18,137.60) --
	(290.16,136.84) --
	(291.14,136.08) --
	(292.12,135.31) --
	(293.10,134.55) --
	(294.08,133.79) --
	(295.06,133.02) --
	(296.04,132.26) --
	(297.03,131.50) --
	(298.01,130.73) --
	(298.99,129.97) --
	(299.97,129.21) --
	(300.95,128.45) --
	(301.93,127.68) --
	(302.91,126.92) --
	(303.89,126.16) --
	(304.87,125.39) --
	(305.85,124.63) --
	(306.84,123.87) --
	(307.82,123.10) --
	(308.80,122.34) --
	(309.78,121.58) --
	(310.76,120.82) --
	(311.74,120.05) --
	(312.72,119.29) --
	(313.70,118.53) --
	(314.45,119.52) --
	(315.15,120.53) --
	(315.83,121.57) --
	(316.46,122.63) --
	(317.06,123.71) --
	(317.63,124.81) --
	(318.15,125.93) --
	(318.64,127.07) --
	(319.08,128.23) --
	(319.49,129.40) --
	(319.86,130.58) --
	(320.18,131.77) --
	(320.47,132.98) --
	(320.71,134.19) --
	(320.91,135.41) --
	(321.07,136.64) --
	(321.19,137.87) --
	(321.26,139.10) --
	(321.29,140.34) --
	(321.28,141.58) --
	(321.23,142.82) --
	(321.13,144.05) --
	(321.00,145.28) --
	(320.82,146.50) --
	(320.60,147.72) --
	(320.33,148.93) --
	(320.03,150.13) --
	(319.68,151.32) --
	(319.30,152.49) --
	(318.87,153.65) --
	(318.40,154.80) --
	(317.90,155.93) --
	(317.35,157.04) --
	(316.77,158.13) --
	(316.15,159.21) --
	(315.50,160.26) --
	(314.81,161.28) --
	(314.08,162.29) --
	(313.32,163.26) --
	(312.53,164.21) --
	(311.71,165.13) --
	(310.85,166.03) --
	(309.96,166.89) --
	(309.05,167.72) --
	(308.10,168.53) --
	(307.13,169.29) --
	(306.14,170.03) --
	(305.12,170.73) --
	(304.07,171.39) --
	(303.01,172.02) --
	(301.92,172.61) --
	(300.81,173.16) --
	(299.69,173.68) --
	(298.55,174.16) --
	(297.39,174.59) --
	(296.22,174.99) --
	(295.03,175.34) --
	(293.83,175.66) --
	(292.63,175.93) --
	(291.41,176.17) --
	(290.19,176.36) --
	(288.96,176.50) --
	(287.73,176.61) --
	(286.49,176.67) --
	(285.25,176.70) --
	(285.25,175.45) --
	(285.25,174.21) --
	(285.25,172.97) --
	(285.25,171.72) --
	(285.25,170.48) --
	(285.25,169.24) --
	(285.25,168.00) --
	(285.25,166.75) --
	(285.25,165.51) --
	(285.25,164.27) --
	(285.25,163.02) --
	(285.25,161.78) --
	(285.25,160.54) --
	(285.25,159.30) --
	(285.25,158.05) --
	(285.25,156.81) --
	(285.25,155.57) --
	(285.25,154.33) --
	(285.25,153.08) --
	(285.25,151.84) --
	(285.25,150.60) --
	(285.25,149.35) --
	(285.25,148.11) --
	(285.25,146.87) --
	(285.25,145.63) --
	(285.25,144.38) --
	(285.25,143.14) --
	(285.25,141.90) --
	(285.25,140.65) --
	(285.25,140.65) --
	cycle;
\definecolor{fillColor}{RGB}{55,126,184}

\path[fill=fillColor] (285.25,140.65) --
	(285.25,141.90) --
	(285.25,143.14) --
	(285.25,144.38) --
	(285.25,145.63) --
	(285.25,146.87) --
	(285.25,148.11) --
	(285.25,149.35) --
	(285.25,150.60) --
	(285.25,151.84) --
	(285.25,153.08) --
	(285.25,154.33) --
	(285.25,155.57) --
	(285.25,156.81) --
	(285.25,158.05) --
	(285.25,159.30) --
	(285.25,160.54) --
	(285.25,161.78) --
	(285.25,163.02) --
	(285.25,164.27) --
	(285.25,165.51) --
	(285.25,166.75) --
	(285.25,168.00) --
	(285.25,169.24) --
	(285.25,170.48) --
	(285.25,171.72) --
	(285.25,172.97) --
	(285.25,174.21) --
	(285.25,175.45) --
	(285.25,176.70) --
	(284.03,176.67) --
	(282.80,176.61) --
	(281.58,176.51) --
	(280.36,176.36) --
	(279.15,176.17) --
	(277.94,175.95) --
	(276.75,175.68) --
	(275.56,175.37) --
	(274.38,175.02) --
	(273.22,174.63) --
	(272.07,174.20) --
	(270.94,173.73) --
	(269.82,173.22) --
	(268.72,172.68) --
	(267.64,172.10) --
	(266.58,171.48) --
	(265.54,170.83) --
	(264.52,170.14) --
	(263.53,169.41) --
	(262.57,168.66) --
	(261.63,167.87) --
	(260.71,167.05) --
	(259.83,166.20) --
	(258.97,165.32) --
	(258.15,164.41) --
	(257.36,163.47) --
	(256.60,162.51) --
	(255.87,161.52) --
	(255.18,160.51) --
	(254.52,159.48) --
	(253.90,158.42) --
	(253.31,157.34) --
	(252.76,156.24) --
	(252.25,155.13) --
	(251.77,154.00) --
	(251.34,152.85) --
	(250.94,151.69) --
	(250.59,150.51) --
	(250.27,149.33) --
	(250.00,148.13) --
	(249.76,146.93) --
	(249.57,145.72) --
	(249.42,144.50) --
	(249.31,143.28) --
	(249.24,142.05) --
	(249.21,140.83) --
	(249.23,139.60) --
	(249.29,138.37) --
	(249.38,137.15) --
	(249.52,135.93) --
	(249.71,134.72) --
	(249.93,133.51) --
	(250.19,132.31) --
	(250.50,131.12) --
	(250.84,129.95) --
	(251.22,128.78) --
	(251.65,127.63) --
	(252.11,126.49) --
	(252.61,125.37) --
	(253.15,124.27) --
	(253.73,123.19) --
	(254.34,122.12) --
	(254.99,121.08) --
	(255.67,120.06) --
	(256.39,119.07) --
	(257.14,118.10) --
	(257.93,117.16) --
	(258.74,116.24) --
	(259.59,115.35) --
	(260.46,114.49) --
	(261.37,113.66) --
	(262.30,112.87) --
	(263.26,112.10) --
	(264.25,111.37) --
	(265.25,110.67) --
	(266.29,110.01) --
	(267.34,109.38) --
	(268.42,108.79) --
	(269.51,108.23) --
	(270.62,107.72) --
	(271.75,107.24) --
	(272.90,106.80) --
	(274.06,106.40) --
	(275.23,106.04) --
	(276.41,105.71) --
	(277.61,105.43) --
	(278.81,105.19) --
	(280.02,104.99) --
	(281.24,104.84) --
	(282.46,104.72) --
	(283.69,104.65) --
	(284.91,104.61) --
	(286.14,104.62) --
	(287.37,104.68) --
	(288.59,104.77) --
	(289.81,104.90) --
	(291.02,105.08) --
	(292.23,105.29) --
	(293.43,105.55) --
	(294.62,105.85) --
	(295.80,106.19) --
	(296.97,106.57) --
	(298.12,106.99) --
	(299.26,107.44) --
	(300.38,107.94) --
	(301.48,108.47) --
	(302.57,109.05) --
	(303.64,109.65) --
	(304.68,110.30) --
	(305.70,110.98) --
	(306.70,111.69) --
	(307.68,112.44) --
	(308.62,113.22) --
	(309.54,114.03) --
	(310.44,114.87) --
	(311.30,115.74) --
	(312.13,116.64) --
	(312.93,117.57) --
	(313.70,118.53) --
	(312.72,119.29) --
	(311.74,120.05) --
	(310.76,120.82) --
	(309.78,121.58) --
	(308.80,122.34) --
	(307.82,123.10) --
	(306.84,123.87) --
	(305.85,124.63) --
	(304.87,125.39) --
	(303.89,126.16) --
	(302.91,126.92) --
	(301.93,127.68) --
	(300.95,128.45) --
	(299.97,129.21) --
	(298.99,129.97) --
	(298.01,130.73) --
	(297.03,131.50) --
	(296.04,132.26) --
	(295.06,133.02) --
	(294.08,133.79) --
	(293.10,134.55) --
	(292.12,135.31) --
	(291.14,136.08) --
	(290.16,136.84) --
	(289.18,137.60) --
	(288.20,138.37) --
	(287.22,139.13) --
	(286.24,139.89) --
	(285.25,140.65) --
	(285.25,140.65) --
	cycle;
\definecolor{drawColor}{RGB}{0,0,0}

\node[text=drawColor,anchor=base,inner sep=0pt, outer sep=0pt, scale=  0.71] at (301.44,146.12) {35.52{\%}};

\node[text=drawColor,anchor=base,inner sep=0pt, outer sep=0pt, scale=  0.71] at (269.07,130.29) {64.48{\%}};
\end{scope}
\begin{scope}
\path[clip] (240.20,  0.00) rectangle (330.31, 90.10);
\definecolor{fillColor}{RGB}{228,26,28}

\path[fill=fillColor] (285.25, 45.05) --
	(284.87, 43.87) --
	(284.48, 42.69) --
	(284.10, 41.51) --
	(283.71, 40.32) --
	(283.33, 39.14) --
	(282.95, 37.96) --
	(282.56, 36.78) --
	(282.18, 35.60) --
	(281.79, 34.42) --
	(281.41, 33.23) --
	(281.02, 32.05) --
	(280.64, 30.87) --
	(280.25, 29.69) --
	(279.87, 28.51) --
	(279.48, 27.33) --
	(279.10, 26.14) --
	(278.71, 24.96) --
	(278.33, 23.78) --
	(277.94, 22.60) --
	(277.56, 21.42) --
	(277.17, 20.24) --
	(276.79, 19.05) --
	(276.40, 17.87) --
	(276.02, 16.69) --
	(275.63, 15.51) --
	(275.25, 14.33) --
	(274.86, 13.15) --
	(274.48, 11.96) --
	(274.09, 10.78) --
	(275.26, 10.42) --
	(276.44, 10.10) --
	(277.63,  9.83) --
	(278.83,  9.59) --
	(280.03,  9.39) --
	(281.24,  9.23) --
	(282.46,  9.12) --
	(283.68,  9.04) --
	(284.90,  9.01) --
	(286.12,  9.02) --
	(287.34,  9.07) --
	(288.56,  9.16) --
	(289.77,  9.29) --
	(290.98,  9.47) --
	(292.18,  9.68) --
	(293.38,  9.94) --
	(294.56, 10.23) --
	(295.74, 10.57) --
	(296.90, 10.94) --
	(298.05, 11.36) --
	(299.18, 11.81) --
	(300.30, 12.30) --
	(301.40, 12.83) --
	(302.48, 13.40) --
	(303.55, 14.00) --
	(304.59, 14.64) --
	(305.61, 15.31) --
	(306.60, 16.01) --
	(307.58, 16.75) --
	(308.52, 17.53) --
	(309.44, 18.33) --
	(310.33, 19.17) --
	(311.19, 20.03) --
	(312.03, 20.92) --
	(312.83, 21.84) --
	(313.60, 22.79) --
	(314.34, 23.76) --
	(315.04, 24.76) --
	(315.71, 25.78) --
	(316.35, 26.83) --
	(316.95, 27.89) --
	(317.51, 28.97) --
	(318.04, 30.08) --
	(318.53, 31.20) --
	(318.98, 32.33) --
	(319.39, 33.48) --
	(319.76, 34.64) --
	(320.09, 35.82) --
	(320.39, 37.00) --
	(320.64, 38.20) --
	(320.85, 39.40) --
	(321.02, 40.61) --
	(321.15, 41.83) --
	(321.24, 43.04) --
	(321.29, 44.26) --
	(321.29, 45.48) --
	(321.26, 46.71) --
	(321.18, 47.92) --
	(321.06, 49.14) --
	(320.90, 50.35) --
	(320.70, 51.56) --
	(320.46, 52.75) --
	(320.18, 53.94) --
	(319.86, 55.12) --
	(319.50, 56.29) --
	(319.10, 57.44) --
	(318.66, 58.58) --
	(318.18, 59.70) --
	(317.67, 60.81) --
	(317.11, 61.90) --
	(316.53, 62.97) --
	(315.90, 64.02) --
	(315.24, 65.05) --
	(314.55, 66.05) --
	(313.82, 67.03) --
	(313.06, 67.99) --
	(312.26, 68.91) --
	(311.44, 69.82) --
	(310.59, 70.69) --
	(309.70, 71.53) --
	(308.79, 72.35) --
	(307.85, 73.13) --
	(306.89, 73.88) --
	(305.90, 74.59) --
	(304.89, 75.28) --
	(303.85, 75.92) --
	(302.79, 76.54) --
	(301.72, 77.11) --
	(300.62, 77.65) --
	(299.51, 78.15) --
	(298.38, 78.62) --
	(297.23, 79.04) --
	(296.08, 79.43) --
	(294.91, 79.78) --
	(293.72, 80.08) --
	(292.53, 80.35) --
	(291.33, 80.58) --
	(290.12, 80.76) --
	(288.91, 80.91) --
	(287.69, 81.01) --
	(286.48, 81.07) --
	(285.25, 81.09) --
	(285.25, 79.85) --
	(285.25, 78.61) --
	(285.25, 77.36) --
	(285.25, 76.12) --
	(285.25, 74.88) --
	(285.25, 73.64) --
	(285.25, 72.39) --
	(285.25, 71.15) --
	(285.25, 69.91) --
	(285.25, 68.66) --
	(285.25, 67.42) --
	(285.25, 66.18) --
	(285.25, 64.94) --
	(285.25, 63.69) --
	(285.25, 62.45) --
	(285.25, 61.21) --
	(285.25, 59.96) --
	(285.25, 58.72) --
	(285.25, 57.48) --
	(285.25, 56.24) --
	(285.25, 54.99) --
	(285.25, 53.75) --
	(285.25, 52.51) --
	(285.25, 51.27) --
	(285.25, 50.02) --
	(285.25, 48.78) --
	(285.25, 47.54) --
	(285.25, 46.29) --
	(285.25, 45.05) --
	(285.25, 45.05) --
	cycle;
\definecolor{fillColor}{RGB}{55,126,184}

\path[fill=fillColor] (285.25, 45.05) --
	(285.25, 46.29) --
	(285.25, 47.54) --
	(285.25, 48.78) --
	(285.25, 50.02) --
	(285.25, 51.27) --
	(285.25, 52.51) --
	(285.25, 53.75) --
	(285.25, 54.99) --
	(285.25, 56.24) --
	(285.25, 57.48) --
	(285.25, 58.72) --
	(285.25, 59.96) --
	(285.25, 61.21) --
	(285.25, 62.45) --
	(285.25, 63.69) --
	(285.25, 64.94) --
	(285.25, 66.18) --
	(285.25, 67.42) --
	(285.25, 68.66) --
	(285.25, 69.91) --
	(285.25, 71.15) --
	(285.25, 72.39) --
	(285.25, 73.64) --
	(285.25, 74.88) --
	(285.25, 76.12) --
	(285.25, 77.36) --
	(285.25, 78.61) --
	(285.25, 79.85) --
	(285.25, 81.09) --
	(284.03, 81.07) --
	(282.80, 81.01) --
	(281.58, 80.90) --
	(280.36, 80.76) --
	(279.15, 80.57) --
	(277.94, 80.34) --
	(276.74, 80.07) --
	(275.56, 79.76) --
	(274.38, 79.41) --
	(273.22, 79.02) --
	(272.07, 78.59) --
	(270.93, 78.12) --
	(269.81, 77.62) --
	(268.71, 77.07) --
	(267.63, 76.49) --
	(266.57, 75.87) --
	(265.53, 75.22) --
	(264.52, 74.53) --
	(263.53, 73.81) --
	(262.56, 73.05) --
	(261.62, 72.26) --
	(260.71, 71.44) --
	(259.82, 70.59) --
	(258.97, 69.71) --
	(258.14, 68.80) --
	(257.35, 67.86) --
	(256.59, 66.90) --
	(255.86, 65.91) --
	(255.17, 64.90) --
	(254.51, 63.86) --
	(253.89, 62.80) --
	(253.30, 61.73) --
	(252.75, 60.63) --
	(252.24, 59.51) --
	(251.77, 58.38) --
	(251.33, 57.23) --
	(250.94, 56.07) --
	(250.58, 54.89) --
	(250.27, 53.71) --
	(249.99, 52.51) --
	(249.76, 51.31) --
	(249.57, 50.09) --
	(249.42, 48.88) --
	(249.31, 47.65) --
	(249.24, 46.43) --
	(249.21, 45.20) --
	(249.23, 43.97) --
	(249.29, 42.75) --
	(249.39, 41.52) --
	(249.53, 40.31) --
	(249.71, 39.09) --
	(249.93, 37.88) --
	(250.20, 36.69) --
	(250.50, 35.50) --
	(250.85, 34.32) --
	(251.23, 33.15) --
	(251.66, 32.00) --
	(252.12, 30.87) --
	(252.62, 29.75) --
	(253.16, 28.64) --
	(253.74, 27.56) --
	(254.36, 26.50) --
	(255.01, 25.46) --
	(255.69, 24.44) --
	(256.41, 23.44) --
	(257.16, 22.47) --
	(257.95, 21.53) --
	(258.76, 20.61) --
	(259.61, 19.73) --
	(260.49, 18.87) --
	(261.39, 18.04) --
	(262.33, 17.24) --
	(263.29, 16.48) --
	(264.27, 15.75) --
	(265.28, 15.05) --
	(266.32, 14.39) --
	(267.37, 13.76) --
	(268.45, 13.17) --
	(269.54, 12.61) --
	(270.66, 12.10) --
	(271.79, 11.62) --
	(272.93, 11.18) --
	(274.09, 10.78) --
	(274.48, 11.96) --
	(274.86, 13.15) --
	(275.25, 14.33) --
	(275.63, 15.51) --
	(276.02, 16.69) --
	(276.40, 17.87) --
	(276.79, 19.05) --
	(277.17, 20.24) --
	(277.56, 21.42) --
	(277.94, 22.60) --
	(278.33, 23.78) --
	(278.71, 24.96) --
	(279.10, 26.14) --
	(279.48, 27.33) --
	(279.87, 28.51) --
	(280.25, 29.69) --
	(280.64, 30.87) --
	(281.02, 32.05) --
	(281.41, 33.23) --
	(281.79, 34.42) --
	(282.18, 35.60) --
	(282.56, 36.78) --
	(282.95, 37.96) --
	(283.33, 39.14) --
	(283.71, 40.32) --
	(284.10, 41.51) --
	(284.48, 42.69) --
	(284.87, 43.87) --
	(285.25, 45.05) --
	(285.25, 45.05) --
	cycle;
\definecolor{drawColor}{RGB}{0,0,0}

\node[text=drawColor,anchor=base,inner sep=0pt, outer sep=0pt, scale=  0.71] at (303.05, 39.78) {55.01{\%}};

\node[text=drawColor,anchor=base,inner sep=0pt, outer sep=0pt, scale=  0.71] at (267.46, 45.43) {44.99{\%}};
\end{scope}
\begin{scope}
\path[clip] (144.60,281.31) rectangle (234.70,289.08);
\definecolor{drawColor}{RGB}{0,0,0}

\node[text=drawColor,anchor=base,inner sep=0pt, outer sep=0pt, scale=  0.70] at (189.65,283) {Migrantes};
\end{scope}
\begin{scope}
\path[clip] (240.20,281.31) rectangle (330.31,289.08);
\definecolor{drawColor}{RGB}{0,0,0}

\node[text=drawColor,anchor=base,inner sep=0pt, outer sep=0pt, scale=  0.70] at (285.25,283) {Nativos};
\end{scope}
\begin{scope}
\path[clip] (330.31,191.21) rectangle (370.68,281.31);
\definecolor{drawColor}{RGB}{0,0,0}

\node[text=drawColor,anchor=base,inner sep=0pt, outer sep=0pt, scale=  0.70] at (350.49,233.23) {\textbf{Región 1}};
\end{scope}
\begin{scope}
\path[clip] (330.31, 95.60) rectangle (370.68,185.71);
\definecolor{drawColor}{RGB}{0,0,0}

\node[text=drawColor,anchor=base,inner sep=0pt, outer sep=0pt, scale=  0.70] at (350.49,137.62) {\textbf{Región 2}};
\end{scope}
\begin{scope}
\path[clip] (330.31,  0.00) rectangle (370.68, 90.10);
\definecolor{drawColor}{RGB}{0,0,0}

\node[text=drawColor,anchor=base,inner sep=0pt, outer sep=0pt, scale=  0.70] at (350.49, 42.02) {\textbf{Región 3}};
\end{scope}
\begin{scope}
\path[clip] (  0.00,  0.00) rectangle (433.62,289.08);
\definecolor{fillColor}{RGB}{228,26,28}

\path[fill=fillColor] ( 63.65,133.76) rectangle ( 76.68,146.79);
\end{scope}
\begin{scope}
\path[clip] (  0.00,  0.00) rectangle (433.62,289.08);
\definecolor{fillColor}{RGB}{55,126,184}

\path[fill=fillColor] ( 63.65,119.30) rectangle ( 76.68,132.34);
\end{scope}
\begin{scope}
\path[clip] (  0.00,  0.00) rectangle (433.62,289.08);
\definecolor{drawColor}{RGB}{0,0,0}

\node[text=drawColor,anchor=base west,inner sep=0pt, outer sep=0pt, scale=  0.60] at ( 82.90,137.24) {No propietario};
\end{scope}
\begin{scope}
\path[clip] (  0.00,  0.00) rectangle (433.62,289.08);
\definecolor{drawColor}{RGB}{0,0,0}

\node[text=drawColor,anchor=base west,inner sep=0pt, outer sep=0pt, scale=  0.60] at ( 82.90,122.79) {Propietario};
\end{scope}
\end{tikzpicture}
 
\end{center}
\begin{flushleft}
\begin{scriptsize}
Fuente: Elaboración propia en base a EPH.\\
Nota: Los migrantes están definidos como personas que vivían hace cinco años en otra provincia. Los nativos están definidos como personas que nacieron y viven en la misma provincia. Las estimaciones corresponden al período desde el segundo trimestre de 2016 hasta el cuarto trimestre de 2019.\end{scriptsize}
\end{flushleft}
\end{figure}
 


%\begin{figure}[ht!]
%\begin{center}
%\caption{\\Percepción de subsidios de los nativos y migrantes por regiones}
%\label{figure:subsidio_mig}
%% Created by tikzDevice version 0.12.3.1 on 2021-07-01 17:52:09
% !TEX encoding = UTF-8 Unicode
\begin{tikzpicture}[x=1pt,y=1pt]
\definecolor{fillColor}{RGB}{255,255,255}
\path[use as bounding box,fill=fillColor,fill opacity=0.00] (0,0) rectangle (433.62,289.08);
\begin{scope}
\path[clip] (  0.00,  0.00) rectangle (433.62,289.08);
\definecolor{drawColor}{RGB}{255,255,255}
\definecolor{fillColor}{RGB}{255,255,255}

\path[draw=drawColor,line width= 0.6pt,line join=round,line cap=round,fill=fillColor] (  0.00,  0.00) rectangle (433.62,289.08);
\end{scope}
\begin{scope}
\path[clip] ( 43.44, 69.96) rectangle (168.00,267.01);
\definecolor{drawColor}{RGB}{255,255,255}

\path[draw=drawColor,line width= 0.3pt,line join=round] ( 43.44,101.31) --
	(168.00,101.31);

\path[draw=drawColor,line width= 0.3pt,line join=round] ( 43.44,146.09) --
	(168.00,146.09);

\path[draw=drawColor,line width= 0.3pt,line join=round] ( 43.44,190.88) --
	(168.00,190.88);

\path[draw=drawColor,line width= 0.3pt,line join=round] ( 43.44,235.66) --
	(168.00,235.66);

\path[draw=drawColor,line width= 0.6pt,line join=round] ( 43.44, 78.92) --
	(168.00, 78.92);

\path[draw=drawColor,line width= 0.6pt,line join=round] ( 43.44,123.70) --
	(168.00,123.70);

\path[draw=drawColor,line width= 0.6pt,line join=round] ( 43.44,168.48) --
	(168.00,168.48);

\path[draw=drawColor,line width= 0.6pt,line join=round] ( 43.44,213.27) --
	(168.00,213.27);

\path[draw=drawColor,line width= 0.6pt,line join=round] ( 43.44,258.05) --
	(168.00,258.05);

\path[draw=drawColor,line width= 0.6pt,line join=round] ( 77.41, 69.96) --
	( 77.41,267.01);

\path[draw=drawColor,line width= 0.6pt,line join=round] (134.03, 69.96) --
	(134.03,267.01);
\definecolor{fillColor}{RGB}{228,26,28}

\path[fill=fillColor] ( 54.77, 93.64) rectangle (100.06,258.05);
\definecolor{fillColor}{RGB}{55,126,184}

\path[fill=fillColor] ( 54.77, 78.92) rectangle (100.06, 93.64);
\definecolor{fillColor}{RGB}{228,26,28}

\path[fill=fillColor] (111.38,101.64) rectangle (156.68,258.05);
\definecolor{fillColor}{RGB}{55,126,184}

\path[fill=fillColor] (111.38, 78.92) rectangle (156.68,101.64);
\definecolor{drawColor}{RGB}{0,0,0}

\node[text=drawColor,anchor=base,inner sep=0pt, outer sep=0pt, scale=  0.85] at ( 77.41,156.47) {91.78{\%}};

\node[text=drawColor,anchor=base,inner sep=0pt, outer sep=0pt, scale=  0.85] at ( 77.41, 81.87) {8.22{\%}};

\node[text=drawColor,anchor=base,inner sep=0pt, outer sep=0pt, scale=  0.85] at (134.03,161.26) {87.32{\%}};

\node[text=drawColor,anchor=base,inner sep=0pt, outer sep=0pt, scale=  0.85] at (134.03, 85.07) {12.68{\%}};
\end{scope}
\begin{scope}
\path[clip] (173.50, 69.96) rectangle (298.06,267.01);
\definecolor{drawColor}{RGB}{255,255,255}

\path[draw=drawColor,line width= 0.3pt,line join=round] (173.50,101.31) --
	(298.06,101.31);

\path[draw=drawColor,line width= 0.3pt,line join=round] (173.50,146.09) --
	(298.06,146.09);

\path[draw=drawColor,line width= 0.3pt,line join=round] (173.50,190.88) --
	(298.06,190.88);

\path[draw=drawColor,line width= 0.3pt,line join=round] (173.50,235.66) --
	(298.06,235.66);

\path[draw=drawColor,line width= 0.6pt,line join=round] (173.50, 78.92) --
	(298.06, 78.92);

\path[draw=drawColor,line width= 0.6pt,line join=round] (173.50,123.70) --
	(298.06,123.70);

\path[draw=drawColor,line width= 0.6pt,line join=round] (173.50,168.48) --
	(298.06,168.48);

\path[draw=drawColor,line width= 0.6pt,line join=round] (173.50,213.27) --
	(298.06,213.27);

\path[draw=drawColor,line width= 0.6pt,line join=round] (173.50,258.05) --
	(298.06,258.05);

\path[draw=drawColor,line width= 0.6pt,line join=round] (207.47, 69.96) --
	(207.47,267.01);

\path[draw=drawColor,line width= 0.6pt,line join=round] (264.09, 69.96) --
	(264.09,267.01);
\definecolor{fillColor}{RGB}{228,26,28}

\path[fill=fillColor] (184.83,105.04) rectangle (230.12,258.05);
\definecolor{fillColor}{RGB}{55,126,184}

\path[fill=fillColor] (184.83, 78.92) rectangle (230.12,105.04);
\definecolor{fillColor}{RGB}{228,26,28}

\path[fill=fillColor] (241.44,119.20) rectangle (286.74,258.05);
\definecolor{fillColor}{RGB}{55,126,184}

\path[fill=fillColor] (241.44, 78.92) rectangle (286.74,119.20);
\definecolor{drawColor}{RGB}{0,0,0}

\node[text=drawColor,anchor=base,inner sep=0pt, outer sep=0pt, scale=  0.85] at (207.47,163.30) {85.42{\%}};

\node[text=drawColor,anchor=base,inner sep=0pt, outer sep=0pt, scale=  0.85] at (207.47, 86.43) {14.58{\%}};

\node[text=drawColor,anchor=base,inner sep=0pt, outer sep=0pt, scale=  0.85] at (264.09,171.80) {77.51{\%}};

\node[text=drawColor,anchor=base,inner sep=0pt, outer sep=0pt, scale=  0.85] at (264.09, 92.09) {22.49{\%}};
\end{scope}
\begin{scope}
\path[clip] (303.56, 69.96) rectangle (428.12,267.01);
\definecolor{drawColor}{RGB}{255,255,255}

\path[draw=drawColor,line width= 0.3pt,line join=round] (303.56,101.31) --
	(428.12,101.31);

\path[draw=drawColor,line width= 0.3pt,line join=round] (303.56,146.09) --
	(428.12,146.09);

\path[draw=drawColor,line width= 0.3pt,line join=round] (303.56,190.88) --
	(428.12,190.88);

\path[draw=drawColor,line width= 0.3pt,line join=round] (303.56,235.66) --
	(428.12,235.66);

\path[draw=drawColor,line width= 0.6pt,line join=round] (303.56, 78.92) --
	(428.12, 78.92);

\path[draw=drawColor,line width= 0.6pt,line join=round] (303.56,123.70) --
	(428.12,123.70);

\path[draw=drawColor,line width= 0.6pt,line join=round] (303.56,168.48) --
	(428.12,168.48);

\path[draw=drawColor,line width= 0.6pt,line join=round] (303.56,213.27) --
	(428.12,213.27);

\path[draw=drawColor,line width= 0.6pt,line join=round] (303.56,258.05) --
	(428.12,258.05);

\path[draw=drawColor,line width= 0.6pt,line join=round] (337.53, 69.96) --
	(337.53,267.01);

\path[draw=drawColor,line width= 0.6pt,line join=round] (394.15, 69.96) --
	(394.15,267.01);
\definecolor{fillColor}{RGB}{228,26,28}

\path[fill=fillColor] (314.88, 98.41) rectangle (360.18,258.05);
\definecolor{fillColor}{RGB}{55,126,184}

\path[fill=fillColor] (314.88, 78.92) rectangle (360.18, 98.41);
\definecolor{fillColor}{RGB}{228,26,28}

\path[fill=fillColor] (371.50, 99.76) rectangle (416.80,258.05);
\definecolor{fillColor}{RGB}{55,126,184}

\path[fill=fillColor] (371.50, 78.92) rectangle (416.80, 99.76);
\definecolor{drawColor}{RGB}{0,0,0}

\node[text=drawColor,anchor=base,inner sep=0pt, outer sep=0pt, scale=  0.85] at (337.53,159.33) {89.12{\%}};

\node[text=drawColor,anchor=base,inner sep=0pt, outer sep=0pt, scale=  0.85] at (337.53, 83.77) {10.88{\%}};

\node[text=drawColor,anchor=base,inner sep=0pt, outer sep=0pt, scale=  0.85] at (394.15,160.14) {88.37{\%}};

\node[text=drawColor,anchor=base,inner sep=0pt, outer sep=0pt, scale=  0.85] at (394.15, 84.31) {11.63{\%}};
\end{scope}
\begin{scope}
\path[clip] ( 43.44,267.01) rectangle (168.00,283.58);
\definecolor{drawColor}{gray}{0.10}

\node[text=drawColor,anchor=base,inner sep=0pt, outer sep=0pt, scale=  0.88] at (105.72,272.26) {Región Centro};
\end{scope}
\begin{scope}
\path[clip] (173.50,267.01) rectangle (298.06,283.58);
\definecolor{drawColor}{gray}{0.10}

\node[text=drawColor,anchor=base,inner sep=0pt, outer sep=0pt, scale=  0.88] at (235.78,272.26) {Región Norte};
\end{scope}
\begin{scope}
\path[clip] (303.56,267.01) rectangle (428.12,283.58);
\definecolor{drawColor}{gray}{0.10}

\node[text=drawColor,anchor=base,inner sep=0pt, outer sep=0pt, scale=  0.88] at (365.84,272.26) {Región Sur};
\end{scope}
\begin{scope}
\path[clip] (  0.00,  0.00) rectangle (433.62,289.08);
\definecolor{drawColor}{gray}{0.20}

\path[draw=drawColor,line width= 0.6pt,line join=round] ( 77.41, 67.21) --
	( 77.41, 69.96);

\path[draw=drawColor,line width= 0.6pt,line join=round] (134.03, 67.21) --
	(134.03, 69.96);
\end{scope}
\begin{scope}
\path[clip] (  0.00,  0.00) rectangle (433.62,289.08);
\definecolor{drawColor}{RGB}{0,0,0}

\node[text=drawColor,anchor=base,inner sep=0pt, outer sep=0pt, scale=  0.88] at ( 77.41, 58.95) {Migrantes};

\node[text=drawColor,anchor=base,inner sep=0pt, outer sep=0pt, scale=  0.88] at (134.03, 58.95) {Nativos};
\end{scope}
\begin{scope}
\path[clip] (  0.00,  0.00) rectangle (433.62,289.08);
\definecolor{drawColor}{gray}{0.20}

\path[draw=drawColor,line width= 0.6pt,line join=round] (207.47, 67.21) --
	(207.47, 69.96);

\path[draw=drawColor,line width= 0.6pt,line join=round] (264.09, 67.21) --
	(264.09, 69.96);
\end{scope}
\begin{scope}
\path[clip] (  0.00,  0.00) rectangle (433.62,289.08);
\definecolor{drawColor}{RGB}{0,0,0}

\node[text=drawColor,anchor=base,inner sep=0pt, outer sep=0pt, scale=  0.88] at (207.47, 58.95) {Migrantes};

\node[text=drawColor,anchor=base,inner sep=0pt, outer sep=0pt, scale=  0.88] at (264.09, 58.95) {Nativos};
\end{scope}
\begin{scope}
\path[clip] (  0.00,  0.00) rectangle (433.62,289.08);
\definecolor{drawColor}{gray}{0.20}

\path[draw=drawColor,line width= 0.6pt,line join=round] (337.53, 67.21) --
	(337.53, 69.96);

\path[draw=drawColor,line width= 0.6pt,line join=round] (394.15, 67.21) --
	(394.15, 69.96);
\end{scope}
\begin{scope}
\path[clip] (  0.00,  0.00) rectangle (433.62,289.08);
\definecolor{drawColor}{RGB}{0,0,0}

\node[text=drawColor,anchor=base,inner sep=0pt, outer sep=0pt, scale=  0.88] at (337.53, 58.95) {Migrantes};

\node[text=drawColor,anchor=base,inner sep=0pt, outer sep=0pt, scale=  0.88] at (394.15, 58.95) {Nativos};
\end{scope}
\begin{scope}
\path[clip] (  0.00,  0.00) rectangle (433.62,289.08);
\definecolor{drawColor}{RGB}{0,0,0}

\node[text=drawColor,anchor=base east,inner sep=0pt, outer sep=0pt, scale=  0.88] at ( 38.49, 75.89) {0{\%}};

\node[text=drawColor,anchor=base east,inner sep=0pt, outer sep=0pt, scale=  0.88] at ( 38.49,120.67) {25{\%}};

\node[text=drawColor,anchor=base east,inner sep=0pt, outer sep=0pt, scale=  0.88] at ( 38.49,165.45) {50{\%}};

\node[text=drawColor,anchor=base east,inner sep=0pt, outer sep=0pt, scale=  0.88] at ( 38.49,210.24) {75{\%}};

\node[text=drawColor,anchor=base east,inner sep=0pt, outer sep=0pt, scale=  0.88] at ( 38.49,255.02) {100{\%}};
\end{scope}
\begin{scope}
\path[clip] (  0.00,  0.00) rectangle (433.62,289.08);
\definecolor{drawColor}{gray}{0.20}

\path[draw=drawColor,line width= 0.6pt,line join=round] ( 40.69, 78.92) --
	( 43.44, 78.92);

\path[draw=drawColor,line width= 0.6pt,line join=round] ( 40.69,123.70) --
	( 43.44,123.70);

\path[draw=drawColor,line width= 0.6pt,line join=round] ( 40.69,168.48) --
	( 43.44,168.48);

\path[draw=drawColor,line width= 0.6pt,line join=round] ( 40.69,213.27) --
	( 43.44,213.27);

\path[draw=drawColor,line width= 0.6pt,line join=round] ( 40.69,258.05) --
	( 43.44,258.05);
\end{scope}
\begin{scope}
\path[clip] (  0.00,  0.00) rectangle (433.62,289.08);
\definecolor{fillColor}{RGB}{255,255,255}

\path[fill=fillColor] (171.04,  5.50) rectangle (300.52, 33.78);
\end{scope}
\begin{scope}
\path[clip] (  0.00,  0.00) rectangle (433.62,289.08);
\definecolor{fillColor}{gray}{0.95}

\path[fill=fillColor] (182.04, 11.00) rectangle (196.50, 28.28);
\end{scope}
\begin{scope}
\path[clip] (  0.00,  0.00) rectangle (433.62,289.08);
\definecolor{fillColor}{RGB}{228,26,28}

\path[fill=fillColor] (182.75, 11.71) rectangle (195.78, 27.56);
\end{scope}
\begin{scope}
\path[clip] (  0.00,  0.00) rectangle (433.62,289.08);
\definecolor{fillColor}{gray}{0.95}

\path[fill=fillColor] (244.18, 11.00) rectangle (258.63, 28.28);
\end{scope}
\begin{scope}
\path[clip] (  0.00,  0.00) rectangle (433.62,289.08);
\definecolor{fillColor}{RGB}{55,126,184}

\path[fill=fillColor] (244.89, 11.71) rectangle (257.92, 27.56);
\end{scope}
\begin{scope}
\path[clip] (  0.00,  0.00) rectangle (433.62,289.08);
\definecolor{drawColor}{RGB}{0,0,0}

\node[text=drawColor,anchor=base west,inner sep=0pt, outer sep=0pt, scale=  0.88] at (202.00, 21.36) {No recibe};

\node[text=drawColor,anchor=base west,inner sep=0pt, outer sep=0pt, scale=  0.88] at (202.00, 11.86) {subsidio};
\end{scope}
\begin{scope}
\path[clip] (  0.00,  0.00) rectangle (433.62,289.08);
\definecolor{drawColor}{RGB}{0,0,0}

\node[text=drawColor,anchor=base west,inner sep=0pt, outer sep=0pt, scale=  0.88] at (264.13, 21.36) {Recibe};

\node[text=drawColor,anchor=base west,inner sep=0pt, outer sep=0pt, scale=  0.88] at (264.13, 11.86) {subsidio};
\end{scope}
\end{tikzpicture}
 
%\end{center}
%\begin{flushleft}
%\begin{scriptsize}
%Fuente: Elaboración propia en base a EPH
%\end{scriptsize}
%\end{flushleft}
%\end{figure}
\subsection{Ocupación y calificación}
La busqueda de acceso a oportunidades laborales o una mayor demanda laboral para determinados niveles de calificación funcionan como incentivos para abandonar la localidad de origen de las personas.

En el cuadro \ref{cuadro:tasaactiv_mig} se pude ver la tasa de actividad de  los migrantes y nativos para las distintas regiones. Los migrantes de la región Sur tienen una tasa de actividad superior a la de los nativos, sin embargo en la región Centro y Norte esta situación es inversa, en donde los nativos forman parte de la fuerza de trabajo en una en una mayor proporción que los migrantes.
\begin{table}[htbp!]
\centering
\caption{\\Tasa de actividad de nativos y migrantes} 
\begin{tabular}{lcc}
  \hline
 Región & Migrantes & Nativos \\ 
  \hline
Centro & 49.87\% & 50.81\% \\ 
Norte & 43.90\% & 45.69\% \\ 
Sur & 55.81\%  & 42.82\% \\
   \hline
\end{tabular}
\label{cuadro:tasaactiv_mig}
\begin{flushleft}
\begin{scriptsize}
Fuente: Elaboración propia en base a EPH.\\
Nota: Los migrantes están definidos como personas que vivían hace cinco años en otra provincia. Los nativos están definidos como personas que nacieron y viven en la misma provincia. Las estimaciones corresponden al período desde el segundo trimestre de 2016 hasta el cuarto trimestre de 2019.
\end{scriptsize}
\end{flushleft}
\end{table}

\begin{figure}[htbp!]
\begin{center}
\caption{\\Calificación de la ocupación los nativos y migrantes}
\label{figure:calif_mig}
% Created by tikzDevice version 0.12.3.1 on 2021-05-24 17:14:22
% !TEX encoding = UTF-8 Unicode
\begin{tikzpicture}[x=1pt,y=1pt]
\definecolor{fillColor}{RGB}{255,255,255}
\path[use as bounding box,fill=fillColor,fill opacity=0.00] (0,0) rectangle (433.62,289.08);
\begin{scope}
\path[clip] (  0.46,153.48) rectangle (128.30,281.31);
\definecolor{fillColor}{RGB}{228,26,28}

\path[fill=fillColor] ( 64.38,217.39) --
	( 65.85,216.42) --
	( 67.32,215.45) --
	( 68.79,214.48) --
	( 70.26,213.50) --
	( 71.73,212.53) --
	( 73.21,211.56) --
	( 74.68,210.59) --
	( 76.15,209.62) --
	( 77.62,208.65) --
	( 79.09,207.67) --
	( 80.56,206.70) --
	( 82.03,205.73) --
	( 83.50,204.76) --
	( 84.98,203.79) --
	( 86.45,202.81) --
	( 87.92,201.84) --
	( 89.39,200.87) --
	( 90.86,199.90) --
	( 92.33,198.93) --
	( 93.80,197.95) --
	( 95.27,196.98) --
	( 96.74,196.01) --
	( 98.22,195.04) --
	( 99.69,194.07) --
	(101.16,193.09) --
	(102.63,192.12) --
	(104.10,191.15) --
	(105.57,190.18) --
	(107.04,189.21) --
	(107.98,190.68) --
	(108.87,192.19) --
	(109.70,193.72) --
	(110.49,195.29) --
	(111.22,196.88) --
	(111.89,198.49) --
	(112.51,200.13) --
	(113.07,201.78) --
	(113.58,203.46) --
	(114.02,205.15) --
	(114.41,206.85) --
	(114.74,208.57) --
	(115.02,210.30) --
	(115.23,212.03) --
	(115.38,213.77) --
	(115.48,215.52) --
	(115.51,217.27) --
	(115.49,219.02) --
	(115.40,220.76) --
	(115.26,222.51) --
	(115.05,224.24) --
	(114.79,225.97) --
	(114.46,227.69) --
	(114.08,229.40) --
	(113.64,231.09) --
	(113.15,232.77) --
	(112.59,234.42) --
	(111.98,236.06) --
	(111.32,237.68) --
	(110.59,239.27) --
	(109.82,240.84) --
	(108.99,242.38) --
	(108.11,243.89) --
	(107.18,245.37) --
	(106.20,246.82) --
	(105.17,248.23) --
	(104.09,249.61) --
	(102.96,250.95) --
	(101.79,252.25) --
	(100.58,253.50) --
	( 99.32,254.72) --
	( 98.03,255.89) --
	( 96.69,257.02) --
	( 95.32,258.10) --
	( 93.91,259.14) --
	( 92.46,260.12) --
	( 90.98,261.06) --
	( 89.48,261.94) --
	( 87.94,262.78) --
	( 86.37,263.55) --
	( 84.78,264.28) --
	( 83.17,264.95) --
	( 81.53,265.56) --
	( 79.87,266.12) --
	( 78.20,266.62) --
	( 76.50,267.07) --
	( 74.80,267.45) --
	( 73.08,267.78) --
	( 71.35,268.05) --
	( 69.62,268.26) --
	( 67.87,268.41) --
	( 66.13,268.50) --
	( 64.38,268.53) --
	( 64.38,266.76) --
	( 64.38,265.00) --
	( 64.38,263.24) --
	( 64.38,261.47) --
	( 64.38,259.71) --
	( 64.38,257.95) --
	( 64.38,256.18) --
	( 64.38,254.42) --
	( 64.38,252.66) --
	( 64.38,250.89) --
	( 64.38,249.13) --
	( 64.38,247.37) --
	( 64.38,245.60) --
	( 64.38,243.84) --
	( 64.38,242.08) --
	( 64.38,240.31) --
	( 64.38,238.55) --
	( 64.38,236.79) --
	( 64.38,235.02) --
	( 64.38,233.26) --
	( 64.38,231.50) --
	( 64.38,229.73) --
	( 64.38,227.97) --
	( 64.38,226.21) --
	( 64.38,224.45) --
	( 64.38,222.68) --
	( 64.38,220.92) --
	( 64.38,219.16) --
	( 64.38,217.39) --
	( 64.38,217.39) --
	cycle;
\definecolor{fillColor}{RGB}{55,126,184}

\path[fill=fillColor] ( 64.38,217.39) --
	( 64.38,219.16) --
	( 64.38,220.92) --
	( 64.38,222.68) --
	( 64.38,224.45) --
	( 64.38,226.21) --
	( 64.38,227.97) --
	( 64.38,229.73) --
	( 64.38,231.50) --
	( 64.38,233.26) --
	( 64.38,235.02) --
	( 64.38,236.79) --
	( 64.38,238.55) --
	( 64.38,240.31) --
	( 64.38,242.08) --
	( 64.38,243.84) --
	( 64.38,245.60) --
	( 64.38,247.37) --
	( 64.38,249.13) --
	( 64.38,250.89) --
	( 64.38,252.66) --
	( 64.38,254.42) --
	( 64.38,256.18) --
	( 64.38,257.95) --
	( 64.38,259.71) --
	( 64.38,261.47) --
	( 64.38,263.24) --
	( 64.38,265.00) --
	( 64.38,266.76) --
	( 64.38,268.53) --
	( 62.65,268.50) --
	( 60.92,268.41) --
	( 59.20,268.26) --
	( 57.48,268.06) --
	( 55.77,267.80) --
	( 54.07,267.47) --
	( 52.38,267.10) --
	( 50.70,266.66) --
	( 49.05,266.17) --
	( 47.40,265.63) --
	( 45.78,265.02) --
	( 44.18,264.37) --
	( 42.60,263.66) --
	( 41.05,262.89) --
	( 39.52,262.08) --
	( 38.03,261.21) --
	( 36.56,260.29) --
	( 35.12,259.33) --
	( 33.72,258.31) --
	( 32.35,257.25) --
	( 31.02,256.15) --
	( 29.73,255.00) --
	( 28.48,253.80) --
	( 27.27,252.57) --
	( 26.10,251.29) --
	( 24.97,249.98) --
	( 23.89,248.63) --
	( 22.86,247.24) --
	( 21.87,245.82) --
	( 20.94,244.36) --
	( 20.05,242.88) --
	( 19.21,241.36) --
	( 18.43,239.82) --
	( 17.69,238.25) --
	( 17.02,236.66) --
	( 16.39,235.05) --
	( 15.82,233.41) --
	( 15.31,231.76) --
	( 14.85,230.09) --
	( 14.45,228.41) --
	( 14.10,226.71) --
	( 13.82,225.01) --
	( 13.59,223.29) --
	( 13.42,221.57) --
	( 13.31,219.84) --
	( 13.25,218.11) --
	( 13.26,216.38) --
	( 13.32,214.66) --
	( 13.44,212.93) --
	( 13.62,211.21) --
	( 13.86,209.49) --
	( 14.16,207.79) --
	( 14.51,206.10) --
	( 14.92,204.42) --
	( 15.39,202.75) --
	( 15.91,201.10) --
	( 16.49,199.47) --
	( 17.12,197.86) --
	( 17.81,196.27) --
	( 18.55,194.71) --
	( 19.35,193.17) --
	( 20.19,191.66) --
	( 21.09,190.18) --
	( 22.03,188.73) --
	( 23.03,187.32) --
	( 24.07,185.93) --
	( 25.16,184.59) --
	( 26.29,183.28) --
	( 27.46,182.01) --
	( 28.68,180.78) --
	( 29.94,179.59) --
	( 31.24,178.45) --
	( 32.58,177.35) --
	( 33.95,176.30) --
	( 35.36,175.29) --
	( 36.80,174.34) --
	( 38.27,173.43) --
	( 39.77,172.57) --
	( 41.30,171.76) --
	( 42.86,171.01) --
	( 44.44,170.31) --
	( 46.05,169.66) --
	( 47.67,169.07) --
	( 49.32,168.53) --
	( 50.98,168.05) --
	( 52.66,167.62) --
	( 54.35,167.25) --
	( 56.05,166.94) --
	( 57.76,166.69) --
	( 59.48,166.49) --
	( 61.21,166.36) --
	( 62.93,166.28) --
	( 64.66,166.26) --
	( 66.39,166.30) --
	( 68.12,166.40) --
	( 69.85,166.55) --
	( 71.56,166.77) --
	( 73.27,167.04) --
	( 74.97,167.37) --
	( 76.66,167.76) --
	( 78.33,168.20) --
	( 79.99,168.70) --
	( 81.62,169.26) --
	( 83.24,169.87) --
	( 84.84,170.53) --
	( 86.41,171.25) --
	( 87.96,172.02) --
	( 89.48,172.85) --
	( 90.98,173.72) --
	( 92.44,174.65) --
	( 93.87,175.62) --
	( 95.27,176.64) --
	( 96.63,177.71) --
	( 97.95,178.82) --
	( 99.24,179.98) --
	(100.48,181.18) --
	(101.69,182.42) --
	(102.85,183.71) --
	(103.97,185.03) --
	(105.04,186.39) --
	(106.06,187.78) --
	(107.04,189.21) --
	(105.57,190.18) --
	(104.10,191.15) --
	(102.63,192.12) --
	(101.16,193.09) --
	( 99.69,194.07) --
	( 98.22,195.04) --
	( 96.74,196.01) --
	( 95.27,196.98) --
	( 93.80,197.95) --
	( 92.33,198.93) --
	( 90.86,199.90) --
	( 89.39,200.87) --
	( 87.92,201.84) --
	( 86.45,202.81) --
	( 84.98,203.79) --
	( 83.50,204.76) --
	( 82.03,205.73) --
	( 80.56,206.70) --
	( 79.09,207.67) --
	( 77.62,208.65) --
	( 76.15,209.62) --
	( 74.68,210.59) --
	( 73.21,211.56) --
	( 71.73,212.53) --
	( 70.26,213.50) --
	( 68.79,214.48) --
	( 67.32,215.45) --
	( 65.85,216.42) --
	( 64.38,217.39) --
	( 64.38,217.39) --
	cycle;
\definecolor{drawColor}{RGB}{0,0,0}

\node[text=drawColor,anchor=base,inner sep=0pt, outer sep=0pt, scale=  0.71] at ( 86.90,227.05) {34.29{\%}};

\node[text=drawColor,anchor=base,inner sep=0pt, outer sep=0pt, scale=  0.71] at ( 41.86,202.83) {65.71{\%}};
\end{scope}
\begin{scope}
\path[clip] (  0.46, 20.14) rectangle (128.30,147.98);
\definecolor{fillColor}{RGB}{228,26,28}

\path[fill=fillColor] ( 64.38, 84.06) --
	( 66.14, 84.11) --
	( 67.90, 84.16) --
	( 69.67, 84.21) --
	( 71.43, 84.26) --
	( 73.19, 84.31) --
	( 74.95, 84.37) --
	( 76.72, 84.42) --
	( 78.48, 84.47) --
	( 80.24, 84.52) --
	( 82.00, 84.57) --
	( 83.77, 84.62) --
	( 85.53, 84.67) --
	( 87.29, 84.72) --
	( 89.05, 84.77) --
	( 90.82, 84.82) --
	( 92.58, 84.87) --
	( 94.34, 84.93) --
	( 96.10, 84.98) --
	( 97.87, 85.03) --
	( 99.63, 85.08) --
	(101.39, 85.13) --
	(103.15, 85.18) --
	(104.92, 85.23) --
	(106.68, 85.28) --
	(108.44, 85.33) --
	(110.20, 85.38) --
	(111.97, 85.43) --
	(113.73, 85.48) --
	(115.49, 85.54) --
	(115.41, 87.29) --
	(115.27, 89.03) --
	(115.07, 90.77) --
	(114.81, 92.51) --
	(114.49, 94.23) --
	(114.11, 95.94) --
	(113.68, 97.64) --
	(113.18, 99.32) --
	(112.63,100.98) --
	(112.02,102.62) --
	(111.36,104.24) --
	(110.64,105.84) --
	(109.87,107.41) --
	(109.04,108.96) --
	(108.16,110.47) --
	(107.23,111.96) --
	(106.25,113.41) --
	(105.22,114.83) --
	(104.14,116.21) --
	(103.02,117.55) --
	(101.85,118.86) --
	(100.63,120.12) --
	( 99.38,121.34) --
	( 98.08,122.52) --
	( 96.74,123.65) --
	( 95.37,124.73) --
	( 93.96,125.77) --
	( 92.51,126.76) --
	( 91.03,127.70) --
	( 89.52,128.59) --
	( 87.98,129.42) --
	( 86.41,130.20) --
	( 84.82,130.93) --
	( 83.20,131.60) --
	( 81.56,132.22) --
	( 79.90,132.78) --
	( 78.22,133.28) --
	( 76.53,133.73) --
	( 74.82,134.12) --
	( 73.10,134.44) --
	( 71.37,134.71) --
	( 69.63,134.92) --
	( 67.88,135.07) --
	( 66.13,135.16) --
	( 64.38,135.19) --
	( 64.38,133.43) --
	( 64.38,131.67) --
	( 64.38,129.90) --
	( 64.38,128.14) --
	( 64.38,126.38) --
	( 64.38,124.61) --
	( 64.38,122.85) --
	( 64.38,121.09) --
	( 64.38,119.32) --
	( 64.38,117.56) --
	( 64.38,115.80) --
	( 64.38,114.03) --
	( 64.38,112.27) --
	( 64.38,110.51) --
	( 64.38,108.75) --
	( 64.38,106.98) --
	( 64.38,105.22) --
	( 64.38,103.46) --
	( 64.38,101.69) --
	( 64.38, 99.93) --
	( 64.38, 98.17) --
	( 64.38, 96.40) --
	( 64.38, 94.64) --
	( 64.38, 92.88) --
	( 64.38, 91.11) --
	( 64.38, 89.35) --
	( 64.38, 87.59) --
	( 64.38, 85.82) --
	( 64.38, 84.06) --
	( 64.38, 84.06) --
	cycle;
\definecolor{fillColor}{RGB}{55,126,184}

\path[fill=fillColor] ( 64.38, 84.06) --
	( 64.38, 85.82) --
	( 64.38, 87.59) --
	( 64.38, 89.35) --
	( 64.38, 91.11) --
	( 64.38, 92.88) --
	( 64.38, 94.64) --
	( 64.38, 96.40) --
	( 64.38, 98.17) --
	( 64.38, 99.93) --
	( 64.38,101.69) --
	( 64.38,103.46) --
	( 64.38,105.22) --
	( 64.38,106.98) --
	( 64.38,108.75) --
	( 64.38,110.51) --
	( 64.38,112.27) --
	( 64.38,114.03) --
	( 64.38,115.80) --
	( 64.38,117.56) --
	( 64.38,119.32) --
	( 64.38,121.09) --
	( 64.38,122.85) --
	( 64.38,124.61) --
	( 64.38,126.38) --
	( 64.38,128.14) --
	( 64.38,129.90) --
	( 64.38,131.67) --
	( 64.38,133.43) --
	( 64.38,135.19) --
	( 62.65,135.16) --
	( 60.92,135.08) --
	( 59.19,134.93) --
	( 57.47,134.72) --
	( 55.76,134.46) --
	( 54.06,134.14) --
	( 52.37,133.76) --
	( 50.69,133.33) --
	( 49.03,132.84) --
	( 47.39,132.29) --
	( 45.77,131.69) --
	( 44.17,131.03) --
	( 42.59,130.32) --
	( 41.03,129.55) --
	( 39.51,128.74) --
	( 38.01,127.87) --
	( 36.54,126.95) --
	( 35.10,125.98) --
	( 33.70,124.97) --
	( 32.33,123.91) --
	( 31.00,122.80) --
	( 29.71,121.65) --
	( 28.46,120.45) --
	( 27.25,119.21) --
	( 26.08,117.94) --
	( 24.95,116.62) --
	( 23.87,115.27) --
	( 22.84,113.88) --
	( 21.85,112.45) --
	( 20.92,111.00) --
	( 20.03,109.51) --
	( 19.19,107.99) --
	( 18.41,106.45) --
	( 17.68,104.88) --
	( 17.00,103.29) --
	( 16.37,101.67) --
	( 15.81,100.04) --
	( 15.29, 98.38) --
	( 14.84, 96.71) --
	( 14.44, 95.03) --
	( 14.09, 93.33) --
	( 13.81, 91.62) --
	( 13.58, 89.91) --
	( 13.41, 88.18) --
	( 13.30, 86.45) --
	( 13.25, 84.72) --
	( 13.26, 82.99) --
	( 13.32, 81.26) --
	( 13.45, 79.53) --
	( 13.63, 77.81) --
	( 13.87, 76.10) --
	( 14.17, 74.39) --
	( 14.53, 72.70) --
	( 14.94, 71.02) --
	( 15.41, 69.35) --
	( 15.93, 67.70) --
	( 16.52, 66.07) --
	( 17.15, 64.46) --
	( 17.84, 62.87) --
	( 18.59, 61.31) --
	( 19.38, 59.77) --
	( 20.23, 58.26) --
	( 21.13, 56.78) --
	( 22.08, 55.33) --
	( 23.08, 53.92) --
	( 24.12, 52.53) --
	( 25.21, 51.19) --
	( 26.35, 49.88) --
	( 27.53, 48.61) --
	( 28.75, 47.39) --
	( 30.01, 46.20) --
	( 31.31, 45.06) --
	( 32.65, 43.96) --
	( 34.03, 42.91) --
	( 35.44, 41.91) --
	( 36.88, 40.95) --
	( 38.36, 40.04) --
	( 39.86, 39.19) --
	( 41.40, 38.38) --
	( 42.96, 37.63) --
	( 44.54, 36.93) --
	( 46.15, 36.29) --
	( 47.78, 35.70) --
	( 49.42, 35.16) --
	( 51.09, 34.69) --
	( 52.77, 34.26) --
	( 54.46, 33.90) --
	( 56.16, 33.59) --
	( 57.88, 33.34) --
	( 59.60, 33.15) --
	( 61.32, 33.02) --
	( 63.05, 32.94) --
	( 64.79, 32.93) --
	( 66.52, 32.97) --
	( 68.24, 33.07) --
	( 69.97, 33.23) --
	( 71.69, 33.45) --
	( 73.40, 33.73) --
	( 75.10, 34.06) --
	( 76.78, 34.45) --
	( 78.45, 34.90) --
	( 80.11, 35.41) --
	( 81.75, 35.97) --
	( 83.37, 36.58) --
	( 84.96, 37.25) --
	( 86.54, 37.98) --
	( 88.08, 38.75) --
	( 89.61, 39.58) --
	( 91.10, 40.46) --
	( 92.56, 41.39) --
	( 93.99, 42.37) --
	( 95.38, 43.40) --
	( 96.74, 44.47) --
	( 98.06, 45.59) --
	( 99.35, 46.75) --
	(100.59, 47.96) --
	(101.79, 49.20) --
	(102.95, 50.49) --
	(104.06, 51.82) --
	(105.13, 53.18) --
	(106.15, 54.58) --
	(107.13, 56.01) --
	(108.05, 57.47) --
	(108.93, 58.96) --
	(109.75, 60.49) --
	(110.53, 62.04) --
	(111.25, 63.61) --
	(111.91, 65.21) --
	(112.52, 66.83) --
	(113.08, 68.47) --
	(113.58, 70.13) --
	(114.02, 71.80) --
	(114.41, 73.49) --
	(114.74, 75.19) --
	(115.01, 76.90) --
	(115.22, 78.62) --
	(115.38, 80.34) --
	(115.47, 82.07) --
	(115.51, 83.80) --
	(115.49, 85.54) --
	(113.73, 85.48) --
	(111.97, 85.43) --
	(110.20, 85.38) --
	(108.44, 85.33) --
	(106.68, 85.28) --
	(104.92, 85.23) --
	(103.15, 85.18) --
	(101.39, 85.13) --
	( 99.63, 85.08) --
	( 97.87, 85.03) --
	( 96.10, 84.98) --
	( 94.34, 84.93) --
	( 92.58, 84.87) --
	( 90.82, 84.82) --
	( 89.05, 84.77) --
	( 87.29, 84.72) --
	( 85.53, 84.67) --
	( 83.77, 84.62) --
	( 82.00, 84.57) --
	( 80.24, 84.52) --
	( 78.48, 84.47) --
	( 76.72, 84.42) --
	( 74.95, 84.37) --
	( 73.19, 84.31) --
	( 71.43, 84.26) --
	( 69.67, 84.21) --
	( 67.90, 84.16) --
	( 66.14, 84.11) --
	( 64.38, 84.06) --
	( 64.38, 84.06) --
	cycle;
\definecolor{drawColor}{RGB}{0,0,0}

\node[text=drawColor,anchor=base,inner sep=0pt, outer sep=0pt, scale=  0.71] at ( 82.19, 99.95) {24.54{\%}};

\node[text=drawColor,anchor=base,inner sep=0pt, outer sep=0pt, scale=  0.71] at ( 46.56, 63.27) {75.46{\%}};
\end{scope}
\begin{scope}
\path[clip] (133.80,153.48) rectangle (261.63,281.31);
\definecolor{fillColor}{RGB}{228,26,28}

\path[fill=fillColor] (197.71,217.39) --
	(199.12,216.34) --
	(200.54,215.29) --
	(201.95,214.23) --
	(203.37,213.18) --
	(204.78,212.12) --
	(206.19,211.07) --
	(207.61,210.02) --
	(209.02,208.96) --
	(210.43,207.91) --
	(211.85,206.86) --
	(213.26,205.80) --
	(214.68,204.75) --
	(216.09,203.69) --
	(217.50,202.64) --
	(218.92,201.59) --
	(220.33,200.53) --
	(221.74,199.48) --
	(223.16,198.43) --
	(224.57,197.37) --
	(225.99,196.32) --
	(227.40,195.27) --
	(228.81,194.21) --
	(230.23,193.16) --
	(231.64,192.10) --
	(233.05,191.05) --
	(234.47,190.00) --
	(235.88,188.94) --
	(237.30,187.89) --
	(238.71,186.84) --
	(239.72,188.25) --
	(240.69,189.69) --
	(241.61,191.17) --
	(242.48,192.68) --
	(243.29,194.22) --
	(244.05,195.78) --
	(244.76,197.37) --
	(245.41,198.98) --
	(246.01,200.61) --
	(246.56,202.27) --
	(247.04,203.94) --
	(247.47,205.62) --
	(247.84,207.32) --
	(248.16,209.03) --
	(248.41,210.75) --
	(248.61,212.48) --
	(248.75,214.22) --
	(248.82,215.95) --
	(248.84,217.69) --
	(248.80,219.43) --
	(248.70,221.17) --
	(248.55,222.90) --
	(248.33,224.63) --
	(248.05,226.34) --
	(247.72,228.05) --
	(247.33,229.75) --
	(246.88,231.43) --
	(246.37,233.09) --
	(245.81,234.74) --
	(245.19,236.36) --
	(244.52,237.97) --
	(243.79,239.55) --
	(243.01,241.10) --
	(242.18,242.63) --
	(241.30,244.13) --
	(240.36,245.59) --
	(239.38,247.03) --
	(238.35,248.43) --
	(237.27,249.79) --
	(236.14,251.12) --
	(234.97,252.41) --
	(233.76,253.65) --
	(232.51,254.86) --
	(231.21,256.02) --
	(229.88,257.14) --
	(228.51,258.21) --
	(227.10,259.23) --
	(225.66,260.21) --
	(224.19,261.14) --
	(222.69,262.01) --
	(221.15,262.83) --
	(219.60,263.61) --
	(218.01,264.32) --
	(216.40,264.99) --
	(214.77,265.59) --
	(213.12,266.15) --
	(211.46,266.64) --
	(209.77,267.08) --
	(208.08,267.46) --
	(206.37,267.79) --
	(204.65,268.05) --
	(202.92,268.26) --
	(201.19,268.41) --
	(199.45,268.50) --
	(197.71,268.53) --
	(197.71,266.76) --
	(197.71,265.00) --
	(197.71,263.24) --
	(197.71,261.47) --
	(197.71,259.71) --
	(197.71,257.95) --
	(197.71,256.18) --
	(197.71,254.42) --
	(197.71,252.66) --
	(197.71,250.89) --
	(197.71,249.13) --
	(197.71,247.37) --
	(197.71,245.60) --
	(197.71,243.84) --
	(197.71,242.08) --
	(197.71,240.31) --
	(197.71,238.55) --
	(197.71,236.79) --
	(197.71,235.02) --
	(197.71,233.26) --
	(197.71,231.50) --
	(197.71,229.73) --
	(197.71,227.97) --
	(197.71,226.21) --
	(197.71,224.45) --
	(197.71,222.68) --
	(197.71,220.92) --
	(197.71,219.16) --
	(197.71,217.39) --
	(197.71,217.39) --
	cycle;
\definecolor{fillColor}{RGB}{55,126,184}

\path[fill=fillColor] (197.71,217.39) --
	(197.71,219.16) --
	(197.71,220.92) --
	(197.71,222.68) --
	(197.71,224.45) --
	(197.71,226.21) --
	(197.71,227.97) --
	(197.71,229.73) --
	(197.71,231.50) --
	(197.71,233.26) --
	(197.71,235.02) --
	(197.71,236.79) --
	(197.71,238.55) --
	(197.71,240.31) --
	(197.71,242.08) --
	(197.71,243.84) --
	(197.71,245.60) --
	(197.71,247.37) --
	(197.71,249.13) --
	(197.71,250.89) --
	(197.71,252.66) --
	(197.71,254.42) --
	(197.71,256.18) --
	(197.71,257.95) --
	(197.71,259.71) --
	(197.71,261.47) --
	(197.71,263.24) --
	(197.71,265.00) --
	(197.71,266.76) --
	(197.71,268.53) --
	(195.98,268.50) --
	(194.24,268.41) --
	(192.52,268.26) --
	(190.79,268.06) --
	(189.08,267.79) --
	(187.37,267.47) --
	(185.68,267.09) --
	(184.00,266.65) --
	(182.34,266.16) --
	(180.69,265.61) --
	(179.07,265.00) --
	(177.46,264.34) --
	(175.88,263.63) --
	(174.32,262.86) --
	(172.79,262.04) --
	(171.29,261.17) --
	(169.82,260.25) --
	(168.39,259.28) --
	(166.98,258.26) --
	(165.61,257.20) --
	(164.28,256.08) --
	(162.99,254.93) --
	(161.73,253.73) --
	(160.52,252.49) --
	(159.35,251.20) --
	(158.23,249.88) --
	(157.15,248.53) --
	(156.12,247.13) --
	(155.13,245.70) --
	(154.19,244.24) --
	(153.31,242.75) --
	(152.47,241.23) --
	(151.69,239.68) --
	(150.96,238.11) --
	(150.29,236.51) --
	(149.67,234.89) --
	(149.10,233.25) --
	(148.59,231.59) --
	(148.14,229.92) --
	(147.74,228.23) --
	(147.40,226.53) --
	(147.12,224.81) --
	(146.90,223.09) --
	(146.73,221.37) --
	(146.63,219.63) --
	(146.58,217.90) --
	(146.59,216.16) --
	(146.66,214.43) --
	(146.79,212.70) --
	(146.98,210.98) --
	(147.23,209.26) --
	(147.53,207.55) --
	(147.90,205.85) --
	(148.32,204.17) --
	(148.79,202.50) --
	(149.33,200.85) --
	(149.92,199.22) --
	(150.56,197.61) --
	(151.26,196.02) --
	(152.01,194.46) --
	(152.82,192.92) --
	(153.67,191.41) --
	(154.58,189.93) --
	(155.53,188.48) --
	(156.54,187.07) --
	(157.59,185.69) --
	(158.69,184.35) --
	(159.83,183.04) --
	(161.02,181.78) --
	(162.25,180.55) --
	(163.52,179.37) --
	(164.83,178.23) --
	(166.18,177.14) --
	(167.56,176.09) --
	(168.98,175.09) --
	(170.43,174.14) --
	(171.91,173.24) --
	(173.43,172.39) --
	(174.97,171.60) --
	(176.53,170.85) --
	(178.13,170.16) --
	(179.74,169.52) --
	(181.37,168.94) --
	(183.03,168.41) --
	(184.70,167.94) --
	(186.38,167.53) --
	(188.08,167.17) --
	(189.79,166.88) --
	(191.51,166.64) --
	(193.23,166.46) --
	(194.96,166.33) --
	(196.70,166.27) --
	(198.43,166.26) --
	(200.17,166.32) --
	(201.90,166.43) --
	(203.62,166.60) --
	(205.34,166.83) --
	(207.06,167.12) --
	(208.76,167.47) --
	(210.44,167.87) --
	(212.12,168.33) --
	(213.77,168.85) --
	(215.41,169.42) --
	(217.03,170.05) --
	(218.62,170.73) --
	(220.19,171.47) --
	(221.74,172.26) --
	(223.26,173.10) --
	(224.74,173.99) --
	(226.20,174.93) --
	(227.62,175.92) --
	(229.01,176.96) --
	(230.37,178.05) --
	(231.68,179.18) --
	(232.96,180.35) --
	(234.20,181.57) --
	(235.39,182.83) --
	(236.54,184.13) --
	(237.65,185.46) --
	(238.71,186.84) --
	(237.30,187.89) --
	(235.88,188.94) --
	(234.47,190.00) --
	(233.05,191.05) --
	(231.64,192.10) --
	(230.23,193.16) --
	(228.81,194.21) --
	(227.40,195.27) --
	(225.99,196.32) --
	(224.57,197.37) --
	(223.16,198.43) --
	(221.74,199.48) --
	(220.33,200.53) --
	(218.92,201.59) --
	(217.50,202.64) --
	(216.09,203.69) --
	(214.68,204.75) --
	(213.26,205.80) --
	(211.85,206.86) --
	(210.43,207.91) --
	(209.02,208.96) --
	(207.61,210.02) --
	(206.19,211.07) --
	(204.78,212.12) --
	(203.37,213.18) --
	(201.95,214.23) --
	(200.54,215.29) --
	(199.12,216.34) --
	(197.71,217.39) --
	(197.71,217.39) --
	cycle;
\definecolor{drawColor}{RGB}{0,0,0}

\node[text=drawColor,anchor=base,inner sep=0pt, outer sep=0pt, scale=  0.71] at (220.56,226.41) {35.19{\%}};

\node[text=drawColor,anchor=base,inner sep=0pt, outer sep=0pt, scale=  0.71] at (174.86,203.48) {64.81{\%}};
\end{scope}
\begin{scope}
\path[clip] (133.80, 20.14) rectangle (261.63,147.98);
\definecolor{fillColor}{RGB}{228,26,28}

\path[fill=fillColor] (197.71, 84.06) --
	(199.47, 84.07) --
	(201.24, 84.07) --
	(203.00, 84.08) --
	(204.76, 84.08) --
	(206.53, 84.09) --
	(208.29, 84.09) --
	(210.05, 84.10) --
	(211.82, 84.10) --
	(213.58, 84.11) --
	(215.34, 84.12) --
	(217.11, 84.12) --
	(218.87, 84.13) --
	(220.63, 84.13) --
	(222.40, 84.14) --
	(224.16, 84.14) --
	(225.92, 84.15) --
	(227.69, 84.15) --
	(229.45, 84.16) --
	(231.21, 84.17) --
	(232.98, 84.17) --
	(234.74, 84.18) --
	(236.50, 84.18) --
	(238.26, 84.19) --
	(240.03, 84.19) --
	(241.79, 84.20) --
	(243.55, 84.20) --
	(245.32, 84.21) --
	(247.08, 84.22) --
	(248.84, 84.22) --
	(248.81, 85.96) --
	(248.71, 87.70) --
	(248.56, 89.44) --
	(248.35, 91.17) --
	(248.08, 92.89) --
	(247.75, 94.60) --
	(247.36, 96.30) --
	(246.91, 97.98) --
	(246.41, 99.65) --
	(245.85,101.30) --
	(245.23,102.93) --
	(244.56,104.54) --
	(243.84,106.12) --
	(243.06,107.68) --
	(242.23,109.22) --
	(241.35,110.72) --
	(240.41,112.19) --
	(239.43,113.63) --
	(238.40,115.03) --
	(237.32,116.40) --
	(236.19,117.73) --
	(235.02,119.02) --
	(233.81,120.27) --
	(232.56,121.48) --
	(231.26,122.65) --
	(229.93,123.77) --
	(228.55,124.84) --
	(227.15,125.87) --
	(225.71,126.85) --
	(224.23,127.78) --
	(222.73,128.66) --
	(221.19,129.48) --
	(219.63,130.26) --
	(218.04,130.98) --
	(216.43,131.64) --
	(214.80,132.25) --
	(213.15,132.81) --
	(211.48,133.30) --
	(209.79,133.75) --
	(208.09,134.13) --
	(206.38,134.45) --
	(204.66,134.72) --
	(202.93,134.93) --
	(201.19,135.07) --
	(199.45,135.16) --
	(197.71,135.19) --
	(197.71,133.43) --
	(197.71,131.67) --
	(197.71,129.90) --
	(197.71,128.14) --
	(197.71,126.38) --
	(197.71,124.61) --
	(197.71,122.85) --
	(197.71,121.09) --
	(197.71,119.32) --
	(197.71,117.56) --
	(197.71,115.80) --
	(197.71,114.03) --
	(197.71,112.27) --
	(197.71,110.51) --
	(197.71,108.75) --
	(197.71,106.98) --
	(197.71,105.22) --
	(197.71,103.46) --
	(197.71,101.69) --
	(197.71, 99.93) --
	(197.71, 98.17) --
	(197.71, 96.40) --
	(197.71, 94.64) --
	(197.71, 92.88) --
	(197.71, 91.11) --
	(197.71, 89.35) --
	(197.71, 87.59) --
	(197.71, 85.82) --
	(197.71, 84.06) --
	(197.71, 84.06) --
	cycle;
\definecolor{fillColor}{RGB}{55,126,184}

\path[fill=fillColor] (197.71, 84.06) --
	(197.71, 85.82) --
	(197.71, 87.59) --
	(197.71, 89.35) --
	(197.71, 91.11) --
	(197.71, 92.88) --
	(197.71, 94.64) --
	(197.71, 96.40) --
	(197.71, 98.17) --
	(197.71, 99.93) --
	(197.71,101.69) --
	(197.71,103.46) --
	(197.71,105.22) --
	(197.71,106.98) --
	(197.71,108.75) --
	(197.71,110.51) --
	(197.71,112.27) --
	(197.71,114.03) --
	(197.71,115.80) --
	(197.71,117.56) --
	(197.71,119.32) --
	(197.71,121.09) --
	(197.71,122.85) --
	(197.71,124.61) --
	(197.71,126.38) --
	(197.71,128.14) --
	(197.71,129.90) --
	(197.71,131.67) --
	(197.71,133.43) --
	(197.71,135.19) --
	(195.98,135.16) --
	(194.24,135.08) --
	(192.52,134.93) --
	(190.79,134.72) --
	(189.08,134.46) --
	(187.37,134.14) --
	(185.68,133.76) --
	(184.00,133.32) --
	(182.34,132.83) --
	(180.70,132.28) --
	(179.07,131.67) --
	(177.47,131.01) --
	(175.88,130.30) --
	(174.33,129.53) --
	(172.80,128.71) --
	(171.30,127.84) --
	(169.83,126.92) --
	(168.39,125.95) --
	(166.99,124.93) --
	(165.62,123.87) --
	(164.29,122.76) --
	(162.99,121.60) --
	(161.74,120.40) --
	(160.53,119.16) --
	(159.36,117.88) --
	(158.23,116.56) --
	(157.16,115.20) --
	(156.12,113.81) --
	(155.14,112.38) --
	(154.20,110.92) --
	(153.32,109.43) --
	(152.48,107.91) --
	(151.70,106.36) --
	(150.97,104.79) --
	(150.29,103.19) --
	(149.67,101.57) --
	(149.10, 99.93) --
	(148.59, 98.27) --
	(148.14, 96.60) --
	(147.74, 94.91) --
	(147.40, 93.21) --
	(147.12, 91.50) --
	(146.90, 89.78) --
	(146.73, 88.05) --
	(146.63, 86.32) --
	(146.58, 84.59) --
	(146.59, 82.85) --
	(146.66, 81.12) --
	(146.79, 79.39) --
	(146.98, 77.66) --
	(147.23, 75.95) --
	(147.53, 74.24) --
	(147.89, 72.54) --
	(148.31, 70.86) --
	(148.79, 69.19) --
	(149.32, 67.54) --
	(149.91, 65.91) --
	(150.55, 64.30) --
	(151.25, 62.71) --
	(152.00, 61.15) --
	(152.80, 59.61) --
	(153.66, 58.10) --
	(154.56, 56.62) --
	(155.52, 55.17) --
	(156.52, 53.76) --
	(157.58, 52.38) --
	(158.67, 51.04) --
	(159.82, 49.73) --
	(161.00, 48.47) --
	(162.23, 47.24) --
	(163.50, 46.06) --
	(164.81, 44.92) --
	(166.16, 43.83) --
	(167.54, 42.78) --
	(168.96, 41.78) --
	(170.41, 40.83) --
	(171.89, 39.93) --
	(173.40, 39.08) --
	(174.94, 38.28) --
	(176.51, 37.53) --
	(178.10, 36.84) --
	(179.71, 36.20) --
	(181.34, 35.62) --
	(183.00, 35.09) --
	(184.66, 34.62) --
	(186.35, 34.21) --
	(188.05, 33.85) --
	(189.76, 33.55) --
	(191.47, 33.31) --
	(193.20, 33.13) --
	(194.93, 33.00) --
	(196.66, 32.94) --
	(198.40, 32.93) --
	(200.13, 32.98) --
	(201.86, 33.10) --
	(203.59, 33.27) --
	(205.31, 33.50) --
	(207.02, 33.78) --
	(208.72, 34.13) --
	(210.40, 34.53) --
	(212.08, 34.99) --
	(213.73, 35.50) --
	(215.37, 36.07) --
	(216.99, 36.70) --
	(218.58, 37.38) --
	(220.16, 38.12) --
	(221.70, 38.90) --
	(223.22, 39.74) --
	(224.71, 40.63) --
	(226.16, 41.58) --
	(227.59, 42.56) --
	(228.98, 43.60) --
	(230.33, 44.69) --
	(231.65, 45.82) --
	(232.93, 46.99) --
	(234.17, 48.20) --
	(235.36, 49.46) --
	(236.51, 50.76) --
	(237.62, 52.09) --
	(238.68, 53.47) --
	(239.69, 54.87) --
	(240.66, 56.31) --
	(241.58, 57.79) --
	(242.44, 59.29) --
	(243.26, 60.82) --
	(244.02, 62.38) --
	(244.73, 63.96) --
	(245.38, 65.57) --
	(245.98, 67.20) --
	(246.53, 68.84) --
	(247.02, 70.51) --
	(247.45, 72.19) --
	(247.82, 73.88) --
	(248.14, 75.59) --
	(248.40, 77.30) --
	(248.60, 79.02) --
	(248.74, 80.75) --
	(248.82, 82.49) --
	(248.84, 84.22) --
	(247.08, 84.22) --
	(245.32, 84.21) --
	(243.55, 84.20) --
	(241.79, 84.20) --
	(240.03, 84.19) --
	(238.26, 84.19) --
	(236.50, 84.18) --
	(234.74, 84.18) --
	(232.98, 84.17) --
	(231.21, 84.17) --
	(229.45, 84.16) --
	(227.69, 84.15) --
	(225.92, 84.15) --
	(224.16, 84.14) --
	(222.40, 84.14) --
	(220.63, 84.13) --
	(218.87, 84.13) --
	(217.11, 84.12) --
	(215.34, 84.12) --
	(213.58, 84.11) --
	(211.82, 84.10) --
	(210.05, 84.10) --
	(208.29, 84.09) --
	(206.53, 84.09) --
	(204.76, 84.08) --
	(203.00, 84.08) --
	(201.24, 84.07) --
	(199.47, 84.07) --
	(197.71, 84.06) --
	(197.71, 84.06) --
	cycle;
\definecolor{drawColor}{RGB}{0,0,0}

\node[text=drawColor,anchor=base,inner sep=0pt, outer sep=0pt, scale=  0.71] at (215.76, 99.72) {24.95{\%}};

\node[text=drawColor,anchor=base,inner sep=0pt, outer sep=0pt, scale=  0.71] at (179.66, 63.50) {75.05{\%}};
\end{scope}
\begin{scope}
\path[clip] (267.13,153.48) rectangle (394.96,281.31);
\definecolor{fillColor}{RGB}{228,26,28}

\path[fill=fillColor] (331.04,217.39) --
	(332.80,217.59) --
	(334.55,217.78) --
	(336.30,217.97) --
	(338.05,218.16) --
	(339.81,218.36) --
	(341.56,218.55) --
	(343.31,218.74) --
	(345.06,218.94) --
	(346.82,219.13) --
	(348.57,219.32) --
	(350.32,219.52) --
	(352.07,219.71) --
	(353.83,219.90) --
	(355.58,220.09) --
	(357.33,220.29) --
	(359.08,220.48) --
	(360.84,220.67) --
	(362.59,220.87) --
	(364.34,221.06) --
	(366.10,221.25) --
	(367.85,221.44) --
	(369.60,221.64) --
	(371.35,221.83) --
	(373.11,222.02) --
	(374.86,222.22) --
	(376.61,222.41) --
	(378.36,222.60) --
	(380.12,222.80) --
	(381.87,222.99) --
	(381.65,224.71) --
	(381.37,226.43) --
	(381.04,228.13) --
	(380.64,229.82) --
	(380.19,231.50) --
	(379.68,233.16) --
	(379.12,234.81) --
	(378.50,236.43) --
	(377.83,238.03) --
	(377.10,239.61) --
	(376.32,241.16) --
	(375.48,242.68) --
	(374.60,244.18) --
	(373.66,245.64) --
	(372.68,247.08) --
	(371.65,248.47) --
	(370.57,249.83) --
	(369.44,251.16) --
	(368.27,252.44) --
	(367.06,253.69) --
	(365.81,254.89) --
	(364.51,256.05) --
	(363.18,257.16) --
	(361.81,258.23) --
	(360.40,259.26) --
	(358.96,260.23) --
	(357.49,261.15) --
	(355.99,262.03) --
	(354.46,262.85) --
	(352.90,263.62) --
	(351.32,264.33) --
	(349.71,264.99) --
	(348.09,265.60) --
	(346.44,266.15) --
	(344.77,266.65) --
	(343.09,267.09) --
	(341.40,267.47) --
	(339.69,267.79) --
	(337.97,268.05) --
	(336.25,268.26) --
	(334.52,268.41) --
	(332.78,268.50) --
	(331.04,268.53) --
	(331.04,266.76) --
	(331.04,265.00) --
	(331.04,263.24) --
	(331.04,261.47) --
	(331.04,259.71) --
	(331.04,257.95) --
	(331.04,256.18) --
	(331.04,254.42) --
	(331.04,252.66) --
	(331.04,250.89) --
	(331.04,249.13) --
	(331.04,247.37) --
	(331.04,245.60) --
	(331.04,243.84) --
	(331.04,242.08) --
	(331.04,240.31) --
	(331.04,238.55) --
	(331.04,236.79) --
	(331.04,235.02) --
	(331.04,233.26) --
	(331.04,231.50) --
	(331.04,229.73) --
	(331.04,227.97) --
	(331.04,226.21) --
	(331.04,224.45) --
	(331.04,222.68) --
	(331.04,220.92) --
	(331.04,219.16) --
	(331.04,217.39) --
	(331.04,217.39) --
	cycle;
\definecolor{fillColor}{RGB}{55,126,184}

\path[fill=fillColor] (331.04,217.39) --
	(331.04,219.16) --
	(331.04,220.92) --
	(331.04,222.68) --
	(331.04,224.45) --
	(331.04,226.21) --
	(331.04,227.97) --
	(331.04,229.73) --
	(331.04,231.50) --
	(331.04,233.26) --
	(331.04,235.02) --
	(331.04,236.79) --
	(331.04,238.55) --
	(331.04,240.31) --
	(331.04,242.08) --
	(331.04,243.84) --
	(331.04,245.60) --
	(331.04,247.37) --
	(331.04,249.13) --
	(331.04,250.89) --
	(331.04,252.66) --
	(331.04,254.42) --
	(331.04,256.18) --
	(331.04,257.95) --
	(331.04,259.71) --
	(331.04,261.47) --
	(331.04,263.24) --
	(331.04,265.00) --
	(331.04,266.76) --
	(331.04,268.53) --
	(329.31,268.50) --
	(327.57,268.41) --
	(325.84,268.26) --
	(324.12,268.05) --
	(322.40,267.79) --
	(320.70,267.47) --
	(319.00,267.09) --
	(317.32,266.65) --
	(315.66,266.16) --
	(314.01,265.61) --
	(312.38,265.00) --
	(310.78,264.34) --
	(309.20,263.62) --
	(307.64,262.85) --
	(306.11,262.03) --
	(304.61,261.16) --
	(303.14,260.24) --
	(301.70,259.27) --
	(300.29,258.25) --
	(298.92,257.18) --
	(297.59,256.07) --
	(296.30,254.91) --
	(295.04,253.71) --
	(293.83,252.46) --
	(292.66,251.18) --
	(291.54,249.86) --
	(290.46,248.50) --
	(289.43,247.10) --
	(288.44,245.67) --
	(287.51,244.21) --
	(286.62,242.71) --
	(285.79,241.19) --
	(285.00,239.64) --
	(284.28,238.07) --
	(283.60,236.47) --
	(282.98,234.84) --
	(282.42,233.20) --
	(281.91,231.54) --
	(281.46,229.87) --
	(281.06,228.18) --
	(280.72,226.47) --
	(280.44,224.76) --
	(280.22,223.04) --
	(280.06,221.31) --
	(279.96,219.57) --
	(279.91,217.84) --
	(279.93,216.10) --
	(280.00,214.37) --
	(280.13,212.64) --
	(280.32,210.91) --
	(280.57,209.19) --
	(280.88,207.48) --
	(281.25,205.79) --
	(281.67,204.10) --
	(282.15,202.43) --
	(282.68,200.78) --
	(283.28,199.15) --
	(283.92,197.54) --
	(284.62,195.95) --
	(285.38,194.39) --
	(286.19,192.85) --
	(287.05,191.34) --
	(287.95,189.86) --
	(288.91,188.41) --
	(289.92,187.00) --
	(290.98,185.62) --
	(292.08,184.28) --
	(293.23,182.98) --
	(294.42,181.71) --
	(295.65,180.49) --
	(296.92,179.31) --
	(298.24,178.17) --
	(299.59,177.08) --
	(300.97,176.04) --
	(302.39,175.04) --
	(303.85,174.09) --
	(305.33,173.19) --
	(306.85,172.35) --
	(308.39,171.55) --
	(309.96,170.81) --
	(311.56,170.12) --
	(313.17,169.48) --
	(314.81,168.91) --
	(316.47,168.38) --
	(318.14,167.92) --
	(319.82,167.51) --
	(321.53,167.15) --
	(323.24,166.86) --
	(324.96,166.62) --
	(326.68,166.45) --
	(328.42,166.33) --
	(330.15,166.27) --
	(331.89,166.27) --
	(333.62,166.32) --
	(335.36,166.44) --
	(337.08,166.62) --
	(338.80,166.85) --
	(340.51,167.14) --
	(342.21,167.50) --
	(343.90,167.90) --
	(345.58,168.37) --
	(347.23,168.89) --
	(348.87,169.47) --
	(350.49,170.10) --
	(352.08,170.79) --
	(353.65,171.53) --
	(355.19,172.32) --
	(356.71,173.17) --
	(358.20,174.07) --
	(359.65,175.01) --
	(361.08,176.01) --
	(362.46,177.05) --
	(363.81,178.14) --
	(365.13,179.28) --
	(366.40,180.46) --
	(367.64,181.68) --
	(368.83,182.94) --
	(369.98,184.24) --
	(371.08,185.58) --
	(372.14,186.96) --
	(373.14,188.37) --
	(374.11,189.82) --
	(375.02,191.30) --
	(375.88,192.81) --
	(376.69,194.34) --
	(377.44,195.91) --
	(378.15,197.49) --
	(378.79,199.10) --
	(379.39,200.74) --
	(379.92,202.39) --
	(380.41,204.06) --
	(380.83,205.74) --
	(381.20,207.44) --
	(381.51,209.15) --
	(381.76,210.86) --
	(381.95,212.59) --
	(382.08,214.32) --
	(382.16,216.05) --
	(382.17,217.79) --
	(382.13,219.53) --
	(382.03,221.26) --
	(381.87,222.99) --
	(380.12,222.80) --
	(378.36,222.60) --
	(376.61,222.41) --
	(374.86,222.22) --
	(373.11,222.02) --
	(371.35,221.83) --
	(369.60,221.64) --
	(367.85,221.44) --
	(366.10,221.25) --
	(364.34,221.06) --
	(362.59,220.87) --
	(360.84,220.67) --
	(359.08,220.48) --
	(357.33,220.29) --
	(355.58,220.09) --
	(353.83,219.90) --
	(352.07,219.71) --
	(350.32,219.52) --
	(348.57,219.32) --
	(346.82,219.13) --
	(345.06,218.94) --
	(343.31,218.74) --
	(341.56,218.55) --
	(339.81,218.36) --
	(338.05,218.16) --
	(336.30,217.97) --
	(334.55,217.78) --
	(332.80,217.59) --
	(331.04,217.39) --
	(331.04,217.39) --
	cycle;
\definecolor{drawColor}{RGB}{0,0,0}

\node[text=drawColor,anchor=base,inner sep=0pt, outer sep=0pt, scale=  0.71] at (348.10,233.98) {23.25{\%}};

\node[text=drawColor,anchor=base,inner sep=0pt, outer sep=0pt, scale=  0.71] at (313.98,195.90) {76.75{\%}};
\end{scope}
\begin{scope}
\path[clip] (267.13, 20.14) rectangle (394.96,147.98);
\definecolor{fillColor}{RGB}{228,26,28}

\path[fill=fillColor] (331.04, 84.06) --
	(332.71, 83.49) --
	(334.38, 82.91) --
	(336.05, 82.34) --
	(337.71, 81.77) --
	(339.38, 81.20) --
	(341.05, 80.62) --
	(342.72, 80.05) --
	(344.38, 79.48) --
	(346.05, 78.90) --
	(347.72, 78.33) --
	(349.39, 77.76) --
	(351.05, 77.18) --
	(352.72, 76.61) --
	(354.39, 76.04) --
	(356.06, 75.46) --
	(357.72, 74.89) --
	(359.39, 74.32) --
	(361.06, 73.75) --
	(362.73, 73.17) --
	(364.39, 72.60) --
	(366.06, 72.03) --
	(367.73, 71.45) --
	(369.40, 70.88) --
	(371.06, 70.31) --
	(372.73, 69.73) --
	(374.40, 69.16) --
	(376.07, 68.59) --
	(377.73, 68.02) --
	(379.40, 67.44) --
	(379.94, 69.09) --
	(380.42, 70.76) --
	(380.84, 72.45) --
	(381.21, 74.14) --
	(381.51, 75.85) --
	(381.76, 77.57) --
	(381.95, 79.30) --
	(382.09, 81.03) --
	(382.16, 82.76) --
	(382.17, 84.50) --
	(382.13, 86.24) --
	(382.03, 87.97) --
	(381.86, 89.70) --
	(381.64, 91.42) --
	(381.36, 93.13) --
	(381.03, 94.84) --
	(380.63, 96.53) --
	(380.18, 98.21) --
	(379.67, 99.87) --
	(379.11,101.51) --
	(378.49,103.13) --
	(377.81,104.73) --
	(377.08,106.31) --
	(376.30,107.86) --
	(375.47,109.38) --
	(374.58,110.87) --
	(373.65,112.34) --
	(372.66,113.77) --
	(371.63,115.16) --
	(370.55,116.52) --
	(369.43,117.84) --
	(368.26,119.13) --
	(367.04,120.37) --
	(365.79,121.57) --
	(364.50,122.73) --
	(363.16,123.85) --
	(361.79,124.91) --
	(360.39,125.93) --
	(358.95,126.91) --
	(357.48,127.83) --
	(355.98,128.70) --
	(354.45,129.52) --
	(352.89,130.29) --
	(351.31,131.01) --
	(349.70,131.67) --
	(348.08,132.27) --
	(346.43,132.82) --
	(344.77,133.32) --
	(343.08,133.76) --
	(341.39,134.14) --
	(339.68,134.46) --
	(337.97,134.72) --
	(336.24,134.93) --
	(334.51,135.08) --
	(332.78,135.16) --
	(331.04,135.19) --
	(331.04,133.43) --
	(331.04,131.67) --
	(331.04,129.90) --
	(331.04,128.14) --
	(331.04,126.38) --
	(331.04,124.61) --
	(331.04,122.85) --
	(331.04,121.09) --
	(331.04,119.32) --
	(331.04,117.56) --
	(331.04,115.80) --
	(331.04,114.03) --
	(331.04,112.27) --
	(331.04,110.51) --
	(331.04,108.75) --
	(331.04,106.98) --
	(331.04,105.22) --
	(331.04,103.46) --
	(331.04,101.69) --
	(331.04, 99.93) --
	(331.04, 98.17) --
	(331.04, 96.40) --
	(331.04, 94.64) --
	(331.04, 92.88) --
	(331.04, 91.11) --
	(331.04, 89.35) --
	(331.04, 87.59) --
	(331.04, 85.82) --
	(331.04, 84.06) --
	(331.04, 84.06) --
	cycle;
\definecolor{fillColor}{RGB}{55,126,184}

\path[fill=fillColor] (331.04, 84.06) --
	(331.04, 85.82) --
	(331.04, 87.59) --
	(331.04, 89.35) --
	(331.04, 91.11) --
	(331.04, 92.88) --
	(331.04, 94.64) --
	(331.04, 96.40) --
	(331.04, 98.17) --
	(331.04, 99.93) --
	(331.04,101.69) --
	(331.04,103.46) --
	(331.04,105.22) --
	(331.04,106.98) --
	(331.04,108.75) --
	(331.04,110.51) --
	(331.04,112.27) --
	(331.04,114.03) --
	(331.04,115.80) --
	(331.04,117.56) --
	(331.04,119.32) --
	(331.04,121.09) --
	(331.04,122.85) --
	(331.04,124.61) --
	(331.04,126.38) --
	(331.04,128.14) --
	(331.04,129.90) --
	(331.04,131.67) --
	(331.04,133.43) --
	(331.04,135.19) --
	(329.31,135.16) --
	(327.57,135.08) --
	(325.84,134.93) --
	(324.12,134.72) --
	(322.40,134.46) --
	(320.70,134.14) --
	(319.00,133.75) --
	(317.32,133.32) --
	(315.66,132.82) --
	(314.01,132.27) --
	(312.38,131.67) --
	(310.78,131.00) --
	(309.19,130.29) --
	(307.64,129.52) --
	(306.11,128.70) --
	(304.60,127.83) --
	(303.13,126.90) --
	(301.69,125.93) --
	(300.29,124.91) --
	(298.92,123.84) --
	(297.59,122.73) --
	(296.29,121.57) --
	(295.04,120.37) --
	(293.83,119.13) --
	(292.66,117.84) --
	(291.53,116.52) --
	(290.45,115.16) --
	(289.42,113.76) --
	(288.44,112.33) --
	(287.50,110.87) --
	(286.62,109.37) --
	(285.78,107.85) --
	(285.00,106.30) --
	(284.27,104.72) --
	(283.60,103.12) --
	(282.98,101.50) --
	(282.41, 99.86) --
	(281.90, 98.20) --
	(281.45, 96.52) --
	(281.06, 94.83) --
	(280.72, 93.13) --
	(280.44, 91.41) --
	(280.22, 89.69) --
	(280.06, 87.96) --
	(279.96, 86.23) --
	(279.91, 84.49) --
	(279.93, 82.76) --
	(280.00, 81.02) --
	(280.13, 79.29) --
	(280.32, 77.56) --
	(280.57, 75.84) --
	(280.88, 74.14) --
	(281.25, 72.44) --
	(281.67, 70.75) --
	(282.15, 69.09) --
	(282.69, 67.43) --
	(283.28, 65.80) --
	(283.93, 64.19) --
	(284.63, 62.60) --
	(285.39, 61.04) --
	(286.19, 59.50) --
	(287.05, 57.99) --
	(287.97, 56.51) --
	(288.93, 55.07) --
	(289.93, 53.65) --
	(290.99, 52.27) --
	(292.09, 50.93) --
	(293.24, 49.63) --
	(294.43, 48.37) --
	(295.67, 47.14) --
	(296.94, 45.96) --
	(298.25, 44.83) --
	(299.60, 43.74) --
	(300.99, 42.69) --
	(302.41, 41.69) --
	(303.87, 40.75) --
	(305.35, 39.85) --
	(306.87, 39.00) --
	(308.42, 38.21) --
	(309.99, 37.47) --
	(311.58, 36.78) --
	(313.20, 36.14) --
	(314.83, 35.56) --
	(316.49, 35.04) --
	(318.16, 34.58) --
	(319.85, 34.17) --
	(321.55, 33.82) --
	(323.26, 33.52) --
	(324.98, 33.29) --
	(326.71, 33.11) --
	(328.44, 32.99) --
	(330.18, 32.94) --
	(331.92, 32.94) --
	(333.65, 32.99) --
	(335.38, 33.11) --
	(337.11, 33.29) --
	(338.83, 33.52) --
	(340.54, 33.82) --
	(342.24, 34.17) --
	(343.93, 34.58) --
	(345.61, 35.05) --
	(347.26, 35.57) --
	(348.90, 36.15) --
	(350.52, 36.78) --
	(352.11, 37.47) --
	(353.68, 38.21) --
	(355.22, 39.01) --
	(356.74, 39.85) --
	(358.23, 40.75) --
	(359.68, 41.70) --
	(361.10, 42.70) --
	(362.49, 43.74) --
	(363.84, 44.83) --
	(365.15, 45.97) --
	(366.43, 47.15) --
	(367.66, 48.37) --
	(368.85, 49.64) --
	(370.00, 50.94) --
	(371.10, 52.28) --
	(372.16, 53.66) --
	(373.17, 55.07) --
	(374.13, 56.52) --
	(375.04, 58.00) --
	(375.90, 59.51) --
	(376.70, 61.05) --
	(377.46, 62.61) --
	(378.16, 64.20) --
	(378.81, 65.81) --
	(379.40, 67.44) --
	(377.73, 68.02) --
	(376.07, 68.59) --
	(374.40, 69.16) --
	(372.73, 69.73) --
	(371.06, 70.31) --
	(369.40, 70.88) --
	(367.73, 71.45) --
	(366.06, 72.03) --
	(364.39, 72.60) --
	(362.73, 73.17) --
	(361.06, 73.75) --
	(359.39, 74.32) --
	(357.72, 74.89) --
	(356.06, 75.46) --
	(354.39, 76.04) --
	(352.72, 76.61) --
	(351.05, 77.18) --
	(349.39, 77.76) --
	(347.72, 78.33) --
	(346.05, 78.90) --
	(344.38, 79.48) --
	(342.72, 80.05) --
	(341.05, 80.62) --
	(339.38, 81.20) --
	(337.71, 81.77) --
	(336.05, 82.34) --
	(334.38, 82.91) --
	(332.71, 83.49) --
	(331.04, 84.06) --
	(331.04, 84.06) --
	cycle;
\definecolor{drawColor}{RGB}{0,0,0}

\node[text=drawColor,anchor=base,inner sep=0pt, outer sep=0pt, scale=  0.71] at (351.85, 96.46) {30.27{\%}};

\node[text=drawColor,anchor=base,inner sep=0pt, outer sep=0pt, scale=  0.71] at (310.23, 66.76) {69.73{\%}};
\end{scope}
\begin{scope}
\path[clip] (  0.46,281.31) rectangle (128.30,289.08);
\definecolor{drawColor}{RGB}{0,0,0}

\node[text=drawColor,anchor=base,inner sep=0pt, outer sep=0pt, scale=  0.70] at ( 64.38,283) {\textbf{Región 1}};
\end{scope}
\begin{scope}
\path[clip] (133.80,281.31) rectangle (261.63,289.08);
\definecolor{drawColor}{RGB}{0,0,0}

\node[text=drawColor,anchor=base,inner sep=0pt, outer sep=0pt, scale=  0.70] at (197.71,283) {\textbf{Región 2}};
\end{scope}
\begin{scope}
\path[clip] (267.13,281.31) rectangle (394.96,289.08);
\definecolor{drawColor}{RGB}{0,0,0}

\node[text=drawColor,anchor=base,inner sep=0pt, outer sep=0pt, scale=  0.70] at (331.04,283) {\textbf{Región 3}};
\end{scope}
\begin{scope}
\path[clip] (394.96,153.48) rectangle (433.16,281.31);
\definecolor{drawColor}{RGB}{0,0,0}

\node[text=drawColor,anchor=base,inner sep=0pt, outer sep=0pt, scale=  0.70] at (414.06,214.36) {Migrantes};
\end{scope}
\begin{scope}
\path[clip] (394.96, 20.14) rectangle (433.16,147.98);
\definecolor{drawColor}{RGB}{0,0,0}

\node[text=drawColor,anchor=base,inner sep=0pt, outer sep=0pt, scale=  0.70] at (414.06, 81.03) {Nativos};
\end{scope}
\begin{scope}
\path[clip] (  0.00,  0.00) rectangle (433.62,289.08);
\definecolor{fillColor}{RGB}{228,26,28}

\path[fill=fillColor] (115.29,  0.71) rectangle (128.33, 13.74);
\end{scope}
\begin{scope}
\path[clip] (  0.00,  0.00) rectangle (433.62,289.08);
\definecolor{fillColor}{RGB}{55,126,184}

\path[fill=fillColor] (202.82,  0.71) rectangle (215.85, 13.74);
\end{scope}
\begin{scope}
\path[clip] (  0.00,  0.00) rectangle (433.62,289.08);
\definecolor{drawColor}{RGB}{0,0,0}

\node[text=drawColor,anchor=base west,inner sep=0pt, outer sep=0pt, scale=  0.60] at (134.54,  4.20) {Calificación alta};
\end{scope}
\begin{scope}
\path[clip] (  0.00,  0.00) rectangle (433.62,289.08);
\definecolor{drawColor}{RGB}{0,0,0}

\node[text=drawColor,anchor=base west,inner sep=0pt, outer sep=0pt, scale=  0.60] at (222.07,  4.20) {Calificación baja};
\end{scope}
\end{tikzpicture}
 
\end{center}
\begin{flushleft}
\begin{scriptsize}
Fuente: Elaboración propia en base a EPH.\\
Nota: Los profesionales y técnicos son considerados de calificación alta mientras que los operativos y no calificados son considerados de calificación baja. Los migrantes están definidos como personas que vivían hace cinco años en otra provincia. Los nativos están definidos como personas que nacieron y viven en la misma provincia. Las estimaciones corresponden al período desde el segundo trimestre de 2016 hasta el cuarto trimestre de 2019.
\end{scriptsize}
\end{flushleft}
\end{figure}

Tomando como referencia a la población ocupada, dentro de la población activa considerada anteriormente, se puede observar en la Figura \ref{figure:calif_mig} la calificación de los empleados nativos y migrantes. Para todas las regiones los migrantes ocupan puestos de mayor calificación que los nativos, en donde es más notoria esta diferencia es en la Región Sur.

Estas diferencias en calificación muestran una migración interna de personas con un nivel de calificación que es más elevado, en promedio, que el de las personas que habitan sus localidades de destino. La migración de personas altamente calificadas puede darse en respuesta de una ineficiencia del mercado laboral de origen para absorber esta mano de obra profesional o técnica, ya sea por falta de puestos laborales como por salarios bajos que no brinden un retorno acorde a la idoneidad de los trabajadores.

\subsection{Prima salarial}
La decisión de la migración está vinculada con la maximización de la utilidad, la cual está intimamente ligada con los ingresos potenciales que puede llegar a percibir la persona. Mientras más elevada sea la brecha salarial entre la región de origen y la de destino, mayor serán los incentivos economicos para migrar.

Se analiza en la Figura \ref{figure:prima_sal} los salarios promedios de los nativos y migrantes de las regiones para distintas calificaciones. 

Las Técnicos, Operativos y No calificados que nacieron y viven en la Región Sur tienen una remuneración mayor que los que decidieron vivir en la Región Norte o Centro. Los profesionales que nacieron en la Región Sur encuentran un retorno superior viviendo en la Región Norte, lo que implicaría a priori un incentivo salarial para migrar hacia la misma.

Las Técnicos, Operativos y No calificados que nacieron y viven en la Región Norte tienen una remuneración menor que los que decidieron vivir en la Región Centro o Sur. Los profesionales que nacieron y viven en la Región Norte poseen una retribución superior que los que decidieron vivir en la región Centro, sin embargo, menor que los que decidieron vivir en la región Sur.

Por último, los No calificados, Operativos y Técnicos que nacieron y viven en la Región Centro tienen un salario superior que los que decidieron vivir en la región Norte. Por otro lado, para las mismas calificaciones mencionadas anteriormente, los que decidieron vivir en el Sur poseen una remuneración mayor en términos relativos a las otras regiones. Los profesionales nacidos en la región Centro tienen una remunreación muy similar independientemente de la región donde habitan, siendo levemente superior la de los que viven en la Region Sur.

\begin{figure}[!htbp]
\begin{center}
\caption{\\Diferencia de ingreso laboral por hora de nativos y migrantes}
\label{figure:prima_sal}
% Created by tikzDevice version 0.12.3.1 on 2021-07-06 19:35:49
% !TEX encoding = UTF-8 Unicode
\begin{tikzpicture}[x=1pt,y=1pt]
\definecolor{fillColor}{RGB}{255,255,255}
\path[use as bounding box,fill=fillColor,fill opacity=0.00] (0,0) rectangle (433.62,216.81);
\begin{scope}
\path[clip] (  0.00,  0.00) rectangle (433.62,216.81);
\definecolor{drawColor}{RGB}{255,255,255}
\definecolor{fillColor}{RGB}{255,255,255}

\path[draw=drawColor,line width= 0.6pt,line join=round,line cap=round,fill=fillColor] (  0.00,  0.00) rectangle (433.62,216.81);
\end{scope}
\begin{scope}
\path[clip] ( 36.11, 78.54) rectangle (129.99,194.74);
\definecolor{drawColor}{RGB}{255,255,255}

\path[draw=drawColor,line width= 0.3pt,line join=round] ( 36.11, 98.91) --
	(129.99, 98.91);

\path[draw=drawColor,line width= 0.3pt,line join=round] ( 36.11,129.09) --
	(129.99,129.09);

\path[draw=drawColor,line width= 0.3pt,line join=round] ( 36.11,159.27) --
	(129.99,159.27);

\path[draw=drawColor,line width= 0.3pt,line join=round] ( 36.11,189.46) --
	(129.99,189.46);

\path[draw=drawColor,line width= 0.6pt,line join=round] ( 36.11, 83.82) --
	(129.99, 83.82);

\path[draw=drawColor,line width= 0.6pt,line join=round] ( 36.11,114.00) --
	(129.99,114.00);

\path[draw=drawColor,line width= 0.6pt,line join=round] ( 36.11,144.18) --
	(129.99,144.18);

\path[draw=drawColor,line width= 0.6pt,line join=round] ( 36.11,174.37) --
	(129.99,174.37);

\path[draw=drawColor,line width= 0.6pt,line join=round] ( 53.71, 78.54) --
	( 53.71,194.74);

\path[draw=drawColor,line width= 0.6pt,line join=round] ( 83.05, 78.54) --
	( 83.05,194.74);

\path[draw=drawColor,line width= 0.6pt,line join=round] (112.39, 78.54) --
	(112.39,194.74);
\definecolor{drawColor}{gray}{0.20}
\definecolor{fillColor}{gray}{0.20}

\path[draw=drawColor,line width= 0.4pt,line join=round,line cap=round,fill=fillColor] ( 53.71,175.69) circle (  1.96);

\path[draw=drawColor,line width= 0.4pt,line join=round,line cap=round,fill=fillColor] ( 53.71,179.73) circle (  1.96);

\path[draw=drawColor,line width= 0.4pt,line join=round,line cap=round,fill=fillColor] ( 53.71,175.69) circle (  1.96);

\path[draw=drawColor,line width= 0.4pt,line join=round,line cap=round,fill=fillColor] ( 53.71,180.28) circle (  1.96);

\path[draw=drawColor,line width= 0.4pt,line join=round,line cap=round,fill=fillColor] ( 53.71,177.77) circle (  1.96);

\path[draw=drawColor,line width= 0.4pt,line join=round,line cap=round,fill=fillColor] ( 53.71,188.21) circle (  1.96);

\path[draw=drawColor,line width= 0.4pt,line join=round,line cap=round,fill=fillColor] ( 53.71,185.23) circle (  1.96);

\path[draw=drawColor,line width= 0.4pt,line join=round,line cap=round,fill=fillColor] ( 53.71,171.51) circle (  1.96);

\path[draw=drawColor,line width= 0.4pt,line join=round,line cap=round,fill=fillColor] ( 53.71,170.05) circle (  1.96);

\path[draw=drawColor,line width= 0.4pt,line join=round,line cap=round,fill=fillColor] ( 53.71,170.05) circle (  1.96);

\path[draw=drawColor,line width= 0.4pt,line join=round,line cap=round,fill=fillColor] ( 53.71,170.05) circle (  1.96);

\path[draw=drawColor,line width= 0.4pt,line join=round,line cap=round,fill=fillColor] ( 53.71,175.80) circle (  1.96);

\path[draw=drawColor,line width= 0.4pt,line join=round,line cap=round,fill=fillColor] ( 53.71,178.12) circle (  1.96);

\path[draw=drawColor,line width= 0.4pt,line join=round,line cap=round,fill=fillColor] ( 53.71,170.46) circle (  1.96);

\path[draw=drawColor,line width= 0.4pt,line join=round,line cap=round,fill=fillColor] ( 53.71,169.49) circle (  1.96);

\path[draw=drawColor,line width= 0.4pt,line join=round,line cap=round,fill=fillColor] ( 53.71,169.49) circle (  1.96);

\path[draw=drawColor,line width= 0.4pt,line join=round,line cap=round,fill=fillColor] ( 53.71,181.85) circle (  1.96);

\path[draw=drawColor,line width= 0.4pt,line join=round,line cap=round,fill=fillColor] ( 53.71,181.85) circle (  1.96);

\path[draw=drawColor,line width= 0.4pt,line join=round,line cap=round,fill=fillColor] ( 53.71,181.85) circle (  1.96);

\path[draw=drawColor,line width= 0.4pt,line join=round,line cap=round,fill=fillColor] ( 53.71,173.45) circle (  1.96);

\path[draw=drawColor,line width= 0.4pt,line join=round,line cap=round,fill=fillColor] ( 53.71,175.32) circle (  1.96);

\path[draw=drawColor,line width= 0.4pt,line join=round,line cap=round,fill=fillColor] ( 53.71,176.95) circle (  1.96);

\path[draw=drawColor,line width= 0.4pt,line join=round,line cap=round,fill=fillColor] ( 53.71,176.95) circle (  1.96);

\path[draw=drawColor,line width= 0.4pt,line join=round,line cap=round,fill=fillColor] ( 53.71,174.01) circle (  1.96);

\path[draw=drawColor,line width= 0.4pt,line join=round,line cap=round,fill=fillColor] ( 53.71,179.83) circle (  1.96);

\path[draw=drawColor,line width= 0.4pt,line join=round,line cap=round,fill=fillColor] ( 53.71,177.87) circle (  1.96);

\path[draw=drawColor,line width= 0.4pt,line join=round,line cap=round,fill=fillColor] ( 53.71,174.74) circle (  1.96);

\path[draw=drawColor,line width= 0.4pt,line join=round,line cap=round,fill=fillColor] ( 53.71,171.60) circle (  1.96);

\path[draw=drawColor,line width= 0.4pt,line join=round,line cap=round,fill=fillColor] ( 53.71,175.26) circle (  1.96);

\path[draw=drawColor,line width= 0.4pt,line join=round,line cap=round,fill=fillColor] ( 53.71,182.48) circle (  1.96);

\path[draw=drawColor,line width= 0.4pt,line join=round,line cap=round,fill=fillColor] ( 53.71,185.30) circle (  1.96);

\path[draw=drawColor,line width= 0.4pt,line join=round,line cap=round,fill=fillColor] ( 53.71,185.30) circle (  1.96);

\path[draw=drawColor,line width= 0.4pt,line join=round,line cap=round,fill=fillColor] ( 53.71,185.30) circle (  1.96);

\path[draw=drawColor,line width= 0.4pt,line join=round,line cap=round,fill=fillColor] ( 53.71,182.48) circle (  1.96);

\path[draw=drawColor,line width= 0.4pt,line join=round,line cap=round,fill=fillColor] ( 53.71,178.53) circle (  1.96);

\path[draw=drawColor,line width= 0.4pt,line join=round,line cap=round,fill=fillColor] ( 53.71,182.48) circle (  1.96);

\path[draw=drawColor,line width= 0.4pt,line join=round,line cap=round,fill=fillColor] ( 53.71,172.61) circle (  1.96);

\path[draw=drawColor,line width= 0.4pt,line join=round,line cap=round,fill=fillColor] ( 53.71,173.68) circle (  1.96);

\path[draw=drawColor,line width= 0.4pt,line join=round,line cap=round,fill=fillColor] ( 53.71,184.46) circle (  1.96);

\path[draw=drawColor,line width= 0.4pt,line join=round,line cap=round,fill=fillColor] ( 53.71,184.46) circle (  1.96);

\path[draw=drawColor,line width= 0.4pt,line join=round,line cap=round,fill=fillColor] ( 53.71,180.87) circle (  1.96);

\path[draw=drawColor,line width= 0.4pt,line join=round,line cap=round,fill=fillColor] ( 53.71,177.27) circle (  1.96);

\path[draw=drawColor,line width= 0.4pt,line join=round,line cap=round,fill=fillColor] ( 53.71,173.68) circle (  1.96);

\path[draw=drawColor,line width= 0.4pt,line join=round,line cap=round,fill=fillColor] ( 53.71,188.65) circle (  1.96);

\path[draw=drawColor,line width= 0.4pt,line join=round,line cap=round,fill=fillColor] ( 53.71,171.50) circle (  1.96);

\path[draw=drawColor,line width= 0.4pt,line join=round,line cap=round,fill=fillColor] ( 53.71,176.98) circle (  1.96);

\path[draw=drawColor,line width= 0.4pt,line join=round,line cap=round,fill=fillColor] ( 53.71,189.04) circle (  1.96);

\path[draw=drawColor,line width= 0.4pt,line join=round,line cap=round,fill=fillColor] ( 53.71,171.50) circle (  1.96);

\path[draw=drawColor,line width= 0.4pt,line join=round,line cap=round,fill=fillColor] ( 53.71,169.22) circle (  1.96);

\path[draw=drawColor,line width= 0.4pt,line join=round,line cap=round,fill=fillColor] ( 53.71,175.88) circle (  1.96);

\path[draw=drawColor,line width= 0.4pt,line join=round,line cap=round,fill=fillColor] ( 53.71,185.28) circle (  1.96);

\path[draw=drawColor,line width= 0.4pt,line join=round,line cap=round,fill=fillColor] ( 53.71,177.76) circle (  1.96);

\path[draw=drawColor,line width= 0.4pt,line join=round,line cap=round,fill=fillColor] ( 53.71,185.28) circle (  1.96);

\path[draw=drawColor,line width= 0.4pt,line join=round,line cap=round,fill=fillColor] ( 53.71,189.04) circle (  1.96);

\path[draw=drawColor,line width= 0.4pt,line join=round,line cap=round,fill=fillColor] ( 53.71,173.78) circle (  1.96);

\path[draw=drawColor,line width= 0.4pt,line join=round,line cap=round,fill=fillColor] ( 53.71,182.38) circle (  1.96);

\path[draw=drawColor,line width= 0.4pt,line join=round,line cap=round,fill=fillColor] ( 53.71,175.08) circle (  1.96);

\path[draw=drawColor,line width= 0.4pt,line join=round,line cap=round,fill=fillColor] ( 53.71,179.64) circle (  1.96);

\path[draw=drawColor,line width= 0.4pt,line join=round,line cap=round,fill=fillColor] ( 53.71,181.60) circle (  1.96);

\path[draw=drawColor,line width= 0.4pt,line join=round,line cap=round,fill=fillColor] ( 53.71,175.08) circle (  1.96);

\path[draw=drawColor,line width= 0.4pt,line join=round,line cap=round,fill=fillColor] ( 53.71,184.21) circle (  1.96);

\path[draw=drawColor,line width= 0.4pt,line join=round,line cap=round,fill=fillColor] ( 53.71,186.49) circle (  1.96);

\path[draw=drawColor,line width= 0.4pt,line join=round,line cap=round,fill=fillColor] ( 53.71,186.49) circle (  1.96);

\path[draw=drawColor,line width= 0.4pt,line join=round,line cap=round,fill=fillColor] ( 53.71,175.08) circle (  1.96);

\path[draw=drawColor,line width= 0.4pt,line join=round,line cap=round,fill=fillColor] ( 53.71,182.69) circle (  1.96);

\path[draw=drawColor,line width= 0.4pt,line join=round,line cap=round,fill=fillColor] ( 53.71,180.83) circle (  1.96);

\path[draw=drawColor,line width= 0.4pt,line join=round,line cap=round,fill=fillColor] ( 53.71,182.04) circle (  1.96);

\path[draw=drawColor,line width= 0.4pt,line join=round,line cap=round,fill=fillColor] ( 53.71,174.77) circle (  1.96);

\path[draw=drawColor,line width= 0.4pt,line join=round,line cap=round,fill=fillColor] ( 53.71,179.31) circle (  1.96);

\path[draw=drawColor,line width= 0.4pt,line join=round,line cap=round,fill=fillColor] ( 53.71,171.13) circle (  1.96);

\path[draw=drawColor,line width= 0.4pt,line join=round,line cap=round,fill=fillColor] ( 53.71,187.30) circle (  1.96);

\path[draw=drawColor,line width= 0.4pt,line join=round,line cap=round,fill=fillColor] ( 53.71,174.37) circle (  1.96);

\path[draw=drawColor,line width= 0.4pt,line join=round,line cap=round,fill=fillColor] ( 53.71,171.85) circle (  1.96);

\path[draw=drawColor,line width= 0.4pt,line join=round,line cap=round,fill=fillColor] ( 53.71,178.14) circle (  1.96);

\path[draw=drawColor,line width= 0.4pt,line join=round,line cap=round,fill=fillColor] ( 53.71,178.14) circle (  1.96);

\path[draw=drawColor,line width= 0.4pt,line join=round,line cap=round,fill=fillColor] ( 53.71,184.43) circle (  1.96);

\path[draw=drawColor,line width= 0.4pt,line join=round,line cap=round,fill=fillColor] ( 53.71,184.43) circle (  1.96);

\path[draw=drawColor,line width= 0.4pt,line join=round,line cap=round,fill=fillColor] ( 53.71,174.37) circle (  1.96);

\path[draw=drawColor,line width= 0.4pt,line join=round,line cap=round,fill=fillColor] ( 53.71,178.14) circle (  1.96);

\path[draw=drawColor,line width= 0.4pt,line join=round,line cap=round,fill=fillColor] ( 53.71,171.85) circle (  1.96);

\path[draw=drawColor,line width= 0.4pt,line join=round,line cap=round,fill=fillColor] ( 53.71,171.85) circle (  1.96);

\path[draw=drawColor,line width= 0.4pt,line join=round,line cap=round,fill=fillColor] ( 53.71,178.14) circle (  1.96);

\path[draw=drawColor,line width= 0.4pt,line join=round,line cap=round,fill=fillColor] ( 53.71,174.37) circle (  1.96);

\path[draw=drawColor,line width= 0.4pt,line join=round,line cap=round,fill=fillColor] ( 53.71,170.05) circle (  1.96);

\path[draw=drawColor,line width= 0.4pt,line join=round,line cap=round,fill=fillColor] ( 53.71,184.43) circle (  1.96);

\path[draw=drawColor,line width= 0.4pt,line join=round,line cap=round,fill=fillColor] ( 53.71,184.43) circle (  1.96);

\path[draw=drawColor,line width= 0.4pt,line join=round,line cap=round,fill=fillColor] ( 53.71,178.14) circle (  1.96);

\path[draw=drawColor,line width= 0.4pt,line join=round,line cap=round,fill=fillColor] ( 53.71,174.37) circle (  1.96);

\path[draw=drawColor,line width= 0.4pt,line join=round,line cap=round,fill=fillColor] ( 53.71,189.46) circle (  1.96);

\path[draw=drawColor,line width= 0.4pt,line join=round,line cap=round,fill=fillColor] ( 53.71,189.46) circle (  1.96);

\path[draw=drawColor,line width= 0.4pt,line join=round,line cap=round,fill=fillColor] ( 53.71,174.37) circle (  1.96);

\path[draw=drawColor,line width= 0.4pt,line join=round,line cap=round,fill=fillColor] ( 53.71,174.37) circle (  1.96);

\path[draw=drawColor,line width= 0.6pt,line join=round] ( 53.71,132.45) -- ( 53.71,168.40);

\path[draw=drawColor,line width= 0.6pt,line join=round] ( 53.71,108.34) -- ( 53.71, 86.65);
\definecolor{fillColor}{RGB}{228,26,28}

\path[draw=drawColor,line width= 0.6pt,line join=round,line cap=round,fill=fillColor] ( 42.71,132.45) --
	( 42.71,108.34) --
	( 64.71,108.34) --
	( 64.71,132.45) --
	( 42.71,132.45) --
	cycle;

\path[draw=drawColor,line width= 1.1pt,line join=round] ( 42.71,118.89) -- ( 64.71,118.89);

\path[draw=drawColor,line width= 0.6pt,line join=round] ( 83.05,133.93) -- ( 83.05,138.58);

\path[draw=drawColor,line width= 0.6pt,line join=round] ( 83.05, 92.07) -- ( 83.05, 86.27);
\definecolor{fillColor}{RGB}{55,126,184}

\path[draw=drawColor,line width= 0.6pt,line join=round,line cap=round,fill=fillColor] ( 72.05,133.93) --
	( 72.05, 92.07) --
	( 94.05, 92.07) --
	( 94.05,133.93) --
	( 72.05,133.93) --
	cycle;

\path[draw=drawColor,line width= 1.1pt,line join=round] ( 72.05,104.24) -- ( 94.05,104.24);

\path[draw=drawColor,line width= 0.6pt,line join=round] (112.39,123.75) -- (112.39,132.59);

\path[draw=drawColor,line width= 0.6pt,line join=round] (112.39, 97.91) -- (112.39, 92.28);
\definecolor{fillColor}{RGB}{77,175,74}

\path[draw=drawColor,line width= 0.6pt,line join=round,line cap=round,fill=fillColor] (101.39,123.75) --
	(101.39, 97.91) --
	(123.39, 97.91) --
	(123.39,123.75) --
	(101.39,123.75) --
	cycle;

\path[draw=drawColor,line width= 1.1pt,line join=round] (101.39,104.39) -- (123.39,104.39);
\end{scope}
\begin{scope}
\path[clip] (135.49, 78.54) rectangle (229.37,194.74);
\definecolor{drawColor}{RGB}{255,255,255}

\path[draw=drawColor,line width= 0.3pt,line join=round] (135.49, 98.91) --
	(229.37, 98.91);

\path[draw=drawColor,line width= 0.3pt,line join=round] (135.49,129.09) --
	(229.37,129.09);

\path[draw=drawColor,line width= 0.3pt,line join=round] (135.49,159.27) --
	(229.37,159.27);

\path[draw=drawColor,line width= 0.3pt,line join=round] (135.49,189.46) --
	(229.37,189.46);

\path[draw=drawColor,line width= 0.6pt,line join=round] (135.49, 83.82) --
	(229.37, 83.82);

\path[draw=drawColor,line width= 0.6pt,line join=round] (135.49,114.00) --
	(229.37,114.00);

\path[draw=drawColor,line width= 0.6pt,line join=round] (135.49,144.18) --
	(229.37,144.18);

\path[draw=drawColor,line width= 0.6pt,line join=round] (135.49,174.37) --
	(229.37,174.37);

\path[draw=drawColor,line width= 0.6pt,line join=round] (153.09, 78.54) --
	(153.09,194.74);

\path[draw=drawColor,line width= 0.6pt,line join=round] (182.43, 78.54) --
	(182.43,194.74);

\path[draw=drawColor,line width= 0.6pt,line join=round] (211.76, 78.54) --
	(211.76,194.74);
\definecolor{drawColor}{gray}{0.20}

\path[draw=drawColor,line width= 0.6pt,line join=round] (153.09,147.83) -- (153.09,189.46);

\path[draw=drawColor,line width= 0.6pt,line join=round] (153.09,116.31) -- (153.09, 84.91);
\definecolor{fillColor}{RGB}{228,26,28}

\path[draw=drawColor,line width= 0.6pt,line join=round,line cap=round,fill=fillColor] (142.09,147.83) --
	(142.09,116.31) --
	(164.09,116.31) --
	(164.09,147.83) --
	(142.09,147.83) --
	cycle;

\path[draw=drawColor,line width= 1.1pt,line join=round] (142.09,130.99) -- (164.09,130.99);
\definecolor{fillColor}{gray}{0.20}

\path[draw=drawColor,line width= 0.4pt,line join=round,line cap=round,fill=fillColor] (182.43,175.80) circle (  1.96);

\path[draw=drawColor,line width= 0.4pt,line join=round,line cap=round,fill=fillColor] (182.43,171.83) circle (  1.96);

\path[draw=drawColor,line width= 0.4pt,line join=round,line cap=round,fill=fillColor] (182.43,167.48) circle (  1.96);

\path[draw=drawColor,line width= 0.6pt,line join=round] (182.43,126.69) -- (182.43,156.97);

\path[draw=drawColor,line width= 0.6pt,line join=round] (182.43,104.72) -- (182.43, 92.64);
\definecolor{fillColor}{RGB}{55,126,184}

\path[draw=drawColor,line width= 0.6pt,line join=round,line cap=round,fill=fillColor] (171.43,126.69) --
	(171.43,104.72) --
	(193.43,104.72) --
	(193.43,126.69) --
	(171.43,126.69) --
	cycle;

\path[draw=drawColor,line width= 1.1pt,line join=round] (171.43,117.46) -- (193.43,117.46);
\definecolor{fillColor}{gray}{0.20}

\path[draw=drawColor,line width= 0.4pt,line join=round,line cap=round,fill=fillColor] (211.76,186.75) circle (  1.96);

\path[draw=drawColor,line width= 0.4pt,line join=round,line cap=round,fill=fillColor] (211.76,179.83) circle (  1.96);

\path[draw=drawColor,line width= 0.4pt,line join=round,line cap=round,fill=fillColor] (211.76,184.65) circle (  1.96);

\path[draw=drawColor,line width= 0.4pt,line join=round,line cap=round,fill=fillColor] (211.76,175.08) circle (  1.96);

\path[draw=drawColor,line width= 0.6pt,line join=round] (211.76,126.65) -- (211.76,146.46);

\path[draw=drawColor,line width= 0.6pt,line join=round] (211.76,102.00) -- (211.76, 91.65);
\definecolor{fillColor}{RGB}{77,175,74}

\path[draw=drawColor,line width= 0.6pt,line join=round,line cap=round,fill=fillColor] (200.76,126.65) --
	(200.76,102.00) --
	(222.76,102.00) --
	(222.76,126.65) --
	(200.76,126.65) --
	cycle;

\path[draw=drawColor,line width= 1.1pt,line join=round] (200.76,117.02) -- (222.76,117.02);
\end{scope}
\begin{scope}
\path[clip] (234.87, 78.54) rectangle (328.74,194.74);
\definecolor{drawColor}{RGB}{255,255,255}

\path[draw=drawColor,line width= 0.3pt,line join=round] (234.87, 98.91) --
	(328.74, 98.91);

\path[draw=drawColor,line width= 0.3pt,line join=round] (234.87,129.09) --
	(328.74,129.09);

\path[draw=drawColor,line width= 0.3pt,line join=round] (234.87,159.27) --
	(328.74,159.27);

\path[draw=drawColor,line width= 0.3pt,line join=round] (234.87,189.46) --
	(328.74,189.46);

\path[draw=drawColor,line width= 0.6pt,line join=round] (234.87, 83.82) --
	(328.74, 83.82);

\path[draw=drawColor,line width= 0.6pt,line join=round] (234.87,114.00) --
	(328.74,114.00);

\path[draw=drawColor,line width= 0.6pt,line join=round] (234.87,144.18) --
	(328.74,144.18);

\path[draw=drawColor,line width= 0.6pt,line join=round] (234.87,174.37) --
	(328.74,174.37);

\path[draw=drawColor,line width= 0.6pt,line join=round] (252.47, 78.54) --
	(252.47,194.74);

\path[draw=drawColor,line width= 0.6pt,line join=round] (281.80, 78.54) --
	(281.80,194.74);

\path[draw=drawColor,line width= 0.6pt,line join=round] (311.14, 78.54) --
	(311.14,194.74);
\definecolor{drawColor}{gray}{0.20}

\path[draw=drawColor,line width= 0.6pt,line join=round] (252.47,160.94) -- (252.47,189.46);

\path[draw=drawColor,line width= 0.6pt,line join=round] (252.47,126.93) -- (252.47, 84.98);
\definecolor{fillColor}{RGB}{228,26,28}

\path[draw=drawColor,line width= 0.6pt,line join=round,line cap=round,fill=fillColor] (241.47,160.94) --
	(241.47,126.93) --
	(263.47,126.93) --
	(263.47,160.94) --
	(241.47,160.94) --
	cycle;

\path[draw=drawColor,line width= 1.1pt,line join=round] (241.47,143.79) -- (263.47,143.79);

\path[draw=drawColor,line width= 0.6pt,line join=round] (281.80,160.82) -- (281.80,166.69);

\path[draw=drawColor,line width= 0.6pt,line join=round] (281.80,126.65) -- (281.80,112.38);
\definecolor{fillColor}{RGB}{55,126,184}

\path[draw=drawColor,line width= 0.6pt,line join=round,line cap=round,fill=fillColor] (270.80,160.82) --
	(270.80,126.65) --
	(292.81,126.65) --
	(292.81,160.82) --
	(270.80,160.82) --
	cycle;

\path[draw=drawColor,line width= 1.1pt,line join=round] (270.80,128.66) -- (292.81,128.66);

\path[draw=drawColor,line width= 0.6pt,line join=round] (311.14,155.92) -- (311.14,186.11);

\path[draw=drawColor,line width= 0.6pt,line join=round] (311.14,124.05) -- (311.14, 92.59);
\definecolor{fillColor}{RGB}{77,175,74}

\path[draw=drawColor,line width= 0.6pt,line join=round,line cap=round,fill=fillColor] (300.14,155.92) --
	(300.14,124.05) --
	(322.14,124.05) --
	(322.14,155.92) --
	(300.14,155.92) --
	cycle;

\path[draw=drawColor,line width= 1.1pt,line join=round] (300.14,138.90) -- (322.14,138.90);
\end{scope}
\begin{scope}
\path[clip] (334.24, 78.54) rectangle (428.12,194.74);
\definecolor{drawColor}{RGB}{255,255,255}

\path[draw=drawColor,line width= 0.3pt,line join=round] (334.24, 98.91) --
	(428.12, 98.91);

\path[draw=drawColor,line width= 0.3pt,line join=round] (334.24,129.09) --
	(428.12,129.09);

\path[draw=drawColor,line width= 0.3pt,line join=round] (334.24,159.27) --
	(428.12,159.27);

\path[draw=drawColor,line width= 0.3pt,line join=round] (334.24,189.46) --
	(428.12,189.46);

\path[draw=drawColor,line width= 0.6pt,line join=round] (334.24, 83.82) --
	(428.12, 83.82);

\path[draw=drawColor,line width= 0.6pt,line join=round] (334.24,114.00) --
	(428.12,114.00);

\path[draw=drawColor,line width= 0.6pt,line join=round] (334.24,144.18) --
	(428.12,144.18);

\path[draw=drawColor,line width= 0.6pt,line join=round] (334.24,174.37) --
	(428.12,174.37);

\path[draw=drawColor,line width= 0.6pt,line join=round] (351.84, 78.54) --
	(351.84,194.74);

\path[draw=drawColor,line width= 0.6pt,line join=round] (381.18, 78.54) --
	(381.18,194.74);

\path[draw=drawColor,line width= 0.6pt,line join=round] (410.52, 78.54) --
	(410.52,194.74);
\definecolor{drawColor}{gray}{0.20}

\path[draw=drawColor,line width= 0.6pt,line join=round] (351.84,170.52) -- (351.84,189.46);

\path[draw=drawColor,line width= 0.6pt,line join=round] (351.84,136.02) -- (351.84, 86.45);
\definecolor{fillColor}{RGB}{228,26,28}

\path[draw=drawColor,line width= 0.6pt,line join=round,line cap=round,fill=fillColor] (340.84,170.52) --
	(340.84,136.02) --
	(362.85,136.02) --
	(362.85,170.52) --
	(340.84,170.52) --
	cycle;

\path[draw=drawColor,line width= 1.1pt,line join=round] (340.84,152.80) -- (362.85,152.80);

\path[draw=drawColor,line width= 0.6pt,line join=round] (381.18,150.75) -- (381.18,166.12);

\path[draw=drawColor,line width= 0.6pt,line join=round] (381.18,114.62) -- (381.18,104.76);
\definecolor{fillColor}{RGB}{55,126,184}

\path[draw=drawColor,line width= 0.6pt,line join=round,line cap=round,fill=fillColor] (370.18,150.75) --
	(370.18,114.62) --
	(392.18,114.62) --
	(392.18,150.75) --
	(370.18,150.75) --
	cycle;

\path[draw=drawColor,line width= 1.1pt,line join=round] (370.18,148.15) -- (392.18,148.15);

\path[draw=drawColor,line width= 0.6pt,line join=round] (410.52,173.57) -- (410.52,183.77);

\path[draw=drawColor,line width= 0.6pt,line join=round] (410.52,149.90) -- (410.52,137.36);
\definecolor{fillColor}{RGB}{77,175,74}

\path[draw=drawColor,line width= 0.6pt,line join=round,line cap=round,fill=fillColor] (399.52,173.57) --
	(399.52,149.90) --
	(421.52,149.90) --
	(421.52,173.57) --
	(399.52,173.57) --
	cycle;

\path[draw=drawColor,line width= 1.1pt,line join=round] (399.52,162.75) -- (421.52,162.75);
\end{scope}
\begin{scope}
\path[clip] ( 36.11,194.74) rectangle (129.99,211.31);
\definecolor{drawColor}{RGB}{0,0,0}

\node[text=drawColor,anchor=base,inner sep=0pt, outer sep=0pt, scale=  0.70] at ( 83.05,199.99) {No Calificados};
\end{scope}
\begin{scope}
\path[clip] (135.49,194.74) rectangle (229.37,211.31);
\definecolor{drawColor}{RGB}{0,0,0}

\node[text=drawColor,anchor=base,inner sep=0pt, outer sep=0pt, scale=  0.70] at (182.43,199.99) {Operativos};
\end{scope}
\begin{scope}
\path[clip] (234.87,194.74) rectangle (328.74,211.31);
\definecolor{drawColor}{RGB}{0,0,0}

\node[text=drawColor,anchor=base,inner sep=0pt, outer sep=0pt, scale=  0.70] at (281.80,199.99) {Técnicos};
\end{scope}
\begin{scope}
\path[clip] (334.24,194.74) rectangle (428.12,211.31);
\definecolor{drawColor}{RGB}{0,0,0}

\node[text=drawColor,anchor=base,inner sep=0pt, outer sep=0pt, scale=  0.70] at (381.18,199.99) {Profesionales};
\end{scope}
\begin{scope}
\path[clip] (  0.00,  0.00) rectangle (433.62,216.81);
\definecolor{drawColor}{gray}{0.20}

\path[draw=drawColor,line width= 0.6pt,line join=round] ( 53.71, 75.79) --
	( 53.71, 78.54);

\path[draw=drawColor,line width= 0.6pt,line join=round] ( 83.05, 75.79) --
	( 83.05, 78.54);

\path[draw=drawColor,line width= 0.6pt,line join=round] (112.39, 75.79) --
	(112.39, 78.54);
\end{scope}
\begin{scope}
\path[clip] (  0.00,  0.00) rectangle (433.62,216.81);
\definecolor{drawColor}{RGB}{0,0,0}

\node[text=drawColor,rotate= 90.00,anchor=base,inner sep=0pt, outer sep=0pt, scale=  0.60] at ( 59.77, 45.78) {\textbf{Nacio y vive}};

\node[text=drawColor,rotate= 90.00,anchor=base,inner sep=0pt, outer sep=0pt, scale=  0.60] at ( 69.28, 45.78) {\textbf{en Sur}};

\node[text=drawColor,rotate= 90.00,anchor=base,inner sep=0pt, outer sep=0pt, scale=  0.60] at ( 89.11, 45.78) {Nacio en Sur};

\node[text=drawColor,rotate= 90.00,anchor=base,inner sep=0pt, outer sep=0pt, scale=  0.60] at ( 98.61, 45.78) {vive en Centro};

\node[text=drawColor,rotate= 90.00,anchor=base,inner sep=0pt, outer sep=0pt, scale=  0.60] at (118.45, 45.78) {Nacio en Sur};

\node[text=drawColor,rotate= 90.00,anchor=base,inner sep=0pt, outer sep=0pt, scale=  0.60] at (127.95, 45.78) {vive en Norte};
\end{scope}
\begin{scope}
\path[clip] (  0.00,  0.00) rectangle (433.62,216.81);
\definecolor{drawColor}{gray}{0.20}

\path[draw=drawColor,line width= 0.6pt,line join=round] (153.09, 75.79) --
	(153.09, 78.54);

\path[draw=drawColor,line width= 0.6pt,line join=round] (182.43, 75.79) --
	(182.43, 78.54);

\path[draw=drawColor,line width= 0.6pt,line join=round] (211.76, 75.79) --
	(211.76, 78.54);
\end{scope}
\begin{scope}
\path[clip] (  0.00,  0.00) rectangle (433.62,216.81);
\definecolor{drawColor}{RGB}{0,0,0}

\node[text=drawColor,rotate= 90.00,anchor=base,inner sep=0pt, outer sep=0pt, scale=  0.60] at (159.15, 45.78) {\textbf{Nacio y vive}};

\node[text=drawColor,rotate= 90.00,anchor=base,inner sep=0pt, outer sep=0pt, scale=  0.60] at (168.66, 45.78) {\textbf{en Sur}};

\node[text=drawColor,rotate= 90.00,anchor=base,inner sep=0pt, outer sep=0pt, scale=  0.60] at (188.49, 45.78) {Nacio en Sur};

\node[text=drawColor,rotate= 90.00,anchor=base,inner sep=0pt, outer sep=0pt, scale=  0.60] at (197.99, 45.78) {vive en Centro};

\node[text=drawColor,rotate= 90.00,anchor=base,inner sep=0pt, outer sep=0pt, scale=  0.60] at (217.82, 45.78) {Nacio en Sur};

\node[text=drawColor,rotate= 90.00,anchor=base,inner sep=0pt, outer sep=0pt, scale=  0.60] at (227.33, 45.78) {vive en Norte};
\end{scope}
\begin{scope}
\path[clip] (  0.00,  0.00) rectangle (433.62,216.81);
\definecolor{drawColor}{gray}{0.20}

\path[draw=drawColor,line width= 0.6pt,line join=round] (252.47, 75.79) --
	(252.47, 78.54);

\path[draw=drawColor,line width= 0.6pt,line join=round] (281.80, 75.79) --
	(281.80, 78.54);

\path[draw=drawColor,line width= 0.6pt,line join=round] (311.14, 75.79) --
	(311.14, 78.54);
\end{scope}
\begin{scope}
\path[clip] (  0.00,  0.00) rectangle (433.62,216.81);
\definecolor{drawColor}{RGB}{0,0,0}

\node[text=drawColor,rotate= 90.00,anchor=base,inner sep=0pt, outer sep=0pt, scale=  0.60] at (258.53, 45.78) {\textbf{Nacio y vive}};

\node[text=drawColor,rotate= 90.00,anchor=base,inner sep=0pt, outer sep=0pt, scale=  0.60] at (268.03, 45.78) {\textbf{en Sur}};

\node[text=drawColor,rotate= 90.00,anchor=base,inner sep=0pt, outer sep=0pt, scale=  0.60] at (287.86, 45.78) {Nacio en Sur};

\node[text=drawColor,rotate= 90.00,anchor=base,inner sep=0pt, outer sep=0pt, scale=  0.60] at (297.37, 45.78) {vive en Centro};

\node[text=drawColor,rotate= 90.00,anchor=base,inner sep=0pt, outer sep=0pt, scale=  0.60] at (317.20, 45.78) {Nacio en Sur};

\node[text=drawColor,rotate= 90.00,anchor=base,inner sep=0pt, outer sep=0pt, scale=  0.60] at (326.71, 45.78) {vive en Norte};
\end{scope}
\begin{scope}
\path[clip] (  0.00,  0.00) rectangle (433.62,216.81);
\definecolor{drawColor}{gray}{0.20}

\path[draw=drawColor,line width= 0.6pt,line join=round] (351.84, 75.79) --
	(351.84, 78.54);

\path[draw=drawColor,line width= 0.6pt,line join=round] (381.18, 75.79) --
	(381.18, 78.54);

\path[draw=drawColor,line width= 0.6pt,line join=round] (410.52, 75.79) --
	(410.52, 78.54);
\end{scope}
\begin{scope}
\path[clip] (  0.00,  0.00) rectangle (433.62,216.81);
\definecolor{drawColor}{RGB}{0,0,0}

\node[text=drawColor,rotate= 90.00,anchor=base,inner sep=0pt, outer sep=0pt, scale=  0.60] at (357.91, 45.78) {\textbf{Nacio y vive}};

\node[text=drawColor,rotate= 90.00,anchor=base,inner sep=0pt, outer sep=0pt, scale=  0.60] at (367.41, 45.78) {\textbf{en Sur}};

\node[text=drawColor,rotate= 90.00,anchor=base,inner sep=0pt, outer sep=0pt, scale=  0.60] at (387.24, 45.78) {Nacio en Sur};

\node[text=drawColor,rotate= 90.00,anchor=base,inner sep=0pt, outer sep=0pt, scale=  0.60] at (396.75, 45.78) {vive en Centro};

\node[text=drawColor,rotate= 90.00,anchor=base,inner sep=0pt, outer sep=0pt, scale=  0.60] at (416.58, 45.78) {Nacio en Sur};

\node[text=drawColor,rotate= 90.00,anchor=base,inner sep=0pt, outer sep=0pt, scale=  0.60] at (426.08, 45.78) {vive en Norte};
\end{scope}
\begin{scope}
\path[clip] (  0.00,  0.00) rectangle (433.62,216.81);
\definecolor{drawColor}{RGB}{0,0,0}

\node[text=drawColor,anchor=base east,inner sep=0pt, outer sep=0pt, scale=  0.60] at ( 31.16, 80.79) {0};

\node[text=drawColor,anchor=base east,inner sep=0pt, outer sep=0pt, scale=  0.60] at ( 31.16,110.97) {200};

\node[text=drawColor,anchor=base east,inner sep=0pt, outer sep=0pt, scale=  0.60] at ( 31.16,141.15) {400};

\node[text=drawColor,anchor=base east,inner sep=0pt, outer sep=0pt, scale=  0.60] at ( 31.16,171.34) {600};
\end{scope}
\begin{scope}
\path[clip] (  0.00,  0.00) rectangle (433.62,216.81);
\definecolor{drawColor}{gray}{0.20}

\path[draw=drawColor,line width= 0.6pt,line join=round] ( 33.36, 83.82) --
	( 36.11, 83.82);

\path[draw=drawColor,line width= 0.6pt,line join=round] ( 33.36,114.00) --
	( 36.11,114.00);

\path[draw=drawColor,line width= 0.6pt,line join=round] ( 33.36,144.18) --
	( 36.11,144.18);

\path[draw=drawColor,line width= 0.6pt,line join=round] ( 33.36,174.37) --
	( 36.11,174.37);
\end{scope}
\begin{scope}
\path[clip] (  0.00,  0.00) rectangle (433.62,216.81);
\definecolor{drawColor}{RGB}{0,0,0}

\node[text=drawColor,rotate= 90.00,anchor=base,inner sep=0pt, outer sep=0pt, scale=  0.60] at ( 13.08,136.64) {Ingreso laboral por hora};
\end{scope}
\end{tikzpicture}
 
% Created by tikzDevice version 0.12.3.1 on 2021-07-16 19:43:39
% !TEX encoding = UTF-8 Unicode
\begin{tikzpicture}[x=1pt,y=1pt]
\definecolor{fillColor}{RGB}{255,255,255}
\path[use as bounding box,fill=fillColor,fill opacity=0.00] (0,0) rectangle (361.35,180.67);
\begin{scope}
\path[clip] (  0.00,  0.00) rectangle (361.35,180.67);
\definecolor{drawColor}{RGB}{255,255,255}
\definecolor{fillColor}{RGB}{255,255,255}

\path[draw=drawColor,line width= 0.6pt,line join=round,line cap=round,fill=fillColor] (  0.00,  0.00) rectangle (361.35,180.68);
\end{scope}
\begin{scope}
\path[clip] ( 36.11, 78.54) rectangle (111.92,158.60);
\definecolor{drawColor}{RGB}{255,255,255}

\path[draw=drawColor,line width= 0.3pt,line join=round] ( 36.11, 92.57) --
	(111.92, 92.57);

\path[draw=drawColor,line width= 0.3pt,line join=round] ( 36.11,113.37) --
	(111.92,113.37);

\path[draw=drawColor,line width= 0.3pt,line join=round] ( 36.11,134.17) --
	(111.92,134.17);

\path[draw=drawColor,line width= 0.3pt,line join=round] ( 36.11,154.96) --
	(111.92,154.96);

\path[draw=drawColor,line width= 0.6pt,line join=round] ( 36.11, 82.18) --
	(111.92, 82.18);

\path[draw=drawColor,line width= 0.6pt,line join=round] ( 36.11,102.97) --
	(111.92,102.97);

\path[draw=drawColor,line width= 0.6pt,line join=round] ( 36.11,123.77) --
	(111.92,123.77);

\path[draw=drawColor,line width= 0.6pt,line join=round] ( 36.11,144.57) --
	(111.92,144.57);

\path[draw=drawColor,line width= 0.6pt,line join=round] ( 50.33, 78.54) --
	( 50.33,158.60);

\path[draw=drawColor,line width= 0.6pt,line join=round] ( 74.02, 78.54) --
	( 74.02,158.60);

\path[draw=drawColor,line width= 0.6pt,line join=round] ( 97.71, 78.54) --
	( 97.71,158.60);
\definecolor{drawColor}{gray}{0.20}
\definecolor{fillColor}{gray}{0.20}

\path[draw=drawColor,line width= 0.4pt,line join=round,line cap=round,fill=fillColor] ( 50.33,130.29) circle (  1.96);

\path[draw=drawColor,line width= 0.4pt,line join=round,line cap=round,fill=fillColor] ( 50.33,123.32) circle (  1.96);

\path[draw=drawColor,line width= 0.4pt,line join=round,line cap=round,fill=fillColor] ( 50.33,134.93) circle (  1.96);

\path[draw=drawColor,line width= 0.4pt,line join=round,line cap=round,fill=fillColor] ( 50.33,132.82) circle (  1.96);

\path[draw=drawColor,line width= 0.4pt,line join=round,line cap=round,fill=fillColor] ( 50.33,129.65) circle (  1.96);

\path[draw=drawColor,line width= 0.4pt,line join=round,line cap=round,fill=fillColor] ( 50.33,132.82) circle (  1.96);

\path[draw=drawColor,line width= 0.4pt,line join=round,line cap=round,fill=fillColor] ( 50.33,139.15) circle (  1.96);

\path[draw=drawColor,line width= 0.4pt,line join=round,line cap=round,fill=fillColor] ( 50.33,125.86) circle (  1.96);

\path[draw=drawColor,line width= 0.4pt,line join=round,line cap=round,fill=fillColor] ( 50.33,129.65) circle (  1.96);

\path[draw=drawColor,line width= 0.4pt,line join=round,line cap=round,fill=fillColor] ( 50.33,129.65) circle (  1.96);

\path[draw=drawColor,line width= 0.4pt,line join=round,line cap=round,fill=fillColor] ( 50.33,124.11) circle (  1.96);

\path[draw=drawColor,line width= 0.4pt,line join=round,line cap=round,fill=fillColor] ( 50.33,132.82) circle (  1.96);

\path[draw=drawColor,line width= 0.4pt,line join=round,line cap=round,fill=fillColor] ( 50.33,136.87) circle (  1.96);

\path[draw=drawColor,line width= 0.4pt,line join=round,line cap=round,fill=fillColor] ( 50.33,127.75) circle (  1.96);

\path[draw=drawColor,line width= 0.4pt,line join=round,line cap=round,fill=fillColor] ( 50.33,140.41) circle (  1.96);

\path[draw=drawColor,line width= 0.4pt,line join=round,line cap=round,fill=fillColor] ( 50.33,132.82) circle (  1.96);

\path[draw=drawColor,line width= 0.4pt,line join=round,line cap=round,fill=fillColor] ( 50.33,127.75) circle (  1.96);

\path[draw=drawColor,line width= 0.4pt,line join=round,line cap=round,fill=fillColor] ( 50.33,127.75) circle (  1.96);

\path[draw=drawColor,line width= 0.4pt,line join=round,line cap=round,fill=fillColor] ( 50.33,139.15) circle (  1.96);

\path[draw=drawColor,line width= 0.4pt,line join=round,line cap=round,fill=fillColor] ( 50.33,148.64) circle (  1.96);

\path[draw=drawColor,line width= 0.4pt,line join=round,line cap=round,fill=fillColor] ( 50.33,127.75) circle (  1.96);

\path[draw=drawColor,line width= 0.4pt,line join=round,line cap=round,fill=fillColor] ( 50.33,129.65) circle (  1.96);

\path[draw=drawColor,line width= 0.4pt,line join=round,line cap=round,fill=fillColor] ( 50.33,137.32) circle (  1.96);

\path[draw=drawColor,line width= 0.4pt,line join=round,line cap=round,fill=fillColor] ( 50.33,133.97) circle (  1.96);

\path[draw=drawColor,line width= 0.4pt,line join=round,line cap=round,fill=fillColor] ( 50.33,125.34) circle (  1.96);

\path[draw=drawColor,line width= 0.4pt,line join=round,line cap=round,fill=fillColor] ( 50.33,146.91) circle (  1.96);

\path[draw=drawColor,line width= 0.4pt,line join=round,line cap=round,fill=fillColor] ( 50.33,125.34) circle (  1.96);

\path[draw=drawColor,line width= 0.4pt,line join=round,line cap=round,fill=fillColor] ( 50.33,139.72) circle (  1.96);

\path[draw=drawColor,line width= 0.4pt,line join=round,line cap=round,fill=fillColor] ( 50.33,125.34) circle (  1.96);

\path[draw=drawColor,line width= 0.4pt,line join=round,line cap=round,fill=fillColor] ( 50.33,127.06) circle (  1.96);

\path[draw=drawColor,line width= 0.4pt,line join=round,line cap=round,fill=fillColor] ( 50.33,131.09) circle (  1.96);

\path[draw=drawColor,line width= 0.4pt,line join=round,line cap=round,fill=fillColor] ( 50.33,125.34) circle (  1.96);

\path[draw=drawColor,line width= 0.4pt,line join=round,line cap=round,fill=fillColor] ( 50.33,123.90) circle (  1.96);

\path[draw=drawColor,line width= 0.4pt,line join=round,line cap=round,fill=fillColor] ( 50.33,127.13) circle (  1.96);

\path[draw=drawColor,line width= 0.4pt,line join=round,line cap=round,fill=fillColor] ( 50.33,125.34) circle (  1.96);

\path[draw=drawColor,line width= 0.4pt,line join=round,line cap=round,fill=fillColor] ( 50.33,130.73) circle (  1.96);

\path[draw=drawColor,line width= 0.4pt,line join=round,line cap=round,fill=fillColor] ( 50.33,123.28) circle (  1.96);

\path[draw=drawColor,line width= 0.4pt,line join=round,line cap=round,fill=fillColor] ( 50.33,131.22) circle (  1.96);

\path[draw=drawColor,line width= 0.4pt,line join=round,line cap=round,fill=fillColor] ( 50.33,124.14) circle (  1.96);

\path[draw=drawColor,line width= 0.4pt,line join=round,line cap=round,fill=fillColor] ( 50.33,137.42) circle (  1.96);

\path[draw=drawColor,line width= 0.4pt,line join=round,line cap=round,fill=fillColor] ( 50.33,136.12) circle (  1.96);

\path[draw=drawColor,line width= 0.4pt,line join=round,line cap=round,fill=fillColor] ( 50.33,130.13) circle (  1.96);

\path[draw=drawColor,line width= 0.4pt,line join=round,line cap=round,fill=fillColor] ( 50.33,127.73) circle (  1.96);

\path[draw=drawColor,line width= 0.4pt,line join=round,line cap=round,fill=fillColor] ( 50.33,125.34) circle (  1.96);

\path[draw=drawColor,line width= 0.4pt,line join=round,line cap=round,fill=fillColor] ( 50.33,132.53) circle (  1.96);

\path[draw=drawColor,line width= 0.4pt,line join=round,line cap=round,fill=fillColor] ( 50.33,138.82) circle (  1.96);

\path[draw=drawColor,line width= 0.4pt,line join=round,line cap=round,fill=fillColor] ( 50.33,154.11) circle (  1.96);

\path[draw=drawColor,line width= 0.4pt,line join=round,line cap=round,fill=fillColor] ( 50.33,125.34) circle (  1.96);

\path[draw=drawColor,line width= 0.4pt,line join=round,line cap=round,fill=fillColor] ( 50.33,125.34) circle (  1.96);

\path[draw=drawColor,line width= 0.4pt,line join=round,line cap=round,fill=fillColor] ( 50.33,131.09) circle (  1.96);

\path[draw=drawColor,line width= 0.4pt,line join=round,line cap=round,fill=fillColor] ( 50.33,127.06) circle (  1.96);

\path[draw=drawColor,line width= 0.4pt,line join=round,line cap=round,fill=fillColor] ( 50.33,136.12) circle (  1.96);

\path[draw=drawColor,line width= 0.4pt,line join=round,line cap=round,fill=fillColor] ( 50.33,136.12) circle (  1.96);

\path[draw=drawColor,line width= 0.4pt,line join=round,line cap=round,fill=fillColor] ( 50.33,125.34) circle (  1.96);

\path[draw=drawColor,line width= 0.4pt,line join=round,line cap=round,fill=fillColor] ( 50.33,123.54) circle (  1.96);

\path[draw=drawColor,line width= 0.4pt,line join=round,line cap=round,fill=fillColor] ( 50.33,125.34) circle (  1.96);

\path[draw=drawColor,line width= 0.4pt,line join=round,line cap=round,fill=fillColor] ( 50.33,154.11) circle (  1.96);

\path[draw=drawColor,line width= 0.4pt,line join=round,line cap=round,fill=fillColor] ( 50.33,123.61) circle (  1.96);

\path[draw=drawColor,line width= 0.4pt,line join=round,line cap=round,fill=fillColor] ( 50.33,133.97) circle (  1.96);

\path[draw=drawColor,line width= 0.4pt,line join=round,line cap=round,fill=fillColor] ( 50.33,139.22) circle (  1.96);

\path[draw=drawColor,line width= 0.4pt,line join=round,line cap=round,fill=fillColor] ( 50.33,124.76) circle (  1.96);

\path[draw=drawColor,line width= 0.4pt,line join=round,line cap=round,fill=fillColor] ( 50.33,122.92) circle (  1.96);

\path[draw=drawColor,line width= 0.4pt,line join=round,line cap=round,fill=fillColor] ( 50.33,129.71) circle (  1.96);

\path[draw=drawColor,line width= 0.4pt,line join=round,line cap=round,fill=fillColor] ( 50.33,129.71) circle (  1.96);

\path[draw=drawColor,line width= 0.4pt,line join=round,line cap=round,fill=fillColor] ( 50.33,129.71) circle (  1.96);

\path[draw=drawColor,line width= 0.4pt,line join=round,line cap=round,fill=fillColor] ( 50.33,129.71) circle (  1.96);

\path[draw=drawColor,line width= 0.4pt,line join=round,line cap=round,fill=fillColor] ( 50.33,129.71) circle (  1.96);

\path[draw=drawColor,line width= 0.4pt,line join=round,line cap=round,fill=fillColor] ( 50.33,129.71) circle (  1.96);

\path[draw=drawColor,line width= 0.4pt,line join=round,line cap=round,fill=fillColor] ( 50.33,129.71) circle (  1.96);

\path[draw=drawColor,line width= 0.4pt,line join=round,line cap=round,fill=fillColor] ( 50.33,141.59) circle (  1.96);

\path[draw=drawColor,line width= 0.4pt,line join=round,line cap=round,fill=fillColor] ( 50.33,123.77) circle (  1.96);

\path[draw=drawColor,line width= 0.4pt,line join=round,line cap=round,fill=fillColor] ( 50.33,124.01) circle (  1.96);

\path[draw=drawColor,line width= 0.4pt,line join=round,line cap=round,fill=fillColor] ( 50.33,141.59) circle (  1.96);

\path[draw=drawColor,line width= 0.4pt,line join=round,line cap=round,fill=fillColor] ( 50.33,129.71) circle (  1.96);

\path[draw=drawColor,line width= 0.4pt,line join=round,line cap=round,fill=fillColor] ( 50.33,153.48) circle (  1.96);

\path[draw=drawColor,line width= 0.4pt,line join=round,line cap=round,fill=fillColor] ( 50.33,126.14) circle (  1.96);

\path[draw=drawColor,line width= 0.4pt,line join=round,line cap=round,fill=fillColor] ( 50.33,123.77) circle (  1.96);

\path[draw=drawColor,line width= 0.4pt,line join=round,line cap=round,fill=fillColor] ( 50.33,139.22) circle (  1.96);

\path[draw=drawColor,line width= 0.4pt,line join=round,line cap=round,fill=fillColor] ( 50.33,141.59) circle (  1.96);

\path[draw=drawColor,line width= 0.4pt,line join=round,line cap=round,fill=fillColor] ( 50.33,127.45) circle (  1.96);

\path[draw=drawColor,line width= 0.4pt,line join=round,line cap=round,fill=fillColor] ( 50.33,124.96) circle (  1.96);

\path[draw=drawColor,line width= 0.4pt,line join=round,line cap=round,fill=fillColor] ( 50.33,124.96) circle (  1.96);

\path[draw=drawColor,line width= 0.4pt,line join=round,line cap=round,fill=fillColor] ( 50.33,133.27) circle (  1.96);

\path[draw=drawColor,line width= 0.4pt,line join=round,line cap=round,fill=fillColor] ( 50.33,125.91) circle (  1.96);

\path[draw=drawColor,line width= 0.4pt,line join=round,line cap=round,fill=fillColor] ( 50.33,129.71) circle (  1.96);

\path[draw=drawColor,line width= 0.4pt,line join=round,line cap=round,fill=fillColor] ( 50.33,124.96) circle (  1.96);

\path[draw=drawColor,line width= 0.4pt,line join=round,line cap=round,fill=fillColor] ( 50.33,139.22) circle (  1.96);

\path[draw=drawColor,line width= 0.4pt,line join=round,line cap=round,fill=fillColor] ( 50.33,142.38) circle (  1.96);

\path[draw=drawColor,line width= 0.4pt,line join=round,line cap=round,fill=fillColor] ( 50.33,126.54) circle (  1.96);

\path[draw=drawColor,line width= 0.4pt,line join=round,line cap=round,fill=fillColor] ( 50.33,128.52) circle (  1.96);

\path[draw=drawColor,line width= 0.4pt,line join=round,line cap=round,fill=fillColor] ( 50.33,124.96) circle (  1.96);

\path[draw=drawColor,line width= 0.4pt,line join=round,line cap=round,fill=fillColor] ( 50.33,135.65) circle (  1.96);

\path[draw=drawColor,line width= 0.4pt,line join=round,line cap=round,fill=fillColor] ( 50.33,133.67) circle (  1.96);

\path[draw=drawColor,line width= 0.4pt,line join=round,line cap=round,fill=fillColor] ( 50.33,129.71) circle (  1.96);

\path[draw=drawColor,line width= 0.4pt,line join=round,line cap=round,fill=fillColor] ( 50.33,143.97) circle (  1.96);

\path[draw=drawColor,line width= 0.4pt,line join=round,line cap=round,fill=fillColor] ( 50.33,141.59) circle (  1.96);

\path[draw=drawColor,line width= 0.4pt,line join=round,line cap=round,fill=fillColor] ( 50.33,153.48) circle (  1.96);

\path[draw=drawColor,line width= 0.4pt,line join=round,line cap=round,fill=fillColor] ( 50.33,128.00) circle (  1.96);

\path[draw=drawColor,line width= 0.4pt,line join=round,line cap=round,fill=fillColor] ( 50.33,124.10) circle (  1.96);

\path[draw=drawColor,line width= 0.4pt,line join=round,line cap=round,fill=fillColor] ( 50.33,131.92) circle (  1.96);

\path[draw=drawColor,line width= 0.4pt,line join=round,line cap=round,fill=fillColor] ( 50.33,124.63) circle (  1.96);

\path[draw=drawColor,line width= 0.4pt,line join=round,line cap=round,fill=fillColor] ( 50.33,135.24) circle (  1.96);

\path[draw=drawColor,line width= 0.4pt,line join=round,line cap=round,fill=fillColor] ( 50.33,127.66) circle (  1.96);

\path[draw=drawColor,line width= 0.4pt,line join=round,line cap=round,fill=fillColor] ( 50.33,132.59) circle (  1.96);

\path[draw=drawColor,line width= 0.4pt,line join=round,line cap=round,fill=fillColor] ( 50.33,135.24) circle (  1.96);

\path[draw=drawColor,line width= 0.4pt,line join=round,line cap=round,fill=fillColor] ( 50.33,124.63) circle (  1.96);

\path[draw=drawColor,line width= 0.4pt,line join=round,line cap=round,fill=fillColor] ( 50.33,129.93) circle (  1.96);

\path[draw=drawColor,line width= 0.4pt,line join=round,line cap=round,fill=fillColor] ( 50.33,128.00) circle (  1.96);

\path[draw=drawColor,line width= 0.4pt,line join=round,line cap=round,fill=fillColor] ( 50.33,124.63) circle (  1.96);

\path[draw=drawColor,line width= 0.4pt,line join=round,line cap=round,fill=fillColor] ( 50.33,128.61) circle (  1.96);

\path[draw=drawColor,line width= 0.4pt,line join=round,line cap=round,fill=fillColor] ( 50.33,148.50) circle (  1.96);

\path[draw=drawColor,line width= 0.4pt,line join=round,line cap=round,fill=fillColor] ( 50.33,148.50) circle (  1.96);

\path[draw=drawColor,line width= 0.4pt,line join=round,line cap=round,fill=fillColor] ( 50.33,126.39) circle (  1.96);

\path[draw=drawColor,line width= 0.4pt,line join=round,line cap=round,fill=fillColor] ( 50.33,148.50) circle (  1.96);

\path[draw=drawColor,line width= 0.4pt,line join=round,line cap=round,fill=fillColor] ( 50.33,143.20) circle (  1.96);

\path[draw=drawColor,line width= 0.4pt,line join=round,line cap=round,fill=fillColor] ( 50.33,124.63) circle (  1.96);

\path[draw=drawColor,line width= 0.4pt,line join=round,line cap=round,fill=fillColor] ( 50.33,148.50) circle (  1.96);

\path[draw=drawColor,line width= 0.4pt,line join=round,line cap=round,fill=fillColor] ( 50.33,152.93) circle (  1.96);

\path[draw=drawColor,line width= 0.4pt,line join=round,line cap=round,fill=fillColor] ( 50.33,125.29) circle (  1.96);

\path[draw=drawColor,line width= 0.4pt,line join=round,line cap=round,fill=fillColor] ( 50.33,135.24) circle (  1.96);

\path[draw=drawColor,line width= 0.4pt,line join=round,line cap=round,fill=fillColor] ( 50.33,124.10) circle (  1.96);

\path[draw=drawColor,line width= 0.4pt,line join=round,line cap=round,fill=fillColor] ( 50.33,124.63) circle (  1.96);

\path[draw=drawColor,line width= 0.4pt,line join=round,line cap=round,fill=fillColor] ( 50.33,126.39) circle (  1.96);

\path[draw=drawColor,line width= 0.4pt,line join=round,line cap=round,fill=fillColor] ( 50.33,126.39) circle (  1.96);

\path[draw=drawColor,line width= 0.4pt,line join=round,line cap=round,fill=fillColor] ( 50.33,128.16) circle (  1.96);

\path[draw=drawColor,line width= 0.4pt,line join=round,line cap=round,fill=fillColor] ( 50.33,125.71) circle (  1.96);

\path[draw=drawColor,line width= 0.4pt,line join=round,line cap=round,fill=fillColor] ( 50.33,129.40) circle (  1.96);

\path[draw=drawColor,line width= 0.4pt,line join=round,line cap=round,fill=fillColor] ( 50.33,148.59) circle (  1.96);

\path[draw=drawColor,line width= 0.4pt,line join=round,line cap=round,fill=fillColor] ( 50.33,137.27) circle (  1.96);

\path[draw=drawColor,line width= 0.4pt,line join=round,line cap=round,fill=fillColor] ( 50.33,137.27) circle (  1.96);

\path[draw=drawColor,line width= 0.4pt,line join=round,line cap=round,fill=fillColor] ( 50.33,141.21) circle (  1.96);

\path[draw=drawColor,line width= 0.4pt,line join=round,line cap=round,fill=fillColor] ( 50.33,124.81) circle (  1.96);

\path[draw=drawColor,line width= 0.4pt,line join=round,line cap=round,fill=fillColor] ( 50.33,147.77) circle (  1.96);

\path[draw=drawColor,line width= 0.4pt,line join=round,line cap=round,fill=fillColor] ( 50.33,129.40) circle (  1.96);

\path[draw=drawColor,line width= 0.4pt,line join=round,line cap=round,fill=fillColor] ( 50.33,126.45) circle (  1.96);

\path[draw=drawColor,line width= 0.4pt,line join=round,line cap=round,fill=fillColor] ( 50.33,139.24) circle (  1.96);

\path[draw=drawColor,line width= 0.4pt,line join=round,line cap=round,fill=fillColor] ( 50.33,126.45) circle (  1.96);

\path[draw=drawColor,line width= 0.4pt,line join=round,line cap=round,fill=fillColor] ( 50.33,131.37) circle (  1.96);

\path[draw=drawColor,line width= 0.4pt,line join=round,line cap=round,fill=fillColor] ( 50.33,126.45) circle (  1.96);

\path[draw=drawColor,line width= 0.4pt,line join=round,line cap=round,fill=fillColor] ( 50.33,129.40) circle (  1.96);

\path[draw=drawColor,line width= 0.4pt,line join=round,line cap=round,fill=fillColor] ( 50.33,131.37) circle (  1.96);

\path[draw=drawColor,line width= 0.4pt,line join=round,line cap=round,fill=fillColor] ( 50.33,129.40) circle (  1.96);

\path[draw=drawColor,line width= 0.4pt,line join=round,line cap=round,fill=fillColor] ( 50.33,124.81) circle (  1.96);

\path[draw=drawColor,line width= 0.4pt,line join=round,line cap=round,fill=fillColor] ( 50.33,129.40) circle (  1.96);

\path[draw=drawColor,line width= 0.4pt,line join=round,line cap=round,fill=fillColor] ( 50.33,131.37) circle (  1.96);

\path[draw=drawColor,line width= 0.4pt,line join=round,line cap=round,fill=fillColor] ( 50.33,129.40) circle (  1.96);

\path[draw=drawColor,line width= 0.4pt,line join=round,line cap=round,fill=fillColor] ( 50.33,127.04) circle (  1.96);

\path[draw=drawColor,line width= 0.4pt,line join=round,line cap=round,fill=fillColor] ( 50.33,141.21) circle (  1.96);

\path[draw=drawColor,line width= 0.4pt,line join=round,line cap=round,fill=fillColor] ( 50.33,137.27) circle (  1.96);

\path[draw=drawColor,line width= 0.4pt,line join=round,line cap=round,fill=fillColor] ( 50.33,126.45) circle (  1.96);

\path[draw=drawColor,line width= 0.4pt,line join=round,line cap=round,fill=fillColor] ( 50.33,126.45) circle (  1.96);

\path[draw=drawColor,line width= 0.4pt,line join=round,line cap=round,fill=fillColor] ( 50.33,127.43) circle (  1.96);

\path[draw=drawColor,line width= 0.4pt,line join=round,line cap=round,fill=fillColor] ( 50.33,124.81) circle (  1.96);

\path[draw=drawColor,line width= 0.4pt,line join=round,line cap=round,fill=fillColor] ( 50.33,153.01) circle (  1.96);

\path[draw=drawColor,line width= 0.4pt,line join=round,line cap=round,fill=fillColor] ( 50.33,129.46) circle (  1.96);

\path[draw=drawColor,line width= 0.4pt,line join=round,line cap=round,fill=fillColor] ( 50.33,130.81) circle (  1.96);

\path[draw=drawColor,line width= 0.4pt,line join=round,line cap=round,fill=fillColor] ( 50.33,149.72) circle (  1.96);

\path[draw=drawColor,line width= 0.4pt,line join=round,line cap=round,fill=fillColor] ( 50.33,129.01) circle (  1.96);

\path[draw=drawColor,line width= 0.4pt,line join=round,line cap=round,fill=fillColor] ( 50.33,124.96) circle (  1.96);

\path[draw=drawColor,line width= 0.4pt,line join=round,line cap=round,fill=fillColor] ( 50.33,124.39) circle (  1.96);

\path[draw=drawColor,line width= 0.4pt,line join=round,line cap=round,fill=fillColor] ( 50.33,129.46) circle (  1.96);

\path[draw=drawColor,line width= 0.4pt,line join=round,line cap=round,fill=fillColor] ( 50.33,134.95) circle (  1.96);

\path[draw=drawColor,line width= 0.4pt,line join=round,line cap=round,fill=fillColor] ( 50.33,134.95) circle (  1.96);

\path[draw=drawColor,line width= 0.4pt,line join=round,line cap=round,fill=fillColor] ( 50.33,129.46) circle (  1.96);

\path[draw=drawColor,line width= 0.4pt,line join=round,line cap=round,fill=fillColor] ( 50.33,124.96) circle (  1.96);

\path[draw=drawColor,line width= 0.4pt,line join=round,line cap=round,fill=fillColor] ( 50.33,125.16) circle (  1.96);

\path[draw=drawColor,line width= 0.4pt,line join=round,line cap=round,fill=fillColor] ( 50.33,124.39) circle (  1.96);

\path[draw=drawColor,line width= 0.4pt,line join=round,line cap=round,fill=fillColor] ( 50.33,132.84) circle (  1.96);

\path[draw=drawColor,line width= 0.4pt,line join=round,line cap=round,fill=fillColor] ( 50.33,124.39) circle (  1.96);

\path[draw=drawColor,line width= 0.4pt,line join=round,line cap=round,fill=fillColor] ( 50.33,136.21) circle (  1.96);

\path[draw=drawColor,line width= 0.4pt,line join=round,line cap=round,fill=fillColor] ( 50.33,143.72) circle (  1.96);

\path[draw=drawColor,line width= 0.4pt,line join=round,line cap=round,fill=fillColor] ( 50.33,127.21) circle (  1.96);

\path[draw=drawColor,line width= 0.4pt,line join=round,line cap=round,fill=fillColor] ( 50.33,132.84) circle (  1.96);

\path[draw=drawColor,line width= 0.4pt,line join=round,line cap=round,fill=fillColor] ( 50.33,123.19) circle (  1.96);

\path[draw=drawColor,line width= 0.4pt,line join=round,line cap=round,fill=fillColor] ( 50.33,122.99) circle (  1.96);

\path[draw=drawColor,line width= 0.4pt,line join=round,line cap=round,fill=fillColor] ( 50.33,127.21) circle (  1.96);

\path[draw=drawColor,line width= 0.4pt,line join=round,line cap=round,fill=fillColor] ( 50.33,131.30) circle (  1.96);

\path[draw=drawColor,line width= 0.4pt,line join=round,line cap=round,fill=fillColor] ( 50.33,144.10) circle (  1.96);

\path[draw=drawColor,line width= 0.4pt,line join=round,line cap=round,fill=fillColor] ( 50.33,132.84) circle (  1.96);

\path[draw=drawColor,line width= 0.4pt,line join=round,line cap=round,fill=fillColor] ( 50.33,145.22) circle (  1.96);

\path[draw=drawColor,line width= 0.4pt,line join=round,line cap=round,fill=fillColor] ( 50.33,124.39) circle (  1.96);

\path[draw=drawColor,line width= 0.4pt,line join=round,line cap=round,fill=fillColor] ( 50.33,127.21) circle (  1.96);

\path[draw=drawColor,line width= 0.4pt,line join=round,line cap=round,fill=fillColor] ( 50.33,124.39) circle (  1.96);

\path[draw=drawColor,line width= 0.4pt,line join=round,line cap=round,fill=fillColor] ( 50.33,129.46) circle (  1.96);

\path[draw=drawColor,line width= 0.4pt,line join=round,line cap=round,fill=fillColor] ( 50.33,132.84) circle (  1.96);

\path[draw=drawColor,line width= 0.4pt,line join=round,line cap=round,fill=fillColor] ( 50.33,123.19) circle (  1.96);

\path[draw=drawColor,line width= 0.4pt,line join=round,line cap=round,fill=fillColor] ( 50.33,152.54) circle (  1.96);

\path[draw=drawColor,line width= 0.4pt,line join=round,line cap=round,fill=fillColor] ( 50.33,146.35) circle (  1.96);

\path[draw=drawColor,line width= 0.4pt,line join=round,line cap=round,fill=fillColor] ( 50.33,124.39) circle (  1.96);

\path[draw=drawColor,line width= 0.4pt,line join=round,line cap=round,fill=fillColor] ( 50.33,124.39) circle (  1.96);

\path[draw=drawColor,line width= 0.4pt,line join=round,line cap=round,fill=fillColor] ( 50.33,154.23) circle (  1.96);

\path[draw=drawColor,line width= 0.4pt,line join=round,line cap=round,fill=fillColor] ( 50.33,153.10) circle (  1.96);

\path[draw=drawColor,line width= 0.4pt,line join=round,line cap=round,fill=fillColor] ( 50.33,127.21) circle (  1.96);

\path[draw=drawColor,line width= 0.4pt,line join=round,line cap=round,fill=fillColor] ( 50.33,136.21) circle (  1.96);

\path[draw=drawColor,line width= 0.4pt,line join=round,line cap=round,fill=fillColor] ( 50.33,138.47) circle (  1.96);

\path[draw=drawColor,line width= 0.4pt,line join=round,line cap=round,fill=fillColor] ( 50.33,129.46) circle (  1.96);

\path[draw=drawColor,line width= 0.4pt,line join=round,line cap=round,fill=fillColor] ( 50.33,124.39) circle (  1.96);

\path[draw=drawColor,line width= 0.4pt,line join=round,line cap=round,fill=fillColor] ( 50.33,146.35) circle (  1.96);

\path[draw=drawColor,line width= 0.4pt,line join=round,line cap=round,fill=fillColor] ( 50.33,130.81) circle (  1.96);

\path[draw=drawColor,line width= 0.4pt,line join=round,line cap=round,fill=fillColor] ( 50.33,136.21) circle (  1.96);

\path[draw=drawColor,line width= 0.4pt,line join=round,line cap=round,fill=fillColor] ( 50.33,133.51) circle (  1.96);

\path[draw=drawColor,line width= 0.4pt,line join=round,line cap=round,fill=fillColor] ( 50.33,131.71) circle (  1.96);

\path[draw=drawColor,line width= 0.4pt,line join=round,line cap=round,fill=fillColor] ( 50.33,145.22) circle (  1.96);

\path[draw=drawColor,line width= 0.4pt,line join=round,line cap=round,fill=fillColor] ( 50.33,142.22) circle (  1.96);

\path[draw=drawColor,line width= 0.4pt,line join=round,line cap=round,fill=fillColor] ( 50.33,127.54) circle (  1.96);

\path[draw=drawColor,line width= 0.4pt,line join=round,line cap=round,fill=fillColor] ( 50.33,129.43) circle (  1.96);

\path[draw=drawColor,line width= 0.4pt,line join=round,line cap=round,fill=fillColor] ( 50.33,132.58) circle (  1.96);

\path[draw=drawColor,line width= 0.4pt,line join=round,line cap=round,fill=fillColor] ( 50.33,139.64) circle (  1.96);

\path[draw=drawColor,line width= 0.4pt,line join=round,line cap=round,fill=fillColor] ( 50.33,123.22) circle (  1.96);

\path[draw=drawColor,line width= 0.4pt,line join=round,line cap=round,fill=fillColor] ( 50.33,127.54) circle (  1.96);

\path[draw=drawColor,line width= 0.4pt,line join=round,line cap=round,fill=fillColor] ( 50.33,127.54) circle (  1.96);

\path[draw=drawColor,line width= 0.4pt,line join=round,line cap=round,fill=fillColor] ( 50.33,126.28) circle (  1.96);

\path[draw=drawColor,line width= 0.4pt,line join=round,line cap=round,fill=fillColor] ( 50.33,142.03) circle (  1.96);

\path[draw=drawColor,line width= 0.4pt,line join=round,line cap=round,fill=fillColor] ( 50.33,132.58) circle (  1.96);

\path[draw=drawColor,line width= 0.4pt,line join=round,line cap=round,fill=fillColor] ( 50.33,136.18) circle (  1.96);

\path[draw=drawColor,line width= 0.4pt,line join=round,line cap=round,fill=fillColor] ( 50.33,138.88) circle (  1.96);

\path[draw=drawColor,line width= 0.4pt,line join=round,line cap=round,fill=fillColor] ( 50.33,124.18) circle (  1.96);

\path[draw=drawColor,line width= 0.4pt,line join=round,line cap=round,fill=fillColor] ( 50.33,126.28) circle (  1.96);

\path[draw=drawColor,line width= 0.4pt,line join=round,line cap=round,fill=fillColor] ( 50.33,138.88) circle (  1.96);

\path[draw=drawColor,line width= 0.4pt,line join=round,line cap=round,fill=fillColor] ( 50.33,129.43) circle (  1.96);

\path[draw=drawColor,line width= 0.4pt,line join=round,line cap=round,fill=fillColor] ( 50.33,138.88) circle (  1.96);

\path[draw=drawColor,line width= 0.4pt,line join=round,line cap=round,fill=fillColor] ( 50.33,125.02) circle (  1.96);

\path[draw=drawColor,line width= 0.4pt,line join=round,line cap=round,fill=fillColor] ( 50.33,150.22) circle (  1.96);

\path[draw=drawColor,line width= 0.4pt,line join=round,line cap=round,fill=fillColor] ( 50.33,132.58) circle (  1.96);

\path[draw=drawColor,line width= 0.4pt,line join=round,line cap=round,fill=fillColor] ( 50.33,132.58) circle (  1.96);

\path[draw=drawColor,line width= 0.4pt,line join=round,line cap=round,fill=fillColor] ( 50.33,125.38) circle (  1.96);

\path[draw=drawColor,line width= 0.4pt,line join=round,line cap=round,fill=fillColor] ( 50.33,138.88) circle (  1.96);

\path[draw=drawColor,line width= 0.4pt,line join=round,line cap=round,fill=fillColor] ( 50.33,123.13) circle (  1.96);

\path[draw=drawColor,line width= 0.4pt,line join=round,line cap=round,fill=fillColor] ( 50.33,124.70) circle (  1.96);

\path[draw=drawColor,line width= 0.4pt,line join=round,line cap=round,fill=fillColor] ( 50.33,123.13) circle (  1.96);

\path[draw=drawColor,line width= 0.4pt,line join=round,line cap=round,fill=fillColor] ( 50.33,124.18) circle (  1.96);

\path[draw=drawColor,line width= 0.4pt,line join=round,line cap=round,fill=fillColor] ( 50.33,146.74) circle (  1.96);

\path[draw=drawColor,line width= 0.4pt,line join=round,line cap=round,fill=fillColor] ( 50.33,127.54) circle (  1.96);

\path[draw=drawColor,line width= 0.4pt,line join=round,line cap=round,fill=fillColor] ( 50.33,127.54) circle (  1.96);

\path[draw=drawColor,line width= 0.4pt,line join=round,line cap=round,fill=fillColor] ( 50.33,126.03) circle (  1.96);

\path[draw=drawColor,line width= 0.4pt,line join=round,line cap=round,fill=fillColor] ( 50.33,130.26) circle (  1.96);

\path[draw=drawColor,line width= 0.4pt,line join=round,line cap=round,fill=fillColor] ( 50.33,126.03) circle (  1.96);

\path[draw=drawColor,line width= 0.4pt,line join=round,line cap=round,fill=fillColor] ( 50.33,127.54) circle (  1.96);

\path[draw=drawColor,line width= 0.4pt,line join=round,line cap=round,fill=fillColor] ( 50.33,124.21) circle (  1.96);

\path[draw=drawColor,line width= 0.4pt,line join=round,line cap=round,fill=fillColor] ( 50.33,123.42) circle (  1.96);

\path[draw=drawColor,line width= 0.4pt,line join=round,line cap=round,fill=fillColor] ( 50.33,134.34) circle (  1.96);

\path[draw=drawColor,line width= 0.4pt,line join=round,line cap=round,fill=fillColor] ( 50.33,123.76) circle (  1.96);

\path[draw=drawColor,line width= 0.4pt,line join=round,line cap=round,fill=fillColor] ( 50.33,138.88) circle (  1.96);

\path[draw=drawColor,line width= 0.4pt,line join=round,line cap=round,fill=fillColor] ( 50.33,136.61) circle (  1.96);

\path[draw=drawColor,line width= 0.4pt,line join=round,line cap=round,fill=fillColor] ( 50.33,142.66) circle (  1.96);

\path[draw=drawColor,line width= 0.4pt,line join=round,line cap=round,fill=fillColor] ( 50.33,126.28) circle (  1.96);

\path[draw=drawColor,line width= 0.4pt,line join=round,line cap=round,fill=fillColor] ( 50.33,126.59) circle (  1.96);

\path[draw=drawColor,line width= 0.4pt,line join=round,line cap=round,fill=fillColor] ( 50.33,136.61) circle (  1.96);

\path[draw=drawColor,line width= 0.4pt,line join=round,line cap=round,fill=fillColor] ( 50.33,127.54) circle (  1.96);

\path[draw=drawColor,line width= 0.4pt,line join=round,line cap=round,fill=fillColor] ( 50.33,140.98) circle (  1.96);

\path[draw=drawColor,line width= 0.4pt,line join=round,line cap=round,fill=fillColor] ( 50.33,123.76) circle (  1.96);

\path[draw=drawColor,line width= 0.4pt,line join=round,line cap=round,fill=fillColor] ( 50.33,123.13) circle (  1.96);

\path[draw=drawColor,line width= 0.4pt,line join=round,line cap=round,fill=fillColor] ( 50.33,127.54) circle (  1.96);

\path[draw=drawColor,line width= 0.4pt,line join=round,line cap=round,fill=fillColor] ( 50.33,125.02) circle (  1.96);

\path[draw=drawColor,line width= 0.4pt,line join=round,line cap=round,fill=fillColor] ( 50.33,138.88) circle (  1.96);

\path[draw=drawColor,line width= 0.4pt,line join=round,line cap=round,fill=fillColor] ( 50.33,125.68) circle (  1.96);

\path[draw=drawColor,line width= 0.4pt,line join=round,line cap=round,fill=fillColor] ( 50.33,129.08) circle (  1.96);

\path[draw=drawColor,line width= 0.4pt,line join=round,line cap=round,fill=fillColor] ( 50.33,133.45) circle (  1.96);

\path[draw=drawColor,line width= 0.4pt,line join=round,line cap=round,fill=fillColor] ( 50.33,133.16) circle (  1.96);

\path[draw=drawColor,line width= 0.4pt,line join=round,line cap=round,fill=fillColor] ( 50.33,139.96) circle (  1.96);

\path[draw=drawColor,line width= 0.4pt,line join=round,line cap=round,fill=fillColor] ( 50.33,133.45) circle (  1.96);

\path[draw=drawColor,line width= 0.4pt,line join=round,line cap=round,fill=fillColor] ( 50.33,122.96) circle (  1.96);

\path[draw=drawColor,line width= 0.4pt,line join=round,line cap=round,fill=fillColor] ( 50.33,131.12) circle (  1.96);

\path[draw=drawColor,line width= 0.4pt,line join=round,line cap=round,fill=fillColor] ( 50.33,122.96) circle (  1.96);

\path[draw=drawColor,line width= 0.4pt,line join=round,line cap=round,fill=fillColor] ( 50.33,136.56) circle (  1.96);

\path[draw=drawColor,line width= 0.4pt,line join=round,line cap=round,fill=fillColor] ( 50.33,122.96) circle (  1.96);

\path[draw=drawColor,line width= 0.4pt,line join=round,line cap=round,fill=fillColor] ( 50.33,122.96) circle (  1.96);

\path[draw=drawColor,line width= 0.4pt,line join=round,line cap=round,fill=fillColor] ( 50.33,133.84) circle (  1.96);

\path[draw=drawColor,line width= 0.4pt,line join=round,line cap=round,fill=fillColor] ( 50.33,131.12) circle (  1.96);

\path[draw=drawColor,line width= 0.4pt,line join=round,line cap=round,fill=fillColor] ( 50.33,122.96) circle (  1.96);

\path[draw=drawColor,line width= 0.4pt,line join=round,line cap=round,fill=fillColor] ( 50.33,122.96) circle (  1.96);

\path[draw=drawColor,line width= 0.4pt,line join=round,line cap=round,fill=fillColor] ( 50.33,122.96) circle (  1.96);

\path[draw=drawColor,line width= 0.4pt,line join=round,line cap=round,fill=fillColor] ( 50.33,126.67) circle (  1.96);

\path[draw=drawColor,line width= 0.4pt,line join=round,line cap=round,fill=fillColor] ( 50.33,122.96) circle (  1.96);

\path[draw=drawColor,line width= 0.4pt,line join=round,line cap=round,fill=fillColor] ( 50.33,131.12) circle (  1.96);

\path[draw=drawColor,line width= 0.4pt,line join=round,line cap=round,fill=fillColor] ( 50.33,140.64) circle (  1.96);

\path[draw=drawColor,line width= 0.4pt,line join=round,line cap=round,fill=fillColor] ( 50.33,133.16) circle (  1.96);

\path[draw=drawColor,line width= 0.4pt,line join=round,line cap=round,fill=fillColor] ( 50.33,129.76) circle (  1.96);

\path[draw=drawColor,line width= 0.4pt,line join=round,line cap=round,fill=fillColor] ( 50.33,139.28) circle (  1.96);

\path[draw=drawColor,line width= 0.4pt,line join=round,line cap=round,fill=fillColor] ( 50.33,125.68) circle (  1.96);

\path[draw=drawColor,line width= 0.4pt,line join=round,line cap=round,fill=fillColor] ( 50.33,128.79) circle (  1.96);

\path[draw=drawColor,line width= 0.4pt,line join=round,line cap=round,fill=fillColor] ( 50.33,128.79) circle (  1.96);

\path[draw=drawColor,line width= 0.4pt,line join=round,line cap=round,fill=fillColor] ( 50.33,138.26) circle (  1.96);

\path[draw=drawColor,line width= 0.4pt,line join=round,line cap=round,fill=fillColor] ( 50.33,122.96) circle (  1.96);

\path[draw=drawColor,line width= 0.4pt,line join=round,line cap=round,fill=fillColor] ( 50.33,122.96) circle (  1.96);

\path[draw=drawColor,line width= 0.4pt,line join=round,line cap=round,fill=fillColor] ( 50.33,132.94) circle (  1.96);

\path[draw=drawColor,line width= 0.4pt,line join=round,line cap=round,fill=fillColor] ( 50.33,126.23) circle (  1.96);

\path[draw=drawColor,line width= 0.4pt,line join=round,line cap=round,fill=fillColor] ( 50.33,127.53) circle (  1.96);

\path[draw=drawColor,line width= 0.4pt,line join=round,line cap=round,fill=fillColor] ( 50.33,127.50) circle (  1.96);

\path[draw=drawColor,line width= 0.4pt,line join=round,line cap=round,fill=fillColor] ( 50.33,122.96) circle (  1.96);

\path[draw=drawColor,line width= 0.4pt,line join=round,line cap=round,fill=fillColor] ( 50.33,122.96) circle (  1.96);

\path[draw=drawColor,line width= 0.4pt,line join=round,line cap=round,fill=fillColor] ( 50.33,131.12) circle (  1.96);

\path[draw=drawColor,line width= 0.4pt,line join=round,line cap=round,fill=fillColor] ( 50.33,122.96) circle (  1.96);

\path[draw=drawColor,line width= 0.4pt,line join=round,line cap=round,fill=fillColor] ( 50.33,126.23) circle (  1.96);

\path[draw=drawColor,line width= 0.4pt,line join=round,line cap=round,fill=fillColor] ( 50.33,124.66) circle (  1.96);

\path[draw=drawColor,line width= 0.4pt,line join=round,line cap=round,fill=fillColor] ( 50.33,131.12) circle (  1.96);

\path[draw=drawColor,line width= 0.4pt,line join=round,line cap=round,fill=fillColor] ( 50.33,131.12) circle (  1.96);

\path[draw=drawColor,line width= 0.4pt,line join=round,line cap=round,fill=fillColor] ( 50.33,137.65) circle (  1.96);

\path[draw=drawColor,line width= 0.4pt,line join=round,line cap=round,fill=fillColor] ( 50.33,125.68) circle (  1.96);

\path[draw=drawColor,line width= 0.4pt,line join=round,line cap=round,fill=fillColor] ( 50.33,122.96) circle (  1.96);

\path[draw=drawColor,line width= 0.4pt,line join=round,line cap=round,fill=fillColor] ( 50.33,125.00) circle (  1.96);

\path[draw=drawColor,line width= 0.4pt,line join=round,line cap=round,fill=fillColor] ( 50.33,133.30) circle (  1.96);

\path[draw=drawColor,line width= 0.4pt,line join=round,line cap=round,fill=fillColor] ( 50.33,131.06) circle (  1.96);

\path[draw=drawColor,line width= 0.4pt,line join=round,line cap=round,fill=fillColor] ( 50.33,127.86) circle (  1.96);

\path[draw=drawColor,line width= 0.4pt,line join=round,line cap=round,fill=fillColor] ( 50.33,122.96) circle (  1.96);

\path[draw=drawColor,line width= 0.4pt,line join=round,line cap=round,fill=fillColor] ( 50.33,132.21) circle (  1.96);

\path[draw=drawColor,line width= 0.4pt,line join=round,line cap=round,fill=fillColor] ( 50.33,150.16) circle (  1.96);

\path[draw=drawColor,line width= 0.4pt,line join=round,line cap=round,fill=fillColor] ( 50.33,145.63) circle (  1.96);

\path[draw=drawColor,line width= 0.4pt,line join=round,line cap=round,fill=fillColor] ( 50.33,122.96) circle (  1.96);

\path[draw=drawColor,line width= 0.4pt,line join=round,line cap=round,fill=fillColor] ( 50.33,124.13) circle (  1.96);

\path[draw=drawColor,line width= 0.4pt,line join=round,line cap=round,fill=fillColor] ( 50.33,127.50) circle (  1.96);

\path[draw=drawColor,line width= 0.4pt,line join=round,line cap=round,fill=fillColor] ( 50.33,143.36) circle (  1.96);

\path[draw=drawColor,line width= 0.4pt,line join=round,line cap=round,fill=fillColor] ( 50.33,131.12) circle (  1.96);

\path[draw=drawColor,line width= 0.4pt,line join=round,line cap=round,fill=fillColor] ( 50.33,132.03) circle (  1.96);

\path[draw=drawColor,line width= 0.4pt,line join=round,line cap=round,fill=fillColor] ( 50.33,122.96) circle (  1.96);

\path[draw=drawColor,line width= 0.4pt,line join=round,line cap=round,fill=fillColor] ( 50.33,124.60) circle (  1.96);

\path[draw=drawColor,line width= 0.4pt,line join=round,line cap=round,fill=fillColor] ( 50.33,125.68) circle (  1.96);

\path[draw=drawColor,line width= 0.4pt,line join=round,line cap=round,fill=fillColor] ( 50.33,131.12) circle (  1.96);

\path[draw=drawColor,line width= 0.4pt,line join=round,line cap=round,fill=fillColor] ( 50.33,136.56) circle (  1.96);

\path[draw=drawColor,line width= 0.4pt,line join=round,line cap=round,fill=fillColor] ( 50.33,135.20) circle (  1.96);

\path[draw=drawColor,line width= 0.4pt,line join=round,line cap=round,fill=fillColor] ( 50.33,144.18) circle (  1.96);

\path[draw=drawColor,line width= 0.4pt,line join=round,line cap=round,fill=fillColor] ( 50.33,134.39) circle (  1.96);

\path[draw=drawColor,line width= 0.4pt,line join=round,line cap=round,fill=fillColor] ( 50.33,151.52) circle (  1.96);

\path[draw=drawColor,line width= 0.4pt,line join=round,line cap=round,fill=fillColor] ( 50.33,127.04) circle (  1.96);

\path[draw=drawColor,line width= 0.4pt,line join=round,line cap=round,fill=fillColor] ( 50.33,136.56) circle (  1.96);

\path[draw=drawColor,line width= 0.4pt,line join=round,line cap=round,fill=fillColor] ( 50.33,137.65) circle (  1.96);

\path[draw=drawColor,line width= 0.4pt,line join=round,line cap=round,fill=fillColor] ( 50.33,135.27) circle (  1.96);

\path[draw=drawColor,line width= 0.4pt,line join=round,line cap=round,fill=fillColor] ( 50.33,127.68) circle (  1.96);

\path[draw=drawColor,line width= 0.4pt,line join=round,line cap=round,fill=fillColor] ( 50.33,127.68) circle (  1.96);

\path[draw=drawColor,line width= 0.4pt,line join=round,line cap=round,fill=fillColor] ( 50.33,123.11) circle (  1.96);

\path[draw=drawColor,line width= 0.4pt,line join=round,line cap=round,fill=fillColor] ( 50.33,123.11) circle (  1.96);

\path[draw=drawColor,line width= 0.4pt,line join=round,line cap=round,fill=fillColor] ( 50.33,151.52) circle (  1.96);

\path[draw=drawColor,line width= 0.4pt,line join=round,line cap=round,fill=fillColor] ( 50.33,151.52) circle (  1.96);

\path[draw=drawColor,line width= 0.4pt,line join=round,line cap=round,fill=fillColor] ( 50.33,127.32) circle (  1.96);

\path[draw=drawColor,line width= 0.4pt,line join=round,line cap=round,fill=fillColor] ( 50.33,127.32) circle (  1.96);

\path[draw=drawColor,line width= 0.4pt,line join=round,line cap=round,fill=fillColor] ( 50.33,125.52) circle (  1.96);

\path[draw=drawColor,line width= 0.4pt,line join=round,line cap=round,fill=fillColor] ( 50.33,127.25) circle (  1.96);

\path[draw=drawColor,line width= 0.4pt,line join=round,line cap=round,fill=fillColor] ( 50.33,127.25) circle (  1.96);

\path[draw=drawColor,line width= 0.4pt,line join=round,line cap=round,fill=fillColor] ( 50.33,139.96) circle (  1.96);

\path[draw=drawColor,line width= 0.4pt,line join=round,line cap=round,fill=fillColor] ( 50.33,139.96) circle (  1.96);

\path[draw=drawColor,line width= 0.4pt,line join=round,line cap=round,fill=fillColor] ( 50.33,134.19) circle (  1.96);

\path[draw=drawColor,line width= 0.4pt,line join=round,line cap=round,fill=fillColor] ( 50.33,136.35) circle (  1.96);

\path[draw=drawColor,line width= 0.4pt,line join=round,line cap=round,fill=fillColor] ( 50.33,125.04) circle (  1.96);

\path[draw=drawColor,line width= 0.4pt,line join=round,line cap=round,fill=fillColor] ( 50.33,125.04) circle (  1.96);

\path[draw=drawColor,line width= 0.4pt,line join=round,line cap=round,fill=fillColor] ( 50.33,144.78) circle (  1.96);

\path[draw=drawColor,line width= 0.4pt,line join=round,line cap=round,fill=fillColor] ( 50.33,144.78) circle (  1.96);

\path[draw=drawColor,line width= 0.4pt,line join=round,line cap=round,fill=fillColor] ( 50.33,136.35) circle (  1.96);

\path[draw=drawColor,line width= 0.4pt,line join=round,line cap=round,fill=fillColor] ( 50.33,136.35) circle (  1.96);

\path[draw=drawColor,line width= 0.4pt,line join=round,line cap=round,fill=fillColor] ( 50.33,125.52) circle (  1.96);

\path[draw=drawColor,line width= 0.4pt,line join=round,line cap=round,fill=fillColor] ( 50.33,125.52) circle (  1.96);

\path[draw=drawColor,line width= 0.4pt,line join=round,line cap=round,fill=fillColor] ( 50.33,125.52) circle (  1.96);

\path[draw=drawColor,line width= 0.4pt,line join=round,line cap=round,fill=fillColor] ( 50.33,125.52) circle (  1.96);

\path[draw=drawColor,line width= 0.4pt,line join=round,line cap=round,fill=fillColor] ( 50.33,125.52) circle (  1.96);

\path[draw=drawColor,line width= 0.4pt,line join=round,line cap=round,fill=fillColor] ( 50.33,125.52) circle (  1.96);

\path[draw=drawColor,line width= 0.4pt,line join=round,line cap=round,fill=fillColor] ( 50.33,141.28) circle (  1.96);

\path[draw=drawColor,line width= 0.4pt,line join=round,line cap=round,fill=fillColor] ( 50.33,131.43) circle (  1.96);

\path[draw=drawColor,line width= 0.4pt,line join=round,line cap=round,fill=fillColor] ( 50.33,139.96) circle (  1.96);

\path[draw=drawColor,line width= 0.4pt,line join=round,line cap=round,fill=fillColor] ( 50.33,125.52) circle (  1.96);

\path[draw=drawColor,line width= 0.4pt,line join=round,line cap=round,fill=fillColor] ( 50.33,131.43) circle (  1.96);

\path[draw=drawColor,line width= 0.4pt,line join=round,line cap=round,fill=fillColor] ( 50.33,147.19) circle (  1.96);

\path[draw=drawColor,line width= 0.4pt,line join=round,line cap=round,fill=fillColor] ( 50.33,125.52) circle (  1.96);

\path[draw=drawColor,line width= 0.4pt,line join=round,line cap=round,fill=fillColor] ( 50.33,128.41) circle (  1.96);

\path[draw=drawColor,line width= 0.4pt,line join=round,line cap=round,fill=fillColor] ( 50.33,130.33) circle (  1.96);

\path[draw=drawColor,line width= 0.4pt,line join=round,line cap=round,fill=fillColor] ( 50.33,128.41) circle (  1.96);

\path[draw=drawColor,line width= 0.4pt,line join=round,line cap=round,fill=fillColor] ( 50.33,136.35) circle (  1.96);

\path[draw=drawColor,line width= 0.4pt,line join=round,line cap=round,fill=fillColor] ( 50.33,125.52) circle (  1.96);

\path[draw=drawColor,line width= 0.4pt,line join=round,line cap=round,fill=fillColor] ( 50.33,123.11) circle (  1.96);

\path[draw=drawColor,line width= 0.4pt,line join=round,line cap=round,fill=fillColor] ( 50.33,125.52) circle (  1.96);

\path[draw=drawColor,line width= 0.4pt,line join=round,line cap=round,fill=fillColor] ( 50.33,122.81) circle (  1.96);

\path[draw=drawColor,line width= 0.4pt,line join=round,line cap=round,fill=fillColor] ( 50.33,125.52) circle (  1.96);

\path[draw=drawColor,line width= 0.4pt,line join=round,line cap=round,fill=fillColor] ( 50.33,125.52) circle (  1.96);

\path[draw=drawColor,line width= 0.4pt,line join=round,line cap=round,fill=fillColor] ( 50.33,138.76) circle (  1.96);

\path[draw=drawColor,line width= 0.4pt,line join=round,line cap=round,fill=fillColor] ( 50.33,134.19) circle (  1.96);

\path[draw=drawColor,line width= 0.4pt,line join=round,line cap=round,fill=fillColor] ( 50.33,134.19) circle (  1.96);

\path[draw=drawColor,line width= 0.4pt,line join=round,line cap=round,fill=fillColor] ( 50.33,134.19) circle (  1.96);

\path[draw=drawColor,line width= 0.4pt,line join=round,line cap=round,fill=fillColor] ( 50.33,125.52) circle (  1.96);

\path[draw=drawColor,line width= 0.4pt,line join=round,line cap=round,fill=fillColor] ( 50.33,154.41) circle (  1.96);

\path[draw=drawColor,line width= 0.4pt,line join=round,line cap=round,fill=fillColor] ( 50.33,134.19) circle (  1.96);

\path[draw=drawColor,line width= 0.4pt,line join=round,line cap=round,fill=fillColor] ( 50.33,134.19) circle (  1.96);

\path[draw=drawColor,line width= 0.4pt,line join=round,line cap=round,fill=fillColor] ( 50.33,134.19) circle (  1.96);

\path[draw=drawColor,line width= 0.4pt,line join=round,line cap=round,fill=fillColor] ( 50.33,125.52) circle (  1.96);

\path[draw=drawColor,line width= 0.4pt,line join=round,line cap=round,fill=fillColor] ( 50.33,125.52) circle (  1.96);

\path[draw=drawColor,line width= 0.4pt,line join=round,line cap=round,fill=fillColor] ( 50.33,151.52) circle (  1.96);

\path[draw=drawColor,line width= 0.4pt,line join=round,line cap=round,fill=fillColor] ( 50.33,130.72) circle (  1.96);

\path[draw=drawColor,line width= 0.4pt,line join=round,line cap=round,fill=fillColor] ( 50.33,139.96) circle (  1.96);

\path[draw=drawColor,line width= 0.4pt,line join=round,line cap=round,fill=fillColor] ( 50.33,134.19) circle (  1.96);

\path[draw=drawColor,line width= 0.4pt,line join=round,line cap=round,fill=fillColor] ( 50.33,134.19) circle (  1.96);

\path[draw=drawColor,line width= 0.4pt,line join=round,line cap=round,fill=fillColor] ( 50.33,131.71) circle (  1.96);

\path[draw=drawColor,line width= 0.4pt,line join=round,line cap=round,fill=fillColor] ( 50.33,141.12) circle (  1.96);

\path[draw=drawColor,line width= 0.4pt,line join=round,line cap=round,fill=fillColor] ( 50.33,125.52) circle (  1.96);

\path[draw=drawColor,line width= 0.4pt,line join=round,line cap=round,fill=fillColor] ( 50.33,131.30) circle (  1.96);

\path[draw=drawColor,line width= 0.4pt,line join=round,line cap=round,fill=fillColor] ( 50.33,131.30) circle (  1.96);

\path[draw=drawColor,line width= 0.4pt,line join=round,line cap=round,fill=fillColor] ( 50.33,125.52) circle (  1.96);

\path[draw=drawColor,line width= 0.4pt,line join=round,line cap=round,fill=fillColor] ( 50.33,125.52) circle (  1.96);

\path[draw=drawColor,line width= 0.4pt,line join=round,line cap=round,fill=fillColor] ( 50.33,134.19) circle (  1.96);

\path[draw=drawColor,line width= 0.4pt,line join=round,line cap=round,fill=fillColor] ( 50.33,136.35) circle (  1.96);

\path[draw=drawColor,line width= 0.4pt,line join=round,line cap=round,fill=fillColor] ( 50.33,130.68) circle (  1.96);

\path[draw=drawColor,line width= 0.4pt,line join=round,line cap=round,fill=fillColor] ( 50.33,123.35) circle (  1.96);

\path[draw=drawColor,line width= 0.4pt,line join=round,line cap=round,fill=fillColor] ( 50.33,125.52) circle (  1.96);

\path[draw=drawColor,line width= 0.4pt,line join=round,line cap=round,fill=fillColor] ( 50.33,125.52) circle (  1.96);

\path[draw=drawColor,line width= 0.4pt,line join=round,line cap=round,fill=fillColor] ( 50.33,127.68) circle (  1.96);

\path[draw=drawColor,line width= 0.4pt,line join=round,line cap=round,fill=fillColor] ( 50.33,125.52) circle (  1.96);

\path[draw=drawColor,line width= 0.4pt,line join=round,line cap=round,fill=fillColor] ( 50.33,125.52) circle (  1.96);

\path[draw=drawColor,line width= 0.4pt,line join=round,line cap=round,fill=fillColor] ( 50.33,151.52) circle (  1.96);

\path[draw=drawColor,line width= 0.4pt,line join=round,line cap=round,fill=fillColor] ( 50.33,130.72) circle (  1.96);

\path[draw=drawColor,line width= 0.4pt,line join=round,line cap=round,fill=fillColor] ( 50.33,134.19) circle (  1.96);

\path[draw=drawColor,line width= 0.4pt,line join=round,line cap=round,fill=fillColor] ( 50.33,128.41) circle (  1.96);

\path[draw=drawColor,line width= 0.4pt,line join=round,line cap=round,fill=fillColor] ( 50.33,125.52) circle (  1.96);

\path[draw=drawColor,line width= 0.4pt,line join=round,line cap=round,fill=fillColor] ( 50.33,125.52) circle (  1.96);

\path[draw=drawColor,line width= 0.4pt,line join=round,line cap=round,fill=fillColor] ( 50.33,125.52) circle (  1.96);

\path[draw=drawColor,line width= 0.4pt,line join=round,line cap=round,fill=fillColor] ( 50.33,125.52) circle (  1.96);

\path[draw=drawColor,line width= 0.4pt,line join=round,line cap=round,fill=fillColor] ( 50.33,123.78) circle (  1.96);

\path[draw=drawColor,line width= 0.4pt,line join=round,line cap=round,fill=fillColor] ( 50.33,134.19) circle (  1.96);

\path[draw=drawColor,line width= 0.4pt,line join=round,line cap=round,fill=fillColor] ( 50.33,125.52) circle (  1.96);

\path[draw=drawColor,line width= 0.4pt,line join=round,line cap=round,fill=fillColor] ( 50.33,130.33) circle (  1.96);

\path[draw=drawColor,line width= 0.4pt,line join=round,line cap=round,fill=fillColor] ( 50.33,125.52) circle (  1.96);

\path[draw=drawColor,line width= 0.4pt,line join=round,line cap=round,fill=fillColor] ( 50.33,139.96) circle (  1.96);

\path[draw=drawColor,line width= 0.4pt,line join=round,line cap=round,fill=fillColor] ( 50.33,147.19) circle (  1.96);

\path[draw=drawColor,line width= 0.4pt,line join=round,line cap=round,fill=fillColor] ( 50.33,129.13) circle (  1.96);

\path[draw=drawColor,line width= 0.4pt,line join=round,line cap=round,fill=fillColor] ( 50.33,125.81) circle (  1.96);

\path[draw=drawColor,line width= 0.4pt,line join=round,line cap=round,fill=fillColor] ( 50.33,148.63) circle (  1.96);

\path[draw=drawColor,line width= 0.4pt,line join=round,line cap=round,fill=fillColor] ( 50.33,147.19) circle (  1.96);

\path[draw=drawColor,line width= 0.4pt,line join=round,line cap=round,fill=fillColor] ( 50.33,142.85) circle (  1.96);

\path[draw=drawColor,line width= 0.4pt,line join=round,line cap=round,fill=fillColor] ( 50.33,136.35) circle (  1.96);

\path[draw=drawColor,line width= 0.4pt,line join=round,line cap=round,fill=fillColor] ( 50.33,134.19) circle (  1.96);

\path[draw=drawColor,line width= 0.4pt,line join=round,line cap=round,fill=fillColor] ( 50.33,127.49) circle (  1.96);

\path[draw=drawColor,line width= 0.4pt,line join=round,line cap=round,fill=fillColor] ( 50.33,126.48) circle (  1.96);

\path[draw=drawColor,line width= 0.4pt,line join=round,line cap=round,fill=fillColor] ( 50.33,135.04) circle (  1.96);

\path[draw=drawColor,line width= 0.4pt,line join=round,line cap=round,fill=fillColor] ( 50.33,127.49) circle (  1.96);

\path[draw=drawColor,line width= 0.4pt,line join=round,line cap=round,fill=fillColor] ( 50.33,136.55) circle (  1.96);

\path[draw=drawColor,line width= 0.4pt,line join=round,line cap=round,fill=fillColor] ( 50.33,132.93) circle (  1.96);

\path[draw=drawColor,line width= 0.4pt,line join=round,line cap=round,fill=fillColor] ( 50.33,150.14) circle (  1.96);

\path[draw=drawColor,line width= 0.4pt,line join=round,line cap=round,fill=fillColor] ( 50.33,124.47) circle (  1.96);

\path[draw=drawColor,line width= 0.4pt,line join=round,line cap=round,fill=fillColor] ( 50.33,129.55) circle (  1.96);

\path[draw=drawColor,line width= 0.4pt,line join=round,line cap=round,fill=fillColor] ( 50.33,132.02) circle (  1.96);

\path[draw=drawColor,line width= 0.4pt,line join=round,line cap=round,fill=fillColor] ( 50.33,123.71) circle (  1.96);

\path[draw=drawColor,line width= 0.4pt,line join=round,line cap=round,fill=fillColor] ( 50.33,127.49) circle (  1.96);

\path[draw=drawColor,line width= 0.4pt,line join=round,line cap=round,fill=fillColor] ( 50.33,127.49) circle (  1.96);

\path[draw=drawColor,line width= 0.4pt,line join=round,line cap=round,fill=fillColor] ( 50.33,136.55) circle (  1.96);

\path[draw=drawColor,line width= 0.4pt,line join=round,line cap=round,fill=fillColor] ( 50.33,132.02) circle (  1.96);

\path[draw=drawColor,line width= 0.4pt,line join=round,line cap=round,fill=fillColor] ( 50.33,132.02) circle (  1.96);

\path[draw=drawColor,line width= 0.4pt,line join=round,line cap=round,fill=fillColor] ( 50.33,132.93) circle (  1.96);

\path[draw=drawColor,line width= 0.4pt,line join=round,line cap=round,fill=fillColor] ( 50.33,122.96) circle (  1.96);

\path[draw=drawColor,line width= 0.4pt,line join=round,line cap=round,fill=fillColor] ( 50.33,128.94) circle (  1.96);

\path[draw=drawColor,line width= 0.4pt,line join=round,line cap=round,fill=fillColor] ( 50.33,125.43) circle (  1.96);

\path[draw=drawColor,line width= 0.4pt,line join=round,line cap=round,fill=fillColor] ( 50.33,125.43) circle (  1.96);

\path[draw=drawColor,line width= 0.4pt,line join=round,line cap=round,fill=fillColor] ( 50.33,131.11) circle (  1.96);

\path[draw=drawColor,line width= 0.4pt,line join=round,line cap=round,fill=fillColor] ( 50.33,129.75) circle (  1.96);

\path[draw=drawColor,line width= 0.4pt,line join=round,line cap=round,fill=fillColor] ( 50.33,133.96) circle (  1.96);

\path[draw=drawColor,line width= 0.4pt,line join=round,line cap=round,fill=fillColor] ( 50.33,133.96) circle (  1.96);

\path[draw=drawColor,line width= 0.4pt,line join=round,line cap=round,fill=fillColor] ( 50.33,122.78) circle (  1.96);

\path[draw=drawColor,line width= 0.4pt,line join=round,line cap=round,fill=fillColor] ( 50.33,122.78) circle (  1.96);

\path[draw=drawColor,line width= 0.4pt,line join=round,line cap=round,fill=fillColor] ( 50.33,132.56) circle (  1.96);

\path[draw=drawColor,line width= 0.4pt,line join=round,line cap=round,fill=fillColor] ( 50.33,124.47) circle (  1.96);

\path[draw=drawColor,line width= 0.4pt,line join=round,line cap=round,fill=fillColor] ( 50.33,124.47) circle (  1.96);

\path[draw=drawColor,line width= 0.4pt,line join=round,line cap=round,fill=fillColor] ( 50.33,142.59) circle (  1.96);

\path[draw=drawColor,line width= 0.4pt,line join=round,line cap=round,fill=fillColor] ( 50.33,138.82) circle (  1.96);

\path[draw=drawColor,line width= 0.4pt,line join=round,line cap=round,fill=fillColor] ( 50.33,127.49) circle (  1.96);

\path[draw=drawColor,line width= 0.4pt,line join=round,line cap=round,fill=fillColor] ( 50.33,126.23) circle (  1.96);

\path[draw=drawColor,line width= 0.4pt,line join=round,line cap=round,fill=fillColor] ( 50.33,147.88) circle (  1.96);

\path[draw=drawColor,line width= 0.4pt,line join=round,line cap=round,fill=fillColor] ( 50.33,123.37) circle (  1.96);

\path[draw=drawColor,line width= 0.4pt,line join=round,line cap=round,fill=fillColor] ( 50.33,130.01) circle (  1.96);

\path[draw=drawColor,line width= 0.4pt,line join=round,line cap=round,fill=fillColor] ( 50.33,122.86) circle (  1.96);

\path[draw=drawColor,line width= 0.4pt,line join=round,line cap=round,fill=fillColor] ( 50.33,146.37) circle (  1.96);

\path[draw=drawColor,line width= 0.4pt,line join=round,line cap=round,fill=fillColor] ( 50.33,135.73) circle (  1.96);

\path[draw=drawColor,line width= 0.4pt,line join=round,line cap=round,fill=fillColor] ( 50.33,124.97) circle (  1.96);

\path[draw=drawColor,line width= 0.4pt,line join=round,line cap=round,fill=fillColor] ( 50.33,133.67) circle (  1.96);

\path[draw=drawColor,line width= 0.4pt,line join=round,line cap=round,fill=fillColor] ( 50.33,127.49) circle (  1.96);

\path[draw=drawColor,line width= 0.4pt,line join=round,line cap=round,fill=fillColor] ( 50.33,127.49) circle (  1.96);

\path[draw=drawColor,line width= 0.4pt,line join=round,line cap=round,fill=fillColor] ( 50.33,122.96) circle (  1.96);

\path[draw=drawColor,line width= 0.4pt,line join=round,line cap=round,fill=fillColor] ( 50.33,133.67) circle (  1.96);

\path[draw=drawColor,line width= 0.4pt,line join=round,line cap=round,fill=fillColor] ( 50.33,130.51) circle (  1.96);

\path[draw=drawColor,line width= 0.4pt,line join=round,line cap=round,fill=fillColor] ( 50.33,122.96) circle (  1.96);

\path[draw=drawColor,line width= 0.4pt,line join=round,line cap=round,fill=fillColor] ( 50.33,129.30) circle (  1.96);

\path[draw=drawColor,line width= 0.4pt,line join=round,line cap=round,fill=fillColor] ( 50.33,130.72) circle (  1.96);

\path[draw=drawColor,line width= 0.4pt,line join=round,line cap=round,fill=fillColor] ( 50.33,125.22) circle (  1.96);

\path[draw=drawColor,line width= 0.4pt,line join=round,line cap=round,fill=fillColor] ( 50.33,122.96) circle (  1.96);

\path[draw=drawColor,line width= 0.4pt,line join=round,line cap=round,fill=fillColor] ( 50.33,131.85) circle (  1.96);

\path[draw=drawColor,line width= 0.4pt,line join=round,line cap=round,fill=fillColor] ( 50.33,127.49) circle (  1.96);

\path[draw=drawColor,line width= 0.4pt,line join=round,line cap=round,fill=fillColor] ( 50.33,129.38) circle (  1.96);

\path[draw=drawColor,line width= 0.4pt,line join=round,line cap=round,fill=fillColor] ( 50.33,130.51) circle (  1.96);

\path[draw=drawColor,line width= 0.4pt,line join=round,line cap=round,fill=fillColor] ( 50.33,136.55) circle (  1.96);

\path[draw=drawColor,line width= 0.4pt,line join=round,line cap=round,fill=fillColor] ( 50.33,142.59) circle (  1.96);

\path[draw=drawColor,line width= 0.4pt,line join=round,line cap=round,fill=fillColor] ( 50.33,125.31) circle (  1.96);

\path[draw=drawColor,line width= 0.4pt,line join=round,line cap=round,fill=fillColor] ( 50.33,142.59) circle (  1.96);

\path[draw=drawColor,line width= 0.4pt,line join=round,line cap=round,fill=fillColor] ( 50.33,135.04) circle (  1.96);

\path[draw=drawColor,line width= 0.4pt,line join=round,line cap=round,fill=fillColor] ( 50.33,135.04) circle (  1.96);

\path[draw=drawColor,line width= 0.4pt,line join=round,line cap=round,fill=fillColor] ( 50.33,123.37) circle (  1.96);

\path[draw=drawColor,line width= 0.4pt,line join=round,line cap=round,fill=fillColor] ( 50.33,123.37) circle (  1.96);

\path[draw=drawColor,line width= 0.4pt,line join=round,line cap=round,fill=fillColor] ( 50.33,127.49) circle (  1.96);

\path[draw=drawColor,line width= 0.4pt,line join=round,line cap=round,fill=fillColor] ( 50.33,136.55) circle (  1.96);

\path[draw=drawColor,line width= 0.4pt,line join=round,line cap=round,fill=fillColor] ( 50.33,136.55) circle (  1.96);

\path[draw=drawColor,line width= 0.4pt,line join=round,line cap=round,fill=fillColor] ( 50.33,124.47) circle (  1.96);

\path[draw=drawColor,line width= 0.4pt,line join=round,line cap=round,fill=fillColor] ( 50.33,127.45) circle (  1.96);

\path[draw=drawColor,line width= 0.4pt,line join=round,line cap=round,fill=fillColor] ( 50.33,127.45) circle (  1.96);

\path[draw=drawColor,line width= 0.4pt,line join=round,line cap=round,fill=fillColor] ( 50.33,122.91) circle (  1.96);

\path[draw=drawColor,line width= 0.4pt,line join=round,line cap=round,fill=fillColor] ( 50.33,122.91) circle (  1.96);

\path[draw=drawColor,line width= 0.4pt,line join=round,line cap=round,fill=fillColor] ( 50.33,132.29) circle (  1.96);

\path[draw=drawColor,line width= 0.4pt,line join=round,line cap=round,fill=fillColor] ( 50.33,132.48) circle (  1.96);

\path[draw=drawColor,line width= 0.4pt,line join=round,line cap=round,fill=fillColor] ( 50.33,139.82) circle (  1.96);

\path[draw=drawColor,line width= 0.4pt,line join=round,line cap=round,fill=fillColor] ( 50.33,139.82) circle (  1.96);

\path[draw=drawColor,line width= 0.4pt,line join=round,line cap=round,fill=fillColor] ( 50.33,124.62) circle (  1.96);

\path[draw=drawColor,line width= 0.4pt,line join=round,line cap=round,fill=fillColor] ( 50.33,129.34) circle (  1.96);

\path[draw=drawColor,line width= 0.4pt,line join=round,line cap=round,fill=fillColor] ( 50.33,126.70) circle (  1.96);

\path[draw=drawColor,line width= 0.4pt,line join=round,line cap=round,fill=fillColor] ( 50.33,126.70) circle (  1.96);

\path[draw=drawColor,line width= 0.4pt,line join=round,line cap=round,fill=fillColor] ( 50.33,129.34) circle (  1.96);

\path[draw=drawColor,line width= 0.4pt,line join=round,line cap=round,fill=fillColor] ( 50.33,132.48) circle (  1.96);

\path[draw=drawColor,line width= 0.4pt,line join=round,line cap=round,fill=fillColor] ( 50.33,123.05) circle (  1.96);

\path[draw=drawColor,line width= 0.4pt,line join=round,line cap=round,fill=fillColor] ( 50.33,131.96) circle (  1.96);

\path[draw=drawColor,line width= 0.4pt,line join=round,line cap=round,fill=fillColor] ( 50.33,129.34) circle (  1.96);

\path[draw=drawColor,line width= 0.4pt,line join=round,line cap=round,fill=fillColor] ( 50.33,152.92) circle (  1.96);

\path[draw=drawColor,line width= 0.4pt,line join=round,line cap=round,fill=fillColor] ( 50.33,124.85) circle (  1.96);

\path[draw=drawColor,line width= 0.4pt,line join=round,line cap=round,fill=fillColor] ( 50.33,134.24) circle (  1.96);

\path[draw=drawColor,line width= 0.4pt,line join=round,line cap=round,fill=fillColor] ( 50.33,123.05) circle (  1.96);

\path[draw=drawColor,line width= 0.4pt,line join=round,line cap=round,fill=fillColor] ( 50.33,123.05) circle (  1.96);

\path[draw=drawColor,line width= 0.4pt,line join=round,line cap=round,fill=fillColor] ( 50.33,143.49) circle (  1.96);

\path[draw=drawColor,line width= 0.4pt,line join=round,line cap=round,fill=fillColor] ( 50.33,143.49) circle (  1.96);

\path[draw=drawColor,line width= 0.4pt,line join=round,line cap=round,fill=fillColor] ( 50.33,124.62) circle (  1.96);

\path[draw=drawColor,line width= 0.4pt,line join=round,line cap=round,fill=fillColor] ( 50.33,126.39) circle (  1.96);

\path[draw=drawColor,line width= 0.4pt,line join=round,line cap=round,fill=fillColor] ( 50.33,123.05) circle (  1.96);

\path[draw=drawColor,line width= 0.4pt,line join=round,line cap=round,fill=fillColor] ( 50.33,134.58) circle (  1.96);

\path[draw=drawColor,line width= 0.4pt,line join=round,line cap=round,fill=fillColor] ( 50.33,141.13) circle (  1.96);

\path[draw=drawColor,line width= 0.4pt,line join=round,line cap=round,fill=fillColor] ( 50.33,135.63) circle (  1.96);

\path[draw=drawColor,line width= 0.4pt,line join=round,line cap=round,fill=fillColor] ( 50.33,123.05) circle (  1.96);

\path[draw=drawColor,line width= 0.4pt,line join=round,line cap=round,fill=fillColor] ( 50.33,134.58) circle (  1.96);

\path[draw=drawColor,line width= 0.4pt,line join=round,line cap=round,fill=fillColor] ( 50.33,129.34) circle (  1.96);

\path[draw=drawColor,line width= 0.4pt,line join=round,line cap=round,fill=fillColor] ( 50.33,124.62) circle (  1.96);

\path[draw=drawColor,line width= 0.4pt,line join=round,line cap=round,fill=fillColor] ( 50.33,129.34) circle (  1.96);

\path[draw=drawColor,line width= 0.4pt,line join=round,line cap=round,fill=fillColor] ( 50.33,135.23) circle (  1.96);

\path[draw=drawColor,line width= 0.4pt,line join=round,line cap=round,fill=fillColor] ( 50.33,150.78) circle (  1.96);

\path[draw=drawColor,line width= 0.4pt,line join=round,line cap=round,fill=fillColor] ( 50.33,126.39) circle (  1.96);

\path[draw=drawColor,line width= 0.4pt,line join=round,line cap=round,fill=fillColor] ( 50.33,129.34) circle (  1.96);

\path[draw=drawColor,line width= 0.4pt,line join=round,line cap=round,fill=fillColor] ( 50.33,147.02) circle (  1.96);

\path[draw=drawColor,line width= 0.4pt,line join=round,line cap=round,fill=fillColor] ( 50.33,123.44) circle (  1.96);

\path[draw=drawColor,line width= 0.4pt,line join=round,line cap=round,fill=fillColor] ( 50.33,135.94) circle (  1.96);

\path[draw=drawColor,line width= 0.4pt,line join=round,line cap=round,fill=fillColor] ( 50.33,125.41) circle (  1.96);

\path[draw=drawColor,line width= 0.4pt,line join=round,line cap=round,fill=fillColor] ( 50.33,129.34) circle (  1.96);

\path[draw=drawColor,line width= 0.4pt,line join=round,line cap=round,fill=fillColor] ( 50.33,129.34) circle (  1.96);

\path[draw=drawColor,line width= 0.4pt,line join=round,line cap=round,fill=fillColor] ( 50.33,131.22) circle (  1.96);

\path[draw=drawColor,line width= 0.4pt,line join=round,line cap=round,fill=fillColor] ( 50.33,138.77) circle (  1.96);

\path[draw=drawColor,line width= 0.4pt,line join=round,line cap=round,fill=fillColor] ( 50.33,126.98) circle (  1.96);

\path[draw=drawColor,line width= 0.4pt,line join=round,line cap=round,fill=fillColor] ( 50.33,126.98) circle (  1.96);

\path[draw=drawColor,line width= 0.4pt,line join=round,line cap=round,fill=fillColor] ( 50.33,126.98) circle (  1.96);

\path[draw=drawColor,line width= 0.4pt,line join=round,line cap=round,fill=fillColor] ( 50.33,126.98) circle (  1.96);

\path[draw=drawColor,line width= 0.4pt,line join=round,line cap=round,fill=fillColor] ( 50.33,126.19) circle (  1.96);

\path[draw=drawColor,line width= 0.4pt,line join=round,line cap=round,fill=fillColor] ( 50.33,123.44) circle (  1.96);

\path[draw=drawColor,line width= 0.4pt,line join=round,line cap=round,fill=fillColor] ( 50.33,140.03) circle (  1.96);

\path[draw=drawColor,line width= 0.4pt,line join=round,line cap=round,fill=fillColor] ( 50.33,140.03) circle (  1.96);

\path[draw=drawColor,line width= 0.4pt,line join=round,line cap=round,fill=fillColor] ( 50.33,127.09) circle (  1.96);

\path[draw=drawColor,line width= 0.4pt,line join=round,line cap=round,fill=fillColor] ( 50.33,127.09) circle (  1.96);

\path[draw=drawColor,line width= 0.4pt,line join=round,line cap=round,fill=fillColor] ( 50.33,127.45) circle (  1.96);

\path[draw=drawColor,line width= 0.4pt,line join=round,line cap=round,fill=fillColor] ( 50.33,127.45) circle (  1.96);

\path[draw=drawColor,line width= 0.4pt,line join=round,line cap=round,fill=fillColor] ( 50.33,124.10) circle (  1.96);

\path[draw=drawColor,line width= 0.4pt,line join=round,line cap=round,fill=fillColor] ( 50.33,123.44) circle (  1.96);

\path[draw=drawColor,line width= 0.4pt,line join=round,line cap=round,fill=fillColor] ( 50.33,145.06) circle (  1.96);

\path[draw=drawColor,line width= 0.4pt,line join=round,line cap=round,fill=fillColor] ( 50.33,145.06) circle (  1.96);

\path[draw=drawColor,line width= 0.4pt,line join=round,line cap=round,fill=fillColor] ( 50.33,123.44) circle (  1.96);

\path[draw=drawColor,line width= 0.4pt,line join=round,line cap=round,fill=fillColor] ( 50.33,131.30) circle (  1.96);

\path[draw=drawColor,line width= 0.4pt,line join=round,line cap=round,fill=fillColor] ( 50.33,124.62) circle (  1.96);

\path[draw=drawColor,line width= 0.4pt,line join=round,line cap=round,fill=fillColor] ( 50.33,129.34) circle (  1.96);

\path[draw=drawColor,line width= 0.4pt,line join=round,line cap=round,fill=fillColor] ( 50.33,127.45) circle (  1.96);

\path[draw=drawColor,line width= 0.4pt,line join=round,line cap=round,fill=fillColor] ( 50.33,126.98) circle (  1.96);

\path[draw=drawColor,line width= 0.4pt,line join=round,line cap=round,fill=fillColor] ( 50.33,126.98) circle (  1.96);

\path[draw=drawColor,line width= 0.4pt,line join=round,line cap=round,fill=fillColor] ( 50.33,124.62) circle (  1.96);

\path[draw=drawColor,line width= 0.4pt,line join=round,line cap=round,fill=fillColor] ( 50.33,124.62) circle (  1.96);

\path[draw=drawColor,line width= 0.4pt,line join=round,line cap=round,fill=fillColor] ( 50.33,123.95) circle (  1.96);

\path[draw=drawColor,line width= 0.4pt,line join=round,line cap=round,fill=fillColor] ( 50.33,141.59) circle (  1.96);

\path[draw=drawColor,line width= 0.4pt,line join=round,line cap=round,fill=fillColor] ( 50.33,132.31) circle (  1.96);

\path[draw=drawColor,line width= 0.4pt,line join=round,line cap=round,fill=fillColor] ( 50.33,139.47) circle (  1.96);

\path[draw=drawColor,line width= 0.4pt,line join=round,line cap=round,fill=fillColor] ( 50.33,122.91) circle (  1.96);

\path[draw=drawColor,line width= 0.4pt,line join=round,line cap=round,fill=fillColor] ( 50.33,132.31) circle (  1.96);

\path[draw=drawColor,line width= 0.4pt,line join=round,line cap=round,fill=fillColor] ( 50.33,132.31) circle (  1.96);

\path[draw=drawColor,line width= 0.4pt,line join=round,line cap=round,fill=fillColor] ( 50.33,132.31) circle (  1.96);

\path[draw=drawColor,line width= 0.4pt,line join=round,line cap=round,fill=fillColor] ( 50.33,137.88) circle (  1.96);

\path[draw=drawColor,line width= 0.4pt,line join=round,line cap=round,fill=fillColor] ( 50.33,130.22) circle (  1.96);

\path[draw=drawColor,line width= 0.4pt,line join=round,line cap=round,fill=fillColor] ( 50.33,123.19) circle (  1.96);

\path[draw=drawColor,line width= 0.4pt,line join=round,line cap=round,fill=fillColor] ( 50.33,140.66) circle (  1.96);

\path[draw=drawColor,line width= 0.4pt,line join=round,line cap=round,fill=fillColor] ( 50.33,145.98) circle (  1.96);

\path[draw=drawColor,line width= 0.4pt,line join=round,line cap=round,fill=fillColor] ( 50.33,123.95) circle (  1.96);

\path[draw=drawColor,line width= 0.4pt,line join=round,line cap=round,fill=fillColor] ( 50.33,142.34) circle (  1.96);

\path[draw=drawColor,line width= 0.4pt,line join=round,line cap=round,fill=fillColor] ( 50.33,128.01) circle (  1.96);

\path[draw=drawColor,line width= 0.4pt,line join=round,line cap=round,fill=fillColor] ( 50.33,132.31) circle (  1.96);

\path[draw=drawColor,line width= 0.4pt,line join=round,line cap=round,fill=fillColor] ( 50.33,138.99) circle (  1.96);

\path[draw=drawColor,line width= 0.4pt,line join=round,line cap=round,fill=fillColor] ( 50.33,136.49) circle (  1.96);

\path[draw=drawColor,line width= 0.4pt,line join=round,line cap=round,fill=fillColor] ( 50.33,130.30) circle (  1.96);

\path[draw=drawColor,line width= 0.4pt,line join=round,line cap=round,fill=fillColor] ( 50.33,127.75) circle (  1.96);

\path[draw=drawColor,line width= 0.4pt,line join=round,line cap=round,fill=fillColor] ( 50.33,127.30) circle (  1.96);

\path[draw=drawColor,line width= 0.4pt,line join=round,line cap=round,fill=fillColor] ( 50.33,141.42) circle (  1.96);

\path[draw=drawColor,line width= 0.4pt,line join=round,line cap=round,fill=fillColor] ( 50.33,132.31) circle (  1.96);

\path[draw=drawColor,line width= 0.4pt,line join=round,line cap=round,fill=fillColor] ( 50.33,123.95) circle (  1.96);

\path[draw=drawColor,line width= 0.4pt,line join=round,line cap=round,fill=fillColor] ( 50.33,132.31) circle (  1.96);

\path[draw=drawColor,line width= 0.4pt,line join=round,line cap=round,fill=fillColor] ( 50.33,132.31) circle (  1.96);

\path[draw=drawColor,line width= 0.4pt,line join=round,line cap=round,fill=fillColor] ( 50.33,123.95) circle (  1.96);

\path[draw=drawColor,line width= 0.4pt,line join=round,line cap=round,fill=fillColor] ( 50.33,129.80) circle (  1.96);

\path[draw=drawColor,line width= 0.4pt,line join=round,line cap=round,fill=fillColor] ( 50.33,132.63) circle (  1.96);

\path[draw=drawColor,line width= 0.4pt,line join=round,line cap=round,fill=fillColor] ( 50.33,123.08) circle (  1.96);

\path[draw=drawColor,line width= 0.4pt,line join=round,line cap=round,fill=fillColor] ( 50.33,134.82) circle (  1.96);

\path[draw=drawColor,line width= 0.4pt,line join=round,line cap=round,fill=fillColor] ( 50.33,124.79) circle (  1.96);

\path[draw=drawColor,line width= 0.4pt,line join=round,line cap=round,fill=fillColor] ( 50.33,132.31) circle (  1.96);

\path[draw=drawColor,line width= 0.4pt,line join=round,line cap=round,fill=fillColor] ( 50.33,138.99) circle (  1.96);

\path[draw=drawColor,line width= 0.4pt,line join=round,line cap=round,fill=fillColor] ( 50.33,123.95) circle (  1.96);

\path[draw=drawColor,line width= 0.4pt,line join=round,line cap=round,fill=fillColor] ( 50.33,123.95) circle (  1.96);

\path[draw=drawColor,line width= 0.4pt,line join=round,line cap=round,fill=fillColor] ( 50.33,124.29) circle (  1.96);

\path[draw=drawColor,line width= 0.4pt,line join=round,line cap=round,fill=fillColor] ( 50.33,129.92) circle (  1.96);

\path[draw=drawColor,line width= 0.4pt,line join=round,line cap=round,fill=fillColor] ( 50.33,144.84) circle (  1.96);

\path[draw=drawColor,line width= 0.4pt,line join=round,line cap=round,fill=fillColor] ( 50.33,123.95) circle (  1.96);

\path[draw=drawColor,line width= 0.4pt,line join=round,line cap=round,fill=fillColor] ( 50.33,128.13) circle (  1.96);

\path[draw=drawColor,line width= 0.4pt,line join=round,line cap=round,fill=fillColor] ( 50.33,132.31) circle (  1.96);

\path[draw=drawColor,line width= 0.4pt,line join=round,line cap=round,fill=fillColor] ( 50.33,138.32) circle (  1.96);

\path[draw=drawColor,line width= 0.4pt,line join=round,line cap=round,fill=fillColor] ( 50.33,134.31) circle (  1.96);

\path[draw=drawColor,line width= 0.4pt,line join=round,line cap=round,fill=fillColor] ( 50.33,125.15) circle (  1.96);

\path[draw=drawColor,line width= 0.4pt,line join=round,line cap=round,fill=fillColor] ( 50.33,134.31) circle (  1.96);

\path[draw=drawColor,line width= 0.4pt,line join=round,line cap=round,fill=fillColor] ( 50.33,125.50) circle (  1.96);

\path[draw=drawColor,line width= 0.4pt,line join=round,line cap=round,fill=fillColor] ( 50.33,127.24) circle (  1.96);

\path[draw=drawColor,line width= 0.4pt,line join=round,line cap=round,fill=fillColor] ( 50.33,126.37) circle (  1.96);

\path[draw=drawColor,line width= 0.4pt,line join=round,line cap=round,fill=fillColor] ( 50.33,134.17) circle (  1.96);

\path[draw=drawColor,line width= 0.4pt,line join=round,line cap=round,fill=fillColor] ( 50.33,126.37) circle (  1.96);

\path[draw=drawColor,line width= 0.4pt,line join=round,line cap=round,fill=fillColor] ( 50.33,128.97) circle (  1.96);

\path[draw=drawColor,line width= 0.4pt,line join=round,line cap=round,fill=fillColor] ( 50.33,123.77) circle (  1.96);

\path[draw=drawColor,line width= 0.4pt,line join=round,line cap=round,fill=fillColor] ( 50.33,123.77) circle (  1.96);

\path[draw=drawColor,line width= 0.4pt,line join=round,line cap=round,fill=fillColor] ( 50.33,138.33) circle (  1.96);

\path[draw=drawColor,line width= 0.4pt,line join=round,line cap=round,fill=fillColor] ( 50.33,123.77) circle (  1.96);

\path[draw=drawColor,line width= 0.4pt,line join=round,line cap=round,fill=fillColor] ( 50.33,134.17) circle (  1.96);

\path[draw=drawColor,line width= 0.4pt,line join=round,line cap=round,fill=fillColor] ( 50.33,138.50) circle (  1.96);

\path[draw=drawColor,line width= 0.4pt,line join=round,line cap=round,fill=fillColor] ( 50.33,123.77) circle (  1.96);

\path[draw=drawColor,line width= 0.4pt,line join=round,line cap=round,fill=fillColor] ( 50.33,123.77) circle (  1.96);

\path[draw=drawColor,line width= 0.4pt,line join=round,line cap=round,fill=fillColor] ( 50.33,123.77) circle (  1.96);

\path[draw=drawColor,line width= 0.4pt,line join=round,line cap=round,fill=fillColor] ( 50.33,123.12) circle (  1.96);

\path[draw=drawColor,line width= 0.4pt,line join=round,line cap=round,fill=fillColor] ( 50.33,137.88) circle (  1.96);

\path[draw=drawColor,line width= 0.4pt,line join=round,line cap=round,fill=fillColor] ( 50.33,128.39) circle (  1.96);

\path[draw=drawColor,line width= 0.4pt,line join=round,line cap=round,fill=fillColor] ( 50.33,141.26) circle (  1.96);

\path[draw=drawColor,line width= 0.4pt,line join=round,line cap=round,fill=fillColor] ( 50.33,134.17) circle (  1.96);

\path[draw=drawColor,line width= 0.4pt,line join=round,line cap=round,fill=fillColor] ( 50.33,138.50) circle (  1.96);

\path[draw=drawColor,line width= 0.4pt,line join=round,line cap=round,fill=fillColor] ( 50.33,123.77) circle (  1.96);

\path[draw=drawColor,line width= 0.4pt,line join=round,line cap=round,fill=fillColor] ( 50.33,129.83) circle (  1.96);

\path[draw=drawColor,line width= 0.4pt,line join=round,line cap=round,fill=fillColor] ( 50.33,130.45) circle (  1.96);

\path[draw=drawColor,line width= 0.4pt,line join=round,line cap=round,fill=fillColor] ( 50.33,129.83) circle (  1.96);

\path[draw=drawColor,line width= 0.4pt,line join=round,line cap=round,fill=fillColor] ( 50.33,131.28) circle (  1.96);

\path[draw=drawColor,line width= 0.4pt,line join=round,line cap=round,fill=fillColor] ( 50.33,123.77) circle (  1.96);

\path[draw=drawColor,line width= 0.4pt,line join=round,line cap=round,fill=fillColor] ( 50.33,137.63) circle (  1.96);

\path[draw=drawColor,line width= 0.4pt,line join=round,line cap=round,fill=fillColor] ( 50.33,123.34) circle (  1.96);

\path[draw=drawColor,line width= 0.4pt,line join=round,line cap=round,fill=fillColor] ( 50.33,123.77) circle (  1.96);

\path[draw=drawColor,line width= 0.4pt,line join=round,line cap=round,fill=fillColor] ( 50.33,131.78) circle (  1.96);

\path[draw=drawColor,line width= 0.4pt,line join=round,line cap=round,fill=fillColor] ( 50.33,144.57) circle (  1.96);

\path[draw=drawColor,line width= 0.4pt,line join=round,line cap=round,fill=fillColor] ( 50.33,127.24) circle (  1.96);

\path[draw=drawColor,line width= 0.4pt,line join=round,line cap=round,fill=fillColor] ( 50.33,123.77) circle (  1.96);

\path[draw=drawColor,line width= 0.4pt,line join=round,line cap=round,fill=fillColor] ( 50.33,123.77) circle (  1.96);

\path[draw=drawColor,line width= 0.4pt,line join=round,line cap=round,fill=fillColor] ( 50.33,127.24) circle (  1.96);

\path[draw=drawColor,line width= 0.4pt,line join=round,line cap=round,fill=fillColor] ( 50.33,132.09) circle (  1.96);

\path[draw=drawColor,line width= 0.4pt,line join=round,line cap=round,fill=fillColor] ( 50.33,123.77) circle (  1.96);

\path[draw=drawColor,line width= 0.4pt,line join=round,line cap=round,fill=fillColor] ( 50.33,126.37) circle (  1.96);

\path[draw=drawColor,line width= 0.4pt,line join=round,line cap=round,fill=fillColor] ( 50.33,134.17) circle (  1.96);

\path[draw=drawColor,line width= 0.4pt,line join=round,line cap=round,fill=fillColor] ( 50.33,129.44) circle (  1.96);

\path[draw=drawColor,line width= 0.4pt,line join=round,line cap=round,fill=fillColor] ( 50.33,134.17) circle (  1.96);

\path[draw=drawColor,line width= 0.4pt,line join=round,line cap=round,fill=fillColor] ( 50.33,123.77) circle (  1.96);

\path[draw=drawColor,line width= 0.4pt,line join=round,line cap=round,fill=fillColor] ( 50.33,123.34) circle (  1.96);

\path[draw=drawColor,line width= 0.4pt,line join=round,line cap=round,fill=fillColor] ( 50.33,151.50) circle (  1.96);

\path[draw=drawColor,line width= 0.4pt,line join=round,line cap=round,fill=fillColor] ( 50.33,123.77) circle (  1.96);

\path[draw=drawColor,line width= 0.4pt,line join=round,line cap=round,fill=fillColor] ( 50.33,127.24) circle (  1.96);

\path[draw=drawColor,line width= 0.4pt,line join=round,line cap=round,fill=fillColor] ( 50.33,144.57) circle (  1.96);

\path[draw=drawColor,line width= 0.4pt,line join=round,line cap=round,fill=fillColor] ( 50.33,147.17) circle (  1.96);

\path[draw=drawColor,line width= 0.4pt,line join=round,line cap=round,fill=fillColor] ( 50.33,123.77) circle (  1.96);

\path[draw=drawColor,line width= 0.4pt,line join=round,line cap=round,fill=fillColor] ( 50.33,125.50) circle (  1.96);

\path[draw=drawColor,line width= 0.4pt,line join=round,line cap=round,fill=fillColor] ( 50.33,129.83) circle (  1.96);

\path[draw=drawColor,line width= 0.4pt,line join=round,line cap=round,fill=fillColor] ( 50.33,128.97) circle (  1.96);

\path[draw=drawColor,line width= 0.4pt,line join=round,line cap=round,fill=fillColor] ( 50.33,123.77) circle (  1.96);

\path[draw=drawColor,line width= 0.4pt,line join=round,line cap=round,fill=fillColor] ( 50.33,134.17) circle (  1.96);

\path[draw=drawColor,line width= 0.4pt,line join=round,line cap=round,fill=fillColor] ( 50.33,128.39) circle (  1.96);

\path[draw=drawColor,line width= 0.4pt,line join=round,line cap=round,fill=fillColor] ( 50.33,123.77) circle (  1.96);

\path[draw=drawColor,line width= 0.4pt,line join=round,line cap=round,fill=fillColor] ( 50.33,130.19) circle (  1.96);

\path[draw=drawColor,line width= 0.4pt,line join=round,line cap=round,fill=fillColor] ( 50.33,130.70) circle (  1.96);

\path[draw=drawColor,line width= 0.4pt,line join=round,line cap=round,fill=fillColor] ( 50.33,128.97) circle (  1.96);

\path[draw=drawColor,line width= 0.4pt,line join=round,line cap=round,fill=fillColor] ( 50.33,125.50) circle (  1.96);

\path[draw=drawColor,line width= 0.4pt,line join=round,line cap=round,fill=fillColor] ( 50.33,127.67) circle (  1.96);

\path[draw=drawColor,line width= 0.4pt,line join=round,line cap=round,fill=fillColor] ( 50.33,123.77) circle (  1.96);

\path[draw=drawColor,line width= 0.4pt,line join=round,line cap=round,fill=fillColor] ( 50.33,134.17) circle (  1.96);

\path[draw=drawColor,line width= 0.4pt,line join=round,line cap=round,fill=fillColor] ( 50.33,123.77) circle (  1.96);

\path[draw=drawColor,line width= 0.4pt,line join=round,line cap=round,fill=fillColor] ( 50.33,123.77) circle (  1.96);

\path[draw=drawColor,line width= 0.4pt,line join=round,line cap=round,fill=fillColor] ( 50.33,125.50) circle (  1.96);

\path[draw=drawColor,line width= 0.4pt,line join=round,line cap=round,fill=fillColor] ( 50.33,134.17) circle (  1.96);

\path[draw=drawColor,line width= 0.4pt,line join=round,line cap=round,fill=fillColor] ( 50.33,123.77) circle (  1.96);

\path[draw=drawColor,line width= 0.4pt,line join=round,line cap=round,fill=fillColor] ( 50.33,132.09) circle (  1.96);

\path[draw=drawColor,line width= 0.4pt,line join=round,line cap=round,fill=fillColor] ( 50.33,141.10) circle (  1.96);

\path[draw=drawColor,line width= 0.4pt,line join=round,line cap=round,fill=fillColor] ( 50.33,149.76) circle (  1.96);

\path[draw=drawColor,line width= 0.4pt,line join=round,line cap=round,fill=fillColor] ( 50.33,144.57) circle (  1.96);

\path[draw=drawColor,line width= 0.4pt,line join=round,line cap=round,fill=fillColor] ( 50.33,144.57) circle (  1.96);

\path[draw=drawColor,line width= 0.4pt,line join=round,line cap=round,fill=fillColor] ( 50.33,131.57) circle (  1.96);

\path[draw=drawColor,line width= 0.4pt,line join=round,line cap=round,fill=fillColor] ( 50.33,123.77) circle (  1.96);

\path[draw=drawColor,line width= 0.4pt,line join=round,line cap=round,fill=fillColor] ( 50.33,128.97) circle (  1.96);

\path[draw=drawColor,line width= 0.4pt,line join=round,line cap=round,fill=fillColor] ( 50.33,123.77) circle (  1.96);

\path[draw=drawColor,line width= 0.4pt,line join=round,line cap=round,fill=fillColor] ( 50.33,134.17) circle (  1.96);

\path[draw=drawColor,line width= 0.4pt,line join=round,line cap=round,fill=fillColor] ( 50.33,126.08) circle (  1.96);

\path[draw=drawColor,line width= 0.4pt,line join=round,line cap=round,fill=fillColor] ( 50.33,123.77) circle (  1.96);

\path[draw=drawColor,line width= 0.4pt,line join=round,line cap=round,fill=fillColor] ( 50.33,148.03) circle (  1.96);

\path[draw=drawColor,line width= 0.4pt,line join=round,line cap=round,fill=fillColor] ( 50.33,125.50) circle (  1.96);

\path[draw=drawColor,line width= 0.4pt,line join=round,line cap=round,fill=fillColor] ( 50.33,134.17) circle (  1.96);

\path[draw=drawColor,line width= 0.4pt,line join=round,line cap=round,fill=fillColor] ( 50.33,123.77) circle (  1.96);

\path[draw=drawColor,line width= 0.4pt,line join=round,line cap=round,fill=fillColor] ( 50.33,123.77) circle (  1.96);

\path[draw=drawColor,line width= 0.4pt,line join=round,line cap=round,fill=fillColor] ( 50.33,129.71) circle (  1.96);

\path[draw=drawColor,line width= 0.4pt,line join=round,line cap=round,fill=fillColor] ( 50.33,128.97) circle (  1.96);

\path[draw=drawColor,line width= 0.4pt,line join=round,line cap=round,fill=fillColor] ( 50.33,123.77) circle (  1.96);

\path[draw=drawColor,line width= 0.4pt,line join=round,line cap=round,fill=fillColor] ( 50.33,123.77) circle (  1.96);

\path[draw=drawColor,line width= 0.4pt,line join=round,line cap=round,fill=fillColor] ( 50.33,123.77) circle (  1.96);

\path[draw=drawColor,line width= 0.4pt,line join=round,line cap=round,fill=fillColor] ( 50.33,144.57) circle (  1.96);

\path[draw=drawColor,line width= 0.4pt,line join=round,line cap=round,fill=fillColor] ( 50.33,123.77) circle (  1.96);

\path[draw=drawColor,line width= 0.4pt,line join=round,line cap=round,fill=fillColor] ( 50.33,153.66) circle (  1.96);

\path[draw=drawColor,line width= 0.4pt,line join=round,line cap=round,fill=fillColor] ( 50.33,124.42) circle (  1.96);

\path[draw=drawColor,line width= 0.4pt,line join=round,line cap=round,fill=fillColor] ( 50.33,123.77) circle (  1.96);

\path[draw=drawColor,line width= 0.4pt,line join=round,line cap=round,fill=fillColor] ( 50.33,127.24) circle (  1.96);

\path[draw=drawColor,line width= 0.4pt,line join=round,line cap=round,fill=fillColor] ( 50.33,144.57) circle (  1.96);

\path[draw=drawColor,line width= 0.4pt,line join=round,line cap=round,fill=fillColor] ( 50.33,134.17) circle (  1.96);

\path[draw=drawColor,line width= 0.4pt,line join=round,line cap=round,fill=fillColor] ( 50.33,125.50) circle (  1.96);

\path[draw=drawColor,line width= 0.4pt,line join=round,line cap=round,fill=fillColor] ( 50.33,123.77) circle (  1.96);

\path[draw=drawColor,line width= 0.4pt,line join=round,line cap=round,fill=fillColor] ( 50.33,134.17) circle (  1.96);

\path[draw=drawColor,line width= 0.4pt,line join=round,line cap=round,fill=fillColor] ( 50.33,125.50) circle (  1.96);

\path[draw=drawColor,line width= 0.4pt,line join=round,line cap=round,fill=fillColor] ( 50.33,125.43) circle (  1.96);

\path[draw=drawColor,line width= 0.4pt,line join=round,line cap=round,fill=fillColor] ( 50.33,139.94) circle (  1.96);

\path[draw=drawColor,line width= 0.4pt,line join=round,line cap=round,fill=fillColor] ( 50.33,123.77) circle (  1.96);

\path[draw=drawColor,line width= 0.4pt,line join=round,line cap=round,fill=fillColor] ( 50.33,147.17) circle (  1.96);

\path[draw=drawColor,line width= 0.4pt,line join=round,line cap=round,fill=fillColor] ( 50.33,144.57) circle (  1.96);

\path[draw=drawColor,line width= 0.4pt,line join=round,line cap=round,fill=fillColor] ( 50.33,125.50) circle (  1.96);

\path[draw=drawColor,line width= 0.6pt,line join=round] ( 50.33,105.48) -- ( 50.33,122.71);

\path[draw=drawColor,line width= 0.6pt,line join=round] ( 50.33, 93.98) -- ( 50.33, 83.08);
\definecolor{fillColor}{RGB}{228,26,28}

\path[draw=drawColor,line width= 0.6pt,line join=round,line cap=round,fill=fillColor] ( 41.44,105.48) --
	( 41.44, 93.98) --
	( 59.21, 93.98) --
	( 59.21,105.48) --
	( 41.44,105.48) --
	cycle;

\path[draw=drawColor,line width= 1.1pt,line join=round] ( 41.44, 99.06) -- ( 59.21, 99.06);

\path[draw=drawColor,line width= 0.6pt,line join=round] ( 74.02,108.79) -- ( 74.02,119.07);

\path[draw=drawColor,line width= 0.6pt,line join=round] ( 74.02, 94.05) -- ( 74.02, 82.54);
\definecolor{fillColor}{RGB}{55,126,184}

\path[draw=drawColor,line width= 0.6pt,line join=round,line cap=round,fill=fillColor] ( 65.13,108.79) --
	( 65.13, 94.05) --
	( 82.90, 94.05) --
	( 82.90,108.79) --
	( 65.13,108.79) --
	cycle;

\path[draw=drawColor,line width= 1.1pt,line join=round] ( 65.13,101.31) -- ( 82.90,101.31);
\definecolor{fillColor}{gray}{0.20}

\path[draw=drawColor,line width= 0.4pt,line join=round,line cap=round,fill=fillColor] ( 97.71,150.00) circle (  1.96);

\path[draw=drawColor,line width= 0.6pt,line join=round] ( 97.71,122.04) -- ( 97.71,143.36);

\path[draw=drawColor,line width= 0.6pt,line join=round] ( 97.71,105.00) -- ( 97.71, 89.08);
\definecolor{fillColor}{RGB}{77,175,74}

\path[draw=drawColor,line width= 0.6pt,line join=round,line cap=round,fill=fillColor] ( 88.82,122.04) --
	( 88.82,105.00) --
	(106.59,105.00) --
	(106.59,122.04) --
	( 88.82,122.04) --
	cycle;

\path[draw=drawColor,line width= 1.1pt,line join=round] ( 88.82,110.53) -- (106.59,110.53);
\end{scope}
\begin{scope}
\path[clip] (117.42, 78.54) rectangle (193.23,158.60);
\definecolor{drawColor}{RGB}{255,255,255}

\path[draw=drawColor,line width= 0.3pt,line join=round] (117.42, 92.57) --
	(193.23, 92.57);

\path[draw=drawColor,line width= 0.3pt,line join=round] (117.42,113.37) --
	(193.23,113.37);

\path[draw=drawColor,line width= 0.3pt,line join=round] (117.42,134.17) --
	(193.23,134.17);

\path[draw=drawColor,line width= 0.3pt,line join=round] (117.42,154.96) --
	(193.23,154.96);

\path[draw=drawColor,line width= 0.6pt,line join=round] (117.42, 82.18) --
	(193.23, 82.18);

\path[draw=drawColor,line width= 0.6pt,line join=round] (117.42,102.97) --
	(193.23,102.97);

\path[draw=drawColor,line width= 0.6pt,line join=round] (117.42,123.77) --
	(193.23,123.77);

\path[draw=drawColor,line width= 0.6pt,line join=round] (117.42,144.57) --
	(193.23,144.57);

\path[draw=drawColor,line width= 0.6pt,line join=round] (131.64, 78.54) --
	(131.64,158.60);

\path[draw=drawColor,line width= 0.6pt,line join=round] (155.33, 78.54) --
	(155.33,158.60);

\path[draw=drawColor,line width= 0.6pt,line join=round] (179.02, 78.54) --
	(179.02,158.60);
\definecolor{drawColor}{gray}{0.20}
\definecolor{fillColor}{gray}{0.20}

\path[draw=drawColor,line width= 0.4pt,line join=round,line cap=round,fill=fillColor] (131.64,140.20) circle (  1.96);

\path[draw=drawColor,line width= 0.4pt,line join=round,line cap=round,fill=fillColor] (131.64,143.90) circle (  1.96);

\path[draw=drawColor,line width= 0.4pt,line join=round,line cap=round,fill=fillColor] (131.64,139.15) circle (  1.96);

\path[draw=drawColor,line width= 0.4pt,line join=round,line cap=round,fill=fillColor] (131.64,139.15) circle (  1.96);

\path[draw=drawColor,line width= 0.4pt,line join=round,line cap=round,fill=fillColor] (131.64,151.81) circle (  1.96);

\path[draw=drawColor,line width= 0.4pt,line join=round,line cap=round,fill=fillColor] (131.64,139.15) circle (  1.96);

\path[draw=drawColor,line width= 0.4pt,line join=round,line cap=round,fill=fillColor] (131.64,138.45) circle (  1.96);

\path[draw=drawColor,line width= 0.4pt,line join=round,line cap=round,fill=fillColor] (131.64,145.48) circle (  1.96);

\path[draw=drawColor,line width= 0.4pt,line join=round,line cap=round,fill=fillColor] (131.64,144.03) circle (  1.96);

\path[draw=drawColor,line width= 0.4pt,line join=round,line cap=round,fill=fillColor] (131.64,139.15) circle (  1.96);

\path[draw=drawColor,line width= 0.4pt,line join=round,line cap=round,fill=fillColor] (131.64,145.48) circle (  1.96);

\path[draw=drawColor,line width= 0.4pt,line join=round,line cap=round,fill=fillColor] (131.64,141.86) circle (  1.96);

\path[draw=drawColor,line width= 0.4pt,line join=round,line cap=round,fill=fillColor] (131.64,142.95) circle (  1.96);

\path[draw=drawColor,line width= 0.4pt,line join=round,line cap=round,fill=fillColor] (131.64,139.15) circle (  1.96);

\path[draw=drawColor,line width= 0.4pt,line join=round,line cap=round,fill=fillColor] (131.64,138.61) circle (  1.96);

\path[draw=drawColor,line width= 0.4pt,line join=round,line cap=round,fill=fillColor] (131.64,146.75) circle (  1.96);

\path[draw=drawColor,line width= 0.4pt,line join=round,line cap=round,fill=fillColor] (131.64,145.48) circle (  1.96);

\path[draw=drawColor,line width= 0.4pt,line join=round,line cap=round,fill=fillColor] (131.64,139.15) circle (  1.96);

\path[draw=drawColor,line width= 0.4pt,line join=round,line cap=round,fill=fillColor] (131.64,145.48) circle (  1.96);

\path[draw=drawColor,line width= 0.4pt,line join=round,line cap=round,fill=fillColor] (131.64,148.64) circle (  1.96);

\path[draw=drawColor,line width= 0.4pt,line join=round,line cap=round,fill=fillColor] (131.64,145.48) circle (  1.96);

\path[draw=drawColor,line width= 0.4pt,line join=round,line cap=round,fill=fillColor] (131.64,139.15) circle (  1.96);

\path[draw=drawColor,line width= 0.4pt,line join=round,line cap=round,fill=fillColor] (131.64,145.48) circle (  1.96);

\path[draw=drawColor,line width= 0.4pt,line join=round,line cap=round,fill=fillColor] (131.64,153.39) circle (  1.96);

\path[draw=drawColor,line width= 0.4pt,line join=round,line cap=round,fill=fillColor] (131.64,139.15) circle (  1.96);

\path[draw=drawColor,line width= 0.4pt,line join=round,line cap=round,fill=fillColor] (131.64,147.29) circle (  1.96);

\path[draw=drawColor,line width= 0.4pt,line join=round,line cap=round,fill=fillColor] (131.64,139.15) circle (  1.96);

\path[draw=drawColor,line width= 0.4pt,line join=round,line cap=round,fill=fillColor] (131.64,144.57) circle (  1.96);

\path[draw=drawColor,line width= 0.4pt,line join=round,line cap=round,fill=fillColor] (131.64,139.15) circle (  1.96);

\path[draw=drawColor,line width= 0.4pt,line join=round,line cap=round,fill=fillColor] (131.64,153.39) circle (  1.96);

\path[draw=drawColor,line width= 0.4pt,line join=round,line cap=round,fill=fillColor] (131.64,139.15) circle (  1.96);

\path[draw=drawColor,line width= 0.4pt,line join=round,line cap=round,fill=fillColor] (131.64,145.48) circle (  1.96);

\path[draw=drawColor,line width= 0.4pt,line join=round,line cap=round,fill=fillColor] (131.64,139.15) circle (  1.96);

\path[draw=drawColor,line width= 0.4pt,line join=round,line cap=round,fill=fillColor] (131.64,143.90) circle (  1.96);

\path[draw=drawColor,line width= 0.4pt,line join=round,line cap=round,fill=fillColor] (131.64,141.52) circle (  1.96);

\path[draw=drawColor,line width= 0.4pt,line join=round,line cap=round,fill=fillColor] (131.64,154.34) circle (  1.96);

\path[draw=drawColor,line width= 0.4pt,line join=round,line cap=round,fill=fillColor] (131.64,142.95) circle (  1.96);

\path[draw=drawColor,line width= 0.4pt,line join=round,line cap=round,fill=fillColor] (131.64,141.86) circle (  1.96);

\path[draw=drawColor,line width= 0.4pt,line join=round,line cap=round,fill=fillColor] (131.64,145.48) circle (  1.96);

\path[draw=drawColor,line width= 0.4pt,line join=round,line cap=round,fill=fillColor] (131.64,153.39) circle (  1.96);

\path[draw=drawColor,line width= 0.4pt,line join=round,line cap=round,fill=fillColor] (131.64,145.48) circle (  1.96);

\path[draw=drawColor,line width= 0.4pt,line join=round,line cap=round,fill=fillColor] (131.64,147.29) circle (  1.96);

\path[draw=drawColor,line width= 0.4pt,line join=round,line cap=round,fill=fillColor] (131.64,149.02) circle (  1.96);

\path[draw=drawColor,line width= 0.4pt,line join=round,line cap=round,fill=fillColor] (131.64,138.39) circle (  1.96);

\path[draw=drawColor,line width= 0.4pt,line join=round,line cap=round,fill=fillColor] (131.64,139.15) circle (  1.96);

\path[draw=drawColor,line width= 0.4pt,line join=round,line cap=round,fill=fillColor] (131.64,145.48) circle (  1.96);

\path[draw=drawColor,line width= 0.4pt,line join=round,line cap=round,fill=fillColor] (131.64,145.48) circle (  1.96);

\path[draw=drawColor,line width= 0.4pt,line join=round,line cap=round,fill=fillColor] (131.64,138.96) circle (  1.96);

\path[draw=drawColor,line width= 0.4pt,line join=round,line cap=round,fill=fillColor] (131.64,143.83) circle (  1.96);

\path[draw=drawColor,line width= 0.4pt,line join=round,line cap=round,fill=fillColor] (131.64,143.83) circle (  1.96);

\path[draw=drawColor,line width= 0.4pt,line join=round,line cap=round,fill=fillColor] (131.64,139.72) circle (  1.96);

\path[draw=drawColor,line width= 0.4pt,line join=round,line cap=round,fill=fillColor] (131.64,139.72) circle (  1.96);

\path[draw=drawColor,line width= 0.4pt,line join=round,line cap=round,fill=fillColor] (131.64,154.11) circle (  1.96);

\path[draw=drawColor,line width= 0.4pt,line join=round,line cap=round,fill=fillColor] (131.64,145.12) circle (  1.96);

\path[draw=drawColor,line width= 0.4pt,line join=round,line cap=round,fill=fillColor] (131.64,152.46) circle (  1.96);

\path[draw=drawColor,line width= 0.4pt,line join=round,line cap=round,fill=fillColor] (131.64,142.60) circle (  1.96);

\path[draw=drawColor,line width= 0.4pt,line join=round,line cap=round,fill=fillColor] (131.64,142.60) circle (  1.96);

\path[draw=drawColor,line width= 0.4pt,line join=round,line cap=round,fill=fillColor] (131.64,141.52) circle (  1.96);

\path[draw=drawColor,line width= 0.4pt,line join=round,line cap=round,fill=fillColor] (131.64,139.72) circle (  1.96);

\path[draw=drawColor,line width= 0.4pt,line join=round,line cap=round,fill=fillColor] (131.64,149.31) circle (  1.96);

\path[draw=drawColor,line width= 0.4pt,line join=round,line cap=round,fill=fillColor] (131.64,142.12) circle (  1.96);

\path[draw=drawColor,line width= 0.4pt,line join=round,line cap=round,fill=fillColor] (131.64,154.87) circle (  1.96);

\path[draw=drawColor,line width= 0.4pt,line join=round,line cap=round,fill=fillColor] (131.64,154.11) circle (  1.96);

\path[draw=drawColor,line width= 0.4pt,line join=round,line cap=round,fill=fillColor] (131.64,139.72) circle (  1.96);

\path[draw=drawColor,line width= 0.4pt,line join=round,line cap=round,fill=fillColor] (131.64,150.00) circle (  1.96);

\path[draw=drawColor,line width= 0.4pt,line join=round,line cap=round,fill=fillColor] (131.64,146.91) circle (  1.96);

\path[draw=drawColor,line width= 0.4pt,line join=round,line cap=round,fill=fillColor] (131.64,141.03) circle (  1.96);

\path[draw=drawColor,line width= 0.4pt,line join=round,line cap=round,fill=fillColor] (131.64,139.72) circle (  1.96);

\path[draw=drawColor,line width= 0.4pt,line join=round,line cap=round,fill=fillColor] (131.64,139.72) circle (  1.96);

\path[draw=drawColor,line width= 0.4pt,line join=round,line cap=round,fill=fillColor] (131.64,154.11) circle (  1.96);

\path[draw=drawColor,line width= 0.4pt,line join=round,line cap=round,fill=fillColor] (131.64,146.91) circle (  1.96);

\path[draw=drawColor,line width= 0.4pt,line join=round,line cap=round,fill=fillColor] (131.64,139.72) circle (  1.96);

\path[draw=drawColor,line width= 0.4pt,line join=round,line cap=round,fill=fillColor] (131.64,151.23) circle (  1.96);

\path[draw=drawColor,line width= 0.4pt,line join=round,line cap=round,fill=fillColor] (131.64,154.11) circle (  1.96);

\path[draw=drawColor,line width= 0.4pt,line join=round,line cap=round,fill=fillColor] (131.64,141.37) circle (  1.96);

\path[draw=drawColor,line width= 0.4pt,line join=round,line cap=round,fill=fillColor] (131.64,139.72) circle (  1.96);

\path[draw=drawColor,line width= 0.4pt,line join=round,line cap=round,fill=fillColor] (131.64,152.96) circle (  1.96);

\path[draw=drawColor,line width= 0.4pt,line join=round,line cap=round,fill=fillColor] (131.64,150.00) circle (  1.96);

\path[draw=drawColor,line width= 0.4pt,line join=round,line cap=round,fill=fillColor] (131.64,139.72) circle (  1.96);

\path[draw=drawColor,line width= 0.4pt,line join=round,line cap=round,fill=fillColor] (131.64,151.23) circle (  1.96);

\path[draw=drawColor,line width= 0.4pt,line join=round,line cap=round,fill=fillColor] (131.64,146.91) circle (  1.96);

\path[draw=drawColor,line width= 0.4pt,line join=round,line cap=round,fill=fillColor] (131.64,139.72) circle (  1.96);

\path[draw=drawColor,line width= 0.4pt,line join=round,line cap=round,fill=fillColor] (131.64,144.76) circle (  1.96);

\path[draw=drawColor,line width= 0.4pt,line join=round,line cap=round,fill=fillColor] (131.64,143.83) circle (  1.96);

\path[draw=drawColor,line width= 0.4pt,line join=round,line cap=round,fill=fillColor] (131.64,143.83) circle (  1.96);

\path[draw=drawColor,line width= 0.4pt,line join=round,line cap=round,fill=fillColor] (131.64,139.72) circle (  1.96);

\path[draw=drawColor,line width= 0.4pt,line join=round,line cap=round,fill=fillColor] (131.64,138.28) circle (  1.96);

\path[draw=drawColor,line width= 0.4pt,line join=round,line cap=round,fill=fillColor] (131.64,149.31) circle (  1.96);

\path[draw=drawColor,line width= 0.4pt,line join=round,line cap=round,fill=fillColor] (131.64,146.91) circle (  1.96);

\path[draw=drawColor,line width= 0.4pt,line join=round,line cap=round,fill=fillColor] (131.64,146.91) circle (  1.96);

\path[draw=drawColor,line width= 0.4pt,line join=round,line cap=round,fill=fillColor] (131.64,139.72) circle (  1.96);

\path[draw=drawColor,line width= 0.4pt,line join=round,line cap=round,fill=fillColor] (131.64,138.76) circle (  1.96);

\path[draw=drawColor,line width= 0.4pt,line join=round,line cap=round,fill=fillColor] (131.64,141.59) circle (  1.96);

\path[draw=drawColor,line width= 0.4pt,line join=round,line cap=round,fill=fillColor] (131.64,141.59) circle (  1.96);

\path[draw=drawColor,line width= 0.4pt,line join=round,line cap=round,fill=fillColor] (131.64,145.55) circle (  1.96);

\path[draw=drawColor,line width= 0.4pt,line join=round,line cap=round,fill=fillColor] (131.64,153.48) circle (  1.96);

\path[draw=drawColor,line width= 0.4pt,line join=round,line cap=round,fill=fillColor] (131.64,149.51) circle (  1.96);

\path[draw=drawColor,line width= 0.4pt,line join=round,line cap=round,fill=fillColor] (131.64,138.76) circle (  1.96);

\path[draw=drawColor,line width= 0.4pt,line join=round,line cap=round,fill=fillColor] (131.64,145.55) circle (  1.96);

\path[draw=drawColor,line width= 0.4pt,line join=round,line cap=round,fill=fillColor] (131.64,140.27) circle (  1.96);

\path[draw=drawColor,line width= 0.4pt,line join=round,line cap=round,fill=fillColor] (131.64,151.50) circle (  1.96);

\path[draw=drawColor,line width= 0.4pt,line join=round,line cap=round,fill=fillColor] (131.64,140.27) circle (  1.96);

\path[draw=drawColor,line width= 0.4pt,line join=round,line cap=round,fill=fillColor] (131.64,141.59) circle (  1.96);

\path[draw=drawColor,line width= 0.4pt,line join=round,line cap=round,fill=fillColor] (131.64,153.48) circle (  1.96);

\path[draw=drawColor,line width= 0.4pt,line join=round,line cap=round,fill=fillColor] (131.64,145.55) circle (  1.96);

\path[draw=drawColor,line width= 0.4pt,line join=round,line cap=round,fill=fillColor] (131.64,143.87) circle (  1.96);

\path[draw=drawColor,line width= 0.4pt,line join=round,line cap=round,fill=fillColor] (131.64,141.59) circle (  1.96);

\path[draw=drawColor,line width= 0.4pt,line join=round,line cap=round,fill=fillColor] (131.64,141.59) circle (  1.96);

\path[draw=drawColor,line width= 0.4pt,line join=round,line cap=round,fill=fillColor] (131.64,139.22) circle (  1.96);

\path[draw=drawColor,line width= 0.4pt,line join=round,line cap=round,fill=fillColor] (131.64,141.59) circle (  1.96);

\path[draw=drawColor,line width= 0.4pt,line join=round,line cap=round,fill=fillColor] (131.64,141.59) circle (  1.96);

\path[draw=drawColor,line width= 0.4pt,line join=round,line cap=round,fill=fillColor] (131.64,145.55) circle (  1.96);

\path[draw=drawColor,line width= 0.4pt,line join=round,line cap=round,fill=fillColor] (131.64,139.22) circle (  1.96);

\path[draw=drawColor,line width= 0.4pt,line join=round,line cap=round,fill=fillColor] (131.64,147.53) circle (  1.96);

\path[draw=drawColor,line width= 0.4pt,line join=round,line cap=round,fill=fillColor] (131.64,139.22) circle (  1.96);

\path[draw=drawColor,line width= 0.4pt,line join=round,line cap=round,fill=fillColor] (131.64,153.48) circle (  1.96);

\path[draw=drawColor,line width= 0.4pt,line join=round,line cap=round,fill=fillColor] (131.64,141.59) circle (  1.96);

\path[draw=drawColor,line width= 0.4pt,line join=round,line cap=round,fill=fillColor] (131.64,148.72) circle (  1.96);

\path[draw=drawColor,line width= 0.4pt,line join=round,line cap=round,fill=fillColor] (131.64,144.65) circle (  1.96);

\path[draw=drawColor,line width= 0.4pt,line join=round,line cap=round,fill=fillColor] (131.64,141.59) circle (  1.96);

\path[draw=drawColor,line width= 0.4pt,line join=round,line cap=round,fill=fillColor] (131.64,145.55) circle (  1.96);

\path[draw=drawColor,line width= 0.4pt,line join=round,line cap=round,fill=fillColor] (131.64,150.08) circle (  1.96);

\path[draw=drawColor,line width= 0.4pt,line join=round,line cap=round,fill=fillColor] (131.64,138.76) circle (  1.96);

\path[draw=drawColor,line width= 0.4pt,line join=round,line cap=round,fill=fillColor] (131.64,141.59) circle (  1.96);

\path[draw=drawColor,line width= 0.4pt,line join=round,line cap=round,fill=fillColor] (131.64,141.59) circle (  1.96);

\path[draw=drawColor,line width= 0.4pt,line join=round,line cap=round,fill=fillColor] (131.64,148.72) circle (  1.96);

\path[draw=drawColor,line width= 0.4pt,line join=round,line cap=round,fill=fillColor] (131.64,145.55) circle (  1.96);

\path[draw=drawColor,line width= 0.4pt,line join=round,line cap=round,fill=fillColor] (131.64,142.38) circle (  1.96);

\path[draw=drawColor,line width= 0.4pt,line join=round,line cap=round,fill=fillColor] (131.64,146.99) circle (  1.96);

\path[draw=drawColor,line width= 0.4pt,line join=round,line cap=round,fill=fillColor] (131.64,153.48) circle (  1.96);

\path[draw=drawColor,line width= 0.4pt,line join=round,line cap=round,fill=fillColor] (131.64,148.72) circle (  1.96);

\path[draw=drawColor,line width= 0.4pt,line join=round,line cap=round,fill=fillColor] (131.64,139.22) circle (  1.96);

\path[draw=drawColor,line width= 0.4pt,line join=round,line cap=round,fill=fillColor] (131.64,153.48) circle (  1.96);

\path[draw=drawColor,line width= 0.4pt,line join=round,line cap=round,fill=fillColor] (131.64,141.59) circle (  1.96);

\path[draw=drawColor,line width= 0.4pt,line join=round,line cap=round,fill=fillColor] (131.64,150.83) circle (  1.96);

\path[draw=drawColor,line width= 0.4pt,line join=round,line cap=round,fill=fillColor] (131.64,142.91) circle (  1.96);

\path[draw=drawColor,line width= 0.4pt,line join=round,line cap=round,fill=fillColor] (131.64,145.55) circle (  1.96);

\path[draw=drawColor,line width= 0.4pt,line join=round,line cap=round,fill=fillColor] (131.64,143.29) circle (  1.96);

\path[draw=drawColor,line width= 0.4pt,line join=round,line cap=round,fill=fillColor] (131.64,149.02) circle (  1.96);

\path[draw=drawColor,line width= 0.4pt,line join=round,line cap=round,fill=fillColor] (131.64,150.83) circle (  1.96);

\path[draw=drawColor,line width= 0.4pt,line join=round,line cap=round,fill=fillColor] (131.64,145.55) circle (  1.96);

\path[draw=drawColor,line width= 0.4pt,line join=round,line cap=round,fill=fillColor] (131.64,153.48) circle (  1.96);

\path[draw=drawColor,line width= 0.4pt,line join=round,line cap=round,fill=fillColor] (131.64,145.55) circle (  1.96);

\path[draw=drawColor,line width= 0.4pt,line join=round,line cap=round,fill=fillColor] (131.64,142.38) circle (  1.96);

\path[draw=drawColor,line width= 0.4pt,line join=round,line cap=round,fill=fillColor] (131.64,141.59) circle (  1.96);

\path[draw=drawColor,line width= 0.4pt,line join=round,line cap=round,fill=fillColor] (131.64,153.48) circle (  1.96);

\path[draw=drawColor,line width= 0.4pt,line join=round,line cap=round,fill=fillColor] (131.64,153.48) circle (  1.96);

\path[draw=drawColor,line width= 0.4pt,line join=round,line cap=round,fill=fillColor] (131.64,141.59) circle (  1.96);

\path[draw=drawColor,line width= 0.4pt,line join=round,line cap=round,fill=fillColor] (131.64,145.55) circle (  1.96);

\path[draw=drawColor,line width= 0.4pt,line join=round,line cap=round,fill=fillColor] (131.64,142.38) circle (  1.96);

\path[draw=drawColor,line width= 0.4pt,line join=round,line cap=round,fill=fillColor] (131.64,142.91) circle (  1.96);

\path[draw=drawColor,line width= 0.4pt,line join=round,line cap=round,fill=fillColor] (131.64,141.59) circle (  1.96);

\path[draw=drawColor,line width= 0.4pt,line join=round,line cap=round,fill=fillColor] (131.64,139.22) circle (  1.96);

\path[draw=drawColor,line width= 0.4pt,line join=round,line cap=round,fill=fillColor] (131.64,141.59) circle (  1.96);

\path[draw=drawColor,line width= 0.4pt,line join=round,line cap=round,fill=fillColor] (131.64,139.22) circle (  1.96);

\path[draw=drawColor,line width= 0.4pt,line join=round,line cap=round,fill=fillColor] (131.64,139.22) circle (  1.96);

\path[draw=drawColor,line width= 0.4pt,line join=round,line cap=round,fill=fillColor] (131.64,153.48) circle (  1.96);

\path[draw=drawColor,line width= 0.4pt,line join=round,line cap=round,fill=fillColor] (131.64,153.48) circle (  1.96);

\path[draw=drawColor,line width= 0.4pt,line join=round,line cap=round,fill=fillColor] (131.64,148.72) circle (  1.96);

\path[draw=drawColor,line width= 0.4pt,line join=round,line cap=round,fill=fillColor] (131.64,151.89) circle (  1.96);

\path[draw=drawColor,line width= 0.4pt,line join=round,line cap=round,fill=fillColor] (131.64,141.59) circle (  1.96);

\path[draw=drawColor,line width= 0.4pt,line join=round,line cap=round,fill=fillColor] (131.64,145.55) circle (  1.96);

\path[draw=drawColor,line width= 0.4pt,line join=round,line cap=round,fill=fillColor] (131.64,138.76) circle (  1.96);

\path[draw=drawColor,line width= 0.4pt,line join=round,line cap=round,fill=fillColor] (131.64,153.48) circle (  1.96);

\path[draw=drawColor,line width= 0.4pt,line join=round,line cap=round,fill=fillColor] (131.64,141.59) circle (  1.96);

\path[draw=drawColor,line width= 0.4pt,line join=round,line cap=round,fill=fillColor] (131.64,141.59) circle (  1.96);

\path[draw=drawColor,line width= 0.4pt,line join=round,line cap=round,fill=fillColor] (131.64,153.48) circle (  1.96);

\path[draw=drawColor,line width= 0.4pt,line join=round,line cap=round,fill=fillColor] (131.64,145.55) circle (  1.96);

\path[draw=drawColor,line width= 0.4pt,line join=round,line cap=round,fill=fillColor] (131.64,141.59) circle (  1.96);

\path[draw=drawColor,line width= 0.4pt,line join=round,line cap=round,fill=fillColor] (131.64,138.27) circle (  1.96);

\path[draw=drawColor,line width= 0.4pt,line join=round,line cap=round,fill=fillColor] (131.64,139.22) circle (  1.96);

\path[draw=drawColor,line width= 0.4pt,line join=round,line cap=round,fill=fillColor] (131.64,145.55) circle (  1.96);

\path[draw=drawColor,line width= 0.4pt,line join=round,line cap=round,fill=fillColor] (131.64,141.59) circle (  1.96);

\path[draw=drawColor,line width= 0.4pt,line join=round,line cap=round,fill=fillColor] (131.64,150.83) circle (  1.96);

\path[draw=drawColor,line width= 0.4pt,line join=round,line cap=round,fill=fillColor] (131.64,141.59) circle (  1.96);

\path[draw=drawColor,line width= 0.4pt,line join=round,line cap=round,fill=fillColor] (131.64,141.59) circle (  1.96);

\path[draw=drawColor,line width= 0.4pt,line join=round,line cap=round,fill=fillColor] (131.64,141.87) circle (  1.96);

\path[draw=drawColor,line width= 0.4pt,line join=round,line cap=round,fill=fillColor] (131.64,148.50) circle (  1.96);

\path[draw=drawColor,line width= 0.4pt,line join=round,line cap=round,fill=fillColor] (131.64,140.54) circle (  1.96);

\path[draw=drawColor,line width= 0.4pt,line join=round,line cap=round,fill=fillColor] (131.64,148.50) circle (  1.96);

\path[draw=drawColor,line width= 0.4pt,line join=round,line cap=round,fill=fillColor] (131.64,144.08) circle (  1.96);

\path[draw=drawColor,line width= 0.4pt,line join=round,line cap=round,fill=fillColor] (131.64,144.08) circle (  1.96);

\path[draw=drawColor,line width= 0.4pt,line join=round,line cap=round,fill=fillColor] (131.64,148.50) circle (  1.96);

\path[draw=drawColor,line width= 0.4pt,line join=round,line cap=round,fill=fillColor] (131.64,145.85) circle (  1.96);

\path[draw=drawColor,line width= 0.4pt,line join=round,line cap=round,fill=fillColor] (131.64,139.03) circle (  1.96);

\path[draw=drawColor,line width= 0.4pt,line join=round,line cap=round,fill=fillColor] (131.64,138.78) circle (  1.96);

\path[draw=drawColor,line width= 0.4pt,line join=round,line cap=round,fill=fillColor] (131.64,138.78) circle (  1.96);

\path[draw=drawColor,line width= 0.4pt,line join=round,line cap=round,fill=fillColor] (131.64,148.50) circle (  1.96);

\path[draw=drawColor,line width= 0.4pt,line join=round,line cap=round,fill=fillColor] (131.64,148.50) circle (  1.96);

\path[draw=drawColor,line width= 0.4pt,line join=round,line cap=round,fill=fillColor] (131.64,140.42) circle (  1.96);

\path[draw=drawColor,line width= 0.4pt,line join=round,line cap=round,fill=fillColor] (131.64,140.54) circle (  1.96);

\path[draw=drawColor,line width= 0.4pt,line join=round,line cap=round,fill=fillColor] (131.64,143.20) circle (  1.96);

\path[draw=drawColor,line width= 0.4pt,line join=round,line cap=round,fill=fillColor] (131.64,148.50) circle (  1.96);

\path[draw=drawColor,line width= 0.4pt,line join=round,line cap=round,fill=fillColor] (131.64,139.66) circle (  1.96);

\path[draw=drawColor,line width= 0.4pt,line join=round,line cap=round,fill=fillColor] (131.64,141.61) circle (  1.96);

\path[draw=drawColor,line width= 0.4pt,line join=round,line cap=round,fill=fillColor] (131.64,148.50) circle (  1.96);

\path[draw=drawColor,line width= 0.4pt,line join=round,line cap=round,fill=fillColor] (131.64,148.50) circle (  1.96);

\path[draw=drawColor,line width= 0.4pt,line join=round,line cap=round,fill=fillColor] (131.64,144.08) circle (  1.96);

\path[draw=drawColor,line width= 0.4pt,line join=round,line cap=round,fill=fillColor] (131.64,140.21) circle (  1.96);

\path[draw=drawColor,line width= 0.4pt,line join=round,line cap=round,fill=fillColor] (131.64,141.87) circle (  1.96);

\path[draw=drawColor,line width= 0.4pt,line join=round,line cap=round,fill=fillColor] (131.64,144.08) circle (  1.96);

\path[draw=drawColor,line width= 0.4pt,line join=round,line cap=round,fill=fillColor] (131.64,151.16) circle (  1.96);

\path[draw=drawColor,line width= 0.4pt,line join=round,line cap=round,fill=fillColor] (131.64,151.16) circle (  1.96);

\path[draw=drawColor,line width= 0.4pt,line join=round,line cap=round,fill=fillColor] (131.64,141.87) circle (  1.96);

\path[draw=drawColor,line width= 0.4pt,line join=round,line cap=round,fill=fillColor] (131.64,148.50) circle (  1.96);

\path[draw=drawColor,line width= 0.4pt,line join=round,line cap=round,fill=fillColor] (131.64,149.98) circle (  1.96);

\path[draw=drawColor,line width= 0.4pt,line join=round,line cap=round,fill=fillColor] (131.64,152.93) circle (  1.96);

\path[draw=drawColor,line width= 0.4pt,line join=round,line cap=round,fill=fillColor] (131.64,139.03) circle (  1.96);

\path[draw=drawColor,line width= 0.4pt,line join=round,line cap=round,fill=fillColor] (131.64,152.93) circle (  1.96);

\path[draw=drawColor,line width= 0.4pt,line join=round,line cap=round,fill=fillColor] (131.64,148.50) circle (  1.96);

\path[draw=drawColor,line width= 0.4pt,line join=round,line cap=round,fill=fillColor] (131.64,142.82) circle (  1.96);

\path[draw=drawColor,line width= 0.4pt,line join=round,line cap=round,fill=fillColor] (131.64,148.50) circle (  1.96);

\path[draw=drawColor,line width= 0.4pt,line join=round,line cap=round,fill=fillColor] (131.64,142.47) circle (  1.96);

\path[draw=drawColor,line width= 0.4pt,line join=round,line cap=round,fill=fillColor] (131.64,148.50) circle (  1.96);

\path[draw=drawColor,line width= 0.4pt,line join=round,line cap=round,fill=fillColor] (131.64,141.61) circle (  1.96);

\path[draw=drawColor,line width= 0.4pt,line join=round,line cap=round,fill=fillColor] (131.64,153.81) circle (  1.96);

\path[draw=drawColor,line width= 0.4pt,line join=round,line cap=round,fill=fillColor] (131.64,140.54) circle (  1.96);

\path[draw=drawColor,line width= 0.4pt,line join=round,line cap=round,fill=fillColor] (131.64,145.85) circle (  1.96);

\path[draw=drawColor,line width= 0.4pt,line join=round,line cap=round,fill=fillColor] (131.64,142.67) circle (  1.96);

\path[draw=drawColor,line width= 0.4pt,line join=round,line cap=round,fill=fillColor] (131.64,141.87) circle (  1.96);

\path[draw=drawColor,line width= 0.4pt,line join=round,line cap=round,fill=fillColor] (131.64,141.21) circle (  1.96);

\path[draw=drawColor,line width= 0.4pt,line join=round,line cap=round,fill=fillColor] (131.64,153.01) circle (  1.96);

\path[draw=drawColor,line width= 0.4pt,line join=round,line cap=round,fill=fillColor] (131.64,147.11) circle (  1.96);

\path[draw=drawColor,line width= 0.4pt,line join=round,line cap=round,fill=fillColor] (131.64,152.28) circle (  1.96);

\path[draw=drawColor,line width= 0.4pt,line join=round,line cap=round,fill=fillColor] (131.64,148.59) circle (  1.96);

\path[draw=drawColor,line width= 0.4pt,line join=round,line cap=round,fill=fillColor] (131.64,141.21) circle (  1.96);

\path[draw=drawColor,line width= 0.4pt,line join=round,line cap=round,fill=fillColor] (131.64,145.63) circle (  1.96);

\path[draw=drawColor,line width= 0.4pt,line join=round,line cap=round,fill=fillColor] (131.64,141.21) circle (  1.96);

\path[draw=drawColor,line width= 0.4pt,line join=round,line cap=round,fill=fillColor] (131.64,145.14) circle (  1.96);

\path[draw=drawColor,line width= 0.4pt,line join=round,line cap=round,fill=fillColor] (131.64,147.11) circle (  1.96);

\path[draw=drawColor,line width= 0.4pt,line join=round,line cap=round,fill=fillColor] (131.64,139.24) circle (  1.96);

\path[draw=drawColor,line width= 0.4pt,line join=round,line cap=round,fill=fillColor] (131.64,141.21) circle (  1.96);

\path[draw=drawColor,line width= 0.4pt,line join=round,line cap=round,fill=fillColor] (131.64,151.05) circle (  1.96);

\path[draw=drawColor,line width= 0.4pt,line join=round,line cap=round,fill=fillColor] (131.64,149.64) circle (  1.96);

\path[draw=drawColor,line width= 0.4pt,line join=round,line cap=round,fill=fillColor] (131.64,153.01) circle (  1.96);

\path[draw=drawColor,line width= 0.4pt,line join=round,line cap=round,fill=fillColor] (131.64,151.05) circle (  1.96);

\path[draw=drawColor,line width= 0.4pt,line join=round,line cap=round,fill=fillColor] (131.64,149.64) circle (  1.96);

\path[draw=drawColor,line width= 0.4pt,line join=round,line cap=round,fill=fillColor] (131.64,141.21) circle (  1.96);

\path[draw=drawColor,line width= 0.4pt,line join=round,line cap=round,fill=fillColor] (131.64,151.05) circle (  1.96);

\path[draw=drawColor,line width= 0.4pt,line join=round,line cap=round,fill=fillColor] (131.64,141.21) circle (  1.96);

\path[draw=drawColor,line width= 0.4pt,line join=round,line cap=round,fill=fillColor] (131.64,141.21) circle (  1.96);

\path[draw=drawColor,line width= 0.4pt,line join=round,line cap=round,fill=fillColor] (131.64,153.01) circle (  1.96);

\path[draw=drawColor,line width= 0.4pt,line join=round,line cap=round,fill=fillColor] (131.64,141.21) circle (  1.96);

\path[draw=drawColor,line width= 0.4pt,line join=round,line cap=round,fill=fillColor] (131.64,141.21) circle (  1.96);

\path[draw=drawColor,line width= 0.4pt,line join=round,line cap=round,fill=fillColor] (131.64,144.58) circle (  1.96);

\path[draw=drawColor,line width= 0.4pt,line join=round,line cap=round,fill=fillColor] (131.64,141.21) circle (  1.96);

\path[draw=drawColor,line width= 0.4pt,line join=round,line cap=round,fill=fillColor] (131.64,143.67) circle (  1.96);

\path[draw=drawColor,line width= 0.4pt,line join=round,line cap=round,fill=fillColor] (131.64,139.98) circle (  1.96);

\path[draw=drawColor,line width= 0.4pt,line join=round,line cap=round,fill=fillColor] (131.64,141.21) circle (  1.96);

\path[draw=drawColor,line width= 0.4pt,line join=round,line cap=round,fill=fillColor] (131.64,141.21) circle (  1.96);

\path[draw=drawColor,line width= 0.4pt,line join=round,line cap=round,fill=fillColor] (131.64,149.64) circle (  1.96);

\path[draw=drawColor,line width= 0.4pt,line join=round,line cap=round,fill=fillColor] (131.64,151.05) circle (  1.96);

\path[draw=drawColor,line width= 0.4pt,line join=round,line cap=round,fill=fillColor] (131.64,141.21) circle (  1.96);

\path[draw=drawColor,line width= 0.4pt,line join=round,line cap=round,fill=fillColor] (131.64,141.21) circle (  1.96);

\path[draw=drawColor,line width= 0.4pt,line join=round,line cap=round,fill=fillColor] (131.64,143.39) circle (  1.96);

\path[draw=drawColor,line width= 0.4pt,line join=round,line cap=round,fill=fillColor] (131.64,151.05) circle (  1.96);

\path[draw=drawColor,line width= 0.4pt,line join=round,line cap=round,fill=fillColor] (131.64,141.21) circle (  1.96);

\path[draw=drawColor,line width= 0.4pt,line join=round,line cap=round,fill=fillColor] (131.64,141.21) circle (  1.96);

\path[draw=drawColor,line width= 0.4pt,line join=round,line cap=round,fill=fillColor] (131.64,145.42) circle (  1.96);

\path[draw=drawColor,line width= 0.4pt,line join=round,line cap=round,fill=fillColor] (131.64,141.21) circle (  1.96);

\path[draw=drawColor,line width= 0.4pt,line join=round,line cap=round,fill=fillColor] (131.64,141.21) circle (  1.96);

\path[draw=drawColor,line width= 0.4pt,line join=round,line cap=round,fill=fillColor] (131.64,141.21) circle (  1.96);

\path[draw=drawColor,line width= 0.4pt,line join=round,line cap=round,fill=fillColor] (131.64,151.05) circle (  1.96);

\path[draw=drawColor,line width= 0.4pt,line join=round,line cap=round,fill=fillColor] (131.64,154.32) circle (  1.96);

\path[draw=drawColor,line width= 0.4pt,line join=round,line cap=round,fill=fillColor] (131.64,149.26) circle (  1.96);

\path[draw=drawColor,line width= 0.4pt,line join=round,line cap=round,fill=fillColor] (131.64,151.05) circle (  1.96);

\path[draw=drawColor,line width= 0.4pt,line join=round,line cap=round,fill=fillColor] (131.64,139.10) circle (  1.96);

\path[draw=drawColor,line width= 0.4pt,line join=round,line cap=round,fill=fillColor] (131.64,141.21) circle (  1.96);

\path[draw=drawColor,line width= 0.4pt,line join=round,line cap=round,fill=fillColor] (131.64,139.24) circle (  1.96);

\path[draw=drawColor,line width= 0.4pt,line join=round,line cap=round,fill=fillColor] (131.64,150.43) circle (  1.96);

\path[draw=drawColor,line width= 0.4pt,line join=round,line cap=round,fill=fillColor] (131.64,143.67) circle (  1.96);

\path[draw=drawColor,line width= 0.4pt,line join=round,line cap=round,fill=fillColor] (131.64,141.21) circle (  1.96);

\path[draw=drawColor,line width= 0.4pt,line join=round,line cap=round,fill=fillColor] (131.64,141.21) circle (  1.96);

\path[draw=drawColor,line width= 0.4pt,line join=round,line cap=round,fill=fillColor] (131.64,142.89) circle (  1.96);

\path[draw=drawColor,line width= 0.4pt,line join=round,line cap=round,fill=fillColor] (131.64,147.77) circle (  1.96);

\path[draw=drawColor,line width= 0.4pt,line join=round,line cap=round,fill=fillColor] (131.64,148.15) circle (  1.96);

\path[draw=drawColor,line width= 0.4pt,line join=round,line cap=round,fill=fillColor] (131.64,141.21) circle (  1.96);

\path[draw=drawColor,line width= 0.4pt,line join=round,line cap=round,fill=fillColor] (131.64,141.21) circle (  1.96);

\path[draw=drawColor,line width= 0.4pt,line join=round,line cap=round,fill=fillColor] (131.64,141.21) circle (  1.96);

\path[draw=drawColor,line width= 0.4pt,line join=round,line cap=round,fill=fillColor] (131.64,141.21) circle (  1.96);

\path[draw=drawColor,line width= 0.4pt,line join=round,line cap=round,fill=fillColor] (131.64,146.74) circle (  1.96);

\path[draw=drawColor,line width= 0.4pt,line join=round,line cap=round,fill=fillColor] (131.64,141.21) circle (  1.96);

\path[draw=drawColor,line width= 0.4pt,line join=round,line cap=round,fill=fillColor] (131.64,144.90) circle (  1.96);

\path[draw=drawColor,line width= 0.4pt,line join=round,line cap=round,fill=fillColor] (131.64,141.21) circle (  1.96);

\path[draw=drawColor,line width= 0.4pt,line join=round,line cap=round,fill=fillColor] (131.64,141.21) circle (  1.96);

\path[draw=drawColor,line width= 0.4pt,line join=round,line cap=round,fill=fillColor] (131.64,149.64) circle (  1.96);

\path[draw=drawColor,line width= 0.4pt,line join=round,line cap=round,fill=fillColor] (131.64,141.21) circle (  1.96);

\path[draw=drawColor,line width= 0.4pt,line join=round,line cap=round,fill=fillColor] (131.64,141.21) circle (  1.96);

\path[draw=drawColor,line width= 0.4pt,line join=round,line cap=round,fill=fillColor] (131.64,149.08) circle (  1.96);

\path[draw=drawColor,line width= 0.4pt,line join=round,line cap=round,fill=fillColor] (131.64,141.21) circle (  1.96);

\path[draw=drawColor,line width= 0.4pt,line join=round,line cap=round,fill=fillColor] (131.64,141.21) circle (  1.96);

\path[draw=drawColor,line width= 0.4pt,line join=round,line cap=round,fill=fillColor] (131.64,153.01) circle (  1.96);

\path[draw=drawColor,line width= 0.4pt,line join=round,line cap=round,fill=fillColor] (131.64,141.21) circle (  1.96);

\path[draw=drawColor,line width= 0.4pt,line join=round,line cap=round,fill=fillColor] (131.64,144.16) circle (  1.96);

\path[draw=drawColor,line width= 0.4pt,line join=round,line cap=round,fill=fillColor] (131.64,141.21) circle (  1.96);

\path[draw=drawColor,line width= 0.4pt,line join=round,line cap=round,fill=fillColor] (131.64,149.26) circle (  1.96);

\path[draw=drawColor,line width= 0.4pt,line join=round,line cap=round,fill=fillColor] (131.64,141.21) circle (  1.96);

\path[draw=drawColor,line width= 0.4pt,line join=round,line cap=round,fill=fillColor] (131.64,140.07) circle (  1.96);

\path[draw=drawColor,line width= 0.4pt,line join=round,line cap=round,fill=fillColor] (131.64,142.49) circle (  1.96);

\path[draw=drawColor,line width= 0.4pt,line join=round,line cap=round,fill=fillColor] (131.64,149.72) circle (  1.96);

\path[draw=drawColor,line width= 0.4pt,line join=round,line cap=round,fill=fillColor] (131.64,140.07) circle (  1.96);

\path[draw=drawColor,line width= 0.4pt,line join=round,line cap=round,fill=fillColor] (131.64,149.72) circle (  1.96);

\path[draw=drawColor,line width= 0.4pt,line join=round,line cap=round,fill=fillColor] (131.64,141.28) circle (  1.96);

\path[draw=drawColor,line width= 0.4pt,line join=round,line cap=round,fill=fillColor] (131.64,138.47) circle (  1.96);

\path[draw=drawColor,line width= 0.4pt,line join=round,line cap=round,fill=fillColor] (131.64,142.97) circle (  1.96);

\path[draw=drawColor,line width= 0.4pt,line join=round,line cap=round,fill=fillColor] (131.64,138.47) circle (  1.96);

\path[draw=drawColor,line width= 0.4pt,line join=round,line cap=round,fill=fillColor] (131.64,139.59) circle (  1.96);

\path[draw=drawColor,line width= 0.4pt,line join=round,line cap=round,fill=fillColor] (131.64,140.72) circle (  1.96);

\path[draw=drawColor,line width= 0.4pt,line join=round,line cap=round,fill=fillColor] (131.64,149.72) circle (  1.96);

\path[draw=drawColor,line width= 0.4pt,line join=round,line cap=round,fill=fillColor] (131.64,149.72) circle (  1.96);

\path[draw=drawColor,line width= 0.4pt,line join=round,line cap=round,fill=fillColor] (131.64,140.72) circle (  1.96);

\path[draw=drawColor,line width= 0.4pt,line join=round,line cap=round,fill=fillColor] (131.64,145.22) circle (  1.96);

\path[draw=drawColor,line width= 0.4pt,line join=round,line cap=round,fill=fillColor] (131.64,138.47) circle (  1.96);

\path[draw=drawColor,line width= 0.4pt,line join=round,line cap=round,fill=fillColor] (131.64,149.72) circle (  1.96);

\path[draw=drawColor,line width= 0.4pt,line join=round,line cap=round,fill=fillColor] (131.64,154.23) circle (  1.96);

\path[draw=drawColor,line width= 0.4pt,line join=round,line cap=round,fill=fillColor] (131.64,149.72) circle (  1.96);

\path[draw=drawColor,line width= 0.4pt,line join=round,line cap=round,fill=fillColor] (131.64,147.85) circle (  1.96);

\path[draw=drawColor,line width= 0.4pt,line join=round,line cap=round,fill=fillColor] (131.64,149.72) circle (  1.96);

\path[draw=drawColor,line width= 0.4pt,line join=round,line cap=round,fill=fillColor] (131.64,145.97) circle (  1.96);

\path[draw=drawColor,line width= 0.4pt,line join=round,line cap=round,fill=fillColor] (131.64,144.32) circle (  1.96);

\path[draw=drawColor,line width= 0.4pt,line join=round,line cap=round,fill=fillColor] (131.64,148.12) circle (  1.96);

\path[draw=drawColor,line width= 0.4pt,line join=round,line cap=round,fill=fillColor] (131.64,145.50) circle (  1.96);

\path[draw=drawColor,line width= 0.4pt,line join=round,line cap=round,fill=fillColor] (131.64,138.47) circle (  1.96);

\path[draw=drawColor,line width= 0.4pt,line join=round,line cap=round,fill=fillColor] (131.64,144.72) circle (  1.96);

\path[draw=drawColor,line width= 0.4pt,line join=round,line cap=round,fill=fillColor] (131.64,149.72) circle (  1.96);

\path[draw=drawColor,line width= 0.4pt,line join=round,line cap=round,fill=fillColor] (131.64,149.72) circle (  1.96);

\path[draw=drawColor,line width= 0.4pt,line join=round,line cap=round,fill=fillColor] (131.64,154.55) circle (  1.96);

\path[draw=drawColor,line width= 0.4pt,line join=round,line cap=round,fill=fillColor] (131.64,138.47) circle (  1.96);

\path[draw=drawColor,line width= 0.4pt,line join=round,line cap=round,fill=fillColor] (131.64,138.47) circle (  1.96);

\path[draw=drawColor,line width= 0.4pt,line join=round,line cap=round,fill=fillColor] (131.64,149.72) circle (  1.96);

\path[draw=drawColor,line width= 0.4pt,line join=round,line cap=round,fill=fillColor] (131.64,149.72) circle (  1.96);

\path[draw=drawColor,line width= 0.4pt,line join=round,line cap=round,fill=fillColor] (131.64,149.72) circle (  1.96);

\path[draw=drawColor,line width= 0.4pt,line join=round,line cap=round,fill=fillColor] (131.64,140.72) circle (  1.96);

\path[draw=drawColor,line width= 0.4pt,line join=round,line cap=round,fill=fillColor] (131.64,138.47) circle (  1.96);

\path[draw=drawColor,line width= 0.4pt,line join=round,line cap=round,fill=fillColor] (131.64,154.23) circle (  1.96);

\path[draw=drawColor,line width= 0.4pt,line join=round,line cap=round,fill=fillColor] (131.64,149.72) circle (  1.96);

\path[draw=drawColor,line width= 0.4pt,line join=round,line cap=round,fill=fillColor] (131.64,138.47) circle (  1.96);

\path[draw=drawColor,line width= 0.4pt,line join=round,line cap=round,fill=fillColor] (131.64,138.47) circle (  1.96);

\path[draw=drawColor,line width= 0.4pt,line join=round,line cap=round,fill=fillColor] (131.64,149.72) circle (  1.96);

\path[draw=drawColor,line width= 0.4pt,line join=round,line cap=round,fill=fillColor] (131.64,152.54) circle (  1.96);

\path[draw=drawColor,line width= 0.4pt,line join=round,line cap=round,fill=fillColor] (131.64,149.72) circle (  1.96);

\path[draw=drawColor,line width= 0.4pt,line join=round,line cap=round,fill=fillColor] (131.64,138.47) circle (  1.96);

\path[draw=drawColor,line width= 0.4pt,line join=round,line cap=round,fill=fillColor] (131.64,138.47) circle (  1.96);

\path[draw=drawColor,line width= 0.4pt,line join=round,line cap=round,fill=fillColor] (131.64,140.72) circle (  1.96);

\path[draw=drawColor,line width= 0.4pt,line join=round,line cap=round,fill=fillColor] (131.64,146.35) circle (  1.96);

\path[draw=drawColor,line width= 0.4pt,line join=round,line cap=round,fill=fillColor] (131.64,138.47) circle (  1.96);

\path[draw=drawColor,line width= 0.4pt,line join=round,line cap=round,fill=fillColor] (131.64,149.72) circle (  1.96);

\path[draw=drawColor,line width= 0.4pt,line join=round,line cap=round,fill=fillColor] (131.64,139.59) circle (  1.96);

\path[draw=drawColor,line width= 0.4pt,line join=round,line cap=round,fill=fillColor] (131.64,141.28) circle (  1.96);

\path[draw=drawColor,line width= 0.4pt,line join=round,line cap=round,fill=fillColor] (131.64,141.28) circle (  1.96);

\path[draw=drawColor,line width= 0.4pt,line join=round,line cap=round,fill=fillColor] (131.64,149.72) circle (  1.96);

\path[draw=drawColor,line width= 0.4pt,line join=round,line cap=round,fill=fillColor] (131.64,141.28) circle (  1.96);

\path[draw=drawColor,line width= 0.4pt,line join=round,line cap=round,fill=fillColor] (131.64,145.22) circle (  1.96);

\path[draw=drawColor,line width= 0.4pt,line join=round,line cap=round,fill=fillColor] (131.64,138.47) circle (  1.96);

\path[draw=drawColor,line width= 0.4pt,line join=round,line cap=round,fill=fillColor] (131.64,138.47) circle (  1.96);

\path[draw=drawColor,line width= 0.4pt,line join=round,line cap=round,fill=fillColor] (131.64,138.47) circle (  1.96);

\path[draw=drawColor,line width= 0.4pt,line join=round,line cap=round,fill=fillColor] (131.64,145.50) circle (  1.96);

\path[draw=drawColor,line width= 0.4pt,line join=round,line cap=round,fill=fillColor] (131.64,145.50) circle (  1.96);

\path[draw=drawColor,line width= 0.4pt,line join=round,line cap=round,fill=fillColor] (131.64,153.10) circle (  1.96);

\path[draw=drawColor,line width= 0.4pt,line join=round,line cap=round,fill=fillColor] (131.64,141.28) circle (  1.96);

\path[draw=drawColor,line width= 0.4pt,line join=round,line cap=round,fill=fillColor] (131.64,142.97) circle (  1.96);

\path[draw=drawColor,line width= 0.4pt,line join=round,line cap=round,fill=fillColor] (131.64,149.72) circle (  1.96);

\path[draw=drawColor,line width= 0.4pt,line join=round,line cap=round,fill=fillColor] (131.64,149.72) circle (  1.96);

\path[draw=drawColor,line width= 0.4pt,line join=round,line cap=round,fill=fillColor] (131.64,141.28) circle (  1.96);

\path[draw=drawColor,line width= 0.4pt,line join=round,line cap=round,fill=fillColor] (131.64,141.28) circle (  1.96);

\path[draw=drawColor,line width= 0.4pt,line join=round,line cap=round,fill=fillColor] (131.64,149.72) circle (  1.96);

\path[draw=drawColor,line width= 0.4pt,line join=round,line cap=round,fill=fillColor] (131.64,138.47) circle (  1.96);

\path[draw=drawColor,line width= 0.4pt,line join=round,line cap=round,fill=fillColor] (131.64,140.72) circle (  1.96);

\path[draw=drawColor,line width= 0.4pt,line join=round,line cap=round,fill=fillColor] (131.64,149.72) circle (  1.96);

\path[draw=drawColor,line width= 0.4pt,line join=round,line cap=round,fill=fillColor] (131.64,152.43) circle (  1.96);

\path[draw=drawColor,line width= 0.4pt,line join=round,line cap=round,fill=fillColor] (131.64,138.47) circle (  1.96);

\path[draw=drawColor,line width= 0.4pt,line join=round,line cap=round,fill=fillColor] (131.64,149.72) circle (  1.96);

\path[draw=drawColor,line width= 0.4pt,line join=round,line cap=round,fill=fillColor] (131.64,144.32) circle (  1.96);

\path[draw=drawColor,line width= 0.4pt,line join=round,line cap=round,fill=fillColor] (131.64,149.72) circle (  1.96);

\path[draw=drawColor,line width= 0.4pt,line join=round,line cap=round,fill=fillColor] (131.64,146.35) circle (  1.96);

\path[draw=drawColor,line width= 0.4pt,line join=round,line cap=round,fill=fillColor] (131.64,144.10) circle (  1.96);

\path[draw=drawColor,line width= 0.4pt,line join=round,line cap=round,fill=fillColor] (131.64,138.47) circle (  1.96);

\path[draw=drawColor,line width= 0.4pt,line join=round,line cap=round,fill=fillColor] (131.64,141.62) circle (  1.96);

\path[draw=drawColor,line width= 0.4pt,line join=round,line cap=round,fill=fillColor] (131.64,139.59) circle (  1.96);

\path[draw=drawColor,line width= 0.4pt,line join=round,line cap=round,fill=fillColor] (131.64,143.93) circle (  1.96);

\path[draw=drawColor,line width= 0.4pt,line join=round,line cap=round,fill=fillColor] (131.64,141.28) circle (  1.96);

\path[draw=drawColor,line width= 0.4pt,line join=round,line cap=round,fill=fillColor] (131.64,138.47) circle (  1.96);

\path[draw=drawColor,line width= 0.4pt,line join=round,line cap=round,fill=fillColor] (131.64,145.22) circle (  1.96);

\path[draw=drawColor,line width= 0.4pt,line join=round,line cap=round,fill=fillColor] (131.64,144.32) circle (  1.96);

\path[draw=drawColor,line width= 0.4pt,line join=round,line cap=round,fill=fillColor] (131.64,139.59) circle (  1.96);

\path[draw=drawColor,line width= 0.4pt,line join=round,line cap=round,fill=fillColor] (131.64,142.97) circle (  1.96);

\path[draw=drawColor,line width= 0.4pt,line join=round,line cap=round,fill=fillColor] (131.64,145.50) circle (  1.96);

\path[draw=drawColor,line width= 0.4pt,line join=round,line cap=round,fill=fillColor] (131.64,138.47) circle (  1.96);

\path[draw=drawColor,line width= 0.4pt,line join=round,line cap=round,fill=fillColor] (131.64,149.72) circle (  1.96);

\path[draw=drawColor,line width= 0.4pt,line join=round,line cap=round,fill=fillColor] (131.64,138.47) circle (  1.96);

\path[draw=drawColor,line width= 0.4pt,line join=round,line cap=round,fill=fillColor] (131.64,138.47) circle (  1.96);

\path[draw=drawColor,line width= 0.4pt,line join=round,line cap=round,fill=fillColor] (131.64,138.47) circle (  1.96);

\path[draw=drawColor,line width= 0.4pt,line join=round,line cap=round,fill=fillColor] (131.64,145.22) circle (  1.96);

\path[draw=drawColor,line width= 0.4pt,line join=round,line cap=round,fill=fillColor] (131.64,138.47) circle (  1.96);

\path[draw=drawColor,line width= 0.4pt,line join=round,line cap=round,fill=fillColor] (131.64,138.88) circle (  1.96);

\path[draw=drawColor,line width= 0.4pt,line join=round,line cap=round,fill=fillColor] (131.64,153.06) circle (  1.96);

\path[draw=drawColor,line width= 0.4pt,line join=round,line cap=round,fill=fillColor] (131.64,153.06) circle (  1.96);

\path[draw=drawColor,line width= 0.4pt,line join=round,line cap=round,fill=fillColor] (131.64,140.33) circle (  1.96);

\path[draw=drawColor,line width= 0.4pt,line join=round,line cap=round,fill=fillColor] (131.64,145.18) circle (  1.96);

\path[draw=drawColor,line width= 0.4pt,line join=round,line cap=round,fill=fillColor] (131.64,138.88) circle (  1.96);

\path[draw=drawColor,line width= 0.4pt,line join=round,line cap=round,fill=fillColor] (131.64,144.55) circle (  1.96);

\path[draw=drawColor,line width= 0.4pt,line join=round,line cap=round,fill=fillColor] (131.64,142.66) circle (  1.96);

\path[draw=drawColor,line width= 0.4pt,line join=round,line cap=round,fill=fillColor] (131.64,138.88) circle (  1.96);

\path[draw=drawColor,line width= 0.4pt,line join=round,line cap=round,fill=fillColor] (131.64,138.88) circle (  1.96);

\path[draw=drawColor,line width= 0.4pt,line join=round,line cap=round,fill=fillColor] (131.64,150.22) circle (  1.96);

\path[draw=drawColor,line width= 0.4pt,line join=round,line cap=round,fill=fillColor] (131.64,138.88) circle (  1.96);

\path[draw=drawColor,line width= 0.4pt,line join=round,line cap=round,fill=fillColor] (131.64,138.88) circle (  1.96);

\path[draw=drawColor,line width= 0.4pt,line join=round,line cap=round,fill=fillColor] (131.64,142.66) circle (  1.96);

\path[draw=drawColor,line width= 0.4pt,line join=round,line cap=round,fill=fillColor] (131.64,138.88) circle (  1.96);

\path[draw=drawColor,line width= 0.4pt,line join=round,line cap=round,fill=fillColor] (131.64,145.18) circle (  1.96);

\path[draw=drawColor,line width= 0.4pt,line join=round,line cap=round,fill=fillColor] (131.64,153.06) circle (  1.96);

\path[draw=drawColor,line width= 0.4pt,line join=round,line cap=round,fill=fillColor] (131.64,147.70) circle (  1.96);

\path[draw=drawColor,line width= 0.4pt,line join=round,line cap=round,fill=fillColor] (131.64,142.66) circle (  1.96);

\path[draw=drawColor,line width= 0.4pt,line join=round,line cap=round,fill=fillColor] (131.64,152.74) circle (  1.96);

\path[draw=drawColor,line width= 0.4pt,line join=round,line cap=round,fill=fillColor] (131.64,145.18) circle (  1.96);

\path[draw=drawColor,line width= 0.4pt,line join=round,line cap=round,fill=fillColor] (131.64,142.66) circle (  1.96);

\path[draw=drawColor,line width= 0.4pt,line join=round,line cap=round,fill=fillColor] (131.64,141.24) circle (  1.96);

\path[draw=drawColor,line width= 0.4pt,line join=round,line cap=round,fill=fillColor] (131.64,146.44) circle (  1.96);

\path[draw=drawColor,line width= 0.4pt,line join=round,line cap=round,fill=fillColor] (131.64,142.66) circle (  1.96);

\path[draw=drawColor,line width= 0.4pt,line join=round,line cap=round,fill=fillColor] (131.64,142.66) circle (  1.96);

\path[draw=drawColor,line width= 0.4pt,line join=round,line cap=round,fill=fillColor] (131.64,145.18) circle (  1.96);

\path[draw=drawColor,line width= 0.4pt,line join=round,line cap=round,fill=fillColor] (131.64,140.68) circle (  1.96);

\path[draw=drawColor,line width= 0.4pt,line join=round,line cap=round,fill=fillColor] (131.64,145.69) circle (  1.96);

\path[draw=drawColor,line width= 0.4pt,line join=round,line cap=round,fill=fillColor] (131.64,145.18) circle (  1.96);

\path[draw=drawColor,line width= 0.4pt,line join=round,line cap=round,fill=fillColor] (131.64,138.88) circle (  1.96);

\path[draw=drawColor,line width= 0.4pt,line join=round,line cap=round,fill=fillColor] (131.64,153.06) circle (  1.96);

\path[draw=drawColor,line width= 0.4pt,line join=round,line cap=round,fill=fillColor] (131.64,140.23) circle (  1.96);

\path[draw=drawColor,line width= 0.4pt,line join=round,line cap=round,fill=fillColor] (131.64,146.44) circle (  1.96);

\path[draw=drawColor,line width= 0.4pt,line join=round,line cap=round,fill=fillColor] (131.64,145.18) circle (  1.96);

\path[draw=drawColor,line width= 0.4pt,line join=round,line cap=round,fill=fillColor] (131.64,141.87) circle (  1.96);

\path[draw=drawColor,line width= 0.4pt,line join=round,line cap=round,fill=fillColor] (131.64,141.24) circle (  1.96);

\path[draw=drawColor,line width= 0.4pt,line join=round,line cap=round,fill=fillColor] (131.64,146.98) circle (  1.96);

\path[draw=drawColor,line width= 0.4pt,line join=round,line cap=round,fill=fillColor] (131.64,148.33) circle (  1.96);

\path[draw=drawColor,line width= 0.4pt,line join=round,line cap=round,fill=fillColor] (131.64,149.38) circle (  1.96);

\path[draw=drawColor,line width= 0.4pt,line join=round,line cap=round,fill=fillColor] (131.64,138.88) circle (  1.96);

\path[draw=drawColor,line width= 0.4pt,line join=round,line cap=round,fill=fillColor] (131.64,148.33) circle (  1.96);

\path[draw=drawColor,line width= 0.4pt,line join=round,line cap=round,fill=fillColor] (131.64,141.24) circle (  1.96);

\path[draw=drawColor,line width= 0.4pt,line join=round,line cap=round,fill=fillColor] (131.64,138.88) circle (  1.96);

\path[draw=drawColor,line width= 0.4pt,line join=round,line cap=round,fill=fillColor] (131.64,138.88) circle (  1.96);

\path[draw=drawColor,line width= 0.4pt,line join=round,line cap=round,fill=fillColor] (131.64,153.06) circle (  1.96);

\path[draw=drawColor,line width= 0.4pt,line join=round,line cap=round,fill=fillColor] (131.64,145.18) circle (  1.96);

\path[draw=drawColor,line width= 0.4pt,line join=round,line cap=round,fill=fillColor] (131.64,138.88) circle (  1.96);

\path[draw=drawColor,line width= 0.4pt,line join=round,line cap=round,fill=fillColor] (131.64,141.24) circle (  1.96);

\path[draw=drawColor,line width= 0.4pt,line join=round,line cap=round,fill=fillColor] (131.64,151.48) circle (  1.96);

\path[draw=drawColor,line width= 0.4pt,line join=round,line cap=round,fill=fillColor] (131.64,140.14) circle (  1.96);

\path[draw=drawColor,line width= 0.4pt,line join=round,line cap=round,fill=fillColor] (131.64,145.18) circle (  1.96);

\path[draw=drawColor,line width= 0.4pt,line join=round,line cap=round,fill=fillColor] (131.64,146.12) circle (  1.96);

\path[draw=drawColor,line width= 0.4pt,line join=round,line cap=round,fill=fillColor] (131.64,146.98) circle (  1.96);

\path[draw=drawColor,line width= 0.4pt,line join=round,line cap=round,fill=fillColor] (131.64,145.18) circle (  1.96);

\path[draw=drawColor,line width= 0.4pt,line join=round,line cap=round,fill=fillColor] (131.64,138.88) circle (  1.96);

\path[draw=drawColor,line width= 0.4pt,line join=round,line cap=round,fill=fillColor] (131.64,139.64) circle (  1.96);

\path[draw=drawColor,line width= 0.4pt,line join=round,line cap=round,fill=fillColor] (131.64,146.98) circle (  1.96);

\path[draw=drawColor,line width= 0.4pt,line join=round,line cap=round,fill=fillColor] (131.64,140.14) circle (  1.96);

\path[draw=drawColor,line width= 0.4pt,line join=round,line cap=round,fill=fillColor] (131.64,145.18) circle (  1.96);

\path[draw=drawColor,line width= 0.4pt,line join=round,line cap=round,fill=fillColor] (131.64,138.88) circle (  1.96);

\path[draw=drawColor,line width= 0.4pt,line join=round,line cap=round,fill=fillColor] (131.64,138.88) circle (  1.96);

\path[draw=drawColor,line width= 0.4pt,line join=round,line cap=round,fill=fillColor] (131.64,142.66) circle (  1.96);

\path[draw=drawColor,line width= 0.4pt,line join=round,line cap=round,fill=fillColor] (131.64,151.30) circle (  1.96);

\path[draw=drawColor,line width= 0.4pt,line join=round,line cap=round,fill=fillColor] (131.64,142.66) circle (  1.96);

\path[draw=drawColor,line width= 0.4pt,line join=round,line cap=round,fill=fillColor] (131.64,138.88) circle (  1.96);

\path[draw=drawColor,line width= 0.4pt,line join=round,line cap=round,fill=fillColor] (131.64,142.66) circle (  1.96);

\path[draw=drawColor,line width= 0.4pt,line join=round,line cap=round,fill=fillColor] (131.64,145.18) circle (  1.96);

\path[draw=drawColor,line width= 0.4pt,line join=round,line cap=round,fill=fillColor] (131.64,140.98) circle (  1.96);

\path[draw=drawColor,line width= 0.4pt,line join=round,line cap=round,fill=fillColor] (131.64,145.18) circle (  1.96);

\path[draw=drawColor,line width= 0.4pt,line join=round,line cap=round,fill=fillColor] (131.64,153.06) circle (  1.96);

\path[draw=drawColor,line width= 0.4pt,line join=round,line cap=round,fill=fillColor] (131.64,138.88) circle (  1.96);

\path[draw=drawColor,line width= 0.4pt,line join=round,line cap=round,fill=fillColor] (131.64,142.66) circle (  1.96);

\path[draw=drawColor,line width= 0.4pt,line join=round,line cap=round,fill=fillColor] (131.64,152.74) circle (  1.96);

\path[draw=drawColor,line width= 0.4pt,line join=round,line cap=round,fill=fillColor] (131.64,152.74) circle (  1.96);

\path[draw=drawColor,line width= 0.4pt,line join=round,line cap=round,fill=fillColor] (131.64,142.66) circle (  1.96);

\path[draw=drawColor,line width= 0.4pt,line join=round,line cap=round,fill=fillColor] (131.64,151.48) circle (  1.96);

\path[draw=drawColor,line width= 0.4pt,line join=round,line cap=round,fill=fillColor] (131.64,149.38) circle (  1.96);

\path[draw=drawColor,line width= 0.4pt,line join=round,line cap=round,fill=fillColor] (131.64,142.32) circle (  1.96);

\path[draw=drawColor,line width= 0.4pt,line join=round,line cap=round,fill=fillColor] (131.64,143.61) circle (  1.96);

\path[draw=drawColor,line width= 0.4pt,line join=round,line cap=round,fill=fillColor] (131.64,142.66) circle (  1.96);

\path[draw=drawColor,line width= 0.4pt,line join=round,line cap=round,fill=fillColor] (131.64,150.22) circle (  1.96);

\path[draw=drawColor,line width= 0.4pt,line join=round,line cap=round,fill=fillColor] (131.64,145.18) circle (  1.96);

\path[draw=drawColor,line width= 0.4pt,line join=round,line cap=round,fill=fillColor] (131.64,149.38) circle (  1.96);

\path[draw=drawColor,line width= 0.4pt,line join=round,line cap=round,fill=fillColor] (131.64,150.22) circle (  1.96);

\path[draw=drawColor,line width= 0.4pt,line join=round,line cap=round,fill=fillColor] (131.64,151.48) circle (  1.96);

\path[draw=drawColor,line width= 0.4pt,line join=round,line cap=round,fill=fillColor] (131.64,145.18) circle (  1.96);

\path[draw=drawColor,line width= 0.4pt,line join=round,line cap=round,fill=fillColor] (131.64,148.71) circle (  1.96);

\path[draw=drawColor,line width= 0.4pt,line join=round,line cap=round,fill=fillColor] (131.64,142.66) circle (  1.96);

\path[draw=drawColor,line width= 0.4pt,line join=round,line cap=round,fill=fillColor] (131.64,145.18) circle (  1.96);

\path[draw=drawColor,line width= 0.4pt,line join=round,line cap=round,fill=fillColor] (131.64,150.91) circle (  1.96);

\path[draw=drawColor,line width= 0.4pt,line join=round,line cap=round,fill=fillColor] (131.64,151.73) circle (  1.96);

\path[draw=drawColor,line width= 0.4pt,line join=round,line cap=round,fill=fillColor] (131.64,138.88) circle (  1.96);

\path[draw=drawColor,line width= 0.4pt,line join=round,line cap=round,fill=fillColor] (131.64,153.06) circle (  1.96);

\path[draw=drawColor,line width= 0.4pt,line join=round,line cap=round,fill=fillColor] (131.64,145.18) circle (  1.96);

\path[draw=drawColor,line width= 0.4pt,line join=round,line cap=round,fill=fillColor] (131.64,144.55) circle (  1.96);

\path[draw=drawColor,line width= 0.4pt,line join=round,line cap=round,fill=fillColor] (131.64,138.88) circle (  1.96);

\path[draw=drawColor,line width= 0.4pt,line join=round,line cap=round,fill=fillColor] (131.64,140.46) circle (  1.96);

\path[draw=drawColor,line width= 0.4pt,line join=round,line cap=round,fill=fillColor] (131.64,138.88) circle (  1.96);

\path[draw=drawColor,line width= 0.4pt,line join=round,line cap=round,fill=fillColor] (131.64,149.38) circle (  1.96);

\path[draw=drawColor,line width= 0.4pt,line join=round,line cap=round,fill=fillColor] (131.64,148.71) circle (  1.96);

\path[draw=drawColor,line width= 0.4pt,line join=round,line cap=round,fill=fillColor] (131.64,145.69) circle (  1.96);

\path[draw=drawColor,line width= 0.4pt,line join=round,line cap=round,fill=fillColor] (131.64,145.18) circle (  1.96);

\path[draw=drawColor,line width= 0.4pt,line join=round,line cap=round,fill=fillColor] (131.64,138.88) circle (  1.96);

\path[draw=drawColor,line width= 0.4pt,line join=round,line cap=round,fill=fillColor] (131.64,145.69) circle (  1.96);

\path[draw=drawColor,line width= 0.4pt,line join=round,line cap=round,fill=fillColor] (131.64,152.11) circle (  1.96);

\path[draw=drawColor,line width= 0.4pt,line join=round,line cap=round,fill=fillColor] (131.64,141.24) circle (  1.96);

\path[draw=drawColor,line width= 0.4pt,line join=round,line cap=round,fill=fillColor] (131.64,139.28) circle (  1.96);

\path[draw=drawColor,line width= 0.4pt,line join=round,line cap=round,fill=fillColor] (131.64,151.52) circle (  1.96);

\path[draw=drawColor,line width= 0.4pt,line join=round,line cap=round,fill=fillColor] (131.64,139.28) circle (  1.96);

\path[draw=drawColor,line width= 0.4pt,line join=round,line cap=round,fill=fillColor] (131.64,140.45) circle (  1.96);

\path[draw=drawColor,line width= 0.4pt,line join=round,line cap=round,fill=fillColor] (131.64,140.45) circle (  1.96);

\path[draw=drawColor,line width= 0.4pt,line join=round,line cap=round,fill=fillColor] (131.64,152.10) circle (  1.96);

\path[draw=drawColor,line width= 0.4pt,line join=round,line cap=round,fill=fillColor] (131.64,150.16) circle (  1.96);

\path[draw=drawColor,line width= 0.4pt,line join=round,line cap=round,fill=fillColor] (131.64,145.63) circle (  1.96);

\path[draw=drawColor,line width= 0.4pt,line join=round,line cap=round,fill=fillColor] (131.64,151.06) circle (  1.96);

\path[draw=drawColor,line width= 0.4pt,line join=round,line cap=round,fill=fillColor] (131.64,150.16) circle (  1.96);

\path[draw=drawColor,line width= 0.4pt,line join=round,line cap=round,fill=fillColor] (131.64,150.16) circle (  1.96);

\path[draw=drawColor,line width= 0.4pt,line join=round,line cap=round,fill=fillColor] (131.64,145.91) circle (  1.96);

\path[draw=drawColor,line width= 0.4pt,line join=round,line cap=round,fill=fillColor] (131.64,143.36) circle (  1.96);

\path[draw=drawColor,line width= 0.4pt,line join=round,line cap=round,fill=fillColor] (131.64,147.44) circle (  1.96);

\path[draw=drawColor,line width= 0.4pt,line join=round,line cap=round,fill=fillColor] (131.64,144.72) circle (  1.96);

\path[draw=drawColor,line width= 0.4pt,line join=round,line cap=round,fill=fillColor] (131.64,140.45) circle (  1.96);

\path[draw=drawColor,line width= 0.4pt,line join=round,line cap=round,fill=fillColor] (131.64,143.36) circle (  1.96);

\path[draw=drawColor,line width= 0.4pt,line join=round,line cap=round,fill=fillColor] (131.64,143.36) circle (  1.96);

\path[draw=drawColor,line width= 0.4pt,line join=round,line cap=round,fill=fillColor] (131.64,153.56) circle (  1.96);

\path[draw=drawColor,line width= 0.4pt,line join=round,line cap=round,fill=fillColor] (131.64,139.28) circle (  1.96);

\path[draw=drawColor,line width= 0.4pt,line join=round,line cap=round,fill=fillColor] (131.64,138.83) circle (  1.96);

\path[draw=drawColor,line width= 0.4pt,line join=round,line cap=round,fill=fillColor] (131.64,153.56) circle (  1.96);

\path[draw=drawColor,line width= 0.4pt,line join=round,line cap=round,fill=fillColor] (131.64,143.36) circle (  1.96);

\path[draw=drawColor,line width= 0.4pt,line join=round,line cap=round,fill=fillColor] (131.64,141.66) circle (  1.96);

\path[draw=drawColor,line width= 0.4pt,line join=round,line cap=round,fill=fillColor] (131.64,143.36) circle (  1.96);

\path[draw=drawColor,line width= 0.4pt,line join=round,line cap=round,fill=fillColor] (131.64,150.16) circle (  1.96);

\path[draw=drawColor,line width= 0.4pt,line join=round,line cap=round,fill=fillColor] (131.64,140.45) circle (  1.96);

\path[draw=drawColor,line width= 0.4pt,line join=round,line cap=round,fill=fillColor] (131.64,143.36) circle (  1.96);

\path[draw=drawColor,line width= 0.4pt,line join=round,line cap=round,fill=fillColor] (131.64,145.91) circle (  1.96);

\path[draw=drawColor,line width= 0.4pt,line join=round,line cap=round,fill=fillColor] (131.64,143.36) circle (  1.96);

\path[draw=drawColor,line width= 0.4pt,line join=round,line cap=round,fill=fillColor] (131.64,143.36) circle (  1.96);

\path[draw=drawColor,line width= 0.4pt,line join=round,line cap=round,fill=fillColor] (131.64,150.16) circle (  1.96);

\path[draw=drawColor,line width= 0.4pt,line join=round,line cap=round,fill=fillColor] (131.64,149.48) circle (  1.96);

\path[draw=drawColor,line width= 0.4pt,line join=round,line cap=round,fill=fillColor] (131.64,138.83) circle (  1.96);

\path[draw=drawColor,line width= 0.4pt,line join=round,line cap=round,fill=fillColor] (131.64,142.60) circle (  1.96);

\path[draw=drawColor,line width= 0.4pt,line join=round,line cap=round,fill=fillColor] (131.64,143.36) circle (  1.96);

\path[draw=drawColor,line width= 0.4pt,line join=round,line cap=round,fill=fillColor] (131.64,143.36) circle (  1.96);

\path[draw=drawColor,line width= 0.4pt,line join=round,line cap=round,fill=fillColor] (131.64,142.78) circle (  1.96);

\path[draw=drawColor,line width= 0.4pt,line join=round,line cap=round,fill=fillColor] (131.64,142.60) circle (  1.96);

\path[draw=drawColor,line width= 0.4pt,line join=round,line cap=round,fill=fillColor] (131.64,147.44) circle (  1.96);

\path[draw=drawColor,line width= 0.4pt,line join=round,line cap=round,fill=fillColor] (131.64,143.36) circle (  1.96);

\path[draw=drawColor,line width= 0.4pt,line join=round,line cap=round,fill=fillColor] (131.64,153.56) circle (  1.96);

\path[draw=drawColor,line width= 0.4pt,line join=round,line cap=round,fill=fillColor] (131.64,142.54) circle (  1.96);

\path[draw=drawColor,line width= 0.4pt,line join=round,line cap=round,fill=fillColor] (131.64,143.36) circle (  1.96);

\path[draw=drawColor,line width= 0.4pt,line join=round,line cap=round,fill=fillColor] (131.64,150.16) circle (  1.96);

\path[draw=drawColor,line width= 0.4pt,line join=round,line cap=round,fill=fillColor] (131.64,139.28) circle (  1.96);

\path[draw=drawColor,line width= 0.4pt,line join=round,line cap=round,fill=fillColor] (131.64,139.28) circle (  1.96);

\path[draw=drawColor,line width= 0.4pt,line join=round,line cap=round,fill=fillColor] (131.64,147.44) circle (  1.96);

\path[draw=drawColor,line width= 0.4pt,line join=round,line cap=round,fill=fillColor] (131.64,141.66) circle (  1.96);

\path[draw=drawColor,line width= 0.4pt,line join=round,line cap=round,fill=fillColor] (131.64,150.16) circle (  1.96);

\path[draw=drawColor,line width= 0.4pt,line join=round,line cap=round,fill=fillColor] (131.64,143.36) circle (  1.96);

\path[draw=drawColor,line width= 0.4pt,line join=round,line cap=round,fill=fillColor] (131.64,143.36) circle (  1.96);

\path[draw=drawColor,line width= 0.4pt,line join=round,line cap=round,fill=fillColor] (131.64,147.44) circle (  1.96);

\path[draw=drawColor,line width= 0.4pt,line join=round,line cap=round,fill=fillColor] (131.64,153.96) circle (  1.96);

\path[draw=drawColor,line width= 0.4pt,line join=round,line cap=round,fill=fillColor] (131.64,139.28) circle (  1.96);

\path[draw=drawColor,line width= 0.4pt,line join=round,line cap=round,fill=fillColor] (131.64,141.32) circle (  1.96);

\path[draw=drawColor,line width= 0.4pt,line join=round,line cap=round,fill=fillColor] (131.64,145.91) circle (  1.96);

\path[draw=drawColor,line width= 0.4pt,line join=round,line cap=round,fill=fillColor] (131.64,143.36) circle (  1.96);

\path[draw=drawColor,line width= 0.4pt,line join=round,line cap=round,fill=fillColor] (131.64,150.16) circle (  1.96);

\path[draw=drawColor,line width= 0.4pt,line join=round,line cap=round,fill=fillColor] (131.64,153.56) circle (  1.96);

\path[draw=drawColor,line width= 0.4pt,line join=round,line cap=round,fill=fillColor] (131.64,150.16) circle (  1.96);

\path[draw=drawColor,line width= 0.4pt,line join=round,line cap=round,fill=fillColor] (131.64,140.64) circle (  1.96);

\path[draw=drawColor,line width= 0.4pt,line join=round,line cap=round,fill=fillColor] (131.64,142.00) circle (  1.96);

\path[draw=drawColor,line width= 0.4pt,line join=round,line cap=round,fill=fillColor] (131.64,141.66) circle (  1.96);

\path[draw=drawColor,line width= 0.4pt,line join=round,line cap=round,fill=fillColor] (131.64,141.66) circle (  1.96);

\path[draw=drawColor,line width= 0.4pt,line join=round,line cap=round,fill=fillColor] (131.64,145.63) circle (  1.96);

\path[draw=drawColor,line width= 0.4pt,line join=round,line cap=round,fill=fillColor] (131.64,151.86) circle (  1.96);

\path[draw=drawColor,line width= 0.4pt,line join=round,line cap=round,fill=fillColor] (131.64,138.83) circle (  1.96);

\path[draw=drawColor,line width= 0.4pt,line join=round,line cap=round,fill=fillColor] (131.64,153.56) circle (  1.96);

\path[draw=drawColor,line width= 0.4pt,line join=round,line cap=round,fill=fillColor] (131.64,143.36) circle (  1.96);

\path[draw=drawColor,line width= 0.4pt,line join=round,line cap=round,fill=fillColor] (131.64,139.28) circle (  1.96);

\path[draw=drawColor,line width= 0.4pt,line join=round,line cap=round,fill=fillColor] (131.64,150.16) circle (  1.96);

\path[draw=drawColor,line width= 0.4pt,line join=round,line cap=round,fill=fillColor] (131.64,146.13) circle (  1.96);

\path[draw=drawColor,line width= 0.4pt,line join=round,line cap=round,fill=fillColor] (131.64,150.16) circle (  1.96);

\path[draw=drawColor,line width= 0.4pt,line join=round,line cap=round,fill=fillColor] (131.64,147.44) circle (  1.96);

\path[draw=drawColor,line width= 0.4pt,line join=round,line cap=round,fill=fillColor] (131.64,145.63) circle (  1.96);

\path[draw=drawColor,line width= 0.4pt,line join=round,line cap=round,fill=fillColor] (131.64,146.76) circle (  1.96);

\path[draw=drawColor,line width= 0.4pt,line join=round,line cap=round,fill=fillColor] (131.64,147.44) circle (  1.96);

\path[draw=drawColor,line width= 0.4pt,line join=round,line cap=round,fill=fillColor] (131.64,150.16) circle (  1.96);

\path[draw=drawColor,line width= 0.4pt,line join=round,line cap=round,fill=fillColor] (131.64,150.16) circle (  1.96);

\path[draw=drawColor,line width= 0.4pt,line join=round,line cap=round,fill=fillColor] (131.64,140.45) circle (  1.96);

\path[draw=drawColor,line width= 0.4pt,line join=round,line cap=round,fill=fillColor] (131.64,139.96) circle (  1.96);

\path[draw=drawColor,line width= 0.4pt,line join=round,line cap=round,fill=fillColor] (131.64,140.91) circle (  1.96);

\path[draw=drawColor,line width= 0.4pt,line join=round,line cap=round,fill=fillColor] (131.64,142.00) circle (  1.96);

\path[draw=drawColor,line width= 0.4pt,line join=round,line cap=round,fill=fillColor] (131.64,140.50) circle (  1.96);

\path[draw=drawColor,line width= 0.4pt,line join=round,line cap=round,fill=fillColor] (131.64,150.16) circle (  1.96);

\path[draw=drawColor,line width= 0.4pt,line join=round,line cap=round,fill=fillColor] (131.64,141.09) circle (  1.96);

\path[draw=drawColor,line width= 0.4pt,line join=round,line cap=round,fill=fillColor] (131.64,145.63) circle (  1.96);

\path[draw=drawColor,line width= 0.4pt,line join=round,line cap=round,fill=fillColor] (131.64,139.28) circle (  1.96);

\path[draw=drawColor,line width= 0.4pt,line join=round,line cap=round,fill=fillColor] (131.64,147.44) circle (  1.96);

\path[draw=drawColor,line width= 0.4pt,line join=round,line cap=round,fill=fillColor] (131.64,144.72) circle (  1.96);

\path[draw=drawColor,line width= 0.4pt,line join=round,line cap=round,fill=fillColor] (131.64,150.16) circle (  1.96);

\path[draw=drawColor,line width= 0.4pt,line join=round,line cap=round,fill=fillColor] (131.64,145.63) circle (  1.96);

\path[draw=drawColor,line width= 0.4pt,line join=round,line cap=round,fill=fillColor] (131.64,138.26) circle (  1.96);

\path[draw=drawColor,line width= 0.4pt,line join=round,line cap=round,fill=fillColor] (131.64,143.36) circle (  1.96);

\path[draw=drawColor,line width= 0.4pt,line join=round,line cap=round,fill=fillColor] (131.64,141.66) circle (  1.96);

\path[draw=drawColor,line width= 0.4pt,line join=round,line cap=round,fill=fillColor] (131.64,151.52) circle (  1.96);

\path[draw=drawColor,line width= 0.4pt,line join=round,line cap=round,fill=fillColor] (131.64,151.86) circle (  1.96);

\path[draw=drawColor,line width= 0.4pt,line join=round,line cap=round,fill=fillColor] (131.64,143.36) circle (  1.96);

\path[draw=drawColor,line width= 0.4pt,line join=round,line cap=round,fill=fillColor] (131.64,145.63) circle (  1.96);

\path[draw=drawColor,line width= 0.4pt,line join=round,line cap=round,fill=fillColor] (131.64,139.28) circle (  1.96);

\path[draw=drawColor,line width= 0.4pt,line join=round,line cap=round,fill=fillColor] (131.64,151.52) circle (  1.96);

\path[draw=drawColor,line width= 0.4pt,line join=round,line cap=round,fill=fillColor] (131.64,144.18) circle (  1.96);

\path[draw=drawColor,line width= 0.4pt,line join=round,line cap=round,fill=fillColor] (131.64,143.36) circle (  1.96);

\path[draw=drawColor,line width= 0.4pt,line join=round,line cap=round,fill=fillColor] (131.64,146.76) circle (  1.96);

\path[draw=drawColor,line width= 0.4pt,line join=round,line cap=round,fill=fillColor] (131.64,143.36) circle (  1.96);

\path[draw=drawColor,line width= 0.4pt,line join=round,line cap=round,fill=fillColor] (131.64,143.36) circle (  1.96);

\path[draw=drawColor,line width= 0.4pt,line join=round,line cap=round,fill=fillColor] (131.64,140.91) circle (  1.96);

\path[draw=drawColor,line width= 0.4pt,line join=round,line cap=round,fill=fillColor] (131.64,140.91) circle (  1.96);

\path[draw=drawColor,line width= 0.4pt,line join=round,line cap=round,fill=fillColor] (131.64,144.18) circle (  1.96);

\path[draw=drawColor,line width= 0.4pt,line join=round,line cap=round,fill=fillColor] (131.64,147.44) circle (  1.96);

\path[draw=drawColor,line width= 0.4pt,line join=round,line cap=round,fill=fillColor] (131.64,147.44) circle (  1.96);

\path[draw=drawColor,line width= 0.4pt,line join=round,line cap=round,fill=fillColor] (131.64,139.28) circle (  1.96);

\path[draw=drawColor,line width= 0.4pt,line join=round,line cap=round,fill=fillColor] (131.64,143.36) circle (  1.96);

\path[draw=drawColor,line width= 0.4pt,line join=round,line cap=round,fill=fillColor] (131.64,140.91) circle (  1.96);

\path[draw=drawColor,line width= 0.4pt,line join=round,line cap=round,fill=fillColor] (131.64,145.74) circle (  1.96);

\path[draw=drawColor,line width= 0.4pt,line join=round,line cap=round,fill=fillColor] (131.64,145.74) circle (  1.96);

\path[draw=drawColor,line width= 0.4pt,line join=round,line cap=round,fill=fillColor] (131.64,139.96) circle (  1.96);

\path[draw=drawColor,line width= 0.4pt,line join=round,line cap=round,fill=fillColor] (131.64,139.96) circle (  1.96);

\path[draw=drawColor,line width= 0.4pt,line join=round,line cap=round,fill=fillColor] (131.64,141.77) circle (  1.96);

\path[draw=drawColor,line width= 0.4pt,line join=round,line cap=round,fill=fillColor] (131.64,141.77) circle (  1.96);

\path[draw=drawColor,line width= 0.4pt,line join=round,line cap=round,fill=fillColor] (131.64,147.19) circle (  1.96);

\path[draw=drawColor,line width= 0.4pt,line join=round,line cap=round,fill=fillColor] (131.64,147.19) circle (  1.96);

\path[draw=drawColor,line width= 0.4pt,line join=round,line cap=round,fill=fillColor] (131.64,142.37) circle (  1.96);

\path[draw=drawColor,line width= 0.4pt,line join=round,line cap=round,fill=fillColor] (131.64,142.37) circle (  1.96);

\path[draw=drawColor,line width= 0.4pt,line join=round,line cap=round,fill=fillColor] (131.64,154.41) circle (  1.96);

\path[draw=drawColor,line width= 0.4pt,line join=round,line cap=round,fill=fillColor] (131.64,154.41) circle (  1.96);

\path[draw=drawColor,line width= 0.4pt,line join=round,line cap=round,fill=fillColor] (131.64,147.19) circle (  1.96);

\path[draw=drawColor,line width= 0.4pt,line join=round,line cap=round,fill=fillColor] (131.64,154.41) circle (  1.96);

\path[draw=drawColor,line width= 0.4pt,line join=round,line cap=round,fill=fillColor] (131.64,154.41) circle (  1.96);

\path[draw=drawColor,line width= 0.4pt,line join=round,line cap=round,fill=fillColor] (131.64,147.19) circle (  1.96);

\path[draw=drawColor,line width= 0.4pt,line join=round,line cap=round,fill=fillColor] (131.64,147.19) circle (  1.96);

\path[draw=drawColor,line width= 0.4pt,line join=round,line cap=round,fill=fillColor] (131.64,147.19) circle (  1.96);

\path[draw=drawColor,line width= 0.4pt,line join=round,line cap=round,fill=fillColor] (131.64,139.96) circle (  1.96);

\path[draw=drawColor,line width= 0.4pt,line join=round,line cap=round,fill=fillColor] (131.64,139.96) circle (  1.96);

\path[draw=drawColor,line width= 0.4pt,line join=round,line cap=round,fill=fillColor] (131.64,147.19) circle (  1.96);

\path[draw=drawColor,line width= 0.4pt,line join=round,line cap=round,fill=fillColor] (131.64,147.19) circle (  1.96);

\path[draw=drawColor,line width= 0.4pt,line join=round,line cap=round,fill=fillColor] (131.64,139.96) circle (  1.96);

\path[draw=drawColor,line width= 0.4pt,line join=round,line cap=round,fill=fillColor] (131.64,139.96) circle (  1.96);

\path[draw=drawColor,line width= 0.4pt,line join=round,line cap=round,fill=fillColor] (131.64,145.74) circle (  1.96);

\path[draw=drawColor,line width= 0.4pt,line join=round,line cap=round,fill=fillColor] (131.64,145.74) circle (  1.96);

\path[draw=drawColor,line width= 0.4pt,line join=round,line cap=round,fill=fillColor] (131.64,144.09) circle (  1.96);

\path[draw=drawColor,line width= 0.4pt,line join=round,line cap=round,fill=fillColor] (131.64,139.96) circle (  1.96);

\path[draw=drawColor,line width= 0.4pt,line join=round,line cap=round,fill=fillColor] (131.64,150.80) circle (  1.96);

\path[draw=drawColor,line width= 0.4pt,line join=round,line cap=round,fill=fillColor] (131.64,151.52) circle (  1.96);

\path[draw=drawColor,line width= 0.4pt,line join=round,line cap=round,fill=fillColor] (131.64,142.85) circle (  1.96);

\path[draw=drawColor,line width= 0.4pt,line join=round,line cap=round,fill=fillColor] (131.64,138.52) circle (  1.96);

\path[draw=drawColor,line width= 0.4pt,line join=round,line cap=round,fill=fillColor] (131.64,150.22) circle (  1.96);

\path[draw=drawColor,line width= 0.4pt,line join=round,line cap=round,fill=fillColor] (131.64,148.63) circle (  1.96);

\path[draw=drawColor,line width= 0.4pt,line join=round,line cap=round,fill=fillColor] (131.64,151.52) circle (  1.96);

\path[draw=drawColor,line width= 0.4pt,line join=round,line cap=round,fill=fillColor] (131.64,154.41) circle (  1.96);

\path[draw=drawColor,line width= 0.4pt,line join=round,line cap=round,fill=fillColor] (131.64,143.58) circle (  1.96);

\path[draw=drawColor,line width= 0.4pt,line join=round,line cap=round,fill=fillColor] (131.64,139.96) circle (  1.96);

\path[draw=drawColor,line width= 0.4pt,line join=round,line cap=round,fill=fillColor] (131.64,151.52) circle (  1.96);

\path[draw=drawColor,line width= 0.4pt,line join=round,line cap=round,fill=fillColor] (131.64,139.96) circle (  1.96);

\path[draw=drawColor,line width= 0.4pt,line join=round,line cap=round,fill=fillColor] (131.64,139.96) circle (  1.96);

\path[draw=drawColor,line width= 0.4pt,line join=round,line cap=round,fill=fillColor] (131.64,144.09) circle (  1.96);

\path[draw=drawColor,line width= 0.4pt,line join=round,line cap=round,fill=fillColor] (131.64,142.85) circle (  1.96);

\path[draw=drawColor,line width= 0.4pt,line join=round,line cap=round,fill=fillColor] (131.64,147.19) circle (  1.96);

\path[draw=drawColor,line width= 0.4pt,line join=round,line cap=round,fill=fillColor] (131.64,139.20) circle (  1.96);

\path[draw=drawColor,line width= 0.4pt,line join=round,line cap=round,fill=fillColor] (131.64,141.28) circle (  1.96);

\path[draw=drawColor,line width= 0.4pt,line join=round,line cap=round,fill=fillColor] (131.64,142.85) circle (  1.96);

\path[draw=drawColor,line width= 0.4pt,line join=round,line cap=round,fill=fillColor] (131.64,139.96) circle (  1.96);

\path[draw=drawColor,line width= 0.4pt,line join=round,line cap=round,fill=fillColor] (131.64,144.09) circle (  1.96);

\path[draw=drawColor,line width= 0.4pt,line join=round,line cap=round,fill=fillColor] (131.64,147.19) circle (  1.96);

\path[draw=drawColor,line width= 0.4pt,line join=round,line cap=round,fill=fillColor] (131.64,139.96) circle (  1.96);

\path[draw=drawColor,line width= 0.4pt,line join=round,line cap=round,fill=fillColor] (131.64,154.41) circle (  1.96);

\path[draw=drawColor,line width= 0.4pt,line join=round,line cap=round,fill=fillColor] (131.64,147.19) circle (  1.96);

\path[draw=drawColor,line width= 0.4pt,line join=round,line cap=round,fill=fillColor] (131.64,142.85) circle (  1.96);

\path[draw=drawColor,line width= 0.4pt,line join=round,line cap=round,fill=fillColor] (131.64,139.96) circle (  1.96);

\path[draw=drawColor,line width= 0.4pt,line join=round,line cap=round,fill=fillColor] (131.64,139.96) circle (  1.96);

\path[draw=drawColor,line width= 0.4pt,line join=round,line cap=round,fill=fillColor] (131.64,149.05) circle (  1.96);

\path[draw=drawColor,line width= 0.4pt,line join=round,line cap=round,fill=fillColor] (131.64,141.77) circle (  1.96);

\path[draw=drawColor,line width= 0.4pt,line join=round,line cap=round,fill=fillColor] (131.64,148.63) circle (  1.96);

\path[draw=drawColor,line width= 0.4pt,line join=round,line cap=round,fill=fillColor] (131.64,139.96) circle (  1.96);

\path[draw=drawColor,line width= 0.4pt,line join=round,line cap=round,fill=fillColor] (131.64,142.37) circle (  1.96);

\path[draw=drawColor,line width= 0.4pt,line join=round,line cap=round,fill=fillColor] (131.64,139.96) circle (  1.96);

\path[draw=drawColor,line width= 0.4pt,line join=round,line cap=round,fill=fillColor] (131.64,141.12) circle (  1.96);

\path[draw=drawColor,line width= 0.4pt,line join=round,line cap=round,fill=fillColor] (131.64,151.52) circle (  1.96);

\path[draw=drawColor,line width= 0.4pt,line join=round,line cap=round,fill=fillColor] (131.64,153.69) circle (  1.96);

\path[draw=drawColor,line width= 0.4pt,line join=round,line cap=round,fill=fillColor] (131.64,148.05) circle (  1.96);

\path[draw=drawColor,line width= 0.4pt,line join=round,line cap=round,fill=fillColor] (131.64,148.63) circle (  1.96);

\path[draw=drawColor,line width= 0.4pt,line join=round,line cap=round,fill=fillColor] (131.64,148.63) circle (  1.96);

\path[draw=drawColor,line width= 0.4pt,line join=round,line cap=round,fill=fillColor] (131.64,147.19) circle (  1.96);

\path[draw=drawColor,line width= 0.4pt,line join=round,line cap=round,fill=fillColor] (131.64,154.41) circle (  1.96);

\path[draw=drawColor,line width= 0.4pt,line join=round,line cap=round,fill=fillColor] (131.64,144.09) circle (  1.96);

\path[draw=drawColor,line width= 0.4pt,line join=round,line cap=round,fill=fillColor] (131.64,145.74) circle (  1.96);

\path[draw=drawColor,line width= 0.4pt,line join=round,line cap=round,fill=fillColor] (131.64,141.12) circle (  1.96);

\path[draw=drawColor,line width= 0.4pt,line join=round,line cap=round,fill=fillColor] (131.64,139.96) circle (  1.96);

\path[draw=drawColor,line width= 0.4pt,line join=round,line cap=round,fill=fillColor] (131.64,140.77) circle (  1.96);

\path[draw=drawColor,line width= 0.4pt,line join=round,line cap=round,fill=fillColor] (131.64,140.77) circle (  1.96);

\path[draw=drawColor,line width= 0.4pt,line join=round,line cap=round,fill=fillColor] (131.64,147.19) circle (  1.96);

\path[draw=drawColor,line width= 0.4pt,line join=round,line cap=round,fill=fillColor] (131.64,147.19) circle (  1.96);

\path[draw=drawColor,line width= 0.4pt,line join=round,line cap=round,fill=fillColor] (131.64,142.85) circle (  1.96);

\path[draw=drawColor,line width= 0.4pt,line join=round,line cap=round,fill=fillColor] (131.64,139.82) circle (  1.96);

\path[draw=drawColor,line width= 0.4pt,line join=round,line cap=round,fill=fillColor] (131.64,147.19) circle (  1.96);

\path[draw=drawColor,line width= 0.4pt,line join=round,line cap=round,fill=fillColor] (131.64,145.69) circle (  1.96);

\path[draw=drawColor,line width= 0.4pt,line join=round,line cap=round,fill=fillColor] (131.64,139.60) circle (  1.96);

\path[draw=drawColor,line width= 0.4pt,line join=round,line cap=round,fill=fillColor] (131.64,150.80) circle (  1.96);

\path[draw=drawColor,line width= 0.4pt,line join=round,line cap=round,fill=fillColor] (131.64,139.96) circle (  1.96);

\path[draw=drawColor,line width= 0.4pt,line join=round,line cap=round,fill=fillColor] (131.64,139.96) circle (  1.96);

\path[draw=drawColor,line width= 0.4pt,line join=round,line cap=round,fill=fillColor] (131.64,144.59) circle (  1.96);

\path[draw=drawColor,line width= 0.4pt,line join=round,line cap=round,fill=fillColor] (131.64,154.41) circle (  1.96);

\path[draw=drawColor,line width= 0.4pt,line join=round,line cap=round,fill=fillColor] (131.64,154.41) circle (  1.96);

\path[draw=drawColor,line width= 0.4pt,line join=round,line cap=round,fill=fillColor] (131.64,154.41) circle (  1.96);

\path[draw=drawColor,line width= 0.4pt,line join=round,line cap=round,fill=fillColor] (131.64,139.14) circle (  1.96);

\path[draw=drawColor,line width= 0.4pt,line join=round,line cap=round,fill=fillColor] (131.64,139.14) circle (  1.96);

\path[draw=drawColor,line width= 0.4pt,line join=round,line cap=round,fill=fillColor] (131.64,139.96) circle (  1.96);

\path[draw=drawColor,line width= 0.4pt,line join=round,line cap=round,fill=fillColor] (131.64,139.96) circle (  1.96);

\path[draw=drawColor,line width= 0.4pt,line join=round,line cap=round,fill=fillColor] (131.64,145.91) circle (  1.96);

\path[draw=drawColor,line width= 0.4pt,line join=round,line cap=round,fill=fillColor] (131.64,138.52) circle (  1.96);

\path[draw=drawColor,line width= 0.4pt,line join=round,line cap=round,fill=fillColor] (131.64,154.41) circle (  1.96);

\path[draw=drawColor,line width= 0.4pt,line join=round,line cap=round,fill=fillColor] (131.64,154.41) circle (  1.96);

\path[draw=drawColor,line width= 0.4pt,line join=round,line cap=round,fill=fillColor] (131.64,142.37) circle (  1.96);

\path[draw=drawColor,line width= 0.4pt,line join=round,line cap=round,fill=fillColor] (131.64,142.37) circle (  1.96);

\path[draw=drawColor,line width= 0.4pt,line join=round,line cap=round,fill=fillColor] (131.64,145.38) circle (  1.96);

\path[draw=drawColor,line width= 0.4pt,line join=round,line cap=round,fill=fillColor] (131.64,145.38) circle (  1.96);

\path[draw=drawColor,line width= 0.4pt,line join=round,line cap=round,fill=fillColor] (131.64,141.77) circle (  1.96);

\path[draw=drawColor,line width= 0.4pt,line join=round,line cap=round,fill=fillColor] (131.64,142.85) circle (  1.96);

\path[draw=drawColor,line width= 0.4pt,line join=round,line cap=round,fill=fillColor] (131.64,142.85) circle (  1.96);

\path[draw=drawColor,line width= 0.4pt,line join=round,line cap=round,fill=fillColor] (131.64,139.96) circle (  1.96);

\path[draw=drawColor,line width= 0.4pt,line join=round,line cap=round,fill=fillColor] (131.64,152.61) circle (  1.96);

\path[draw=drawColor,line width= 0.4pt,line join=round,line cap=round,fill=fillColor] (131.64,147.19) circle (  1.96);

\path[draw=drawColor,line width= 0.4pt,line join=round,line cap=round,fill=fillColor] (131.64,151.52) circle (  1.96);

\path[draw=drawColor,line width= 0.4pt,line join=round,line cap=round,fill=fillColor] (131.64,141.12) circle (  1.96);

\path[draw=drawColor,line width= 0.4pt,line join=round,line cap=round,fill=fillColor] (131.64,151.52) circle (  1.96);

\path[draw=drawColor,line width= 0.4pt,line join=round,line cap=round,fill=fillColor] (131.64,139.96) circle (  1.96);

\path[draw=drawColor,line width= 0.4pt,line join=round,line cap=round,fill=fillColor] (131.64,139.96) circle (  1.96);

\path[draw=drawColor,line width= 0.4pt,line join=round,line cap=round,fill=fillColor] (131.64,144.09) circle (  1.96);

\path[draw=drawColor,line width= 0.4pt,line join=round,line cap=round,fill=fillColor] (131.64,139.77) circle (  1.96);

\path[draw=drawColor,line width= 0.4pt,line join=round,line cap=round,fill=fillColor] (131.64,139.77) circle (  1.96);

\path[draw=drawColor,line width= 0.4pt,line join=round,line cap=round,fill=fillColor] (131.64,142.85) circle (  1.96);

\path[draw=drawColor,line width= 0.4pt,line join=round,line cap=round,fill=fillColor] (131.64,142.85) circle (  1.96);

\path[draw=drawColor,line width= 0.4pt,line join=round,line cap=round,fill=fillColor] (131.64,144.59) circle (  1.96);

\path[draw=drawColor,line width= 0.4pt,line join=round,line cap=round,fill=fillColor] (131.64,139.96) circle (  1.96);

\path[draw=drawColor,line width= 0.4pt,line join=round,line cap=round,fill=fillColor] (131.64,151.52) circle (  1.96);

\path[draw=drawColor,line width= 0.4pt,line join=round,line cap=round,fill=fillColor] (131.64,151.52) circle (  1.96);

\path[draw=drawColor,line width= 0.4pt,line join=round,line cap=round,fill=fillColor] (131.64,151.52) circle (  1.96);

\path[draw=drawColor,line width= 0.4pt,line join=round,line cap=round,fill=fillColor] (131.64,154.41) circle (  1.96);

\path[draw=drawColor,line width= 0.4pt,line join=round,line cap=round,fill=fillColor] (131.64,154.41) circle (  1.96);

\path[draw=drawColor,line width= 0.4pt,line join=round,line cap=round,fill=fillColor] (131.64,141.12) circle (  1.96);

\path[draw=drawColor,line width= 0.4pt,line join=round,line cap=round,fill=fillColor] (131.64,143.58) circle (  1.96);

\path[draw=drawColor,line width= 0.4pt,line join=round,line cap=round,fill=fillColor] (131.64,139.96) circle (  1.96);

\path[draw=drawColor,line width= 0.4pt,line join=round,line cap=round,fill=fillColor] (131.64,140.81) circle (  1.96);

\path[draw=drawColor,line width= 0.4pt,line join=round,line cap=round,fill=fillColor] (131.64,138.52) circle (  1.96);

\path[draw=drawColor,line width= 0.4pt,line join=round,line cap=round,fill=fillColor] (131.64,143.58) circle (  1.96);

\path[draw=drawColor,line width= 0.4pt,line join=round,line cap=round,fill=fillColor] (131.64,144.59) circle (  1.96);

\path[draw=drawColor,line width= 0.4pt,line join=round,line cap=round,fill=fillColor] (131.64,145.74) circle (  1.96);

\path[draw=drawColor,line width= 0.4pt,line join=round,line cap=round,fill=fillColor] (131.64,147.19) circle (  1.96);

\path[draw=drawColor,line width= 0.4pt,line join=round,line cap=round,fill=fillColor] (131.64,139.96) circle (  1.96);

\path[draw=drawColor,line width= 0.4pt,line join=round,line cap=round,fill=fillColor] (131.64,150.80) circle (  1.96);

\path[draw=drawColor,line width= 0.4pt,line join=round,line cap=round,fill=fillColor] (131.64,145.38) circle (  1.96);

\path[draw=drawColor,line width= 0.4pt,line join=round,line cap=round,fill=fillColor] (131.64,148.05) circle (  1.96);

\path[draw=drawColor,line width= 0.4pt,line join=round,line cap=round,fill=fillColor] (131.64,148.05) circle (  1.96);

\path[draw=drawColor,line width= 0.4pt,line join=round,line cap=round,fill=fillColor] (131.64,150.56) circle (  1.96);

\path[draw=drawColor,line width= 0.4pt,line join=round,line cap=round,fill=fillColor] (131.64,139.96) circle (  1.96);

\path[draw=drawColor,line width= 0.4pt,line join=round,line cap=round,fill=fillColor] (131.64,145.74) circle (  1.96);

\path[draw=drawColor,line width= 0.4pt,line join=round,line cap=round,fill=fillColor] (131.64,154.41) circle (  1.96);

\path[draw=drawColor,line width= 0.4pt,line join=round,line cap=round,fill=fillColor] (131.64,147.19) circle (  1.96);

\path[draw=drawColor,line width= 0.4pt,line join=round,line cap=round,fill=fillColor] (131.64,139.96) circle (  1.96);

\path[draw=drawColor,line width= 0.4pt,line join=round,line cap=round,fill=fillColor] (131.64,142.85) circle (  1.96);

\path[draw=drawColor,line width= 0.4pt,line join=round,line cap=round,fill=fillColor] (131.64,144.09) circle (  1.96);

\path[draw=drawColor,line width= 0.4pt,line join=round,line cap=round,fill=fillColor] (131.64,151.52) circle (  1.96);

\path[draw=drawColor,line width= 0.4pt,line join=round,line cap=round,fill=fillColor] (131.64,144.59) circle (  1.96);

\path[draw=drawColor,line width= 0.4pt,line join=round,line cap=round,fill=fillColor] (131.64,145.61) circle (  1.96);

\path[draw=drawColor,line width= 0.4pt,line join=round,line cap=round,fill=fillColor] (131.64,140.70) circle (  1.96);

\path[draw=drawColor,line width= 0.4pt,line join=round,line cap=round,fill=fillColor] (131.64,140.18) circle (  1.96);

\path[draw=drawColor,line width= 0.4pt,line join=round,line cap=round,fill=fillColor] (131.64,150.14) circle (  1.96);

\path[draw=drawColor,line width= 0.4pt,line join=round,line cap=round,fill=fillColor] (131.64,154.67) circle (  1.96);

\path[draw=drawColor,line width= 0.4pt,line join=round,line cap=round,fill=fillColor] (131.64,138.82) circle (  1.96);

\path[draw=drawColor,line width= 0.4pt,line join=round,line cap=round,fill=fillColor] (131.64,139.57) circle (  1.96);

\path[draw=drawColor,line width= 0.4pt,line join=round,line cap=round,fill=fillColor] (131.64,150.14) circle (  1.96);

\path[draw=drawColor,line width= 0.4pt,line join=round,line cap=round,fill=fillColor] (131.64,139.57) circle (  1.96);

\path[draw=drawColor,line width= 0.4pt,line join=round,line cap=round,fill=fillColor] (131.64,138.82) circle (  1.96);

\path[draw=drawColor,line width= 0.4pt,line join=round,line cap=round,fill=fillColor] (131.64,150.14) circle (  1.96);

\path[draw=drawColor,line width= 0.4pt,line join=round,line cap=round,fill=fillColor] (131.64,141.73) circle (  1.96);

\path[draw=drawColor,line width= 0.4pt,line join=round,line cap=round,fill=fillColor] (131.64,142.59) circle (  1.96);

\path[draw=drawColor,line width= 0.4pt,line join=round,line cap=round,fill=fillColor] (131.64,145.61) circle (  1.96);

\path[draw=drawColor,line width= 0.4pt,line join=round,line cap=round,fill=fillColor] (131.64,151.05) circle (  1.96);

\path[draw=drawColor,line width= 0.4pt,line join=round,line cap=round,fill=fillColor] (131.64,142.59) circle (  1.96);

\path[draw=drawColor,line width= 0.4pt,line join=round,line cap=round,fill=fillColor] (131.64,143.80) circle (  1.96);

\path[draw=drawColor,line width= 0.4pt,line join=round,line cap=round,fill=fillColor] (131.64,147.42) circle (  1.96);

\path[draw=drawColor,line width= 0.4pt,line join=round,line cap=round,fill=fillColor] (131.64,140.07) circle (  1.96);

\path[draw=drawColor,line width= 0.4pt,line join=round,line cap=round,fill=fillColor] (131.64,151.65) circle (  1.96);

\path[draw=drawColor,line width= 0.4pt,line join=round,line cap=round,fill=fillColor] (131.64,142.59) circle (  1.96);

\path[draw=drawColor,line width= 0.4pt,line join=round,line cap=round,fill=fillColor] (131.64,145.61) circle (  1.96);

\path[draw=drawColor,line width= 0.4pt,line join=round,line cap=round,fill=fillColor] (131.64,138.82) circle (  1.96);

\path[draw=drawColor,line width= 0.4pt,line join=round,line cap=round,fill=fillColor] (131.64,138.82) circle (  1.96);

\path[draw=drawColor,line width= 0.4pt,line join=round,line cap=round,fill=fillColor] (131.64,140.18) circle (  1.96);

\path[draw=drawColor,line width= 0.4pt,line join=round,line cap=round,fill=fillColor] (131.64,140.18) circle (  1.96);

\path[draw=drawColor,line width= 0.4pt,line join=round,line cap=round,fill=fillColor] (131.64,147.42) circle (  1.96);

\path[draw=drawColor,line width= 0.4pt,line join=round,line cap=round,fill=fillColor] (131.64,138.82) circle (  1.96);

\path[draw=drawColor,line width= 0.4pt,line join=round,line cap=round,fill=fillColor] (131.64,138.82) circle (  1.96);

\path[draw=drawColor,line width= 0.4pt,line join=round,line cap=round,fill=fillColor] (131.64,138.82) circle (  1.96);

\path[draw=drawColor,line width= 0.4pt,line join=round,line cap=round,fill=fillColor] (131.64,138.82) circle (  1.96);

\path[draw=drawColor,line width= 0.4pt,line join=round,line cap=round,fill=fillColor] (131.64,144.16) circle (  1.96);

\path[draw=drawColor,line width= 0.4pt,line join=round,line cap=round,fill=fillColor] (131.64,147.06) circle (  1.96);

\path[draw=drawColor,line width= 0.4pt,line join=round,line cap=round,fill=fillColor] (131.64,147.42) circle (  1.96);

\path[draw=drawColor,line width= 0.4pt,line join=round,line cap=round,fill=fillColor] (131.64,139.18) circle (  1.96);

\path[draw=drawColor,line width= 0.4pt,line join=round,line cap=round,fill=fillColor] (131.64,138.82) circle (  1.96);

\path[draw=drawColor,line width= 0.4pt,line join=round,line cap=round,fill=fillColor] (131.64,151.65) circle (  1.96);

\path[draw=drawColor,line width= 0.4pt,line join=round,line cap=round,fill=fillColor] (131.64,151.65) circle (  1.96);

\path[draw=drawColor,line width= 0.4pt,line join=round,line cap=round,fill=fillColor] (131.64,141.08) circle (  1.96);

\path[draw=drawColor,line width= 0.4pt,line join=round,line cap=round,fill=fillColor] (131.64,141.08) circle (  1.96);

\path[draw=drawColor,line width= 0.4pt,line join=round,line cap=round,fill=fillColor] (131.64,138.82) circle (  1.96);

\path[draw=drawColor,line width= 0.4pt,line join=round,line cap=round,fill=fillColor] (131.64,138.82) circle (  1.96);

\path[draw=drawColor,line width= 0.4pt,line join=round,line cap=round,fill=fillColor] (131.64,138.82) circle (  1.96);

\path[draw=drawColor,line width= 0.4pt,line join=round,line cap=round,fill=fillColor] (131.64,138.82) circle (  1.96);

\path[draw=drawColor,line width= 0.4pt,line join=round,line cap=round,fill=fillColor] (131.64,140.18) circle (  1.96);

\path[draw=drawColor,line width= 0.4pt,line join=round,line cap=round,fill=fillColor] (131.64,142.59) circle (  1.96);

\path[draw=drawColor,line width= 0.4pt,line join=round,line cap=round,fill=fillColor] (131.64,150.14) circle (  1.96);

\path[draw=drawColor,line width= 0.4pt,line join=round,line cap=round,fill=fillColor] (131.64,146.91) circle (  1.96);

\path[draw=drawColor,line width= 0.4pt,line join=round,line cap=round,fill=fillColor] (131.64,142.59) circle (  1.96);

\path[draw=drawColor,line width= 0.4pt,line join=round,line cap=round,fill=fillColor] (131.64,145.61) circle (  1.96);

\path[draw=drawColor,line width= 0.4pt,line join=round,line cap=round,fill=fillColor] (131.64,146.15) circle (  1.96);

\path[draw=drawColor,line width= 0.4pt,line join=round,line cap=round,fill=fillColor] (131.64,141.08) circle (  1.96);

\path[draw=drawColor,line width= 0.4pt,line join=round,line cap=round,fill=fillColor] (131.64,150.14) circle (  1.96);

\path[draw=drawColor,line width= 0.4pt,line join=round,line cap=round,fill=fillColor] (131.64,142.59) circle (  1.96);

\path[draw=drawColor,line width= 0.4pt,line join=round,line cap=round,fill=fillColor] (131.64,142.59) circle (  1.96);

\path[draw=drawColor,line width= 0.4pt,line join=round,line cap=round,fill=fillColor] (131.64,148.26) circle (  1.96);

\path[draw=drawColor,line width= 0.4pt,line join=round,line cap=round,fill=fillColor] (131.64,148.26) circle (  1.96);

\path[draw=drawColor,line width= 0.4pt,line join=round,line cap=round,fill=fillColor] (131.64,138.82) circle (  1.96);

\path[draw=drawColor,line width= 0.4pt,line join=round,line cap=round,fill=fillColor] (131.64,151.65) circle (  1.96);

\path[draw=drawColor,line width= 0.4pt,line join=round,line cap=round,fill=fillColor] (131.64,142.59) circle (  1.96);

\path[draw=drawColor,line width= 0.4pt,line join=round,line cap=round,fill=fillColor] (131.64,140.91) circle (  1.96);

\path[draw=drawColor,line width= 0.4pt,line join=round,line cap=round,fill=fillColor] (131.64,143.35) circle (  1.96);

\path[draw=drawColor,line width= 0.4pt,line join=round,line cap=round,fill=fillColor] (131.64,142.59) circle (  1.96);

\path[draw=drawColor,line width= 0.4pt,line join=round,line cap=round,fill=fillColor] (131.64,149.50) circle (  1.96);

\path[draw=drawColor,line width= 0.4pt,line join=round,line cap=round,fill=fillColor] (131.64,146.91) circle (  1.96);

\path[draw=drawColor,line width= 0.4pt,line join=round,line cap=round,fill=fillColor] (131.64,145.61) circle (  1.96);

\path[draw=drawColor,line width= 0.4pt,line join=round,line cap=round,fill=fillColor] (131.64,145.61) circle (  1.96);

\path[draw=drawColor,line width= 0.4pt,line join=round,line cap=round,fill=fillColor] (131.64,150.14) circle (  1.96);

\path[draw=drawColor,line width= 0.4pt,line join=round,line cap=round,fill=fillColor] (131.64,150.14) circle (  1.96);

\path[draw=drawColor,line width= 0.4pt,line join=round,line cap=round,fill=fillColor] (131.64,154.67) circle (  1.96);

\path[draw=drawColor,line width= 0.4pt,line join=round,line cap=round,fill=fillColor] (131.64,154.67) circle (  1.96);

\path[draw=drawColor,line width= 0.4pt,line join=round,line cap=round,fill=fillColor] (131.64,138.82) circle (  1.96);

\path[draw=drawColor,line width= 0.4pt,line join=round,line cap=round,fill=fillColor] (131.64,141.65) circle (  1.96);

\path[draw=drawColor,line width= 0.4pt,line join=round,line cap=round,fill=fillColor] (131.64,141.65) circle (  1.96);

\path[draw=drawColor,line width= 0.4pt,line join=round,line cap=round,fill=fillColor] (131.64,141.65) circle (  1.96);

\path[draw=drawColor,line width= 0.4pt,line join=round,line cap=round,fill=fillColor] (131.64,141.65) circle (  1.96);

\path[draw=drawColor,line width= 0.4pt,line join=round,line cap=round,fill=fillColor] (131.64,138.82) circle (  1.96);

\path[draw=drawColor,line width= 0.4pt,line join=round,line cap=round,fill=fillColor] (131.64,150.14) circle (  1.96);

\path[draw=drawColor,line width= 0.4pt,line join=round,line cap=round,fill=fillColor] (131.64,150.14) circle (  1.96);

\path[draw=drawColor,line width= 0.4pt,line join=round,line cap=round,fill=fillColor] (131.64,138.82) circle (  1.96);

\path[draw=drawColor,line width= 0.4pt,line join=round,line cap=round,fill=fillColor] (131.64,138.82) circle (  1.96);

\path[draw=drawColor,line width= 0.4pt,line join=round,line cap=round,fill=fillColor] (131.64,138.82) circle (  1.96);

\path[draw=drawColor,line width= 0.4pt,line join=round,line cap=round,fill=fillColor] (131.64,142.59) circle (  1.96);

\path[draw=drawColor,line width= 0.4pt,line join=round,line cap=round,fill=fillColor] (131.64,150.14) circle (  1.96);

\path[draw=drawColor,line width= 0.4pt,line join=round,line cap=round,fill=fillColor] (131.64,143.80) circle (  1.96);

\path[draw=drawColor,line width= 0.4pt,line join=round,line cap=round,fill=fillColor] (131.64,148.63) circle (  1.96);

\path[draw=drawColor,line width= 0.4pt,line join=round,line cap=round,fill=fillColor] (131.64,150.14) circle (  1.96);

\path[draw=drawColor,line width= 0.4pt,line join=round,line cap=round,fill=fillColor] (131.64,151.05) circle (  1.96);

\path[draw=drawColor,line width= 0.4pt,line join=round,line cap=round,fill=fillColor] (131.64,138.82) circle (  1.96);

\path[draw=drawColor,line width= 0.4pt,line join=round,line cap=round,fill=fillColor] (131.64,141.08) circle (  1.96);

\path[draw=drawColor,line width= 0.4pt,line join=round,line cap=round,fill=fillColor] (131.64,147.42) circle (  1.96);

\path[draw=drawColor,line width= 0.4pt,line join=round,line cap=round,fill=fillColor] (131.64,139.57) circle (  1.96);

\path[draw=drawColor,line width= 0.4pt,line join=round,line cap=round,fill=fillColor] (131.64,142.59) circle (  1.96);

\path[draw=drawColor,line width= 0.4pt,line join=round,line cap=round,fill=fillColor] (131.64,138.82) circle (  1.96);

\path[draw=drawColor,line width= 0.4pt,line join=round,line cap=round,fill=fillColor] (131.64,140.18) circle (  1.96);

\path[draw=drawColor,line width= 0.4pt,line join=round,line cap=round,fill=fillColor] (131.64,147.06) circle (  1.96);

\path[draw=drawColor,line width= 0.4pt,line join=round,line cap=round,fill=fillColor] (131.64,150.14) circle (  1.96);

\path[draw=drawColor,line width= 0.4pt,line join=round,line cap=round,fill=fillColor] (131.64,142.59) circle (  1.96);

\path[draw=drawColor,line width= 0.4pt,line join=round,line cap=round,fill=fillColor] (131.64,142.59) circle (  1.96);

\path[draw=drawColor,line width= 0.4pt,line join=round,line cap=round,fill=fillColor] (131.64,148.63) circle (  1.96);

\path[draw=drawColor,line width= 0.4pt,line join=round,line cap=round,fill=fillColor] (131.64,138.82) circle (  1.96);

\path[draw=drawColor,line width= 0.4pt,line join=round,line cap=round,fill=fillColor] (131.64,146.62) circle (  1.96);

\path[draw=drawColor,line width= 0.4pt,line join=round,line cap=round,fill=fillColor] (131.64,138.82) circle (  1.96);

\path[draw=drawColor,line width= 0.4pt,line join=round,line cap=round,fill=fillColor] (131.64,147.42) circle (  1.96);

\path[draw=drawColor,line width= 0.4pt,line join=round,line cap=round,fill=fillColor] (131.64,143.80) circle (  1.96);

\path[draw=drawColor,line width= 0.4pt,line join=round,line cap=round,fill=fillColor] (131.64,142.59) circle (  1.96);

\path[draw=drawColor,line width= 0.4pt,line join=round,line cap=round,fill=fillColor] (131.64,142.59) circle (  1.96);

\path[draw=drawColor,line width= 0.4pt,line join=round,line cap=round,fill=fillColor] (131.64,152.66) circle (  1.96);

\path[draw=drawColor,line width= 0.4pt,line join=round,line cap=round,fill=fillColor] (131.64,152.66) circle (  1.96);

\path[draw=drawColor,line width= 0.4pt,line join=round,line cap=round,fill=fillColor] (131.64,138.82) circle (  1.96);

\path[draw=drawColor,line width= 0.4pt,line join=round,line cap=round,fill=fillColor] (131.64,150.14) circle (  1.96);

\path[draw=drawColor,line width= 0.4pt,line join=round,line cap=round,fill=fillColor] (131.64,141.08) circle (  1.96);

\path[draw=drawColor,line width= 0.4pt,line join=round,line cap=round,fill=fillColor] (131.64,150.14) circle (  1.96);

\path[draw=drawColor,line width= 0.4pt,line join=round,line cap=round,fill=fillColor] (131.64,154.67) circle (  1.96);

\path[draw=drawColor,line width= 0.4pt,line join=round,line cap=round,fill=fillColor] (131.64,154.67) circle (  1.96);

\path[draw=drawColor,line width= 0.4pt,line join=round,line cap=round,fill=fillColor] (131.64,142.59) circle (  1.96);

\path[draw=drawColor,line width= 0.4pt,line join=round,line cap=round,fill=fillColor] (131.64,142.59) circle (  1.96);

\path[draw=drawColor,line width= 0.4pt,line join=round,line cap=round,fill=fillColor] (131.64,152.98) circle (  1.96);

\path[draw=drawColor,line width= 0.4pt,line join=round,line cap=round,fill=fillColor] (131.64,146.15) circle (  1.96);

\path[draw=drawColor,line width= 0.4pt,line join=round,line cap=round,fill=fillColor] (131.64,154.26) circle (  1.96);

\path[draw=drawColor,line width= 0.4pt,line join=round,line cap=round,fill=fillColor] (131.64,154.26) circle (  1.96);

\path[draw=drawColor,line width= 0.4pt,line join=round,line cap=round,fill=fillColor] (131.64,143.49) circle (  1.96);

\path[draw=drawColor,line width= 0.4pt,line join=round,line cap=round,fill=fillColor] (131.64,143.49) circle (  1.96);

\path[draw=drawColor,line width= 0.4pt,line join=round,line cap=round,fill=fillColor] (131.64,152.92) circle (  1.96);

\path[draw=drawColor,line width= 0.4pt,line join=round,line cap=round,fill=fillColor] (131.64,152.92) circle (  1.96);

\path[draw=drawColor,line width= 0.4pt,line join=round,line cap=round,fill=fillColor] (131.64,145.06) circle (  1.96);

\path[draw=drawColor,line width= 0.4pt,line join=round,line cap=round,fill=fillColor] (131.64,145.06) circle (  1.96);

\path[draw=drawColor,line width= 0.4pt,line join=round,line cap=round,fill=fillColor] (131.64,145.06) circle (  1.96);

\path[draw=drawColor,line width= 0.4pt,line join=round,line cap=round,fill=fillColor] (131.64,141.13) circle (  1.96);

\path[draw=drawColor,line width= 0.4pt,line join=round,line cap=round,fill=fillColor] (131.64,141.13) circle (  1.96);

\path[draw=drawColor,line width= 0.4pt,line join=round,line cap=round,fill=fillColor] (131.64,141.13) circle (  1.96);

\path[draw=drawColor,line width= 0.4pt,line join=round,line cap=round,fill=fillColor] (131.64,141.13) circle (  1.96);

\path[draw=drawColor,line width= 0.4pt,line join=round,line cap=round,fill=fillColor] (131.64,138.77) circle (  1.96);

\path[draw=drawColor,line width= 0.4pt,line join=round,line cap=round,fill=fillColor] (131.64,138.77) circle (  1.96);

\path[draw=drawColor,line width= 0.4pt,line join=round,line cap=round,fill=fillColor] (131.64,141.13) circle (  1.96);

\path[draw=drawColor,line width= 0.4pt,line join=round,line cap=round,fill=fillColor] (131.64,145.06) circle (  1.96);

\path[draw=drawColor,line width= 0.4pt,line join=round,line cap=round,fill=fillColor] (131.64,149.55) circle (  1.96);

\path[draw=drawColor,line width= 0.4pt,line join=round,line cap=round,fill=fillColor] (131.64,142.81) circle (  1.96);

\path[draw=drawColor,line width= 0.4pt,line join=round,line cap=round,fill=fillColor] (131.64,142.81) circle (  1.96);

\path[draw=drawColor,line width= 0.4pt,line join=round,line cap=round,fill=fillColor] (131.64,145.06) circle (  1.96);

\path[draw=drawColor,line width= 0.4pt,line join=round,line cap=round,fill=fillColor] (131.64,139.44) circle (  1.96);

\path[draw=drawColor,line width= 0.4pt,line join=round,line cap=round,fill=fillColor] (131.64,145.06) circle (  1.96);

\path[draw=drawColor,line width= 0.4pt,line join=round,line cap=round,fill=fillColor] (131.64,145.06) circle (  1.96);

\path[draw=drawColor,line width= 0.4pt,line join=round,line cap=round,fill=fillColor] (131.64,145.06) circle (  1.96);

\path[draw=drawColor,line width= 0.4pt,line join=round,line cap=round,fill=fillColor] (131.64,148.99) circle (  1.96);

\path[draw=drawColor,line width= 0.4pt,line join=round,line cap=round,fill=fillColor] (131.64,148.99) circle (  1.96);

\path[draw=drawColor,line width= 0.4pt,line join=round,line cap=round,fill=fillColor] (131.64,145.06) circle (  1.96);

\path[draw=drawColor,line width= 0.4pt,line join=round,line cap=round,fill=fillColor] (131.64,145.06) circle (  1.96);

\path[draw=drawColor,line width= 0.4pt,line join=round,line cap=round,fill=fillColor] (131.64,141.13) circle (  1.96);

\path[draw=drawColor,line width= 0.4pt,line join=round,line cap=round,fill=fillColor] (131.64,141.13) circle (  1.96);

\path[draw=drawColor,line width= 0.4pt,line join=round,line cap=round,fill=fillColor] (131.64,149.55) circle (  1.96);

\path[draw=drawColor,line width= 0.4pt,line join=round,line cap=round,fill=fillColor] (131.64,138.77) circle (  1.96);

\path[draw=drawColor,line width= 0.4pt,line join=round,line cap=round,fill=fillColor] (131.64,145.84) circle (  1.96);

\path[draw=drawColor,line width= 0.4pt,line join=round,line cap=round,fill=fillColor] (131.64,145.06) circle (  1.96);

\path[draw=drawColor,line width= 0.4pt,line join=round,line cap=round,fill=fillColor] (131.64,141.13) circle (  1.96);

\path[draw=drawColor,line width= 0.4pt,line join=round,line cap=round,fill=fillColor] (131.64,148.20) circle (  1.96);

\path[draw=drawColor,line width= 0.4pt,line join=round,line cap=round,fill=fillColor] (131.64,141.13) circle (  1.96);

\path[draw=drawColor,line width= 0.4pt,line join=round,line cap=round,fill=fillColor] (131.64,145.06) circle (  1.96);

\path[draw=drawColor,line width= 0.4pt,line join=round,line cap=round,fill=fillColor] (131.64,141.13) circle (  1.96);

\path[draw=drawColor,line width= 0.4pt,line join=round,line cap=round,fill=fillColor] (131.64,141.13) circle (  1.96);

\path[draw=drawColor,line width= 0.4pt,line join=round,line cap=round,fill=fillColor] (131.64,138.77) circle (  1.96);

\path[draw=drawColor,line width= 0.4pt,line join=round,line cap=round,fill=fillColor] (131.64,141.13) circle (  1.96);

\path[draw=drawColor,line width= 0.4pt,line join=round,line cap=round,fill=fillColor] (131.64,138.77) circle (  1.96);

\path[draw=drawColor,line width= 0.4pt,line join=round,line cap=round,fill=fillColor] (131.64,141.13) circle (  1.96);

\path[draw=drawColor,line width= 0.4pt,line join=round,line cap=round,fill=fillColor] (131.64,142.54) circle (  1.96);

\path[draw=drawColor,line width= 0.4pt,line join=round,line cap=round,fill=fillColor] (131.64,138.77) circle (  1.96);

\path[draw=drawColor,line width= 0.4pt,line join=round,line cap=round,fill=fillColor] (131.64,138.77) circle (  1.96);

\path[draw=drawColor,line width= 0.4pt,line join=round,line cap=round,fill=fillColor] (131.64,145.06) circle (  1.96);

\path[draw=drawColor,line width= 0.4pt,line join=round,line cap=round,fill=fillColor] (131.64,138.77) circle (  1.96);

\path[draw=drawColor,line width= 0.4pt,line join=round,line cap=round,fill=fillColor] (131.64,138.77) circle (  1.96);

\path[draw=drawColor,line width= 0.4pt,line join=round,line cap=round,fill=fillColor] (131.64,145.06) circle (  1.96);

\path[draw=drawColor,line width= 0.4pt,line join=round,line cap=round,fill=fillColor] (131.64,153.86) circle (  1.96);

\path[draw=drawColor,line width= 0.4pt,line join=round,line cap=round,fill=fillColor] (131.64,145.06) circle (  1.96);

\path[draw=drawColor,line width= 0.4pt,line join=round,line cap=round,fill=fillColor] (131.64,152.92) circle (  1.96);

\path[draw=drawColor,line width= 0.4pt,line join=round,line cap=round,fill=fillColor] (131.64,141.13) circle (  1.96);

\path[draw=drawColor,line width= 0.4pt,line join=round,line cap=round,fill=fillColor] (131.64,138.77) circle (  1.96);

\path[draw=drawColor,line width= 0.4pt,line join=round,line cap=round,fill=fillColor] (131.64,141.13) circle (  1.96);

\path[draw=drawColor,line width= 0.4pt,line join=round,line cap=round,fill=fillColor] (131.64,141.13) circle (  1.96);

\path[draw=drawColor,line width= 0.4pt,line join=round,line cap=round,fill=fillColor] (131.64,146.32) circle (  1.96);

\path[draw=drawColor,line width= 0.4pt,line join=round,line cap=round,fill=fillColor] (131.64,142.17) circle (  1.96);

\path[draw=drawColor,line width= 0.4pt,line join=round,line cap=round,fill=fillColor] (131.64,141.13) circle (  1.96);

\path[draw=drawColor,line width= 0.4pt,line join=round,line cap=round,fill=fillColor] (131.64,142.54) circle (  1.96);

\path[draw=drawColor,line width= 0.4pt,line join=round,line cap=round,fill=fillColor] (131.64,139.15) circle (  1.96);

\path[draw=drawColor,line width= 0.4pt,line join=round,line cap=round,fill=fillColor] (131.64,142.54) circle (  1.96);

\path[draw=drawColor,line width= 0.4pt,line join=round,line cap=round,fill=fillColor] (131.64,152.25) circle (  1.96);

\path[draw=drawColor,line width= 0.4pt,line join=round,line cap=round,fill=fillColor] (131.64,142.54) circle (  1.96);

\path[draw=drawColor,line width= 0.4pt,line join=round,line cap=round,fill=fillColor] (131.64,152.92) circle (  1.96);

\path[draw=drawColor,line width= 0.4pt,line join=round,line cap=round,fill=fillColor] (131.64,152.92) circle (  1.96);

\path[draw=drawColor,line width= 0.4pt,line join=round,line cap=round,fill=fillColor] (131.64,139.82) circle (  1.96);

\path[draw=drawColor,line width= 0.4pt,line join=round,line cap=round,fill=fillColor] (131.64,145.06) circle (  1.96);

\path[draw=drawColor,line width= 0.4pt,line join=round,line cap=round,fill=fillColor] (131.64,140.93) circle (  1.96);

\path[draw=drawColor,line width= 0.4pt,line join=round,line cap=round,fill=fillColor] (131.64,150.09) circle (  1.96);

\path[draw=drawColor,line width= 0.4pt,line join=round,line cap=round,fill=fillColor] (131.64,145.84) circle (  1.96);

\path[draw=drawColor,line width= 0.4pt,line join=round,line cap=round,fill=fillColor] (131.64,148.20) circle (  1.96);

\path[draw=drawColor,line width= 0.4pt,line join=round,line cap=round,fill=fillColor] (131.64,151.98) circle (  1.96);

\path[draw=drawColor,line width= 0.4pt,line join=round,line cap=round,fill=fillColor] (131.64,146.32) circle (  1.96);

\path[draw=drawColor,line width= 0.4pt,line join=round,line cap=round,fill=fillColor] (131.64,152.92) circle (  1.96);

\path[draw=drawColor,line width= 0.4pt,line join=round,line cap=round,fill=fillColor] (131.64,152.92) circle (  1.96);

\path[draw=drawColor,line width= 0.4pt,line join=round,line cap=round,fill=fillColor] (131.64,152.92) circle (  1.96);

\path[draw=drawColor,line width= 0.4pt,line join=round,line cap=round,fill=fillColor] (131.64,138.77) circle (  1.96);

\path[draw=drawColor,line width= 0.4pt,line join=round,line cap=round,fill=fillColor] (131.64,153.86) circle (  1.96);

\path[draw=drawColor,line width= 0.4pt,line join=round,line cap=round,fill=fillColor] (131.64,145.06) circle (  1.96);

\path[draw=drawColor,line width= 0.4pt,line join=round,line cap=round,fill=fillColor] (131.64,147.02) circle (  1.96);

\path[draw=drawColor,line width= 0.4pt,line join=round,line cap=round,fill=fillColor] (131.64,145.06) circle (  1.96);

\path[draw=drawColor,line width= 0.4pt,line join=round,line cap=round,fill=fillColor] (131.64,152.92) circle (  1.96);

\path[draw=drawColor,line width= 0.4pt,line join=round,line cap=round,fill=fillColor] (131.64,141.13) circle (  1.96);

\path[draw=drawColor,line width= 0.4pt,line join=round,line cap=round,fill=fillColor] (131.64,139.71) circle (  1.96);

\path[draw=drawColor,line width= 0.4pt,line join=round,line cap=round,fill=fillColor] (131.64,141.13) circle (  1.96);

\path[draw=drawColor,line width= 0.4pt,line join=round,line cap=round,fill=fillColor] (131.64,138.77) circle (  1.96);

\path[draw=drawColor,line width= 0.4pt,line join=round,line cap=round,fill=fillColor] (131.64,145.06) circle (  1.96);

\path[draw=drawColor,line width= 0.4pt,line join=round,line cap=round,fill=fillColor] (131.64,141.91) circle (  1.96);

\path[draw=drawColor,line width= 0.4pt,line join=round,line cap=round,fill=fillColor] (131.64,138.77) circle (  1.96);

\path[draw=drawColor,line width= 0.4pt,line join=round,line cap=round,fill=fillColor] (131.64,140.57) circle (  1.96);

\path[draw=drawColor,line width= 0.4pt,line join=round,line cap=round,fill=fillColor] (131.64,138.77) circle (  1.96);

\path[draw=drawColor,line width= 0.4pt,line join=round,line cap=round,fill=fillColor] (131.64,145.06) circle (  1.96);

\path[draw=drawColor,line width= 0.4pt,line join=round,line cap=round,fill=fillColor] (131.64,138.77) circle (  1.96);

\path[draw=drawColor,line width= 0.4pt,line join=round,line cap=round,fill=fillColor] (131.64,143.49) circle (  1.96);

\path[draw=drawColor,line width= 0.4pt,line join=round,line cap=round,fill=fillColor] (131.64,145.06) circle (  1.96);

\path[draw=drawColor,line width= 0.4pt,line join=round,line cap=round,fill=fillColor] (131.64,141.13) circle (  1.96);

\path[draw=drawColor,line width= 0.4pt,line join=round,line cap=round,fill=fillColor] (131.64,138.77) circle (  1.96);

\path[draw=drawColor,line width= 0.4pt,line join=round,line cap=round,fill=fillColor] (131.64,145.06) circle (  1.96);

\path[draw=drawColor,line width= 0.4pt,line join=round,line cap=round,fill=fillColor] (131.64,145.84) circle (  1.96);

\path[draw=drawColor,line width= 0.4pt,line join=round,line cap=round,fill=fillColor] (131.64,138.77) circle (  1.96);

\path[draw=drawColor,line width= 0.4pt,line join=round,line cap=round,fill=fillColor] (131.64,138.77) circle (  1.96);

\path[draw=drawColor,line width= 0.4pt,line join=round,line cap=round,fill=fillColor] (131.64,148.20) circle (  1.96);

\path[draw=drawColor,line width= 0.4pt,line join=round,line cap=round,fill=fillColor] (131.64,138.77) circle (  1.96);

\path[draw=drawColor,line width= 0.4pt,line join=round,line cap=round,fill=fillColor] (131.64,141.13) circle (  1.96);

\path[draw=drawColor,line width= 0.4pt,line join=round,line cap=round,fill=fillColor] (131.64,138.77) circle (  1.96);

\path[draw=drawColor,line width= 0.4pt,line join=round,line cap=round,fill=fillColor] (131.64,142.54) circle (  1.96);

\path[draw=drawColor,line width= 0.4pt,line join=round,line cap=round,fill=fillColor] (131.64,141.13) circle (  1.96);

\path[draw=drawColor,line width= 0.4pt,line join=round,line cap=round,fill=fillColor] (131.64,152.92) circle (  1.96);

\path[draw=drawColor,line width= 0.4pt,line join=round,line cap=round,fill=fillColor] (131.64,141.13) circle (  1.96);

\path[draw=drawColor,line width= 0.4pt,line join=round,line cap=round,fill=fillColor] (131.64,145.06) circle (  1.96);

\path[draw=drawColor,line width= 0.4pt,line join=round,line cap=round,fill=fillColor] (131.64,141.13) circle (  1.96);

\path[draw=drawColor,line width= 0.4pt,line join=round,line cap=round,fill=fillColor] (131.64,138.77) circle (  1.96);

\path[draw=drawColor,line width= 0.4pt,line join=round,line cap=round,fill=fillColor] (131.64,152.92) circle (  1.96);

\path[draw=drawColor,line width= 0.4pt,line join=round,line cap=round,fill=fillColor] (131.64,152.92) circle (  1.96);

\path[draw=drawColor,line width= 0.4pt,line join=round,line cap=round,fill=fillColor] (131.64,138.77) circle (  1.96);

\path[draw=drawColor,line width= 0.4pt,line join=round,line cap=round,fill=fillColor] (131.64,138.77) circle (  1.96);

\path[draw=drawColor,line width= 0.4pt,line join=round,line cap=round,fill=fillColor] (131.64,141.13) circle (  1.96);

\path[draw=drawColor,line width= 0.4pt,line join=round,line cap=round,fill=fillColor] (131.64,145.06) circle (  1.96);

\path[draw=drawColor,line width= 0.4pt,line join=round,line cap=round,fill=fillColor] (131.64,150.30) circle (  1.96);

\path[draw=drawColor,line width= 0.4pt,line join=round,line cap=round,fill=fillColor] (131.64,148.99) circle (  1.96);

\path[draw=drawColor,line width= 0.4pt,line join=round,line cap=round,fill=fillColor] (131.64,140.66) circle (  1.96);

\path[draw=drawColor,line width= 0.4pt,line join=round,line cap=round,fill=fillColor] (131.64,151.35) circle (  1.96);

\path[draw=drawColor,line width= 0.4pt,line join=round,line cap=round,fill=fillColor] (131.64,138.77) circle (  1.96);

\path[draw=drawColor,line width= 0.4pt,line join=round,line cap=round,fill=fillColor] (131.64,152.92) circle (  1.96);

\path[draw=drawColor,line width= 0.4pt,line join=round,line cap=round,fill=fillColor] (131.64,153.86) circle (  1.96);

\path[draw=drawColor,line width= 0.4pt,line join=round,line cap=round,fill=fillColor] (131.64,153.86) circle (  1.96);

\path[draw=drawColor,line width= 0.4pt,line join=round,line cap=round,fill=fillColor] (131.64,141.13) circle (  1.96);

\path[draw=drawColor,line width= 0.4pt,line join=round,line cap=round,fill=fillColor] (131.64,141.13) circle (  1.96);

\path[draw=drawColor,line width= 0.4pt,line join=round,line cap=round,fill=fillColor] (131.64,138.77) circle (  1.96);

\path[draw=drawColor,line width= 0.4pt,line join=round,line cap=round,fill=fillColor] (131.64,138.77) circle (  1.96);

\path[draw=drawColor,line width= 0.4pt,line join=round,line cap=round,fill=fillColor] (131.64,148.20) circle (  1.96);

\path[draw=drawColor,line width= 0.4pt,line join=round,line cap=round,fill=fillColor] (131.64,141.47) circle (  1.96);

\path[draw=drawColor,line width= 0.4pt,line join=round,line cap=round,fill=fillColor] (131.64,148.20) circle (  1.96);

\path[draw=drawColor,line width= 0.4pt,line join=round,line cap=round,fill=fillColor] (131.64,148.20) circle (  1.96);

\path[draw=drawColor,line width= 0.4pt,line join=round,line cap=round,fill=fillColor] (131.64,145.06) circle (  1.96);

\path[draw=drawColor,line width= 0.4pt,line join=round,line cap=round,fill=fillColor] (131.64,139.65) circle (  1.96);

\path[draw=drawColor,line width= 0.4pt,line join=round,line cap=round,fill=fillColor] (131.64,140.66) circle (  1.96);

\path[draw=drawColor,line width= 0.4pt,line join=round,line cap=round,fill=fillColor] (131.64,144.84) circle (  1.96);

\path[draw=drawColor,line width= 0.4pt,line join=round,line cap=round,fill=fillColor] (131.64,139.47) circle (  1.96);

\path[draw=drawColor,line width= 0.4pt,line join=round,line cap=round,fill=fillColor] (131.64,144.84) circle (  1.96);

\path[draw=drawColor,line width= 0.4pt,line join=round,line cap=round,fill=fillColor] (131.64,149.02) circle (  1.96);

\path[draw=drawColor,line width= 0.4pt,line join=round,line cap=round,fill=fillColor] (131.64,144.84) circle (  1.96);

\path[draw=drawColor,line width= 0.4pt,line join=round,line cap=round,fill=fillColor] (131.64,144.84) circle (  1.96);

\path[draw=drawColor,line width= 0.4pt,line join=round,line cap=round,fill=fillColor] (131.64,149.02) circle (  1.96);

\path[draw=drawColor,line width= 0.4pt,line join=round,line cap=round,fill=fillColor] (131.64,147.35) circle (  1.96);

\path[draw=drawColor,line width= 0.4pt,line join=round,line cap=round,fill=fillColor] (131.64,138.58) circle (  1.96);

\path[draw=drawColor,line width= 0.4pt,line join=round,line cap=round,fill=fillColor] (131.64,139.47) circle (  1.96);

\path[draw=drawColor,line width= 0.4pt,line join=round,line cap=round,fill=fillColor] (131.64,145.20) circle (  1.96);

\path[draw=drawColor,line width= 0.4pt,line join=round,line cap=round,fill=fillColor] (131.64,149.02) circle (  1.96);

\path[draw=drawColor,line width= 0.4pt,line join=round,line cap=round,fill=fillColor] (131.64,153.79) circle (  1.96);

\path[draw=drawColor,line width= 0.4pt,line join=round,line cap=round,fill=fillColor] (131.64,149.02) circle (  1.96);

\path[draw=drawColor,line width= 0.4pt,line join=round,line cap=round,fill=fillColor] (131.64,149.02) circle (  1.96);

\path[draw=drawColor,line width= 0.4pt,line join=round,line cap=round,fill=fillColor] (131.64,140.90) circle (  1.96);

\path[draw=drawColor,line width= 0.4pt,line join=round,line cap=round,fill=fillColor] (131.64,150.36) circle (  1.96);

\path[draw=drawColor,line width= 0.4pt,line join=round,line cap=round,fill=fillColor] (131.64,144.84) circle (  1.96);

\path[draw=drawColor,line width= 0.4pt,line join=round,line cap=round,fill=fillColor] (131.64,143.45) circle (  1.96);

\path[draw=drawColor,line width= 0.4pt,line join=round,line cap=round,fill=fillColor] (131.64,144.84) circle (  1.96);

\path[draw=drawColor,line width= 0.4pt,line join=round,line cap=round,fill=fillColor] (131.64,154.37) circle (  1.96);

\path[draw=drawColor,line width= 0.4pt,line join=round,line cap=round,fill=fillColor] (131.64,149.02) circle (  1.96);

\path[draw=drawColor,line width= 0.4pt,line join=round,line cap=round,fill=fillColor] (131.64,138.99) circle (  1.96);

\path[draw=drawColor,line width= 0.4pt,line join=round,line cap=round,fill=fillColor] (131.64,144.84) circle (  1.96);

\path[draw=drawColor,line width= 0.4pt,line join=round,line cap=round,fill=fillColor] (131.64,138.99) circle (  1.96);

\path[draw=drawColor,line width= 0.4pt,line join=round,line cap=round,fill=fillColor] (131.64,144.84) circle (  1.96);

\path[draw=drawColor,line width= 0.4pt,line join=round,line cap=round,fill=fillColor] (131.64,142.34) circle (  1.96);

\path[draw=drawColor,line width= 0.4pt,line join=round,line cap=round,fill=fillColor] (131.64,142.34) circle (  1.96);

\path[draw=drawColor,line width= 0.4pt,line join=round,line cap=round,fill=fillColor] (131.64,142.81) circle (  1.96);

\path[draw=drawColor,line width= 0.4pt,line join=round,line cap=round,fill=fillColor] (131.64,142.34) circle (  1.96);

\path[draw=drawColor,line width= 0.4pt,line join=round,line cap=round,fill=fillColor] (131.64,142.34) circle (  1.96);

\path[draw=drawColor,line width= 0.4pt,line join=round,line cap=round,fill=fillColor] (131.64,142.34) circle (  1.96);

\path[draw=drawColor,line width= 0.4pt,line join=round,line cap=round,fill=fillColor] (131.64,142.34) circle (  1.96);

\path[draw=drawColor,line width= 0.4pt,line join=round,line cap=round,fill=fillColor] (131.64,146.35) circle (  1.96);

\path[draw=drawColor,line width= 0.4pt,line join=round,line cap=round,fill=fillColor] (131.64,147.35) circle (  1.96);

\path[draw=drawColor,line width= 0.4pt,line join=round,line cap=round,fill=fillColor] (131.64,144.84) circle (  1.96);

\path[draw=drawColor,line width= 0.4pt,line join=round,line cap=round,fill=fillColor] (131.64,140.66) circle (  1.96);

\path[draw=drawColor,line width= 0.4pt,line join=round,line cap=round,fill=fillColor] (131.64,144.84) circle (  1.96);

\path[draw=drawColor,line width= 0.4pt,line join=round,line cap=round,fill=fillColor] (131.64,149.02) circle (  1.96);

\path[draw=drawColor,line width= 0.4pt,line join=round,line cap=round,fill=fillColor] (131.64,142.34) circle (  1.96);

\path[draw=drawColor,line width= 0.4pt,line join=round,line cap=round,fill=fillColor] (131.64,153.20) circle (  1.96);

\path[draw=drawColor,line width= 0.4pt,line join=round,line cap=round,fill=fillColor] (131.64,141.16) circle (  1.96);

\path[draw=drawColor,line width= 0.4pt,line join=round,line cap=round,fill=fillColor] (131.64,147.35) circle (  1.96);

\path[draw=drawColor,line width= 0.4pt,line join=round,line cap=round,fill=fillColor] (131.64,143.45) circle (  1.96);

\path[draw=drawColor,line width= 0.4pt,line join=round,line cap=round,fill=fillColor] (131.64,149.02) circle (  1.96);

\path[draw=drawColor,line width= 0.4pt,line join=round,line cap=round,fill=fillColor] (131.64,149.02) circle (  1.96);

\path[draw=drawColor,line width= 0.4pt,line join=round,line cap=round,fill=fillColor] (131.64,142.34) circle (  1.96);

\path[draw=drawColor,line width= 0.4pt,line join=round,line cap=round,fill=fillColor] (131.64,142.34) circle (  1.96);

\path[draw=drawColor,line width= 0.4pt,line join=round,line cap=round,fill=fillColor] (131.64,147.35) circle (  1.96);

\path[draw=drawColor,line width= 0.4pt,line join=round,line cap=round,fill=fillColor] (131.64,153.79) circle (  1.96);

\path[draw=drawColor,line width= 0.4pt,line join=round,line cap=round,fill=fillColor] (131.64,142.34) circle (  1.96);

\path[draw=drawColor,line width= 0.4pt,line join=round,line cap=round,fill=fillColor] (131.64,149.02) circle (  1.96);

\path[draw=drawColor,line width= 0.4pt,line join=round,line cap=round,fill=fillColor] (131.64,144.84) circle (  1.96);

\path[draw=drawColor,line width= 0.4pt,line join=round,line cap=round,fill=fillColor] (131.64,138.58) circle (  1.96);

\path[draw=drawColor,line width= 0.4pt,line join=round,line cap=round,fill=fillColor] (131.64,150.36) circle (  1.96);

\path[draw=drawColor,line width= 0.4pt,line join=round,line cap=round,fill=fillColor] (131.64,141.53) circle (  1.96);

\path[draw=drawColor,line width= 0.4pt,line join=round,line cap=round,fill=fillColor] (131.64,146.35) circle (  1.96);

\path[draw=drawColor,line width= 0.4pt,line join=round,line cap=round,fill=fillColor] (131.64,145.54) circle (  1.96);

\path[draw=drawColor,line width= 0.4pt,line join=round,line cap=round,fill=fillColor] (131.64,151.69) circle (  1.96);

\path[draw=drawColor,line width= 0.4pt,line join=round,line cap=round,fill=fillColor] (131.64,148.75) circle (  1.96);

\path[draw=drawColor,line width= 0.4pt,line join=round,line cap=round,fill=fillColor] (131.64,142.34) circle (  1.96);

\path[draw=drawColor,line width= 0.4pt,line join=round,line cap=round,fill=fillColor] (131.64,142.34) circle (  1.96);

\path[draw=drawColor,line width= 0.4pt,line join=round,line cap=round,fill=fillColor] (131.64,144.84) circle (  1.96);

\path[draw=drawColor,line width= 0.4pt,line join=round,line cap=round,fill=fillColor] (131.64,153.79) circle (  1.96);

\path[draw=drawColor,line width= 0.4pt,line join=round,line cap=round,fill=fillColor] (131.64,149.02) circle (  1.96);

\path[draw=drawColor,line width= 0.4pt,line join=round,line cap=round,fill=fillColor] (131.64,142.34) circle (  1.96);

\path[draw=drawColor,line width= 0.4pt,line join=round,line cap=round,fill=fillColor] (131.64,138.99) circle (  1.96);

\path[draw=drawColor,line width= 0.4pt,line join=round,line cap=round,fill=fillColor] (131.64,142.34) circle (  1.96);

\path[draw=drawColor,line width= 0.4pt,line join=round,line cap=round,fill=fillColor] (131.64,140.66) circle (  1.96);

\path[draw=drawColor,line width= 0.4pt,line join=round,line cap=round,fill=fillColor] (131.64,138.32) circle (  1.96);

\path[draw=drawColor,line width= 0.4pt,line join=round,line cap=round,fill=fillColor] (131.64,142.34) circle (  1.96);

\path[draw=drawColor,line width= 0.4pt,line join=round,line cap=round,fill=fillColor] (131.64,138.32) circle (  1.96);

\path[draw=drawColor,line width= 0.4pt,line join=round,line cap=round,fill=fillColor] (131.64,150.36) circle (  1.96);

\path[draw=drawColor,line width= 0.4pt,line join=round,line cap=round,fill=fillColor] (131.64,142.34) circle (  1.96);

\path[draw=drawColor,line width= 0.4pt,line join=round,line cap=round,fill=fillColor] (131.64,152.36) circle (  1.96);

\path[draw=drawColor,line width= 0.4pt,line join=round,line cap=round,fill=fillColor] (131.64,153.79) circle (  1.96);

\path[draw=drawColor,line width= 0.4pt,line join=round,line cap=round,fill=fillColor] (131.64,142.34) circle (  1.96);

\path[draw=drawColor,line width= 0.4pt,line join=round,line cap=round,fill=fillColor] (131.64,138.32) circle (  1.96);

\path[draw=drawColor,line width= 0.4pt,line join=round,line cap=round,fill=fillColor] (131.64,149.02) circle (  1.96);

\path[draw=drawColor,line width= 0.4pt,line join=round,line cap=round,fill=fillColor] (131.64,154.37) circle (  1.96);

\path[draw=drawColor,line width= 0.4pt,line join=round,line cap=round,fill=fillColor] (131.64,138.32) circle (  1.96);

\path[draw=drawColor,line width= 0.4pt,line join=round,line cap=round,fill=fillColor] (131.64,142.34) circle (  1.96);

\path[draw=drawColor,line width= 0.4pt,line join=round,line cap=round,fill=fillColor] (131.64,147.17) circle (  1.96);

\path[draw=drawColor,line width= 0.4pt,line join=round,line cap=round,fill=fillColor] (131.64,139.94) circle (  1.96);

\path[draw=drawColor,line width= 0.4pt,line join=round,line cap=round,fill=fillColor] (131.64,151.50) circle (  1.96);

\path[draw=drawColor,line width= 0.4pt,line join=round,line cap=round,fill=fillColor] (131.64,153.66) circle (  1.96);

\path[draw=drawColor,line width= 0.4pt,line join=round,line cap=round,fill=fillColor] (131.64,145.31) circle (  1.96);

\path[draw=drawColor,line width= 0.4pt,line join=round,line cap=round,fill=fillColor] (131.64,147.17) circle (  1.96);

\path[draw=drawColor,line width= 0.4pt,line join=round,line cap=round,fill=fillColor] (131.64,147.17) circle (  1.96);

\path[draw=drawColor,line width= 0.4pt,line join=round,line cap=round,fill=fillColor] (131.64,139.94) circle (  1.96);

\path[draw=drawColor,line width= 0.4pt,line join=round,line cap=round,fill=fillColor] (131.64,152.88) circle (  1.96);

\path[draw=drawColor,line width= 0.4pt,line join=round,line cap=round,fill=fillColor] (131.64,147.17) circle (  1.96);

\path[draw=drawColor,line width= 0.4pt,line join=round,line cap=round,fill=fillColor] (131.64,144.57) circle (  1.96);

\path[draw=drawColor,line width= 0.4pt,line join=round,line cap=round,fill=fillColor] (131.64,147.17) circle (  1.96);

\path[draw=drawColor,line width= 0.4pt,line join=round,line cap=round,fill=fillColor] (131.64,147.17) circle (  1.96);

\path[draw=drawColor,line width= 0.4pt,line join=round,line cap=round,fill=fillColor] (131.64,140.41) circle (  1.96);

\path[draw=drawColor,line width= 0.4pt,line join=round,line cap=round,fill=fillColor] (131.64,144.57) circle (  1.96);

\path[draw=drawColor,line width= 0.4pt,line join=round,line cap=round,fill=fillColor] (131.64,153.66) circle (  1.96);

\path[draw=drawColor,line width= 0.4pt,line join=round,line cap=round,fill=fillColor] (131.64,144.57) circle (  1.96);

\path[draw=drawColor,line width= 0.4pt,line join=round,line cap=round,fill=fillColor] (131.64,140.67) circle (  1.96);

\path[draw=drawColor,line width= 0.4pt,line join=round,line cap=round,fill=fillColor] (131.64,141.10) circle (  1.96);

\path[draw=drawColor,line width= 0.4pt,line join=round,line cap=round,fill=fillColor] (131.64,138.50) circle (  1.96);

\path[draw=drawColor,line width= 0.4pt,line join=round,line cap=round,fill=fillColor] (131.64,147.17) circle (  1.96);

\path[draw=drawColor,line width= 0.4pt,line join=round,line cap=round,fill=fillColor] (131.64,144.57) circle (  1.96);

\path[draw=drawColor,line width= 0.4pt,line join=round,line cap=round,fill=fillColor] (131.64,144.57) circle (  1.96);

\path[draw=drawColor,line width= 0.4pt,line join=round,line cap=round,fill=fillColor] (131.64,151.50) circle (  1.96);

\path[draw=drawColor,line width= 0.4pt,line join=round,line cap=round,fill=fillColor] (131.64,144.92) circle (  1.96);

\path[draw=drawColor,line width= 0.4pt,line join=round,line cap=round,fill=fillColor] (131.64,141.26) circle (  1.96);

\path[draw=drawColor,line width= 0.4pt,line join=round,line cap=round,fill=fillColor] (131.64,147.17) circle (  1.96);

\path[draw=drawColor,line width= 0.4pt,line join=round,line cap=round,fill=fillColor] (131.64,141.10) circle (  1.96);

\path[draw=drawColor,line width= 0.4pt,line join=round,line cap=round,fill=fillColor] (131.64,144.57) circle (  1.96);

\path[draw=drawColor,line width= 0.4pt,line join=round,line cap=round,fill=fillColor] (131.64,142.83) circle (  1.96);

\path[draw=drawColor,line width= 0.4pt,line join=round,line cap=round,fill=fillColor] (131.64,147.17) circle (  1.96);

\path[draw=drawColor,line width= 0.4pt,line join=round,line cap=round,fill=fillColor] (131.64,151.50) circle (  1.96);

\path[draw=drawColor,line width= 0.4pt,line join=round,line cap=round,fill=fillColor] (131.64,146.55) circle (  1.96);

\path[draw=drawColor,line width= 0.4pt,line join=round,line cap=round,fill=fillColor] (131.64,151.50) circle (  1.96);

\path[draw=drawColor,line width= 0.4pt,line join=round,line cap=round,fill=fillColor] (131.64,139.94) circle (  1.96);

\path[draw=drawColor,line width= 0.4pt,line join=round,line cap=round,fill=fillColor] (131.64,148.73) circle (  1.96);

\path[draw=drawColor,line width= 0.4pt,line join=round,line cap=round,fill=fillColor] (131.64,138.50) circle (  1.96);

\path[draw=drawColor,line width= 0.4pt,line join=round,line cap=round,fill=fillColor] (131.64,138.50) circle (  1.96);

\path[draw=drawColor,line width= 0.4pt,line join=round,line cap=round,fill=fillColor] (131.64,141.59) circle (  1.96);

\path[draw=drawColor,line width= 0.4pt,line join=round,line cap=round,fill=fillColor] (131.64,144.57) circle (  1.96);

\path[draw=drawColor,line width= 0.4pt,line join=round,line cap=round,fill=fillColor] (131.64,141.10) circle (  1.96);

\path[draw=drawColor,line width= 0.4pt,line join=round,line cap=round,fill=fillColor] (131.64,138.50) circle (  1.96);

\path[draw=drawColor,line width= 0.4pt,line join=round,line cap=round,fill=fillColor] (131.64,147.17) circle (  1.96);

\path[draw=drawColor,line width= 0.4pt,line join=round,line cap=round,fill=fillColor] (131.64,141.10) circle (  1.96);

\path[draw=drawColor,line width= 0.4pt,line join=round,line cap=round,fill=fillColor] (131.64,144.57) circle (  1.96);

\path[draw=drawColor,line width= 0.4pt,line join=round,line cap=round,fill=fillColor] (131.64,141.97) circle (  1.96);

\path[draw=drawColor,line width= 0.4pt,line join=round,line cap=round,fill=fillColor] (131.64,151.50) circle (  1.96);

\path[draw=drawColor,line width= 0.4pt,line join=round,line cap=round,fill=fillColor] (131.64,144.57) circle (  1.96);

\path[draw=drawColor,line width= 0.4pt,line join=round,line cap=round,fill=fillColor] (131.64,139.37) circle (  1.96);

\path[draw=drawColor,line width= 0.4pt,line join=round,line cap=round,fill=fillColor] (131.64,144.57) circle (  1.96);

\path[draw=drawColor,line width= 0.4pt,line join=round,line cap=round,fill=fillColor] (131.64,151.50) circle (  1.96);

\path[draw=drawColor,line width= 0.4pt,line join=round,line cap=round,fill=fillColor] (131.64,142.17) circle (  1.96);

\path[draw=drawColor,line width= 0.4pt,line join=round,line cap=round,fill=fillColor] (131.64,144.57) circle (  1.96);

\path[draw=drawColor,line width= 0.4pt,line join=round,line cap=round,fill=fillColor] (131.64,145.72) circle (  1.96);

\path[draw=drawColor,line width= 0.4pt,line join=round,line cap=round,fill=fillColor] (131.64,144.57) circle (  1.96);

\path[draw=drawColor,line width= 0.4pt,line join=round,line cap=round,fill=fillColor] (131.64,140.67) circle (  1.96);

\path[draw=drawColor,line width= 0.4pt,line join=round,line cap=round,fill=fillColor] (131.64,139.94) circle (  1.96);

\path[draw=drawColor,line width= 0.4pt,line join=round,line cap=round,fill=fillColor] (131.64,141.26) circle (  1.96);

\path[draw=drawColor,line width= 0.4pt,line join=round,line cap=round,fill=fillColor] (131.64,144.57) circle (  1.96);

\path[draw=drawColor,line width= 0.4pt,line join=round,line cap=round,fill=fillColor] (131.64,144.57) circle (  1.96);

\path[draw=drawColor,line width= 0.4pt,line join=round,line cap=round,fill=fillColor] (131.64,147.17) circle (  1.96);

\path[draw=drawColor,line width= 0.4pt,line join=round,line cap=round,fill=fillColor] (131.64,151.50) circle (  1.96);

\path[draw=drawColor,line width= 0.4pt,line join=round,line cap=round,fill=fillColor] (131.64,151.50) circle (  1.96);

\path[draw=drawColor,line width= 0.4pt,line join=round,line cap=round,fill=fillColor] (131.64,144.57) circle (  1.96);

\path[draw=drawColor,line width= 0.4pt,line join=round,line cap=round,fill=fillColor] (131.64,139.37) circle (  1.96);

\path[draw=drawColor,line width= 0.4pt,line join=round,line cap=round,fill=fillColor] (131.64,141.10) circle (  1.96);

\path[draw=drawColor,line width= 0.4pt,line join=round,line cap=round,fill=fillColor] (131.64,144.57) circle (  1.96);

\path[draw=drawColor,line width= 0.4pt,line join=round,line cap=round,fill=fillColor] (131.64,147.17) circle (  1.96);

\path[draw=drawColor,line width= 0.4pt,line join=round,line cap=round,fill=fillColor] (131.64,139.37) circle (  1.96);

\path[draw=drawColor,line width= 0.4pt,line join=round,line cap=round,fill=fillColor] (131.64,144.57) circle (  1.96);

\path[draw=drawColor,line width= 0.4pt,line join=round,line cap=round,fill=fillColor] (131.64,144.57) circle (  1.96);

\path[draw=drawColor,line width= 0.4pt,line join=round,line cap=round,fill=fillColor] (131.64,150.80) circle (  1.96);

\path[draw=drawColor,line width= 0.4pt,line join=round,line cap=round,fill=fillColor] (131.64,139.37) circle (  1.96);

\path[draw=drawColor,line width= 0.4pt,line join=round,line cap=round,fill=fillColor] (131.64,141.59) circle (  1.96);

\path[draw=drawColor,line width= 0.4pt,line join=round,line cap=round,fill=fillColor] (131.64,147.85) circle (  1.96);

\path[draw=drawColor,line width= 0.4pt,line join=round,line cap=round,fill=fillColor] (131.64,144.57) circle (  1.96);

\path[draw=drawColor,line width= 0.4pt,line join=round,line cap=round,fill=fillColor] (131.64,144.57) circle (  1.96);

\path[draw=drawColor,line width= 0.4pt,line join=round,line cap=round,fill=fillColor] (131.64,144.57) circle (  1.96);

\path[draw=drawColor,line width= 0.4pt,line join=round,line cap=round,fill=fillColor] (131.64,144.57) circle (  1.96);

\path[draw=drawColor,line width= 0.4pt,line join=round,line cap=round,fill=fillColor] (131.64,144.57) circle (  1.96);

\path[draw=drawColor,line width= 0.4pt,line join=round,line cap=round,fill=fillColor] (131.64,141.10) circle (  1.96);

\path[draw=drawColor,line width= 0.4pt,line join=round,line cap=round,fill=fillColor] (131.64,154.39) circle (  1.96);

\path[draw=drawColor,line width= 0.4pt,line join=round,line cap=round,fill=fillColor] (131.64,140.41) circle (  1.96);

\path[draw=drawColor,line width= 0.4pt,line join=round,line cap=round,fill=fillColor] (131.64,140.41) circle (  1.96);

\path[draw=drawColor,line width= 0.4pt,line join=round,line cap=round,fill=fillColor] (131.64,144.57) circle (  1.96);

\path[draw=drawColor,line width= 0.4pt,line join=round,line cap=round,fill=fillColor] (131.64,144.57) circle (  1.96);

\path[draw=drawColor,line width= 0.4pt,line join=round,line cap=round,fill=fillColor] (131.64,144.57) circle (  1.96);

\path[draw=drawColor,line width= 0.4pt,line join=round,line cap=round,fill=fillColor] (131.64,144.57) circle (  1.96);

\path[draw=drawColor,line width= 0.4pt,line join=round,line cap=round,fill=fillColor] (131.64,144.57) circle (  1.96);

\path[draw=drawColor,line width= 0.4pt,line join=round,line cap=round,fill=fillColor] (131.64,141.26) circle (  1.96);

\path[draw=drawColor,line width= 0.4pt,line join=round,line cap=round,fill=fillColor] (131.64,141.10) circle (  1.96);

\path[draw=drawColor,line width= 0.4pt,line join=round,line cap=round,fill=fillColor] (131.64,139.37) circle (  1.96);

\path[draw=drawColor,line width= 0.4pt,line join=round,line cap=round,fill=fillColor] (131.64,144.57) circle (  1.96);

\path[draw=drawColor,line width= 0.4pt,line join=round,line cap=round,fill=fillColor] (131.64,144.57) circle (  1.96);

\path[draw=drawColor,line width= 0.4pt,line join=round,line cap=round,fill=fillColor] (131.64,142.83) circle (  1.96);

\path[draw=drawColor,line width= 0.4pt,line join=round,line cap=round,fill=fillColor] (131.64,141.10) circle (  1.96);

\path[draw=drawColor,line width= 0.4pt,line join=round,line cap=round,fill=fillColor] (131.64,141.59) circle (  1.96);

\path[draw=drawColor,line width= 0.4pt,line join=round,line cap=round,fill=fillColor] (131.64,144.57) circle (  1.96);

\path[draw=drawColor,line width= 0.4pt,line join=round,line cap=round,fill=fillColor] (131.64,144.57) circle (  1.96);

\path[draw=drawColor,line width= 0.4pt,line join=round,line cap=round,fill=fillColor] (131.64,147.17) circle (  1.96);

\path[draw=drawColor,line width= 0.4pt,line join=round,line cap=round,fill=fillColor] (131.64,151.50) circle (  1.96);

\path[draw=drawColor,line width= 0.4pt,line join=round,line cap=round,fill=fillColor] (131.64,140.67) circle (  1.96);

\path[draw=drawColor,line width= 0.4pt,line join=round,line cap=round,fill=fillColor] (131.64,144.57) circle (  1.96);

\path[draw=drawColor,line width= 0.4pt,line join=round,line cap=round,fill=fillColor] (131.64,141.10) circle (  1.96);

\path[draw=drawColor,line width= 0.4pt,line join=round,line cap=round,fill=fillColor] (131.64,141.10) circle (  1.96);

\path[draw=drawColor,line width= 0.4pt,line join=round,line cap=round,fill=fillColor] (131.64,147.17) circle (  1.96);

\path[draw=drawColor,line width= 0.4pt,line join=round,line cap=round,fill=fillColor] (131.64,144.57) circle (  1.96);

\path[draw=drawColor,line width= 0.4pt,line join=round,line cap=round,fill=fillColor] (131.64,151.50) circle (  1.96);

\path[draw=drawColor,line width= 0.4pt,line join=round,line cap=round,fill=fillColor] (131.64,154.39) circle (  1.96);

\path[draw=drawColor,line width= 0.4pt,line join=round,line cap=round,fill=fillColor] (131.64,148.61) circle (  1.96);

\path[draw=drawColor,line width= 0.4pt,line join=round,line cap=round,fill=fillColor] (131.64,142.83) circle (  1.96);

\path[draw=drawColor,line width= 0.4pt,line join=round,line cap=round,fill=fillColor] (131.64,147.17) circle (  1.96);

\path[draw=drawColor,line width= 0.4pt,line join=round,line cap=round,fill=fillColor] (131.64,141.59) circle (  1.96);

\path[draw=drawColor,line width= 0.4pt,line join=round,line cap=round,fill=fillColor] (131.64,151.50) circle (  1.96);

\path[draw=drawColor,line width= 0.4pt,line join=round,line cap=round,fill=fillColor] (131.64,147.17) circle (  1.96);

\path[draw=drawColor,line width= 0.4pt,line join=round,line cap=round,fill=fillColor] (131.64,145.72) circle (  1.96);

\path[draw=drawColor,line width= 0.4pt,line join=round,line cap=round,fill=fillColor] (131.64,148.73) circle (  1.96);

\path[draw=drawColor,line width= 0.4pt,line join=round,line cap=round,fill=fillColor] (131.64,147.17) circle (  1.96);

\path[draw=drawColor,line width= 0.4pt,line join=round,line cap=round,fill=fillColor] (131.64,151.50) circle (  1.96);

\path[draw=drawColor,line width= 0.4pt,line join=round,line cap=round,fill=fillColor] (131.64,144.57) circle (  1.96);

\path[draw=drawColor,line width= 0.4pt,line join=round,line cap=round,fill=fillColor] (131.64,139.94) circle (  1.96);

\path[draw=drawColor,line width= 0.4pt,line join=round,line cap=round,fill=fillColor] (131.64,147.17) circle (  1.96);

\path[draw=drawColor,line width= 0.4pt,line join=round,line cap=round,fill=fillColor] (131.64,147.17) circle (  1.96);

\path[draw=drawColor,line width= 0.4pt,line join=round,line cap=round,fill=fillColor] (131.64,139.94) circle (  1.96);

\path[draw=drawColor,line width= 0.4pt,line join=round,line cap=round,fill=fillColor] (131.64,151.50) circle (  1.96);

\path[draw=drawColor,line width= 0.4pt,line join=round,line cap=round,fill=fillColor] (131.64,142.83) circle (  1.96);

\path[draw=drawColor,line width= 0.4pt,line join=round,line cap=round,fill=fillColor] (131.64,151.50) circle (  1.96);

\path[draw=drawColor,line width= 0.4pt,line join=round,line cap=round,fill=fillColor] (131.64,141.59) circle (  1.96);

\path[draw=drawColor,line width= 0.4pt,line join=round,line cap=round,fill=fillColor] (131.64,143.18) circle (  1.96);

\path[draw=drawColor,line width= 0.4pt,line join=round,line cap=round,fill=fillColor] (131.64,151.50) circle (  1.96);

\path[draw=drawColor,line width= 0.4pt,line join=round,line cap=round,fill=fillColor] (131.64,154.39) circle (  1.96);

\path[draw=drawColor,line width= 0.4pt,line join=round,line cap=round,fill=fillColor] (131.64,151.50) circle (  1.96);

\path[draw=drawColor,line width= 0.4pt,line join=round,line cap=round,fill=fillColor] (131.64,140.41) circle (  1.96);

\path[draw=drawColor,line width= 0.4pt,line join=round,line cap=round,fill=fillColor] (131.64,147.54) circle (  1.96);

\path[draw=drawColor,line width= 0.4pt,line join=round,line cap=round,fill=fillColor] (131.64,147.17) circle (  1.96);

\path[draw=drawColor,line width= 0.4pt,line join=round,line cap=round,fill=fillColor] (131.64,144.57) circle (  1.96);

\path[draw=drawColor,line width= 0.4pt,line join=round,line cap=round,fill=fillColor] (131.64,147.17) circle (  1.96);

\path[draw=drawColor,line width= 0.4pt,line join=round,line cap=round,fill=fillColor] (131.64,147.17) circle (  1.96);

\path[draw=drawColor,line width= 0.4pt,line join=round,line cap=round,fill=fillColor] (131.64,141.59) circle (  1.96);

\path[draw=drawColor,line width= 0.4pt,line join=round,line cap=round,fill=fillColor] (131.64,153.48) circle (  1.96);

\path[draw=drawColor,line width= 0.4pt,line join=round,line cap=round,fill=fillColor] (131.64,140.67) circle (  1.96);

\path[draw=drawColor,line width= 0.4pt,line join=round,line cap=round,fill=fillColor] (131.64,139.37) circle (  1.96);

\path[draw=drawColor,line width= 0.4pt,line join=round,line cap=round,fill=fillColor] (131.64,151.50) circle (  1.96);

\path[draw=drawColor,line width= 0.4pt,line join=round,line cap=round,fill=fillColor] (131.64,144.57) circle (  1.96);

\path[draw=drawColor,line width= 0.4pt,line join=round,line cap=round,fill=fillColor] (131.64,151.50) circle (  1.96);

\path[draw=drawColor,line width= 0.4pt,line join=round,line cap=round,fill=fillColor] (131.64,144.57) circle (  1.96);

\path[draw=drawColor,line width= 0.4pt,line join=round,line cap=round,fill=fillColor] (131.64,151.50) circle (  1.96);

\path[draw=drawColor,line width= 0.4pt,line join=round,line cap=round,fill=fillColor] (131.64,147.17) circle (  1.96);

\path[draw=drawColor,line width= 0.4pt,line join=round,line cap=round,fill=fillColor] (131.64,147.17) circle (  1.96);

\path[draw=drawColor,line width= 0.4pt,line join=round,line cap=round,fill=fillColor] (131.64,142.83) circle (  1.96);

\path[draw=drawColor,line width= 0.4pt,line join=round,line cap=round,fill=fillColor] (131.64,140.67) circle (  1.96);

\path[draw=drawColor,line width= 0.4pt,line join=round,line cap=round,fill=fillColor] (131.64,141.59) circle (  1.96);

\path[draw=drawColor,line width= 0.4pt,line join=round,line cap=round,fill=fillColor] (131.64,151.50) circle (  1.96);

\path[draw=drawColor,line width= 0.4pt,line join=round,line cap=round,fill=fillColor] (131.64,138.69) circle (  1.96);

\path[draw=drawColor,line width= 0.4pt,line join=round,line cap=round,fill=fillColor] (131.64,143.08) circle (  1.96);

\path[draw=drawColor,line width= 0.4pt,line join=round,line cap=round,fill=fillColor] (131.64,147.17) circle (  1.96);

\path[draw=drawColor,line width= 0.4pt,line join=round,line cap=round,fill=fillColor] (131.64,141.97) circle (  1.96);

\path[draw=drawColor,line width= 0.4pt,line join=round,line cap=round,fill=fillColor] (131.64,139.94) circle (  1.96);

\path[draw=drawColor,line width= 0.4pt,line join=round,line cap=round,fill=fillColor] (131.64,149.42) circle (  1.96);

\path[draw=drawColor,line width= 0.4pt,line join=round,line cap=round,fill=fillColor] (131.64,141.97) circle (  1.96);

\path[draw=drawColor,line width= 0.4pt,line join=round,line cap=round,fill=fillColor] (131.64,147.17) circle (  1.96);

\path[draw=drawColor,line width= 0.4pt,line join=round,line cap=round,fill=fillColor] (131.64,144.57) circle (  1.96);

\path[draw=drawColor,line width= 0.4pt,line join=round,line cap=round,fill=fillColor] (131.64,147.17) circle (  1.96);

\path[draw=drawColor,line width= 0.4pt,line join=round,line cap=round,fill=fillColor] (131.64,141.59) circle (  1.96);

\path[draw=drawColor,line width= 0.4pt,line join=round,line cap=round,fill=fillColor] (131.64,139.94) circle (  1.96);

\path[draw=drawColor,line width= 0.4pt,line join=round,line cap=round,fill=fillColor] (131.64,151.50) circle (  1.96);

\path[draw=drawColor,line width= 0.4pt,line join=round,line cap=round,fill=fillColor] (131.64,140.41) circle (  1.96);

\path[draw=drawColor,line width= 0.4pt,line join=round,line cap=round,fill=fillColor] (131.64,147.17) circle (  1.96);

\path[draw=drawColor,line width= 0.4pt,line join=round,line cap=round,fill=fillColor] (131.64,144.57) circle (  1.96);

\path[draw=drawColor,line width= 0.4pt,line join=round,line cap=round,fill=fillColor] (131.64,141.10) circle (  1.96);

\path[draw=drawColor,line width= 0.4pt,line join=round,line cap=round,fill=fillColor] (131.64,147.17) circle (  1.96);

\path[draw=drawColor,line width= 0.4pt,line join=round,line cap=round,fill=fillColor] (131.64,151.50) circle (  1.96);

\path[draw=drawColor,line width= 0.4pt,line join=round,line cap=round,fill=fillColor] (131.64,144.57) circle (  1.96);

\path[draw=drawColor,line width= 0.4pt,line join=round,line cap=round,fill=fillColor] (131.64,140.67) circle (  1.96);

\path[draw=drawColor,line width= 0.4pt,line join=round,line cap=round,fill=fillColor] (131.64,142.83) circle (  1.96);

\path[draw=drawColor,line width= 0.6pt,line join=round] (131.64,112.77) -- (131.64,138.26);

\path[draw=drawColor,line width= 0.6pt,line join=round] (131.64, 95.77) -- (131.64, 82.32);
\definecolor{fillColor}{RGB}{228,26,28}

\path[draw=drawColor,line width= 0.6pt,line join=round,line cap=round,fill=fillColor] (122.75,112.77) --
	(122.75, 95.77) --
	(140.52, 95.77) --
	(140.52,112.77) --
	(122.75,112.77) --
	cycle;

\path[draw=drawColor,line width= 1.1pt,line join=round] (122.75,103.40) -- (140.52,103.40);
\definecolor{fillColor}{gray}{0.20}

\path[draw=drawColor,line width= 0.4pt,line join=round,line cap=round,fill=fillColor] (155.33,148.33) circle (  1.96);

\path[draw=drawColor,line width= 0.4pt,line join=round,line cap=round,fill=fillColor] (155.33,147.44) circle (  1.96);

\path[draw=drawColor,line width= 0.4pt,line join=round,line cap=round,fill=fillColor] (155.33,142.59) circle (  1.96);

\path[draw=drawColor,line width= 0.6pt,line join=round] (155.33,116.92) -- (155.33,141.28);

\path[draw=drawColor,line width= 0.6pt,line join=round] (155.33,100.49) -- (155.33, 85.01);
\definecolor{fillColor}{RGB}{55,126,184}

\path[draw=drawColor,line width= 0.6pt,line join=round,line cap=round,fill=fillColor] (146.44,116.92) --
	(146.44,100.49) --
	(164.21,100.49) --
	(164.21,116.92) --
	(146.44,116.92) --
	cycle;

\path[draw=drawColor,line width= 1.1pt,line join=round] (146.44,106.28) -- (164.21,106.28);

\path[draw=drawColor,line width= 0.6pt,line join=round] (179.02,133.16) -- (179.02,154.87);

\path[draw=drawColor,line width= 0.6pt,line join=round] (179.02,106.81) -- (179.02, 86.53);
\definecolor{fillColor}{RGB}{77,175,74}

\path[draw=drawColor,line width= 0.6pt,line join=round,line cap=round,fill=fillColor] (170.13,133.16) --
	(170.13,106.81) --
	(187.90,106.81) --
	(187.90,133.16) --
	(170.13,133.16) --
	cycle;

\path[draw=drawColor,line width= 1.1pt,line join=round] (170.13,116.51) -- (187.90,116.51);
\end{scope}
\begin{scope}
\path[clip] (198.73, 78.54) rectangle (274.54,158.60);
\definecolor{drawColor}{RGB}{255,255,255}

\path[draw=drawColor,line width= 0.3pt,line join=round] (198.73, 92.57) --
	(274.54, 92.57);

\path[draw=drawColor,line width= 0.3pt,line join=round] (198.73,113.37) --
	(274.54,113.37);

\path[draw=drawColor,line width= 0.3pt,line join=round] (198.73,134.17) --
	(274.54,134.17);

\path[draw=drawColor,line width= 0.3pt,line join=round] (198.73,154.96) --
	(274.54,154.96);

\path[draw=drawColor,line width= 0.6pt,line join=round] (198.73, 82.18) --
	(274.54, 82.18);

\path[draw=drawColor,line width= 0.6pt,line join=round] (198.73,102.97) --
	(274.54,102.97);

\path[draw=drawColor,line width= 0.6pt,line join=round] (198.73,123.77) --
	(274.54,123.77);

\path[draw=drawColor,line width= 0.6pt,line join=round] (198.73,144.57) --
	(274.54,144.57);

\path[draw=drawColor,line width= 0.6pt,line join=round] (212.94, 78.54) --
	(212.94,158.60);

\path[draw=drawColor,line width= 0.6pt,line join=round] (236.64, 78.54) --
	(236.64,158.60);

\path[draw=drawColor,line width= 0.6pt,line join=round] (260.33, 78.54) --
	(260.33,158.60);
\definecolor{drawColor}{gray}{0.20}

\path[draw=drawColor,line width= 0.6pt,line join=round] (212.94,125.52) -- (212.94,154.96);

\path[draw=drawColor,line width= 0.6pt,line join=round] (212.94,103.32) -- (212.94, 82.37);
\definecolor{fillColor}{RGB}{228,26,28}

\path[draw=drawColor,line width= 0.6pt,line join=round,line cap=round,fill=fillColor] (204.06,125.52) --
	(204.06,103.32) --
	(221.83,103.32) --
	(221.83,125.52) --
	(204.06,125.52) --
	cycle;

\path[draw=drawColor,line width= 1.1pt,line join=round] (204.06,113.62) -- (221.83,113.62);

\path[draw=drawColor,line width= 0.6pt,line join=round] (236.64,130.14) -- (236.64,148.33);

\path[draw=drawColor,line width= 0.6pt,line join=round] (236.64,107.63) -- (236.64, 91.24);
\definecolor{fillColor}{RGB}{55,126,184}

\path[draw=drawColor,line width= 0.6pt,line join=round,line cap=round,fill=fillColor] (227.75,130.14) --
	(227.75,107.63) --
	(245.52,107.63) --
	(245.52,130.14) --
	(227.75,130.14) --
	cycle;

\path[draw=drawColor,line width= 1.1pt,line join=round] (227.75,122.64) -- (245.52,122.64);
\definecolor{fillColor}{gray}{0.20}

\path[draw=drawColor,line width= 0.4pt,line join=round,line cap=round,fill=fillColor] (260.33, 91.67) circle (  1.96);

\path[draw=drawColor,line width= 0.6pt,line join=round] (260.33,139.44) -- (260.33,154.11);

\path[draw=drawColor,line width= 0.6pt,line join=round] (260.33,122.31) -- (260.33, 98.49);
\definecolor{fillColor}{RGB}{77,175,74}

\path[draw=drawColor,line width= 0.6pt,line join=round,line cap=round,fill=fillColor] (251.44,139.44) --
	(251.44,122.31) --
	(269.21,122.31) --
	(269.21,139.44) --
	(251.44,139.44) --
	cycle;

\path[draw=drawColor,line width= 1.1pt,line join=round] (251.44,132.84) -- (269.21,132.84);
\end{scope}
\begin{scope}
\path[clip] (280.04, 78.54) rectangle (355.85,158.60);
\definecolor{drawColor}{RGB}{255,255,255}

\path[draw=drawColor,line width= 0.3pt,line join=round] (280.04, 92.57) --
	(355.85, 92.57);

\path[draw=drawColor,line width= 0.3pt,line join=round] (280.04,113.37) --
	(355.85,113.37);

\path[draw=drawColor,line width= 0.3pt,line join=round] (280.04,134.17) --
	(355.85,134.17);

\path[draw=drawColor,line width= 0.3pt,line join=round] (280.04,154.96) --
	(355.85,154.96);

\path[draw=drawColor,line width= 0.6pt,line join=round] (280.04, 82.18) --
	(355.85, 82.18);

\path[draw=drawColor,line width= 0.6pt,line join=round] (280.04,102.97) --
	(355.85,102.97);

\path[draw=drawColor,line width= 0.6pt,line join=round] (280.04,123.77) --
	(355.85,123.77);

\path[draw=drawColor,line width= 0.6pt,line join=round] (280.04,144.57) --
	(355.85,144.57);

\path[draw=drawColor,line width= 0.6pt,line join=round] (294.25, 78.54) --
	(294.25,158.60);

\path[draw=drawColor,line width= 0.6pt,line join=round] (317.95, 78.54) --
	(317.95,158.60);

\path[draw=drawColor,line width= 0.6pt,line join=round] (341.64, 78.54) --
	(341.64,158.60);
\definecolor{drawColor}{gray}{0.20}

\path[draw=drawColor,line width= 0.6pt,line join=round] (294.25,134.17) -- (294.25,154.96);

\path[draw=drawColor,line width= 0.6pt,line join=round] (294.25,110.03) -- (294.25, 84.53);
\definecolor{fillColor}{RGB}{228,26,28}

\path[draw=drawColor,line width= 0.6pt,line join=round,line cap=round,fill=fillColor] (285.37,134.17) --
	(285.37,110.03) --
	(303.14,110.03) --
	(303.14,134.17) --
	(285.37,134.17) --
	cycle;

\path[draw=drawColor,line width= 1.1pt,line join=round] (285.37,121.48) -- (303.14,121.48);

\path[draw=drawColor,line width= 0.6pt,line join=round] (317.95,138.18) -- (317.95,152.93);

\path[draw=drawColor,line width= 0.6pt,line join=round] (317.95,110.43) -- (317.95, 85.94);
\definecolor{fillColor}{RGB}{55,126,184}

\path[draw=drawColor,line width= 0.6pt,line join=round,line cap=round,fill=fillColor] (309.06,138.18) --
	(309.06,110.43) --
	(326.83,110.43) --
	(326.83,138.18) --
	(309.06,138.18) --
	cycle;

\path[draw=drawColor,line width= 1.1pt,line join=round] (309.06,119.98) -- (326.83,119.98);

\path[draw=drawColor,line width= 0.6pt,line join=round] (341.64,146.37) -- (341.64,153.61);

\path[draw=drawColor,line width= 0.6pt,line join=round] (341.64,119.67) -- (341.64,108.71);
\definecolor{fillColor}{RGB}{77,175,74}

\path[draw=drawColor,line width= 0.6pt,line join=round,line cap=round,fill=fillColor] (332.75,146.37) --
	(332.75,119.67) --
	(350.52,119.67) --
	(350.52,146.37) --
	(332.75,146.37) --
	cycle;

\path[draw=drawColor,line width= 1.1pt,line join=round] (332.75,142.99) -- (350.52,142.99);
\end{scope}
\begin{scope}
\path[clip] ( 36.11,158.60) rectangle (111.92,175.17);
\definecolor{drawColor}{RGB}{0,0,0}

\node[text=drawColor,anchor=base,inner sep=0pt, outer sep=0pt, scale=  0.70] at ( 74.02,163.86) {No Calificados};
\end{scope}
\begin{scope}
\path[clip] (117.42,158.60) rectangle (193.23,175.17);
\definecolor{drawColor}{RGB}{0,0,0}

\node[text=drawColor,anchor=base,inner sep=0pt, outer sep=0pt, scale=  0.70] at (155.33,163.86) {Operativos};
\end{scope}
\begin{scope}
\path[clip] (198.73,158.60) rectangle (274.54,175.17);
\definecolor{drawColor}{RGB}{0,0,0}

\node[text=drawColor,anchor=base,inner sep=0pt, outer sep=0pt, scale=  0.70] at (236.64,163.86) {Técnicos};
\end{scope}
\begin{scope}
\path[clip] (280.04,158.60) rectangle (355.85,175.17);
\definecolor{drawColor}{RGB}{0,0,0}

\node[text=drawColor,anchor=base,inner sep=0pt, outer sep=0pt, scale=  0.70] at (317.95,163.86) {Profesionales};
\end{scope}
\begin{scope}
\path[clip] (  0.00,  0.00) rectangle (361.35,180.67);
\definecolor{drawColor}{gray}{0.20}

\path[draw=drawColor,line width= 0.6pt,line join=round] ( 50.33, 75.79) --
	( 50.33, 78.54);

\path[draw=drawColor,line width= 0.6pt,line join=round] ( 74.02, 75.79) --
	( 74.02, 78.54);

\path[draw=drawColor,line width= 0.6pt,line join=round] ( 97.71, 75.79) --
	( 97.71, 78.54);
\end{scope}
\begin{scope}
\path[clip] (  0.00,  0.00) rectangle (361.35,180.67);
\definecolor{drawColor}{RGB}{0,0,0}

\node[text=drawColor,rotate= 90.00,anchor=base,inner sep=0pt, outer sep=0pt, scale=  0.60] at ( 54.39, 45.78) {Nació y vive};

\node[text=drawColor,rotate= 90.00,anchor=base,inner sep=0pt, outer sep=0pt, scale=  0.60] at ( 63.89, 45.78) {en Norte};

\node[text=drawColor,rotate= 90.00,anchor=base,inner sep=0pt, outer sep=0pt, scale=  0.60] at ( 78.08, 45.78) {Nació en Norte};

\node[text=drawColor,rotate= 90.00,anchor=base,inner sep=0pt, outer sep=0pt, scale=  0.60] at ( 87.58, 45.78) {vive en Centro};

\node[text=drawColor,rotate= 90.00,anchor=base,inner sep=0pt, outer sep=0pt, scale=  0.60] at (101.77, 45.78) {Nació en Norte};

\node[text=drawColor,rotate= 90.00,anchor=base,inner sep=0pt, outer sep=0pt, scale=  0.60] at (111.27, 45.78) {vive en Sur};
\end{scope}
\begin{scope}
\path[clip] (  0.00,  0.00) rectangle (361.35,180.67);
\definecolor{drawColor}{gray}{0.20}

\path[draw=drawColor,line width= 0.6pt,line join=round] (131.64, 75.79) --
	(131.64, 78.54);

\path[draw=drawColor,line width= 0.6pt,line join=round] (155.33, 75.79) --
	(155.33, 78.54);

\path[draw=drawColor,line width= 0.6pt,line join=round] (179.02, 75.79) --
	(179.02, 78.54);
\end{scope}
\begin{scope}
\path[clip] (  0.00,  0.00) rectangle (361.35,180.67);
\definecolor{drawColor}{RGB}{0,0,0}

\node[text=drawColor,rotate= 90.00,anchor=base,inner sep=0pt, outer sep=0pt, scale=  0.60] at (135.70, 45.78) {Nació y vive};

\node[text=drawColor,rotate= 90.00,anchor=base,inner sep=0pt, outer sep=0pt, scale=  0.60] at (145.20, 45.78) {en Norte};

\node[text=drawColor,rotate= 90.00,anchor=base,inner sep=0pt, outer sep=0pt, scale=  0.60] at (159.39, 45.78) {Nació en Norte};

\node[text=drawColor,rotate= 90.00,anchor=base,inner sep=0pt, outer sep=0pt, scale=  0.60] at (168.89, 45.78) {vive en Centro};

\node[text=drawColor,rotate= 90.00,anchor=base,inner sep=0pt, outer sep=0pt, scale=  0.60] at (183.08, 45.78) {Nació en Norte};

\node[text=drawColor,rotate= 90.00,anchor=base,inner sep=0pt, outer sep=0pt, scale=  0.60] at (192.58, 45.78) {vive en Sur};
\end{scope}
\begin{scope}
\path[clip] (  0.00,  0.00) rectangle (361.35,180.67);
\definecolor{drawColor}{gray}{0.20}

\path[draw=drawColor,line width= 0.6pt,line join=round] (212.94, 75.79) --
	(212.94, 78.54);

\path[draw=drawColor,line width= 0.6pt,line join=round] (236.64, 75.79) --
	(236.64, 78.54);

\path[draw=drawColor,line width= 0.6pt,line join=round] (260.33, 75.79) --
	(260.33, 78.54);
\end{scope}
\begin{scope}
\path[clip] (  0.00,  0.00) rectangle (361.35,180.67);
\definecolor{drawColor}{RGB}{0,0,0}

\node[text=drawColor,rotate= 90.00,anchor=base,inner sep=0pt, outer sep=0pt, scale=  0.60] at (217.01, 45.78) {Nació y vive};

\node[text=drawColor,rotate= 90.00,anchor=base,inner sep=0pt, outer sep=0pt, scale=  0.60] at (226.51, 45.78) {en Norte};

\node[text=drawColor,rotate= 90.00,anchor=base,inner sep=0pt, outer sep=0pt, scale=  0.60] at (240.70, 45.78) {Nació en Norte};

\node[text=drawColor,rotate= 90.00,anchor=base,inner sep=0pt, outer sep=0pt, scale=  0.60] at (250.20, 45.78) {vive en Centro};

\node[text=drawColor,rotate= 90.00,anchor=base,inner sep=0pt, outer sep=0pt, scale=  0.60] at (264.39, 45.78) {Nació en Norte};

\node[text=drawColor,rotate= 90.00,anchor=base,inner sep=0pt, outer sep=0pt, scale=  0.60] at (273.89, 45.78) {vive en Sur};
\end{scope}
\begin{scope}
\path[clip] (  0.00,  0.00) rectangle (361.35,180.67);
\definecolor{drawColor}{gray}{0.20}

\path[draw=drawColor,line width= 0.6pt,line join=round] (294.25, 75.79) --
	(294.25, 78.54);

\path[draw=drawColor,line width= 0.6pt,line join=round] (317.95, 75.79) --
	(317.95, 78.54);

\path[draw=drawColor,line width= 0.6pt,line join=round] (341.64, 75.79) --
	(341.64, 78.54);
\end{scope}
\begin{scope}
\path[clip] (  0.00,  0.00) rectangle (361.35,180.67);
\definecolor{drawColor}{RGB}{0,0,0}

\node[text=drawColor,rotate= 90.00,anchor=base,inner sep=0pt, outer sep=0pt, scale=  0.60] at (298.32, 45.78) {Nació y vive};

\node[text=drawColor,rotate= 90.00,anchor=base,inner sep=0pt, outer sep=0pt, scale=  0.60] at (307.82, 45.78) {en Norte};

\node[text=drawColor,rotate= 90.00,anchor=base,inner sep=0pt, outer sep=0pt, scale=  0.60] at (322.01, 45.78) {Nació en Norte};

\node[text=drawColor,rotate= 90.00,anchor=base,inner sep=0pt, outer sep=0pt, scale=  0.60] at (331.51, 45.78) {vive en Centro};

\node[text=drawColor,rotate= 90.00,anchor=base,inner sep=0pt, outer sep=0pt, scale=  0.60] at (345.70, 45.78) {Nació en Norte};

\node[text=drawColor,rotate= 90.00,anchor=base,inner sep=0pt, outer sep=0pt, scale=  0.60] at (355.20, 45.78) {vive en Sur};
\end{scope}
\begin{scope}
\path[clip] (  0.00,  0.00) rectangle (361.35,180.67);
\definecolor{drawColor}{RGB}{0,0,0}

\node[text=drawColor,anchor=base east,inner sep=0pt, outer sep=0pt, scale=  0.88] at ( 31.16, 79.15) {0};

\node[text=drawColor,anchor=base east,inner sep=0pt, outer sep=0pt, scale=  0.88] at ( 31.16, 99.94) {200};

\node[text=drawColor,anchor=base east,inner sep=0pt, outer sep=0pt, scale=  0.88] at ( 31.16,120.74) {400};

\node[text=drawColor,anchor=base east,inner sep=0pt, outer sep=0pt, scale=  0.88] at ( 31.16,141.54) {600};
\end{scope}
\begin{scope}
\path[clip] (  0.00,  0.00) rectangle (361.35,180.67);
\definecolor{drawColor}{gray}{0.20}

\path[draw=drawColor,line width= 0.6pt,line join=round] ( 33.36, 82.18) --
	( 36.11, 82.18);

\path[draw=drawColor,line width= 0.6pt,line join=round] ( 33.36,102.97) --
	( 36.11,102.97);

\path[draw=drawColor,line width= 0.6pt,line join=round] ( 33.36,123.77) --
	( 36.11,123.77);

\path[draw=drawColor,line width= 0.6pt,line join=round] ( 33.36,144.57) --
	( 36.11,144.57);
\end{scope}
\begin{scope}
\path[clip] (  0.00,  0.00) rectangle (361.35,180.67);
\definecolor{drawColor}{RGB}{0,0,0}

\node[text=drawColor,rotate= 90.00,anchor=base,inner sep=0pt, outer sep=0pt, scale=  0.60] at ( 13.08,118.57) {Ingreso laboral por hora};
\end{scope}
\end{tikzpicture}

% Created by tikzDevice version 0.12.3.1 on 2021-07-16 19:43:40
% !TEX encoding = UTF-8 Unicode
\begin{tikzpicture}[x=1pt,y=1pt]
\definecolor{fillColor}{RGB}{255,255,255}
\path[use as bounding box,fill=fillColor,fill opacity=0.00] (0,0) rectangle (361.35,180.67);
\begin{scope}
\path[clip] (  0.00,  0.00) rectangle (361.35,180.67);
\definecolor{drawColor}{RGB}{255,255,255}
\definecolor{fillColor}{RGB}{255,255,255}

\path[draw=drawColor,line width= 0.6pt,line join=round,line cap=round,fill=fillColor] (  0.00,  0.00) rectangle (361.35,180.68);
\end{scope}
\begin{scope}
\path[clip] ( 36.11, 78.54) rectangle (111.92,158.60);
\definecolor{drawColor}{RGB}{255,255,255}

\path[draw=drawColor,line width= 0.3pt,line join=round] ( 36.11, 92.57) --
	(111.92, 92.57);

\path[draw=drawColor,line width= 0.3pt,line join=round] ( 36.11,113.37) --
	(111.92,113.37);

\path[draw=drawColor,line width= 0.3pt,line join=round] ( 36.11,134.17) --
	(111.92,134.17);

\path[draw=drawColor,line width= 0.3pt,line join=round] ( 36.11,154.96) --
	(111.92,154.96);

\path[draw=drawColor,line width= 0.6pt,line join=round] ( 36.11, 82.18) --
	(111.92, 82.18);

\path[draw=drawColor,line width= 0.6pt,line join=round] ( 36.11,102.97) --
	(111.92,102.97);

\path[draw=drawColor,line width= 0.6pt,line join=round] ( 36.11,123.77) --
	(111.92,123.77);

\path[draw=drawColor,line width= 0.6pt,line join=round] ( 36.11,144.57) --
	(111.92,144.57);

\path[draw=drawColor,line width= 0.6pt,line join=round] ( 50.33, 78.54) --
	( 50.33,158.60);

\path[draw=drawColor,line width= 0.6pt,line join=round] ( 74.02, 78.54) --
	( 74.02,158.60);

\path[draw=drawColor,line width= 0.6pt,line join=round] ( 97.71, 78.54) --
	( 97.71,158.60);
\definecolor{drawColor}{gray}{0.20}
\definecolor{fillColor}{gray}{0.20}

\path[draw=drawColor,line width= 0.4pt,line join=round,line cap=round,fill=fillColor] ( 50.33,131.01) circle (  1.96);

\path[draw=drawColor,line width= 0.4pt,line join=round,line cap=round,fill=fillColor] ( 50.33,130.29) circle (  1.96);

\path[draw=drawColor,line width= 0.4pt,line join=round,line cap=round,fill=fillColor] ( 50.33,145.48) circle (  1.96);

\path[draw=drawColor,line width= 0.4pt,line join=round,line cap=round,fill=fillColor] ( 50.33,153.92) circle (  1.96);

\path[draw=drawColor,line width= 0.4pt,line join=round,line cap=round,fill=fillColor] ( 50.33,132.82) circle (  1.96);

\path[draw=drawColor,line width= 0.4pt,line join=round,line cap=round,fill=fillColor] ( 50.33,136.77) circle (  1.96);

\path[draw=drawColor,line width= 0.4pt,line join=round,line cap=round,fill=fillColor] ( 50.33,132.82) circle (  1.96);

\path[draw=drawColor,line width= 0.4pt,line join=round,line cap=round,fill=fillColor] ( 50.33,131.55) circle (  1.96);

\path[draw=drawColor,line width= 0.4pt,line join=round,line cap=round,fill=fillColor] ( 50.33,130.29) circle (  1.96);

\path[draw=drawColor,line width= 0.4pt,line join=round,line cap=round,fill=fillColor] ( 50.33,139.15) circle (  1.96);

\path[draw=drawColor,line width= 0.4pt,line join=round,line cap=round,fill=fillColor] ( 50.33,134.40) circle (  1.96);

\path[draw=drawColor,line width= 0.4pt,line join=round,line cap=round,fill=fillColor] ( 50.33,131.72) circle (  1.96);

\path[draw=drawColor,line width= 0.4pt,line join=round,line cap=round,fill=fillColor] ( 50.33,139.72) circle (  1.96);

\path[draw=drawColor,line width= 0.4pt,line join=round,line cap=round,fill=fillColor] ( 50.33,133.97) circle (  1.96);

\path[draw=drawColor,line width= 0.4pt,line join=round,line cap=round,fill=fillColor] ( 50.33,131.22) circle (  1.96);

\path[draw=drawColor,line width= 0.4pt,line join=round,line cap=round,fill=fillColor] ( 50.33,137.67) circle (  1.96);

\path[draw=drawColor,line width= 0.4pt,line join=round,line cap=round,fill=fillColor] ( 50.33,139.72) circle (  1.96);

\path[draw=drawColor,line width= 0.4pt,line join=round,line cap=round,fill=fillColor] ( 50.33,152.31) circle (  1.96);

\path[draw=drawColor,line width= 0.4pt,line join=round,line cap=round,fill=fillColor] ( 50.33,139.72) circle (  1.96);

\path[draw=drawColor,line width= 0.4pt,line join=round,line cap=round,fill=fillColor] ( 50.33,154.11) circle (  1.96);

\path[draw=drawColor,line width= 0.4pt,line join=round,line cap=round,fill=fillColor] ( 50.33,130.13) circle (  1.96);

\path[draw=drawColor,line width= 0.4pt,line join=round,line cap=round,fill=fillColor] ( 50.33,146.91) circle (  1.96);

\path[draw=drawColor,line width= 0.4pt,line join=round,line cap=round,fill=fillColor] ( 50.33,130.41) circle (  1.96);

\path[draw=drawColor,line width= 0.4pt,line join=round,line cap=round,fill=fillColor] ( 50.33,130.13) circle (  1.96);

\path[draw=drawColor,line width= 0.4pt,line join=round,line cap=round,fill=fillColor] ( 50.33,133.97) circle (  1.96);

\path[draw=drawColor,line width= 0.4pt,line join=round,line cap=round,fill=fillColor] ( 50.33,151.23) circle (  1.96);

\path[draw=drawColor,line width= 0.4pt,line join=round,line cap=round,fill=fillColor] ( 50.33,136.05) circle (  1.96);

\path[draw=drawColor,line width= 0.4pt,line join=round,line cap=round,fill=fillColor] ( 50.33,139.22) circle (  1.96);

\path[draw=drawColor,line width= 0.4pt,line join=round,line cap=round,fill=fillColor] ( 50.33,153.48) circle (  1.96);

\path[draw=drawColor,line width= 0.4pt,line join=round,line cap=round,fill=fillColor] ( 50.33,136.75) circle (  1.96);

\path[draw=drawColor,line width= 0.4pt,line join=round,line cap=round,fill=fillColor] ( 50.33,153.48) circle (  1.96);

\path[draw=drawColor,line width= 0.4pt,line join=round,line cap=round,fill=fillColor] ( 50.33,137.63) circle (  1.96);

\path[draw=drawColor,line width= 0.4pt,line join=round,line cap=round,fill=fillColor] ( 50.33,144.56) circle (  1.96);

\path[draw=drawColor,line width= 0.4pt,line join=round,line cap=round,fill=fillColor] ( 50.33,137.10) circle (  1.96);

\path[draw=drawColor,line width= 0.4pt,line join=round,line cap=round,fill=fillColor] ( 50.33,137.63) circle (  1.96);

\path[draw=drawColor,line width= 0.4pt,line join=round,line cap=round,fill=fillColor] ( 50.33,139.22) circle (  1.96);

\path[draw=drawColor,line width= 0.4pt,line join=round,line cap=round,fill=fillColor] ( 50.33,145.55) circle (  1.96);

\path[draw=drawColor,line width= 0.4pt,line join=round,line cap=round,fill=fillColor] ( 50.33,132.43) circle (  1.96);

\path[draw=drawColor,line width= 0.4pt,line join=round,line cap=round,fill=fillColor] ( 50.33,149.39) circle (  1.96);

\path[draw=drawColor,line width= 0.4pt,line join=round,line cap=round,fill=fillColor] ( 50.33,145.85) circle (  1.96);

\path[draw=drawColor,line width= 0.4pt,line join=round,line cap=round,fill=fillColor] ( 50.33,135.24) circle (  1.96);

\path[draw=drawColor,line width= 0.4pt,line join=round,line cap=round,fill=fillColor] ( 50.33,131.70) circle (  1.96);

\path[draw=drawColor,line width= 0.4pt,line join=round,line cap=round,fill=fillColor] ( 50.33,135.24) circle (  1.96);

\path[draw=drawColor,line width= 0.4pt,line join=round,line cap=round,fill=fillColor] ( 50.33,148.50) circle (  1.96);

\path[draw=drawColor,line width= 0.4pt,line join=round,line cap=round,fill=fillColor] ( 50.33,137.45) circle (  1.96);

\path[draw=drawColor,line width= 0.4pt,line join=round,line cap=round,fill=fillColor] ( 50.33,135.24) circle (  1.96);

\path[draw=drawColor,line width= 0.4pt,line join=round,line cap=round,fill=fillColor] ( 50.33,137.20) circle (  1.96);

\path[draw=drawColor,line width= 0.4pt,line join=round,line cap=round,fill=fillColor] ( 50.33,139.66) circle (  1.96);

\path[draw=drawColor,line width= 0.4pt,line join=round,line cap=round,fill=fillColor] ( 50.33,133.47) circle (  1.96);

\path[draw=drawColor,line width= 0.4pt,line join=round,line cap=round,fill=fillColor] ( 50.33,140.54) circle (  1.96);

\path[draw=drawColor,line width= 0.4pt,line join=round,line cap=round,fill=fillColor] ( 50.33,145.85) circle (  1.96);

\path[draw=drawColor,line width= 0.4pt,line join=round,line cap=round,fill=fillColor] ( 50.33,152.93) circle (  1.96);

\path[draw=drawColor,line width= 0.4pt,line join=round,line cap=round,fill=fillColor] ( 50.33,135.24) circle (  1.96);

\path[draw=drawColor,line width= 0.4pt,line join=round,line cap=round,fill=fillColor] ( 50.33,133.76) circle (  1.96);

\path[draw=drawColor,line width= 0.4pt,line join=round,line cap=round,fill=fillColor] ( 50.33,132.59) circle (  1.96);

\path[draw=drawColor,line width= 0.4pt,line join=round,line cap=round,fill=fillColor] ( 50.33,135.24) circle (  1.96);

\path[draw=drawColor,line width= 0.4pt,line join=round,line cap=round,fill=fillColor] ( 50.33,135.24) circle (  1.96);

\path[draw=drawColor,line width= 0.4pt,line join=round,line cap=round,fill=fillColor] ( 50.33,135.24) circle (  1.96);

\path[draw=drawColor,line width= 0.4pt,line join=round,line cap=round,fill=fillColor] ( 50.33,130.69) circle (  1.96);

\path[draw=drawColor,line width= 0.4pt,line join=round,line cap=round,fill=fillColor] ( 50.33,131.92) circle (  1.96);

\path[draw=drawColor,line width= 0.4pt,line join=round,line cap=round,fill=fillColor] ( 50.33,131.37) circle (  1.96);

\path[draw=drawColor,line width= 0.4pt,line join=round,line cap=round,fill=fillColor] ( 50.33,133.83) circle (  1.96);

\path[draw=drawColor,line width= 0.4pt,line join=round,line cap=round,fill=fillColor] ( 50.33,138.40) circle (  1.96);

\path[draw=drawColor,line width= 0.4pt,line join=round,line cap=round,fill=fillColor] ( 50.33,132.42) circle (  1.96);

\path[draw=drawColor,line width= 0.4pt,line join=round,line cap=round,fill=fillColor] ( 50.33,137.27) circle (  1.96);

\path[draw=drawColor,line width= 0.4pt,line join=round,line cap=round,fill=fillColor] ( 50.33,135.30) circle (  1.96);

\path[draw=drawColor,line width= 0.4pt,line join=round,line cap=round,fill=fillColor] ( 50.33,133.83) circle (  1.96);

\path[draw=drawColor,line width= 0.4pt,line join=round,line cap=round,fill=fillColor] ( 50.33,132.02) circle (  1.96);

\path[draw=drawColor,line width= 0.4pt,line join=round,line cap=round,fill=fillColor] ( 50.33,141.21) circle (  1.96);

\path[draw=drawColor,line width= 0.4pt,line join=round,line cap=round,fill=fillColor] ( 50.33,131.37) circle (  1.96);

\path[draw=drawColor,line width= 0.4pt,line join=round,line cap=round,fill=fillColor] ( 50.33,131.37) circle (  1.96);

\path[draw=drawColor,line width= 0.4pt,line join=round,line cap=round,fill=fillColor] ( 50.33,149.64) circle (  1.96);

\path[draw=drawColor,line width= 0.4pt,line join=round,line cap=round,fill=fillColor] ( 50.33,153.01) circle (  1.96);

\path[draw=drawColor,line width= 0.4pt,line join=round,line cap=round,fill=fillColor] ( 50.33,149.72) circle (  1.96);

\path[draw=drawColor,line width= 0.4pt,line join=round,line cap=round,fill=fillColor] ( 50.33,138.47) circle (  1.96);

\path[draw=drawColor,line width= 0.4pt,line join=round,line cap=round,fill=fillColor] ( 50.33,149.72) circle (  1.96);

\path[draw=drawColor,line width= 0.4pt,line join=round,line cap=round,fill=fillColor] ( 50.33,136.21) circle (  1.96);

\path[draw=drawColor,line width= 0.4pt,line join=round,line cap=round,fill=fillColor] ( 50.33,132.84) circle (  1.96);

\path[draw=drawColor,line width= 0.4pt,line join=round,line cap=round,fill=fillColor] ( 50.33,132.84) circle (  1.96);

\path[draw=drawColor,line width= 0.4pt,line join=round,line cap=round,fill=fillColor] ( 50.33,135.46) circle (  1.96);

\path[draw=drawColor,line width= 0.4pt,line join=round,line cap=round,fill=fillColor] ( 50.33,149.72) circle (  1.96);

\path[draw=drawColor,line width= 0.4pt,line join=round,line cap=round,fill=fillColor] ( 50.33,131.71) circle (  1.96);

\path[draw=drawColor,line width= 0.4pt,line join=round,line cap=round,fill=fillColor] ( 50.33,134.41) circle (  1.96);

\path[draw=drawColor,line width= 0.4pt,line join=round,line cap=round,fill=fillColor] ( 50.33,142.49) circle (  1.96);

\path[draw=drawColor,line width= 0.4pt,line join=round,line cap=round,fill=fillColor] ( 50.33,138.47) circle (  1.96);

\path[draw=drawColor,line width= 0.4pt,line join=round,line cap=round,fill=fillColor] ( 50.33,139.59) circle (  1.96);

\path[draw=drawColor,line width= 0.4pt,line join=round,line cap=round,fill=fillColor] ( 50.33,147.02) circle (  1.96);

\path[draw=drawColor,line width= 0.4pt,line join=round,line cap=round,fill=fillColor] ( 50.33,154.23) circle (  1.96);

\path[draw=drawColor,line width= 0.4pt,line join=round,line cap=round,fill=fillColor] ( 50.33,138.47) circle (  1.96);

\path[draw=drawColor,line width= 0.4pt,line join=round,line cap=round,fill=fillColor] ( 50.33,132.84) circle (  1.96);

\path[draw=drawColor,line width= 0.4pt,line join=round,line cap=round,fill=fillColor] ( 50.33,132.84) circle (  1.96);

\path[draw=drawColor,line width= 0.4pt,line join=round,line cap=round,fill=fillColor] ( 50.33,138.47) circle (  1.96);

\path[draw=drawColor,line width= 0.4pt,line join=round,line cap=round,fill=fillColor] ( 50.33,130.96) circle (  1.96);

\path[draw=drawColor,line width= 0.4pt,line join=round,line cap=round,fill=fillColor] ( 50.33,132.84) circle (  1.96);

\path[draw=drawColor,line width= 0.4pt,line join=round,line cap=round,fill=fillColor] ( 50.33,136.21) circle (  1.96);

\path[draw=drawColor,line width= 0.4pt,line join=round,line cap=round,fill=fillColor] ( 50.33,152.54) circle (  1.96);

\path[draw=drawColor,line width= 0.4pt,line join=round,line cap=round,fill=fillColor] ( 50.33,136.21) circle (  1.96);

\path[draw=drawColor,line width= 0.4pt,line join=round,line cap=round,fill=fillColor] ( 50.33,134.71) circle (  1.96);

\path[draw=drawColor,line width= 0.4pt,line join=round,line cap=round,fill=fillColor] ( 50.33,136.21) circle (  1.96);

\path[draw=drawColor,line width= 0.4pt,line join=round,line cap=round,fill=fillColor] ( 50.33,132.84) circle (  1.96);

\path[draw=drawColor,line width= 0.4pt,line join=round,line cap=round,fill=fillColor] ( 50.33,138.88) circle (  1.96);

\path[draw=drawColor,line width= 0.4pt,line join=round,line cap=round,fill=fillColor] ( 50.33,140.98) circle (  1.96);

\path[draw=drawColor,line width= 0.4pt,line join=round,line cap=round,fill=fillColor] ( 50.33,132.58) circle (  1.96);

\path[draw=drawColor,line width= 0.4pt,line join=round,line cap=round,fill=fillColor] ( 50.33,138.88) circle (  1.96);

\path[draw=drawColor,line width= 0.4pt,line join=round,line cap=round,fill=fillColor] ( 50.33,134.68) circle (  1.96);

\path[draw=drawColor,line width= 0.4pt,line join=round,line cap=round,fill=fillColor] ( 50.33,132.58) circle (  1.96);

\path[draw=drawColor,line width= 0.4pt,line join=round,line cap=round,fill=fillColor] ( 50.33,132.58) circle (  1.96);

\path[draw=drawColor,line width= 0.4pt,line join=round,line cap=round,fill=fillColor] ( 50.33,132.58) circle (  1.96);

\path[draw=drawColor,line width= 0.4pt,line join=round,line cap=round,fill=fillColor] ( 50.33,136.05) circle (  1.96);

\path[draw=drawColor,line width= 0.4pt,line join=round,line cap=round,fill=fillColor] ( 50.33,132.58) circle (  1.96);

\path[draw=drawColor,line width= 0.4pt,line join=round,line cap=round,fill=fillColor] ( 50.33,138.88) circle (  1.96);

\path[draw=drawColor,line width= 0.4pt,line join=round,line cap=round,fill=fillColor] ( 50.33,140.02) circle (  1.96);

\path[draw=drawColor,line width= 0.4pt,line join=round,line cap=round,fill=fillColor] ( 50.33,148.33) circle (  1.96);

\path[draw=drawColor,line width= 0.4pt,line join=round,line cap=round,fill=fillColor] ( 50.33,135.10) circle (  1.96);

\path[draw=drawColor,line width= 0.4pt,line join=round,line cap=round,fill=fillColor] ( 50.33,152.74) circle (  1.96);

\path[draw=drawColor,line width= 0.4pt,line join=round,line cap=round,fill=fillColor] ( 50.33,132.58) circle (  1.96);

\path[draw=drawColor,line width= 0.4pt,line join=round,line cap=round,fill=fillColor] ( 50.33,138.88) circle (  1.96);

\path[draw=drawColor,line width= 0.4pt,line join=round,line cap=round,fill=fillColor] ( 50.33,132.58) circle (  1.96);

\path[draw=drawColor,line width= 0.4pt,line join=round,line cap=round,fill=fillColor] ( 50.33,135.10) circle (  1.96);

\path[draw=drawColor,line width= 0.4pt,line join=round,line cap=round,fill=fillColor] ( 50.33,145.18) circle (  1.96);

\path[draw=drawColor,line width= 0.4pt,line join=round,line cap=round,fill=fillColor] ( 50.33,151.30) circle (  1.96);

\path[draw=drawColor,line width= 0.4pt,line join=round,line cap=round,fill=fillColor] ( 50.33,132.58) circle (  1.96);

\path[draw=drawColor,line width= 0.4pt,line join=round,line cap=round,fill=fillColor] ( 50.33,137.62) circle (  1.96);

\path[draw=drawColor,line width= 0.4pt,line join=round,line cap=round,fill=fillColor] ( 50.33,142.66) circle (  1.96);

\path[draw=drawColor,line width= 0.4pt,line join=round,line cap=round,fill=fillColor] ( 50.33,133.26) circle (  1.96);

\path[draw=drawColor,line width= 0.4pt,line join=round,line cap=round,fill=fillColor] ( 50.33,138.88) circle (  1.96);

\path[draw=drawColor,line width= 0.4pt,line join=round,line cap=round,fill=fillColor] ( 50.33,147.44) circle (  1.96);

\path[draw=drawColor,line width= 0.4pt,line join=round,line cap=round,fill=fillColor] ( 50.33,132.68) circle (  1.96);

\path[draw=drawColor,line width= 0.4pt,line join=round,line cap=round,fill=fillColor] ( 50.33,133.16) circle (  1.96);

\path[draw=drawColor,line width= 0.4pt,line join=round,line cap=round,fill=fillColor] ( 50.33,143.36) circle (  1.96);

\path[draw=drawColor,line width= 0.4pt,line join=round,line cap=round,fill=fillColor] ( 50.33,130.73) circle (  1.96);

\path[draw=drawColor,line width= 0.4pt,line join=round,line cap=round,fill=fillColor] ( 50.33,131.12) circle (  1.96);

\path[draw=drawColor,line width= 0.4pt,line join=round,line cap=round,fill=fillColor] ( 50.33,131.12) circle (  1.96);

\path[draw=drawColor,line width= 0.4pt,line join=round,line cap=round,fill=fillColor] ( 50.33,136.56) circle (  1.96);

\path[draw=drawColor,line width= 0.4pt,line join=round,line cap=round,fill=fillColor] ( 50.33,131.12) circle (  1.96);

\path[draw=drawColor,line width= 0.4pt,line join=round,line cap=round,fill=fillColor] ( 50.33,145.63) circle (  1.96);

\path[draw=drawColor,line width= 0.4pt,line join=round,line cap=round,fill=fillColor] ( 50.33,152.10) circle (  1.96);

\path[draw=drawColor,line width= 0.4pt,line join=round,line cap=round,fill=fillColor] ( 50.33,143.36) circle (  1.96);

\path[draw=drawColor,line width= 0.4pt,line join=round,line cap=round,fill=fillColor] ( 50.33,134.09) circle (  1.96);

\path[draw=drawColor,line width= 0.4pt,line join=round,line cap=round,fill=fillColor] ( 50.33,131.12) circle (  1.96);

\path[draw=drawColor,line width= 0.4pt,line join=round,line cap=round,fill=fillColor] ( 50.33,140.81) circle (  1.96);

\path[draw=drawColor,line width= 0.4pt,line join=round,line cap=round,fill=fillColor] ( 50.33,143.36) circle (  1.96);

\path[draw=drawColor,line width= 0.4pt,line join=round,line cap=round,fill=fillColor] ( 50.33,131.12) circle (  1.96);

\path[draw=drawColor,line width= 0.4pt,line join=round,line cap=round,fill=fillColor] ( 50.33,131.12) circle (  1.96);

\path[draw=drawColor,line width= 0.4pt,line join=round,line cap=round,fill=fillColor] ( 50.33,131.12) circle (  1.96);

\path[draw=drawColor,line width= 0.4pt,line join=round,line cap=round,fill=fillColor] ( 50.33,150.16) circle (  1.96);

\path[draw=drawColor,line width= 0.4pt,line join=round,line cap=round,fill=fillColor] ( 50.33,131.12) circle (  1.96);

\path[draw=drawColor,line width= 0.4pt,line join=round,line cap=round,fill=fillColor] ( 50.33,143.36) circle (  1.96);

\path[draw=drawColor,line width= 0.4pt,line join=round,line cap=round,fill=fillColor] ( 50.33,141.32) circle (  1.96);

\path[draw=drawColor,line width= 0.4pt,line join=round,line cap=round,fill=fillColor] ( 50.33,142.60) circle (  1.96);

\path[draw=drawColor,line width= 0.4pt,line join=round,line cap=round,fill=fillColor] ( 50.33,150.16) circle (  1.96);

\path[draw=drawColor,line width= 0.4pt,line join=round,line cap=round,fill=fillColor] ( 50.33,140.45) circle (  1.96);

\path[draw=drawColor,line width= 0.4pt,line join=round,line cap=round,fill=fillColor] ( 50.33,131.12) circle (  1.96);

\path[draw=drawColor,line width= 0.4pt,line join=round,line cap=round,fill=fillColor] ( 50.33,136.56) circle (  1.96);

\path[draw=drawColor,line width= 0.4pt,line join=round,line cap=round,fill=fillColor] ( 50.33,131.12) circle (  1.96);

\path[draw=drawColor,line width= 0.4pt,line join=round,line cap=round,fill=fillColor] ( 50.33,143.36) circle (  1.96);

\path[draw=drawColor,line width= 0.4pt,line join=round,line cap=round,fill=fillColor] ( 50.33,150.16) circle (  1.96);

\path[draw=drawColor,line width= 0.4pt,line join=round,line cap=round,fill=fillColor] ( 50.33,135.20) circle (  1.96);

\path[draw=drawColor,line width= 0.4pt,line join=round,line cap=round,fill=fillColor] ( 50.33,139.28) circle (  1.96);

\path[draw=drawColor,line width= 0.4pt,line join=round,line cap=round,fill=fillColor] ( 50.33,136.56) circle (  1.96);

\path[draw=drawColor,line width= 0.4pt,line join=round,line cap=round,fill=fillColor] ( 50.33,147.44) circle (  1.96);

\path[draw=drawColor,line width= 0.4pt,line join=round,line cap=round,fill=fillColor] ( 50.33,147.44) circle (  1.96);

\path[draw=drawColor,line width= 0.4pt,line join=round,line cap=round,fill=fillColor] ( 50.33,134.39) circle (  1.96);

\path[draw=drawColor,line width= 0.4pt,line join=round,line cap=round,fill=fillColor] ( 50.33,136.56) circle (  1.96);

\path[draw=drawColor,line width= 0.4pt,line join=round,line cap=round,fill=fillColor] ( 50.33,151.52) circle (  1.96);

\path[draw=drawColor,line width= 0.4pt,line join=round,line cap=round,fill=fillColor] ( 50.33,131.71) circle (  1.96);

\path[draw=drawColor,line width= 0.4pt,line join=round,line cap=round,fill=fillColor] ( 50.33,134.19) circle (  1.96);

\path[draw=drawColor,line width= 0.4pt,line join=round,line cap=round,fill=fillColor] ( 50.33,136.35) circle (  1.96);

\path[draw=drawColor,line width= 0.4pt,line join=round,line cap=round,fill=fillColor] ( 50.33,136.35) circle (  1.96);

\path[draw=drawColor,line width= 0.4pt,line join=round,line cap=round,fill=fillColor] ( 50.33,142.85) circle (  1.96);

\path[draw=drawColor,line width= 0.4pt,line join=round,line cap=round,fill=fillColor] ( 50.33,134.19) circle (  1.96);

\path[draw=drawColor,line width= 0.4pt,line join=round,line cap=round,fill=fillColor] ( 50.33,139.96) circle (  1.96);

\path[draw=drawColor,line width= 0.4pt,line join=round,line cap=round,fill=fillColor] ( 50.33,131.71) circle (  1.96);

\path[draw=drawColor,line width= 0.4pt,line join=round,line cap=round,fill=fillColor] ( 50.33,139.96) circle (  1.96);

\path[draw=drawColor,line width= 0.4pt,line join=round,line cap=round,fill=fillColor] ( 50.33,141.28) circle (  1.96);

\path[draw=drawColor,line width= 0.4pt,line join=round,line cap=round,fill=fillColor] ( 50.33,134.19) circle (  1.96);

\path[draw=drawColor,line width= 0.4pt,line join=round,line cap=round,fill=fillColor] ( 50.33,146.57) circle (  1.96);

\path[draw=drawColor,line width= 0.4pt,line join=round,line cap=round,fill=fillColor] ( 50.33,147.19) circle (  1.96);

\path[draw=drawColor,line width= 0.4pt,line join=round,line cap=round,fill=fillColor] ( 50.33,139.96) circle (  1.96);

\path[draw=drawColor,line width= 0.4pt,line join=round,line cap=round,fill=fillColor] ( 50.33,141.28) circle (  1.96);

\path[draw=drawColor,line width= 0.4pt,line join=round,line cap=round,fill=fillColor] ( 50.33,131.71) circle (  1.96);

\path[draw=drawColor,line width= 0.4pt,line join=round,line cap=round,fill=fillColor] ( 50.33,132.02) circle (  1.96);

\path[draw=drawColor,line width= 0.4pt,line join=round,line cap=round,fill=fillColor] ( 50.33,138.16) circle (  1.96);

\path[draw=drawColor,line width= 0.4pt,line join=round,line cap=round,fill=fillColor] ( 50.33,144.09) circle (  1.96);

\path[draw=drawColor,line width= 0.4pt,line join=round,line cap=round,fill=fillColor] ( 50.33,151.52) circle (  1.96);

\path[draw=drawColor,line width= 0.4pt,line join=round,line cap=round,fill=fillColor] ( 50.33,151.52) circle (  1.96);

\path[draw=drawColor,line width= 0.4pt,line join=round,line cap=round,fill=fillColor] ( 50.33,139.96) circle (  1.96);

\path[draw=drawColor,line width= 0.4pt,line join=round,line cap=round,fill=fillColor] ( 50.33,135.15) circle (  1.96);

\path[draw=drawColor,line width= 0.4pt,line join=round,line cap=round,fill=fillColor] ( 50.33,132.74) circle (  1.96);

\path[draw=drawColor,line width= 0.4pt,line join=round,line cap=round,fill=fillColor] ( 50.33,149.60) circle (  1.96);

\path[draw=drawColor,line width= 0.4pt,line join=round,line cap=round,fill=fillColor] ( 50.33,136.35) circle (  1.96);

\path[draw=drawColor,line width= 0.4pt,line join=round,line cap=round,fill=fillColor] ( 50.33,138.52) circle (  1.96);

\path[draw=drawColor,line width= 0.4pt,line join=round,line cap=round,fill=fillColor] ( 50.33,134.19) circle (  1.96);

\path[draw=drawColor,line width= 0.4pt,line join=round,line cap=round,fill=fillColor] ( 50.33,138.71) circle (  1.96);

\path[draw=drawColor,line width= 0.4pt,line join=round,line cap=round,fill=fillColor] ( 50.33,147.19) circle (  1.96);

\path[draw=drawColor,line width= 0.4pt,line join=round,line cap=round,fill=fillColor] ( 50.33,144.09) circle (  1.96);

\path[draw=drawColor,line width= 0.4pt,line join=round,line cap=round,fill=fillColor] ( 50.33,147.67) circle (  1.96);

\path[draw=drawColor,line width= 0.4pt,line join=round,line cap=round,fill=fillColor] ( 50.33,132.45) circle (  1.96);

\path[draw=drawColor,line width= 0.4pt,line join=round,line cap=round,fill=fillColor] ( 50.33,139.96) circle (  1.96);

\path[draw=drawColor,line width= 0.4pt,line join=round,line cap=round,fill=fillColor] ( 50.33,136.35) circle (  1.96);

\path[draw=drawColor,line width= 0.4pt,line join=round,line cap=round,fill=fillColor] ( 50.33,154.67) circle (  1.96);

\path[draw=drawColor,line width= 0.4pt,line join=round,line cap=round,fill=fillColor] ( 50.33,154.67) circle (  1.96);

\path[draw=drawColor,line width= 0.4pt,line join=round,line cap=round,fill=fillColor] ( 50.33,132.52) circle (  1.96);

\path[draw=drawColor,line width= 0.4pt,line join=round,line cap=round,fill=fillColor] ( 50.33,136.55) circle (  1.96);

\path[draw=drawColor,line width= 0.4pt,line join=round,line cap=round,fill=fillColor] ( 50.33,154.67) circle (  1.96);

\path[draw=drawColor,line width= 0.4pt,line join=round,line cap=round,fill=fillColor] ( 50.33,132.02) circle (  1.96);

\path[draw=drawColor,line width= 0.4pt,line join=round,line cap=round,fill=fillColor] ( 50.33,130.51) circle (  1.96);

\path[draw=drawColor,line width= 0.4pt,line join=round,line cap=round,fill=fillColor] ( 50.33,138.82) circle (  1.96);

\path[draw=drawColor,line width= 0.4pt,line join=round,line cap=round,fill=fillColor] ( 50.33,130.51) circle (  1.96);

\path[draw=drawColor,line width= 0.4pt,line join=round,line cap=round,fill=fillColor] ( 50.33,132.02) circle (  1.96);

\path[draw=drawColor,line width= 0.4pt,line join=round,line cap=round,fill=fillColor] ( 50.33,139.85) circle (  1.96);

\path[draw=drawColor,line width= 0.4pt,line join=round,line cap=round,fill=fillColor] ( 50.33,130.32) circle (  1.96);

\path[draw=drawColor,line width= 0.4pt,line join=round,line cap=round,fill=fillColor] ( 50.33,132.02) circle (  1.96);

\path[draw=drawColor,line width= 0.4pt,line join=round,line cap=round,fill=fillColor] ( 50.33,133.53) circle (  1.96);

\path[draw=drawColor,line width= 0.4pt,line join=round,line cap=round,fill=fillColor] ( 50.33,138.82) circle (  1.96);

\path[draw=drawColor,line width= 0.4pt,line join=round,line cap=round,fill=fillColor] ( 50.33,154.67) circle (  1.96);

\path[draw=drawColor,line width= 0.4pt,line join=round,line cap=round,fill=fillColor] ( 50.33,142.59) circle (  1.96);

\path[draw=drawColor,line width= 0.4pt,line join=round,line cap=round,fill=fillColor] ( 50.33,142.59) circle (  1.96);

\path[draw=drawColor,line width= 0.4pt,line join=round,line cap=round,fill=fillColor] ( 50.33,136.55) circle (  1.96);

\path[draw=drawColor,line width= 0.4pt,line join=round,line cap=round,fill=fillColor] ( 50.33,137.56) circle (  1.96);

\path[draw=drawColor,line width= 0.4pt,line join=round,line cap=round,fill=fillColor] ( 50.33,139.85) circle (  1.96);

\path[draw=drawColor,line width= 0.4pt,line join=round,line cap=round,fill=fillColor] ( 50.33,136.55) circle (  1.96);

\path[draw=drawColor,line width= 0.4pt,line join=round,line cap=round,fill=fillColor] ( 50.33,150.14) circle (  1.96);

\path[draw=drawColor,line width= 0.4pt,line join=round,line cap=round,fill=fillColor] ( 50.33,131.61) circle (  1.96);

\path[draw=drawColor,line width= 0.4pt,line join=round,line cap=round,fill=fillColor] ( 50.33,130.38) circle (  1.96);

\path[draw=drawColor,line width= 0.4pt,line join=round,line cap=round,fill=fillColor] ( 50.33,138.06) circle (  1.96);

\path[draw=drawColor,line width= 0.4pt,line join=round,line cap=round,fill=fillColor] ( 50.33,136.55) circle (  1.96);

\path[draw=drawColor,line width= 0.4pt,line join=round,line cap=round,fill=fillColor] ( 50.33,138.77) circle (  1.96);

\path[draw=drawColor,line width= 0.4pt,line join=round,line cap=round,fill=fillColor] ( 50.33,141.91) circle (  1.96);

\path[draw=drawColor,line width= 0.4pt,line join=round,line cap=round,fill=fillColor] ( 50.33,142.54) circle (  1.96);

\path[draw=drawColor,line width= 0.4pt,line join=round,line cap=round,fill=fillColor] ( 50.33,145.06) circle (  1.96);

\path[draw=drawColor,line width= 0.4pt,line join=round,line cap=round,fill=fillColor] ( 50.33,143.49) circle (  1.96);

\path[draw=drawColor,line width= 0.4pt,line join=round,line cap=round,fill=fillColor] ( 50.33,131.30) circle (  1.96);

\path[draw=drawColor,line width= 0.4pt,line join=round,line cap=round,fill=fillColor] ( 50.33,131.96) circle (  1.96);

\path[draw=drawColor,line width= 0.4pt,line join=round,line cap=round,fill=fillColor] ( 50.33,148.99) circle (  1.96);

\path[draw=drawColor,line width= 0.4pt,line join=round,line cap=round,fill=fillColor] ( 50.33,131.30) circle (  1.96);

\path[draw=drawColor,line width= 0.4pt,line join=round,line cap=round,fill=fillColor] ( 50.33,141.13) circle (  1.96);

\path[draw=drawColor,line width= 0.4pt,line join=round,line cap=round,fill=fillColor] ( 50.33,145.06) circle (  1.96);

\path[draw=drawColor,line width= 0.4pt,line join=round,line cap=round,fill=fillColor] ( 50.33,141.13) circle (  1.96);

\path[draw=drawColor,line width= 0.4pt,line join=round,line cap=round,fill=fillColor] ( 50.33,145.06) circle (  1.96);

\path[draw=drawColor,line width= 0.4pt,line join=round,line cap=round,fill=fillColor] ( 50.33,132.48) circle (  1.96);

\path[draw=drawColor,line width= 0.4pt,line join=round,line cap=round,fill=fillColor] ( 50.33,138.77) circle (  1.96);

\path[draw=drawColor,line width= 0.4pt,line join=round,line cap=round,fill=fillColor] ( 50.33,141.13) circle (  1.96);

\path[draw=drawColor,line width= 0.4pt,line join=round,line cap=round,fill=fillColor] ( 50.33,136.08) circle (  1.96);

\path[draw=drawColor,line width= 0.4pt,line join=round,line cap=round,fill=fillColor] ( 50.33,134.58) circle (  1.96);

\path[draw=drawColor,line width= 0.4pt,line join=round,line cap=round,fill=fillColor] ( 50.33,136.41) circle (  1.96);

\path[draw=drawColor,line width= 0.4pt,line join=round,line cap=round,fill=fillColor] ( 50.33,134.58) circle (  1.96);

\path[draw=drawColor,line width= 0.4pt,line join=round,line cap=round,fill=fillColor] ( 50.33,145.06) circle (  1.96);

\path[draw=drawColor,line width= 0.4pt,line join=round,line cap=round,fill=fillColor] ( 50.33,136.08) circle (  1.96);

\path[draw=drawColor,line width= 0.4pt,line join=round,line cap=round,fill=fillColor] ( 50.33,138.77) circle (  1.96);

\path[draw=drawColor,line width= 0.4pt,line join=round,line cap=round,fill=fillColor] ( 50.33,135.77) circle (  1.96);

\path[draw=drawColor,line width= 0.4pt,line join=round,line cap=round,fill=fillColor] ( 50.33,141.13) circle (  1.96);

\path[draw=drawColor,line width= 0.4pt,line join=round,line cap=round,fill=fillColor] ( 50.33,149.55) circle (  1.96);

\path[draw=drawColor,line width= 0.4pt,line join=round,line cap=round,fill=fillColor] ( 50.33,145.06) circle (  1.96);

\path[draw=drawColor,line width= 0.4pt,line join=round,line cap=round,fill=fillColor] ( 50.33,154.59) circle (  1.96);

\path[draw=drawColor,line width= 0.4pt,line join=round,line cap=round,fill=fillColor] ( 50.33,132.31) circle (  1.96);

\path[draw=drawColor,line width= 0.4pt,line join=round,line cap=round,fill=fillColor] ( 50.33,139.47) circle (  1.96);

\path[draw=drawColor,line width= 0.4pt,line join=round,line cap=round,fill=fillColor] ( 50.33,138.99) circle (  1.96);

\path[draw=drawColor,line width= 0.4pt,line join=round,line cap=round,fill=fillColor] ( 50.33,134.40) circle (  1.96);

\path[draw=drawColor,line width= 0.4pt,line join=round,line cap=round,fill=fillColor] ( 50.33,132.31) circle (  1.96);

\path[draw=drawColor,line width= 0.4pt,line join=round,line cap=round,fill=fillColor] ( 50.33,142.34) circle (  1.96);

\path[draw=drawColor,line width= 0.4pt,line join=round,line cap=round,fill=fillColor] ( 50.33,132.31) circle (  1.96);

\path[draw=drawColor,line width= 0.4pt,line join=round,line cap=round,fill=fillColor] ( 50.33,130.30) circle (  1.96);

\path[draw=drawColor,line width= 0.4pt,line join=round,line cap=round,fill=fillColor] ( 50.33,132.31) circle (  1.96);

\path[draw=drawColor,line width= 0.4pt,line join=round,line cap=round,fill=fillColor] ( 50.33,130.30) circle (  1.96);

\path[draw=drawColor,line width= 0.4pt,line join=round,line cap=round,fill=fillColor] ( 50.33,134.82) circle (  1.96);

\path[draw=drawColor,line width= 0.4pt,line join=round,line cap=round,fill=fillColor] ( 50.33,132.31) circle (  1.96);

\path[draw=drawColor,line width= 0.4pt,line join=round,line cap=round,fill=fillColor] ( 50.33,132.31) circle (  1.96);

\path[draw=drawColor,line width= 0.4pt,line join=round,line cap=round,fill=fillColor] ( 50.33,132.31) circle (  1.96);

\path[draw=drawColor,line width= 0.4pt,line join=round,line cap=round,fill=fillColor] ( 50.33,142.34) circle (  1.96);

\path[draw=drawColor,line width= 0.4pt,line join=round,line cap=round,fill=fillColor] ( 50.33,140.66) circle (  1.96);

\path[draw=drawColor,line width= 0.4pt,line join=round,line cap=round,fill=fillColor] ( 50.33,132.31) circle (  1.96);

\path[draw=drawColor,line width= 0.4pt,line join=round,line cap=round,fill=fillColor] ( 50.33,146.45) circle (  1.96);

\path[draw=drawColor,line width= 0.4pt,line join=round,line cap=round,fill=fillColor] ( 50.33,142.34) circle (  1.96);

\path[draw=drawColor,line width= 0.4pt,line join=round,line cap=round,fill=fillColor] ( 50.33,132.31) circle (  1.96);

\path[draw=drawColor,line width= 0.4pt,line join=round,line cap=round,fill=fillColor] ( 50.33,144.84) circle (  1.96);

\path[draw=drawColor,line width= 0.4pt,line join=round,line cap=round,fill=fillColor] ( 50.33,139.83) circle (  1.96);

\path[draw=drawColor,line width= 0.4pt,line join=round,line cap=round,fill=fillColor] ( 50.33,154.59) circle (  1.96);

\path[draw=drawColor,line width= 0.4pt,line join=round,line cap=round,fill=fillColor] ( 50.33,142.75) circle (  1.96);

\path[draw=drawColor,line width= 0.4pt,line join=round,line cap=round,fill=fillColor] ( 50.33,140.66) circle (  1.96);

\path[draw=drawColor,line width= 0.4pt,line join=round,line cap=round,fill=fillColor] ( 50.33,134.40) circle (  1.96);

\path[draw=drawColor,line width= 0.4pt,line join=round,line cap=round,fill=fillColor] ( 50.33,147.35) circle (  1.96);

\path[draw=drawColor,line width= 0.4pt,line join=round,line cap=round,fill=fillColor] ( 50.33,134.40) circle (  1.96);

\path[draw=drawColor,line width= 0.4pt,line join=round,line cap=round,fill=fillColor] ( 50.33,132.31) circle (  1.96);

\path[draw=drawColor,line width= 0.4pt,line join=round,line cap=round,fill=fillColor] ( 50.33,140.66) circle (  1.96);

\path[draw=drawColor,line width= 0.4pt,line join=round,line cap=round,fill=fillColor] ( 50.33,132.31) circle (  1.96);

\path[draw=drawColor,line width= 0.4pt,line join=round,line cap=round,fill=fillColor] ( 50.33,142.34) circle (  1.96);

\path[draw=drawColor,line width= 0.4pt,line join=round,line cap=round,fill=fillColor] ( 50.33,132.31) circle (  1.96);

\path[draw=drawColor,line width= 0.4pt,line join=round,line cap=round,fill=fillColor] ( 50.33,135.65) circle (  1.96);

\path[draw=drawColor,line width= 0.4pt,line join=round,line cap=round,fill=fillColor] ( 50.33,141.86) circle (  1.96);

\path[draw=drawColor,line width= 0.4pt,line join=round,line cap=round,fill=fillColor] ( 50.33,135.65) circle (  1.96);

\path[draw=drawColor,line width= 0.4pt,line join=round,line cap=round,fill=fillColor] ( 50.33,139.47) circle (  1.96);

\path[draw=drawColor,line width= 0.4pt,line join=round,line cap=round,fill=fillColor] ( 50.33,141.71) circle (  1.96);

\path[draw=drawColor,line width= 0.4pt,line join=round,line cap=round,fill=fillColor] ( 50.33,135.65) circle (  1.96);

\path[draw=drawColor,line width= 0.4pt,line join=round,line cap=round,fill=fillColor] ( 50.33,132.31) circle (  1.96);

\path[draw=drawColor,line width= 0.4pt,line join=round,line cap=round,fill=fillColor] ( 50.33,137.88) circle (  1.96);

\path[draw=drawColor,line width= 0.4pt,line join=round,line cap=round,fill=fillColor] ( 50.33,151.50) circle (  1.96);

\path[draw=drawColor,line width= 0.4pt,line join=round,line cap=round,fill=fillColor] ( 50.33,134.17) circle (  1.96);

\path[draw=drawColor,line width= 0.4pt,line join=round,line cap=round,fill=fillColor] ( 50.33,134.17) circle (  1.96);

\path[draw=drawColor,line width= 0.4pt,line join=round,line cap=round,fill=fillColor] ( 50.33,137.06) circle (  1.96);

\path[draw=drawColor,line width= 0.4pt,line join=round,line cap=round,fill=fillColor] ( 50.33,154.96) circle (  1.96);

\path[draw=drawColor,line width= 0.4pt,line join=round,line cap=round,fill=fillColor] ( 50.33,139.37) circle (  1.96);

\path[draw=drawColor,line width= 0.4pt,line join=round,line cap=round,fill=fillColor] ( 50.33,151.50) circle (  1.96);

\path[draw=drawColor,line width= 0.4pt,line join=round,line cap=round,fill=fillColor] ( 50.33,136.25) circle (  1.96);

\path[draw=drawColor,line width= 0.4pt,line join=round,line cap=round,fill=fillColor] ( 50.33,134.17) circle (  1.96);

\path[draw=drawColor,line width= 0.4pt,line join=round,line cap=round,fill=fillColor] ( 50.33,136.25) circle (  1.96);

\path[draw=drawColor,line width= 0.4pt,line join=round,line cap=round,fill=fillColor] ( 50.33,131.69) circle (  1.96);

\path[draw=drawColor,line width= 0.4pt,line join=round,line cap=round,fill=fillColor] ( 50.33,134.17) circle (  1.96);

\path[draw=drawColor,line width= 0.4pt,line join=round,line cap=round,fill=fillColor] ( 50.33,130.70) circle (  1.96);

\path[draw=drawColor,line width= 0.4pt,line join=round,line cap=round,fill=fillColor] ( 50.33,142.83) circle (  1.96);

\path[draw=drawColor,line width= 0.4pt,line join=round,line cap=round,fill=fillColor] ( 50.33,139.37) circle (  1.96);

\path[draw=drawColor,line width= 0.4pt,line join=round,line cap=round,fill=fillColor] ( 50.33,134.17) circle (  1.96);

\path[draw=drawColor,line width= 0.4pt,line join=round,line cap=round,fill=fillColor] ( 50.33,134.17) circle (  1.96);

\path[draw=drawColor,line width= 0.4pt,line join=round,line cap=round,fill=fillColor] ( 50.33,145.22) circle (  1.96);

\path[draw=drawColor,line width= 0.4pt,line join=round,line cap=round,fill=fillColor] ( 50.33,134.17) circle (  1.96);

\path[draw=drawColor,line width= 0.4pt,line join=round,line cap=round,fill=fillColor] ( 50.33,134.17) circle (  1.96);

\path[draw=drawColor,line width= 0.4pt,line join=round,line cap=round,fill=fillColor] ( 50.33,142.83) circle (  1.96);

\path[draw=drawColor,line width= 0.4pt,line join=round,line cap=round,fill=fillColor] ( 50.33,134.17) circle (  1.96);

\path[draw=drawColor,line width= 0.4pt,line join=round,line cap=round,fill=fillColor] ( 50.33,144.57) circle (  1.96);

\path[draw=drawColor,line width= 0.4pt,line join=round,line cap=round,fill=fillColor] ( 50.33,136.33) circle (  1.96);

\path[draw=drawColor,line width= 0.4pt,line join=round,line cap=round,fill=fillColor] ( 50.33,134.17) circle (  1.96);

\path[draw=drawColor,line width= 0.4pt,line join=round,line cap=round,fill=fillColor] ( 50.33,134.17) circle (  1.96);

\path[draw=drawColor,line width= 0.4pt,line join=round,line cap=round,fill=fillColor] ( 50.33,140.41) circle (  1.96);

\path[draw=drawColor,line width= 0.4pt,line join=round,line cap=round,fill=fillColor] ( 50.33,132.09) circle (  1.96);

\path[draw=drawColor,line width= 0.4pt,line join=round,line cap=round,fill=fillColor] ( 50.33,134.17) circle (  1.96);

\path[draw=drawColor,line width= 0.4pt,line join=round,line cap=round,fill=fillColor] ( 50.33,134.17) circle (  1.96);

\path[draw=drawColor,line width= 0.4pt,line join=round,line cap=round,fill=fillColor] ( 50.33,134.17) circle (  1.96);

\path[draw=drawColor,line width= 0.4pt,line join=round,line cap=round,fill=fillColor] ( 50.33,134.17) circle (  1.96);

\path[draw=drawColor,line width= 0.4pt,line join=round,line cap=round,fill=fillColor] ( 50.33,152.36) circle (  1.96);

\path[draw=drawColor,line width= 0.4pt,line join=round,line cap=round,fill=fillColor] ( 50.33,151.50) circle (  1.96);

\path[draw=drawColor,line width= 0.4pt,line join=round,line cap=round,fill=fillColor] ( 50.33,136.25) circle (  1.96);

\path[draw=drawColor,line width= 0.4pt,line join=round,line cap=round,fill=fillColor] ( 50.33,134.17) circle (  1.96);

\path[draw=drawColor,line width= 0.4pt,line join=round,line cap=round,fill=fillColor] ( 50.33,140.41) circle (  1.96);

\path[draw=drawColor,line width= 0.4pt,line join=round,line cap=round,fill=fillColor] ( 50.33,144.57) circle (  1.96);

\path[draw=drawColor,line width= 0.4pt,line join=round,line cap=round,fill=fillColor] ( 50.33,139.37) circle (  1.96);

\path[draw=drawColor,line width= 0.4pt,line join=round,line cap=round,fill=fillColor] ( 50.33,134.17) circle (  1.96);

\path[draw=drawColor,line width= 0.4pt,line join=round,line cap=round,fill=fillColor] ( 50.33,130.70) circle (  1.96);

\path[draw=drawColor,line width= 0.4pt,line join=round,line cap=round,fill=fillColor] ( 50.33,140.41) circle (  1.96);

\path[draw=drawColor,line width= 0.4pt,line join=round,line cap=round,fill=fillColor] ( 50.33,134.17) circle (  1.96);

\path[draw=drawColor,line width= 0.4pt,line join=round,line cap=round,fill=fillColor] ( 50.33,130.32) circle (  1.96);

\path[draw=drawColor,line width= 0.4pt,line join=round,line cap=round,fill=fillColor] ( 50.33,153.07) circle (  1.96);

\path[draw=drawColor,line width= 0.4pt,line join=round,line cap=round,fill=fillColor] ( 50.33,131.57) circle (  1.96);

\path[draw=drawColor,line width= 0.4pt,line join=round,line cap=round,fill=fillColor] ( 50.33,144.07) circle (  1.96);

\path[draw=drawColor,line width= 0.4pt,line join=round,line cap=round,fill=fillColor] ( 50.33,134.17) circle (  1.96);

\path[draw=drawColor,line width= 0.4pt,line join=round,line cap=round,fill=fillColor] ( 50.33,140.41) circle (  1.96);

\path[draw=drawColor,line width= 0.4pt,line join=round,line cap=round,fill=fillColor] ( 50.33,135.70) circle (  1.96);

\path[draw=drawColor,line width= 0.4pt,line join=round,line cap=round,fill=fillColor] ( 50.33,131.86) circle (  1.96);

\path[draw=drawColor,line width= 0.4pt,line join=round,line cap=round,fill=fillColor] ( 50.33,139.37) circle (  1.96);

\path[draw=drawColor,line width= 0.4pt,line join=round,line cap=round,fill=fillColor] ( 50.33,151.50) circle (  1.96);

\path[draw=drawColor,line width= 0.4pt,line join=round,line cap=round,fill=fillColor] ( 50.33,144.57) circle (  1.96);

\path[draw=drawColor,line width= 0.4pt,line join=round,line cap=round,fill=fillColor] ( 50.33,141.10) circle (  1.96);

\path[draw=drawColor,line width= 0.4pt,line join=round,line cap=round,fill=fillColor] ( 50.33,134.17) circle (  1.96);

\path[draw=drawColor,line width= 0.4pt,line join=round,line cap=round,fill=fillColor] ( 50.33,131.28) circle (  1.96);

\path[draw=drawColor,line width= 0.4pt,line join=round,line cap=round,fill=fillColor] ( 50.33,147.17) circle (  1.96);

\path[draw=drawColor,line width= 0.4pt,line join=round,line cap=round,fill=fillColor] ( 50.33,134.17) circle (  1.96);

\path[draw=drawColor,line width= 0.4pt,line join=round,line cap=round,fill=fillColor] ( 50.33,151.50) circle (  1.96);

\path[draw=drawColor,line width= 0.4pt,line join=round,line cap=round,fill=fillColor] ( 50.33,130.45) circle (  1.96);

\path[draw=drawColor,line width= 0.4pt,line join=round,line cap=round,fill=fillColor] ( 50.33,144.57) circle (  1.96);

\path[draw=drawColor,line width= 0.4pt,line join=round,line cap=round,fill=fillColor] ( 50.33,134.17) circle (  1.96);

\path[draw=drawColor,line width= 0.4pt,line join=round,line cap=round,fill=fillColor] ( 50.33,131.57) circle (  1.96);

\path[draw=drawColor,line width= 0.4pt,line join=round,line cap=round,fill=fillColor] ( 50.33,147.17) circle (  1.96);

\path[draw=drawColor,line width= 0.4pt,line join=round,line cap=round,fill=fillColor] ( 50.33,134.17) circle (  1.96);

\path[draw=drawColor,line width= 0.4pt,line join=round,line cap=round,fill=fillColor] ( 50.33,139.94) circle (  1.96);

\path[draw=drawColor,line width= 0.4pt,line join=round,line cap=round,fill=fillColor] ( 50.33,139.94) circle (  1.96);

\path[draw=drawColor,line width= 0.4pt,line join=round,line cap=round,fill=fillColor] ( 50.33,139.94) circle (  1.96);

\path[draw=drawColor,line width= 0.4pt,line join=round,line cap=round,fill=fillColor] ( 50.33,134.17) circle (  1.96);

\path[draw=drawColor,line width= 0.4pt,line join=round,line cap=round,fill=fillColor] ( 50.33,149.76) circle (  1.96);

\path[draw=drawColor,line width= 0.4pt,line join=round,line cap=round,fill=fillColor] ( 50.33,147.17) circle (  1.96);

\path[draw=drawColor,line width= 0.4pt,line join=round,line cap=round,fill=fillColor] ( 50.33,134.17) circle (  1.96);

\path[draw=drawColor,line width= 0.4pt,line join=round,line cap=round,fill=fillColor] ( 50.33,134.17) circle (  1.96);

\path[draw=drawColor,line width= 0.4pt,line join=round,line cap=round,fill=fillColor] ( 50.33,131.80) circle (  1.96);

\path[draw=drawColor,line width= 0.4pt,line join=round,line cap=round,fill=fillColor] ( 50.33,131.69) circle (  1.96);

\path[draw=drawColor,line width= 0.4pt,line join=round,line cap=round,fill=fillColor] ( 50.33,141.97) circle (  1.96);

\path[draw=drawColor,line width= 0.4pt,line join=round,line cap=round,fill=fillColor] ( 50.33,132.09) circle (  1.96);

\path[draw=drawColor,line width= 0.6pt,line join=round] ( 50.33,110.47) -- ( 50.33,130.01);

\path[draw=drawColor,line width= 0.6pt,line join=round] ( 50.33, 97.43) -- ( 50.33, 82.45);
\definecolor{fillColor}{RGB}{228,26,28}

\path[draw=drawColor,line width= 0.6pt,line join=round,line cap=round,fill=fillColor] ( 41.44,110.47) --
	( 41.44, 97.43) --
	( 59.21, 97.43) --
	( 59.21,110.47) --
	( 41.44,110.47) --
	cycle;

\path[draw=drawColor,line width= 1.1pt,line join=round] ( 41.44,103.40) -- ( 59.21,103.40);
\definecolor{fillColor}{gray}{0.20}

\path[draw=drawColor,line width= 0.4pt,line join=round,line cap=round,fill=fillColor] ( 74.02,136.56) circle (  1.96);

\path[draw=drawColor,line width= 0.6pt,line join=round] ( 74.02,106.25) -- ( 74.02,125.58);

\path[draw=drawColor,line width= 0.6pt,line join=round] ( 74.02, 93.01) -- ( 74.02, 85.97);
\definecolor{fillColor}{RGB}{55,126,184}

\path[draw=drawColor,line width= 0.6pt,line join=round,line cap=round,fill=fillColor] ( 65.13,106.25) --
	( 65.13, 93.01) --
	( 82.90, 93.01) --
	( 82.90,106.25) --
	( 65.13,106.25) --
	cycle;

\path[draw=drawColor,line width= 1.1pt,line join=round] ( 65.13, 97.76) -- ( 82.90, 97.76);
\definecolor{fillColor}{gray}{0.20}

\path[draw=drawColor,line width= 0.4pt,line join=round,line cap=round,fill=fillColor] ( 97.71,152.68) circle (  1.96);

\path[draw=drawColor,line width= 0.6pt,line join=round] ( 97.71,121.17) -- ( 97.71,148.59);

\path[draw=drawColor,line width= 0.6pt,line join=round] ( 97.71,102.03) -- ( 97.71, 87.81);
\definecolor{fillColor}{RGB}{77,175,74}

\path[draw=drawColor,line width= 0.6pt,line join=round,line cap=round,fill=fillColor] ( 88.82,121.17) --
	( 88.82,102.03) --
	(106.59,102.03) --
	(106.59,121.17) --
	( 88.82,121.17) --
	cycle;

\path[draw=drawColor,line width= 1.1pt,line join=round] ( 88.82,111.18) -- (106.59,111.18);
\end{scope}
\begin{scope}
\path[clip] (117.42, 78.54) rectangle (193.23,158.60);
\definecolor{drawColor}{RGB}{255,255,255}

\path[draw=drawColor,line width= 0.3pt,line join=round] (117.42, 92.57) --
	(193.23, 92.57);

\path[draw=drawColor,line width= 0.3pt,line join=round] (117.42,113.37) --
	(193.23,113.37);

\path[draw=drawColor,line width= 0.3pt,line join=round] (117.42,134.17) --
	(193.23,134.17);

\path[draw=drawColor,line width= 0.3pt,line join=round] (117.42,154.96) --
	(193.23,154.96);

\path[draw=drawColor,line width= 0.6pt,line join=round] (117.42, 82.18) --
	(193.23, 82.18);

\path[draw=drawColor,line width= 0.6pt,line join=round] (117.42,102.97) --
	(193.23,102.97);

\path[draw=drawColor,line width= 0.6pt,line join=round] (117.42,123.77) --
	(193.23,123.77);

\path[draw=drawColor,line width= 0.6pt,line join=round] (117.42,144.57) --
	(193.23,144.57);

\path[draw=drawColor,line width= 0.6pt,line join=round] (131.64, 78.54) --
	(131.64,158.60);

\path[draw=drawColor,line width= 0.6pt,line join=round] (155.33, 78.54) --
	(155.33,158.60);

\path[draw=drawColor,line width= 0.6pt,line join=round] (179.02, 78.54) --
	(179.02,158.60);
\definecolor{drawColor}{gray}{0.20}
\definecolor{fillColor}{gray}{0.20}

\path[draw=drawColor,line width= 0.4pt,line join=round,line cap=round,fill=fillColor] (131.64,147.29) circle (  1.96);

\path[draw=drawColor,line width= 0.4pt,line join=round,line cap=round,fill=fillColor] (131.64,148.64) circle (  1.96);

\path[draw=drawColor,line width= 0.4pt,line join=round,line cap=round,fill=fillColor] (131.64,149.70) circle (  1.96);

\path[draw=drawColor,line width= 0.4pt,line join=round,line cap=round,fill=fillColor] (131.64,147.29) circle (  1.96);

\path[draw=drawColor,line width= 0.4pt,line join=round,line cap=round,fill=fillColor] (131.64,148.64) circle (  1.96);

\path[draw=drawColor,line width= 0.4pt,line join=round,line cap=round,fill=fillColor] (131.64,147.29) circle (  1.96);

\path[draw=drawColor,line width= 0.4pt,line join=round,line cap=round,fill=fillColor] (131.64,148.64) circle (  1.96);

\path[draw=drawColor,line width= 0.4pt,line join=round,line cap=round,fill=fillColor] (131.64,149.70) circle (  1.96);

\path[draw=drawColor,line width= 0.4pt,line join=round,line cap=round,fill=fillColor] (131.64,150.00) circle (  1.96);

\path[draw=drawColor,line width= 0.4pt,line join=round,line cap=round,fill=fillColor] (131.64,148.64) circle (  1.96);

\path[draw=drawColor,line width= 0.4pt,line join=round,line cap=round,fill=fillColor] (131.64,153.39) circle (  1.96);

\path[draw=drawColor,line width= 0.4pt,line join=round,line cap=round,fill=fillColor] (131.64,154.52) circle (  1.96);

\path[draw=drawColor,line width= 0.4pt,line join=round,line cap=round,fill=fillColor] (131.64,147.29) circle (  1.96);

\path[draw=drawColor,line width= 0.4pt,line join=round,line cap=round,fill=fillColor] (131.64,153.39) circle (  1.96);

\path[draw=drawColor,line width= 0.4pt,line join=round,line cap=round,fill=fillColor] (131.64,154.52) circle (  1.96);

\path[draw=drawColor,line width= 0.4pt,line join=round,line cap=round,fill=fillColor] (131.64,151.54) circle (  1.96);

\path[draw=drawColor,line width= 0.4pt,line join=round,line cap=round,fill=fillColor] (131.64,146.91) circle (  1.96);

\path[draw=drawColor,line width= 0.4pt,line join=round,line cap=round,fill=fillColor] (131.64,148.71) circle (  1.96);

\path[draw=drawColor,line width= 0.4pt,line join=round,line cap=round,fill=fillColor] (131.64,154.11) circle (  1.96);

\path[draw=drawColor,line width= 0.4pt,line join=round,line cap=round,fill=fillColor] (131.64,146.91) circle (  1.96);

\path[draw=drawColor,line width= 0.4pt,line join=round,line cap=round,fill=fillColor] (131.64,151.23) circle (  1.96);

\path[draw=drawColor,line width= 0.4pt,line join=round,line cap=round,fill=fillColor] (131.64,149.61) circle (  1.96);

\path[draw=drawColor,line width= 0.4pt,line join=round,line cap=round,fill=fillColor] (131.64,146.91) circle (  1.96);

\path[draw=drawColor,line width= 0.4pt,line join=round,line cap=round,fill=fillColor] (131.64,149.31) circle (  1.96);

\path[draw=drawColor,line width= 0.4pt,line join=round,line cap=round,fill=fillColor] (131.64,151.71) circle (  1.96);

\path[draw=drawColor,line width= 0.4pt,line join=round,line cap=round,fill=fillColor] (131.64,146.91) circle (  1.96);

\path[draw=drawColor,line width= 0.4pt,line join=round,line cap=round,fill=fillColor] (131.64,154.11) circle (  1.96);

\path[draw=drawColor,line width= 0.4pt,line join=round,line cap=round,fill=fillColor] (131.64,150.15) circle (  1.96);

\path[draw=drawColor,line width= 0.4pt,line join=round,line cap=round,fill=fillColor] (131.64,149.31) circle (  1.96);

\path[draw=drawColor,line width= 0.4pt,line join=round,line cap=round,fill=fillColor] (131.64,152.31) circle (  1.96);

\path[draw=drawColor,line width= 0.4pt,line join=round,line cap=round,fill=fillColor] (131.64,154.11) circle (  1.96);

\path[draw=drawColor,line width= 0.4pt,line join=round,line cap=round,fill=fillColor] (131.64,149.31) circle (  1.96);

\path[draw=drawColor,line width= 0.4pt,line join=round,line cap=round,fill=fillColor] (131.64,154.11) circle (  1.96);

\path[draw=drawColor,line width= 0.4pt,line join=round,line cap=round,fill=fillColor] (131.64,151.23) circle (  1.96);

\path[draw=drawColor,line width= 0.4pt,line join=round,line cap=round,fill=fillColor] (131.64,154.11) circle (  1.96);

\path[draw=drawColor,line width= 0.4pt,line join=round,line cap=round,fill=fillColor] (131.64,146.91) circle (  1.96);

\path[draw=drawColor,line width= 0.4pt,line join=round,line cap=round,fill=fillColor] (131.64,146.91) circle (  1.96);

\path[draw=drawColor,line width= 0.4pt,line join=round,line cap=round,fill=fillColor] (131.64,146.91) circle (  1.96);

\path[draw=drawColor,line width= 0.4pt,line join=round,line cap=round,fill=fillColor] (131.64,146.91) circle (  1.96);

\path[draw=drawColor,line width= 0.4pt,line join=round,line cap=round,fill=fillColor] (131.64,146.91) circle (  1.96);

\path[draw=drawColor,line width= 0.4pt,line join=round,line cap=round,fill=fillColor] (131.64,154.11) circle (  1.96);

\path[draw=drawColor,line width= 0.4pt,line join=round,line cap=round,fill=fillColor] (131.64,154.11) circle (  1.96);

\path[draw=drawColor,line width= 0.4pt,line join=round,line cap=round,fill=fillColor] (131.64,149.61) circle (  1.96);

\path[draw=drawColor,line width= 0.4pt,line join=round,line cap=round,fill=fillColor] (131.64,149.61) circle (  1.96);

\path[draw=drawColor,line width= 0.4pt,line join=round,line cap=round,fill=fillColor] (131.64,154.11) circle (  1.96);

\path[draw=drawColor,line width= 0.4pt,line join=round,line cap=round,fill=fillColor] (131.64,146.91) circle (  1.96);

\path[draw=drawColor,line width= 0.4pt,line join=round,line cap=round,fill=fillColor] (131.64,151.50) circle (  1.96);

\path[draw=drawColor,line width= 0.4pt,line join=round,line cap=round,fill=fillColor] (131.64,153.48) circle (  1.96);

\path[draw=drawColor,line width= 0.4pt,line join=round,line cap=round,fill=fillColor] (131.64,146.54) circle (  1.96);

\path[draw=drawColor,line width= 0.4pt,line join=round,line cap=round,fill=fillColor] (131.64,150.08) circle (  1.96);

\path[draw=drawColor,line width= 0.4pt,line join=round,line cap=round,fill=fillColor] (131.64,153.48) circle (  1.96);

\path[draw=drawColor,line width= 0.4pt,line join=round,line cap=round,fill=fillColor] (131.64,148.19) circle (  1.96);

\path[draw=drawColor,line width= 0.4pt,line join=round,line cap=round,fill=fillColor] (131.64,149.02) circle (  1.96);

\path[draw=drawColor,line width= 0.4pt,line join=round,line cap=round,fill=fillColor] (131.64,153.48) circle (  1.96);

\path[draw=drawColor,line width= 0.4pt,line join=round,line cap=round,fill=fillColor] (131.64,153.48) circle (  1.96);

\path[draw=drawColor,line width= 0.4pt,line join=round,line cap=round,fill=fillColor] (131.64,150.08) circle (  1.96);

\path[draw=drawColor,line width= 0.4pt,line join=round,line cap=round,fill=fillColor] (131.64,147.53) circle (  1.96);

\path[draw=drawColor,line width= 0.4pt,line join=round,line cap=round,fill=fillColor] (131.64,148.72) circle (  1.96);

\path[draw=drawColor,line width= 0.4pt,line join=round,line cap=round,fill=fillColor] (131.64,148.19) circle (  1.96);

\path[draw=drawColor,line width= 0.4pt,line join=round,line cap=round,fill=fillColor] (131.64,147.14) circle (  1.96);

\path[draw=drawColor,line width= 0.4pt,line join=round,line cap=round,fill=fillColor] (131.64,148.72) circle (  1.96);

\path[draw=drawColor,line width= 0.4pt,line join=round,line cap=round,fill=fillColor] (131.64,153.48) circle (  1.96);

\path[draw=drawColor,line width= 0.4pt,line join=round,line cap=round,fill=fillColor] (131.64,147.53) circle (  1.96);

\path[draw=drawColor,line width= 0.4pt,line join=round,line cap=round,fill=fillColor] (131.64,148.72) circle (  1.96);

\path[draw=drawColor,line width= 0.4pt,line join=round,line cap=round,fill=fillColor] (131.64,152.60) circle (  1.96);

\path[draw=drawColor,line width= 0.4pt,line join=round,line cap=round,fill=fillColor] (131.64,151.50) circle (  1.96);

\path[draw=drawColor,line width= 0.4pt,line join=round,line cap=round,fill=fillColor] (131.64,153.48) circle (  1.96);

\path[draw=drawColor,line width= 0.4pt,line join=round,line cap=round,fill=fillColor] (131.64,150.83) circle (  1.96);

\path[draw=drawColor,line width= 0.4pt,line join=round,line cap=round,fill=fillColor] (131.64,153.48) circle (  1.96);

\path[draw=drawColor,line width= 0.4pt,line join=round,line cap=round,fill=fillColor] (131.64,153.48) circle (  1.96);

\path[draw=drawColor,line width= 0.4pt,line join=round,line cap=round,fill=fillColor] (131.64,148.19) circle (  1.96);

\path[draw=drawColor,line width= 0.4pt,line join=round,line cap=round,fill=fillColor] (131.64,148.19) circle (  1.96);

\path[draw=drawColor,line width= 0.4pt,line join=round,line cap=round,fill=fillColor] (131.64,153.48) circle (  1.96);

\path[draw=drawColor,line width= 0.4pt,line join=round,line cap=round,fill=fillColor] (131.64,147.03) circle (  1.96);

\path[draw=drawColor,line width= 0.4pt,line join=round,line cap=round,fill=fillColor] (131.64,148.50) circle (  1.96);

\path[draw=drawColor,line width= 0.4pt,line join=round,line cap=round,fill=fillColor] (131.64,148.50) circle (  1.96);

\path[draw=drawColor,line width= 0.4pt,line join=round,line cap=round,fill=fillColor] (131.64,150.40) circle (  1.96);

\path[draw=drawColor,line width= 0.4pt,line join=round,line cap=round,fill=fillColor] (131.64,152.93) circle (  1.96);

\path[draw=drawColor,line width= 0.4pt,line join=round,line cap=round,fill=fillColor] (131.64,152.93) circle (  1.96);

\path[draw=drawColor,line width= 0.4pt,line join=round,line cap=round,fill=fillColor] (131.64,152.93) circle (  1.96);

\path[draw=drawColor,line width= 0.4pt,line join=round,line cap=round,fill=fillColor] (131.64,148.50) circle (  1.96);

\path[draw=drawColor,line width= 0.4pt,line join=round,line cap=round,fill=fillColor] (131.64,148.50) circle (  1.96);

\path[draw=drawColor,line width= 0.4pt,line join=round,line cap=round,fill=fillColor] (131.64,150.40) circle (  1.96);

\path[draw=drawColor,line width= 0.4pt,line join=round,line cap=round,fill=fillColor] (131.64,154.53) circle (  1.96);

\path[draw=drawColor,line width= 0.4pt,line join=round,line cap=round,fill=fillColor] (131.64,148.50) circle (  1.96);

\path[draw=drawColor,line width= 0.4pt,line join=round,line cap=round,fill=fillColor] (131.64,152.93) circle (  1.96);

\path[draw=drawColor,line width= 0.4pt,line join=round,line cap=round,fill=fillColor] (131.64,151.45) circle (  1.96);

\path[draw=drawColor,line width= 0.4pt,line join=round,line cap=round,fill=fillColor] (131.64,148.50) circle (  1.96);

\path[draw=drawColor,line width= 0.4pt,line join=round,line cap=round,fill=fillColor] (131.64,152.93) circle (  1.96);

\path[draw=drawColor,line width= 0.4pt,line join=round,line cap=round,fill=fillColor] (131.64,148.50) circle (  1.96);

\path[draw=drawColor,line width= 0.4pt,line join=round,line cap=round,fill=fillColor] (131.64,150.40) circle (  1.96);

\path[draw=drawColor,line width= 0.4pt,line join=round,line cap=round,fill=fillColor] (131.64,147.15) circle (  1.96);

\path[draw=drawColor,line width= 0.4pt,line join=round,line cap=round,fill=fillColor] (131.64,149.71) circle (  1.96);

\path[draw=drawColor,line width= 0.4pt,line join=round,line cap=round,fill=fillColor] (131.64,148.50) circle (  1.96);

\path[draw=drawColor,line width= 0.4pt,line join=round,line cap=round,fill=fillColor] (131.64,149.39) circle (  1.96);

\path[draw=drawColor,line width= 0.4pt,line join=round,line cap=round,fill=fillColor] (131.64,148.50) circle (  1.96);

\path[draw=drawColor,line width= 0.4pt,line join=round,line cap=round,fill=fillColor] (131.64,148.50) circle (  1.96);

\path[draw=drawColor,line width= 0.4pt,line join=round,line cap=round,fill=fillColor] (131.64,148.50) circle (  1.96);

\path[draw=drawColor,line width= 0.4pt,line join=round,line cap=round,fill=fillColor] (131.64,148.50) circle (  1.96);

\path[draw=drawColor,line width= 0.4pt,line join=round,line cap=round,fill=fillColor] (131.64,148.50) circle (  1.96);

\path[draw=drawColor,line width= 0.4pt,line join=round,line cap=round,fill=fillColor] (131.64,148.50) circle (  1.96);

\path[draw=drawColor,line width= 0.4pt,line join=round,line cap=round,fill=fillColor] (131.64,148.50) circle (  1.96);

\path[draw=drawColor,line width= 0.4pt,line join=round,line cap=round,fill=fillColor] (131.64,148.50) circle (  1.96);

\path[draw=drawColor,line width= 0.4pt,line join=round,line cap=round,fill=fillColor] (131.64,150.40) circle (  1.96);

\path[draw=drawColor,line width= 0.4pt,line join=round,line cap=round,fill=fillColor] (131.64,152.93) circle (  1.96);

\path[draw=drawColor,line width= 0.4pt,line join=round,line cap=round,fill=fillColor] (131.64,152.93) circle (  1.96);

\path[draw=drawColor,line width= 0.4pt,line join=round,line cap=round,fill=fillColor] (131.64,150.40) circle (  1.96);

\path[draw=drawColor,line width= 0.4pt,line join=round,line cap=round,fill=fillColor] (131.64,147.03) circle (  1.96);

\path[draw=drawColor,line width= 0.4pt,line join=round,line cap=round,fill=fillColor] (131.64,148.50) circle (  1.96);

\path[draw=drawColor,line width= 0.4pt,line join=round,line cap=round,fill=fillColor] (131.64,152.93) circle (  1.96);

\path[draw=drawColor,line width= 0.4pt,line join=round,line cap=round,fill=fillColor] (131.64,152.93) circle (  1.96);

\path[draw=drawColor,line width= 0.4pt,line join=round,line cap=round,fill=fillColor] (131.64,149.83) circle (  1.96);

\path[draw=drawColor,line width= 0.4pt,line join=round,line cap=round,fill=fillColor] (131.64,150.40) circle (  1.96);

\path[draw=drawColor,line width= 0.4pt,line join=round,line cap=round,fill=fillColor] (131.64,148.50) circle (  1.96);

\path[draw=drawColor,line width= 0.4pt,line join=round,line cap=round,fill=fillColor] (131.64,148.50) circle (  1.96);

\path[draw=drawColor,line width= 0.4pt,line join=round,line cap=round,fill=fillColor] (131.64,152.93) circle (  1.96);

\path[draw=drawColor,line width= 0.4pt,line join=round,line cap=round,fill=fillColor] (131.64,148.50) circle (  1.96);

\path[draw=drawColor,line width= 0.4pt,line join=round,line cap=round,fill=fillColor] (131.64,152.93) circle (  1.96);

\path[draw=drawColor,line width= 0.4pt,line join=round,line cap=round,fill=fillColor] (131.64,148.50) circle (  1.96);

\path[draw=drawColor,line width= 0.4pt,line join=round,line cap=round,fill=fillColor] (131.64,154.32) circle (  1.96);

\path[draw=drawColor,line width= 0.4pt,line join=round,line cap=round,fill=fillColor] (131.64,151.05) circle (  1.96);

\path[draw=drawColor,line width= 0.4pt,line join=round,line cap=round,fill=fillColor] (131.64,146.13) circle (  1.96);

\path[draw=drawColor,line width= 0.4pt,line join=round,line cap=round,fill=fillColor] (131.64,147.77) circle (  1.96);

\path[draw=drawColor,line width= 0.4pt,line join=round,line cap=round,fill=fillColor] (131.64,149.26) circle (  1.96);

\path[draw=drawColor,line width= 0.4pt,line join=round,line cap=round,fill=fillColor] (131.64,146.74) circle (  1.96);

\path[draw=drawColor,line width= 0.4pt,line join=round,line cap=round,fill=fillColor] (131.64,147.11) circle (  1.96);

\path[draw=drawColor,line width= 0.4pt,line join=round,line cap=round,fill=fillColor] (131.64,153.01) circle (  1.96);

\path[draw=drawColor,line width= 0.4pt,line join=round,line cap=round,fill=fillColor] (131.64,149.64) circle (  1.96);

\path[draw=drawColor,line width= 0.4pt,line join=round,line cap=round,fill=fillColor] (131.64,149.64) circle (  1.96);

\path[draw=drawColor,line width= 0.4pt,line join=round,line cap=round,fill=fillColor] (131.64,148.59) circle (  1.96);

\path[draw=drawColor,line width= 0.4pt,line join=round,line cap=round,fill=fillColor] (131.64,153.01) circle (  1.96);

\path[draw=drawColor,line width= 0.4pt,line join=round,line cap=round,fill=fillColor] (131.64,148.59) circle (  1.96);

\path[draw=drawColor,line width= 0.4pt,line join=round,line cap=round,fill=fillColor] (131.64,153.01) circle (  1.96);

\path[draw=drawColor,line width= 0.4pt,line join=round,line cap=round,fill=fillColor] (131.64,148.59) circle (  1.96);

\path[draw=drawColor,line width= 0.4pt,line join=round,line cap=round,fill=fillColor] (131.64,149.64) circle (  1.96);

\path[draw=drawColor,line width= 0.4pt,line join=round,line cap=round,fill=fillColor] (131.64,146.27) circle (  1.96);

\path[draw=drawColor,line width= 0.4pt,line join=round,line cap=round,fill=fillColor] (131.64,149.64) circle (  1.96);

\path[draw=drawColor,line width= 0.4pt,line join=round,line cap=round,fill=fillColor] (131.64,147.11) circle (  1.96);

\path[draw=drawColor,line width= 0.4pt,line join=round,line cap=round,fill=fillColor] (131.64,147.77) circle (  1.96);

\path[draw=drawColor,line width= 0.4pt,line join=round,line cap=round,fill=fillColor] (131.64,147.77) circle (  1.96);

\path[draw=drawColor,line width= 0.4pt,line join=round,line cap=round,fill=fillColor] (131.64,148.59) circle (  1.96);

\path[draw=drawColor,line width= 0.4pt,line join=round,line cap=round,fill=fillColor] (131.64,147.77) circle (  1.96);

\path[draw=drawColor,line width= 0.4pt,line join=round,line cap=round,fill=fillColor] (131.64,146.57) circle (  1.96);

\path[draw=drawColor,line width= 0.4pt,line join=round,line cap=round,fill=fillColor] (131.64,146.57) circle (  1.96);

\path[draw=drawColor,line width= 0.4pt,line join=round,line cap=round,fill=fillColor] (131.64,147.77) circle (  1.96);

\path[draw=drawColor,line width= 0.4pt,line join=round,line cap=round,fill=fillColor] (131.64,151.62) circle (  1.96);

\path[draw=drawColor,line width= 0.4pt,line join=round,line cap=round,fill=fillColor] (131.64,154.70) circle (  1.96);

\path[draw=drawColor,line width= 0.4pt,line join=round,line cap=round,fill=fillColor] (131.64,149.72) circle (  1.96);

\path[draw=drawColor,line width= 0.4pt,line join=round,line cap=round,fill=fillColor] (131.64,149.72) circle (  1.96);

\path[draw=drawColor,line width= 0.4pt,line join=round,line cap=round,fill=fillColor] (131.64,149.72) circle (  1.96);

\path[draw=drawColor,line width= 0.4pt,line join=round,line cap=round,fill=fillColor] (131.64,149.72) circle (  1.96);

\path[draw=drawColor,line width= 0.4pt,line join=round,line cap=round,fill=fillColor] (131.64,149.72) circle (  1.96);

\path[draw=drawColor,line width= 0.4pt,line join=round,line cap=round,fill=fillColor] (131.64,149.72) circle (  1.96);

\path[draw=drawColor,line width= 0.4pt,line join=round,line cap=round,fill=fillColor] (131.64,148.04) circle (  1.96);

\path[draw=drawColor,line width= 0.4pt,line join=round,line cap=round,fill=fillColor] (131.64,148.32) circle (  1.96);

\path[draw=drawColor,line width= 0.4pt,line join=round,line cap=round,fill=fillColor] (131.64,149.72) circle (  1.96);

\path[draw=drawColor,line width= 0.4pt,line join=round,line cap=round,fill=fillColor] (131.64,151.46) circle (  1.96);

\path[draw=drawColor,line width= 0.4pt,line join=round,line cap=round,fill=fillColor] (131.64,149.72) circle (  1.96);

\path[draw=drawColor,line width= 0.4pt,line join=round,line cap=round,fill=fillColor] (131.64,149.72) circle (  1.96);

\path[draw=drawColor,line width= 0.4pt,line join=round,line cap=round,fill=fillColor] (131.64,154.55) circle (  1.96);

\path[draw=drawColor,line width= 0.4pt,line join=round,line cap=round,fill=fillColor] (131.64,153.10) circle (  1.96);

\path[draw=drawColor,line width= 0.4pt,line join=round,line cap=round,fill=fillColor] (131.64,149.72) circle (  1.96);

\path[draw=drawColor,line width= 0.4pt,line join=round,line cap=round,fill=fillColor] (131.64,149.72) circle (  1.96);

\path[draw=drawColor,line width= 0.4pt,line join=round,line cap=round,fill=fillColor] (131.64,149.72) circle (  1.96);

\path[draw=drawColor,line width= 0.4pt,line join=round,line cap=round,fill=fillColor] (131.64,149.72) circle (  1.96);

\path[draw=drawColor,line width= 0.4pt,line join=round,line cap=round,fill=fillColor] (131.64,149.72) circle (  1.96);

\path[draw=drawColor,line width= 0.4pt,line join=round,line cap=round,fill=fillColor] (131.64,149.72) circle (  1.96);

\path[draw=drawColor,line width= 0.4pt,line join=round,line cap=round,fill=fillColor] (131.64,149.72) circle (  1.96);

\path[draw=drawColor,line width= 0.4pt,line join=round,line cap=round,fill=fillColor] (131.64,149.72) circle (  1.96);

\path[draw=drawColor,line width= 0.4pt,line join=round,line cap=round,fill=fillColor] (131.64,149.72) circle (  1.96);

\path[draw=drawColor,line width= 0.4pt,line join=round,line cap=round,fill=fillColor] (131.64,149.72) circle (  1.96);

\path[draw=drawColor,line width= 0.4pt,line join=round,line cap=round,fill=fillColor] (131.64,151.65) circle (  1.96);

\path[draw=drawColor,line width= 0.4pt,line join=round,line cap=round,fill=fillColor] (131.64,149.72) circle (  1.96);

\path[draw=drawColor,line width= 0.4pt,line join=round,line cap=round,fill=fillColor] (131.64,149.72) circle (  1.96);

\path[draw=drawColor,line width= 0.4pt,line join=round,line cap=round,fill=fillColor] (131.64,146.65) circle (  1.96);

\path[draw=drawColor,line width= 0.4pt,line join=round,line cap=round,fill=fillColor] (131.64,149.72) circle (  1.96);

\path[draw=drawColor,line width= 0.4pt,line join=round,line cap=round,fill=fillColor] (131.64,149.72) circle (  1.96);

\path[draw=drawColor,line width= 0.4pt,line join=round,line cap=round,fill=fillColor] (131.64,149.72) circle (  1.96);

\path[draw=drawColor,line width= 0.4pt,line join=round,line cap=round,fill=fillColor] (131.64,149.72) circle (  1.96);

\path[draw=drawColor,line width= 0.4pt,line join=round,line cap=round,fill=fillColor] (131.64,153.58) circle (  1.96);

\path[draw=drawColor,line width= 0.4pt,line join=round,line cap=round,fill=fillColor] (131.64,149.72) circle (  1.96);

\path[draw=drawColor,line width= 0.4pt,line join=round,line cap=round,fill=fillColor] (131.64,153.10) circle (  1.96);

\path[draw=drawColor,line width= 0.4pt,line join=round,line cap=round,fill=fillColor] (131.64,146.51) circle (  1.96);

\path[draw=drawColor,line width= 0.4pt,line join=round,line cap=round,fill=fillColor] (131.64,149.72) circle (  1.96);

\path[draw=drawColor,line width= 0.4pt,line join=round,line cap=round,fill=fillColor] (131.64,151.65) circle (  1.96);

\path[draw=drawColor,line width= 0.4pt,line join=round,line cap=round,fill=fillColor] (131.64,149.72) circle (  1.96);

\path[draw=drawColor,line width= 0.4pt,line join=round,line cap=round,fill=fillColor] (131.64,149.72) circle (  1.96);

\path[draw=drawColor,line width= 0.4pt,line join=round,line cap=round,fill=fillColor] (131.64,149.72) circle (  1.96);

\path[draw=drawColor,line width= 0.4pt,line join=round,line cap=round,fill=fillColor] (131.64,149.72) circle (  1.96);

\path[draw=drawColor,line width= 0.4pt,line join=round,line cap=round,fill=fillColor] (131.64,149.72) circle (  1.96);

\path[draw=drawColor,line width= 0.4pt,line join=round,line cap=round,fill=fillColor] (131.64,154.79) circle (  1.96);

\path[draw=drawColor,line width= 0.4pt,line join=round,line cap=round,fill=fillColor] (131.64,146.35) circle (  1.96);

\path[draw=drawColor,line width= 0.4pt,line join=round,line cap=round,fill=fillColor] (131.64,149.72) circle (  1.96);

\path[draw=drawColor,line width= 0.4pt,line join=round,line cap=round,fill=fillColor] (131.64,149.72) circle (  1.96);

\path[draw=drawColor,line width= 0.4pt,line join=round,line cap=round,fill=fillColor] (131.64,149.72) circle (  1.96);

\path[draw=drawColor,line width= 0.4pt,line join=round,line cap=round,fill=fillColor] (131.64,153.48) circle (  1.96);

\path[draw=drawColor,line width= 0.4pt,line join=round,line cap=round,fill=fillColor] (131.64,151.98) circle (  1.96);

\path[draw=drawColor,line width= 0.4pt,line join=round,line cap=round,fill=fillColor] (131.64,149.72) circle (  1.96);

\path[draw=drawColor,line width= 0.4pt,line join=round,line cap=round,fill=fillColor] (131.64,146.98) circle (  1.96);

\path[draw=drawColor,line width= 0.4pt,line join=round,line cap=round,fill=fillColor] (131.64,153.06) circle (  1.96);

\path[draw=drawColor,line width= 0.4pt,line join=round,line cap=round,fill=fillColor] (131.64,154.93) circle (  1.96);

\path[draw=drawColor,line width= 0.4pt,line join=round,line cap=round,fill=fillColor] (131.64,151.48) circle (  1.96);

\path[draw=drawColor,line width= 0.4pt,line join=round,line cap=round,fill=fillColor] (131.64,146.98) circle (  1.96);

\path[draw=drawColor,line width= 0.4pt,line join=round,line cap=round,fill=fillColor] (131.64,148.33) circle (  1.96);

\path[draw=drawColor,line width= 0.4pt,line join=round,line cap=round,fill=fillColor] (131.64,148.33) circle (  1.96);

\path[draw=drawColor,line width= 0.4pt,line join=round,line cap=round,fill=fillColor] (131.64,148.33) circle (  1.96);

\path[draw=drawColor,line width= 0.4pt,line join=round,line cap=round,fill=fillColor] (131.64,151.73) circle (  1.96);

\path[draw=drawColor,line width= 0.4pt,line join=round,line cap=round,fill=fillColor] (131.64,146.44) circle (  1.96);

\path[draw=drawColor,line width= 0.4pt,line join=round,line cap=round,fill=fillColor] (131.64,149.38) circle (  1.96);

\path[draw=drawColor,line width= 0.4pt,line join=round,line cap=round,fill=fillColor] (131.64,146.44) circle (  1.96);

\path[draw=drawColor,line width= 0.4pt,line join=round,line cap=round,fill=fillColor] (131.64,148.33) circle (  1.96);

\path[draw=drawColor,line width= 0.4pt,line join=round,line cap=round,fill=fillColor] (131.64,153.06) circle (  1.96);

\path[draw=drawColor,line width= 0.4pt,line join=round,line cap=round,fill=fillColor] (131.64,148.33) circle (  1.96);

\path[draw=drawColor,line width= 0.4pt,line join=round,line cap=round,fill=fillColor] (131.64,154.95) circle (  1.96);

\path[draw=drawColor,line width= 0.4pt,line join=round,line cap=round,fill=fillColor] (131.64,153.06) circle (  1.96);

\path[draw=drawColor,line width= 0.4pt,line join=round,line cap=round,fill=fillColor] (131.64,148.33) circle (  1.96);

\path[draw=drawColor,line width= 0.4pt,line join=round,line cap=round,fill=fillColor] (131.64,153.06) circle (  1.96);

\path[draw=drawColor,line width= 0.4pt,line join=round,line cap=round,fill=fillColor] (131.64,148.33) circle (  1.96);

\path[draw=drawColor,line width= 0.4pt,line join=round,line cap=round,fill=fillColor] (131.64,153.46) circle (  1.96);

\path[draw=drawColor,line width= 0.4pt,line join=round,line cap=round,fill=fillColor] (131.64,147.28) circle (  1.96);

\path[draw=drawColor,line width= 0.4pt,line join=round,line cap=round,fill=fillColor] (131.64,152.38) circle (  1.96);

\path[draw=drawColor,line width= 0.4pt,line join=round,line cap=round,fill=fillColor] (131.64,150.91) circle (  1.96);

\path[draw=drawColor,line width= 0.4pt,line join=round,line cap=round,fill=fillColor] (131.64,150.91) circle (  1.96);

\path[draw=drawColor,line width= 0.4pt,line join=round,line cap=round,fill=fillColor] (131.64,154.87) circle (  1.96);

\path[draw=drawColor,line width= 0.4pt,line join=round,line cap=round,fill=fillColor] (131.64,151.73) circle (  1.96);

\path[draw=drawColor,line width= 0.4pt,line join=round,line cap=round,fill=fillColor] (131.64,149.38) circle (  1.96);

\path[draw=drawColor,line width= 0.4pt,line join=round,line cap=round,fill=fillColor] (131.64,149.38) circle (  1.96);

\path[draw=drawColor,line width= 0.4pt,line join=round,line cap=round,fill=fillColor] (131.64,153.06) circle (  1.96);

\path[draw=drawColor,line width= 0.4pt,line join=round,line cap=round,fill=fillColor] (131.64,151.48) circle (  1.96);

\path[draw=drawColor,line width= 0.4pt,line join=round,line cap=round,fill=fillColor] (131.64,146.98) circle (  1.96);

\path[draw=drawColor,line width= 0.4pt,line join=round,line cap=round,fill=fillColor] (131.64,146.44) circle (  1.96);

\path[draw=drawColor,line width= 0.4pt,line join=round,line cap=round,fill=fillColor] (131.64,150.22) circle (  1.96);

\path[draw=drawColor,line width= 0.4pt,line join=round,line cap=round,fill=fillColor] (131.64,146.44) circle (  1.96);

\path[draw=drawColor,line width= 0.4pt,line join=round,line cap=round,fill=fillColor] (131.64,148.33) circle (  1.96);

\path[draw=drawColor,line width= 0.4pt,line join=round,line cap=round,fill=fillColor] (131.64,146.98) circle (  1.96);

\path[draw=drawColor,line width= 0.4pt,line join=round,line cap=round,fill=fillColor] (131.64,149.38) circle (  1.96);

\path[draw=drawColor,line width= 0.4pt,line join=round,line cap=round,fill=fillColor] (131.64,147.35) circle (  1.96);

\path[draw=drawColor,line width= 0.4pt,line join=round,line cap=round,fill=fillColor] (131.64,154.76) circle (  1.96);

\path[draw=drawColor,line width= 0.4pt,line join=round,line cap=round,fill=fillColor] (131.64,153.06) circle (  1.96);

\path[draw=drawColor,line width= 0.4pt,line join=round,line cap=round,fill=fillColor] (131.64,153.06) circle (  1.96);

\path[draw=drawColor,line width= 0.4pt,line join=round,line cap=round,fill=fillColor] (131.64,148.33) circle (  1.96);

\path[draw=drawColor,line width= 0.4pt,line join=round,line cap=round,fill=fillColor] (131.64,149.48) circle (  1.96);

\path[draw=drawColor,line width= 0.4pt,line join=round,line cap=round,fill=fillColor] (131.64,147.44) circle (  1.96);

\path[draw=drawColor,line width= 0.4pt,line join=round,line cap=round,fill=fillColor] (131.64,153.56) circle (  1.96);

\path[draw=drawColor,line width= 0.4pt,line join=round,line cap=round,fill=fillColor] (131.64,153.56) circle (  1.96);

\path[draw=drawColor,line width= 0.4pt,line join=round,line cap=round,fill=fillColor] (131.64,153.56) circle (  1.96);

\path[draw=drawColor,line width= 0.4pt,line join=round,line cap=round,fill=fillColor] (131.64,153.56) circle (  1.96);

\path[draw=drawColor,line width= 0.4pt,line join=round,line cap=round,fill=fillColor] (131.64,153.56) circle (  1.96);

\path[draw=drawColor,line width= 0.4pt,line join=round,line cap=round,fill=fillColor] (131.64,153.56) circle (  1.96);

\path[draw=drawColor,line width= 0.4pt,line join=round,line cap=round,fill=fillColor] (131.64,147.44) circle (  1.96);

\path[draw=drawColor,line width= 0.4pt,line join=round,line cap=round,fill=fillColor] (131.64,153.56) circle (  1.96);

\path[draw=drawColor,line width= 0.4pt,line join=round,line cap=round,fill=fillColor] (131.64,150.16) circle (  1.96);

\path[draw=drawColor,line width= 0.4pt,line join=round,line cap=round,fill=fillColor] (131.64,154.69) circle (  1.96);

\path[draw=drawColor,line width= 0.4pt,line join=round,line cap=round,fill=fillColor] (131.64,147.44) circle (  1.96);

\path[draw=drawColor,line width= 0.4pt,line join=round,line cap=round,fill=fillColor] (131.64,150.16) circle (  1.96);

\path[draw=drawColor,line width= 0.4pt,line join=round,line cap=round,fill=fillColor] (131.64,147.44) circle (  1.96);

\path[draw=drawColor,line width= 0.4pt,line join=round,line cap=round,fill=fillColor] (131.64,154.69) circle (  1.96);

\path[draw=drawColor,line width= 0.4pt,line join=round,line cap=round,fill=fillColor] (131.64,147.44) circle (  1.96);

\path[draw=drawColor,line width= 0.4pt,line join=round,line cap=round,fill=fillColor] (131.64,153.56) circle (  1.96);

\path[draw=drawColor,line width= 0.4pt,line join=round,line cap=round,fill=fillColor] (131.64,150.16) circle (  1.96);

\path[draw=drawColor,line width= 0.4pt,line join=round,line cap=round,fill=fillColor] (131.64,151.52) circle (  1.96);

\path[draw=drawColor,line width= 0.4pt,line join=round,line cap=round,fill=fillColor] (131.64,153.56) circle (  1.96);

\path[draw=drawColor,line width= 0.4pt,line join=round,line cap=round,fill=fillColor] (131.64,149.48) circle (  1.96);

\path[draw=drawColor,line width= 0.4pt,line join=round,line cap=round,fill=fillColor] (131.64,147.44) circle (  1.96);

\path[draw=drawColor,line width= 0.4pt,line join=round,line cap=round,fill=fillColor] (131.64,150.16) circle (  1.96);

\path[draw=drawColor,line width= 0.4pt,line join=round,line cap=round,fill=fillColor] (131.64,147.44) circle (  1.96);

\path[draw=drawColor,line width= 0.4pt,line join=round,line cap=round,fill=fillColor] (131.64,150.16) circle (  1.96);

\path[draw=drawColor,line width= 0.4pt,line join=round,line cap=round,fill=fillColor] (131.64,150.16) circle (  1.96);

\path[draw=drawColor,line width= 0.4pt,line join=round,line cap=round,fill=fillColor] (131.64,153.56) circle (  1.96);

\path[draw=drawColor,line width= 0.4pt,line join=round,line cap=round,fill=fillColor] (131.64,150.16) circle (  1.96);

\path[draw=drawColor,line width= 0.4pt,line join=round,line cap=round,fill=fillColor] (131.64,146.45) circle (  1.96);

\path[draw=drawColor,line width= 0.4pt,line join=round,line cap=round,fill=fillColor] (131.64,146.76) circle (  1.96);

\path[draw=drawColor,line width= 0.4pt,line join=round,line cap=round,fill=fillColor] (131.64,150.16) circle (  1.96);

\path[draw=drawColor,line width= 0.4pt,line join=round,line cap=round,fill=fillColor] (131.64,151.34) circle (  1.96);

\path[draw=drawColor,line width= 0.4pt,line join=round,line cap=round,fill=fillColor] (131.64,151.76) circle (  1.96);

\path[draw=drawColor,line width= 0.4pt,line join=round,line cap=round,fill=fillColor] (131.64,150.16) circle (  1.96);

\path[draw=drawColor,line width= 0.4pt,line join=round,line cap=round,fill=fillColor] (131.64,154.69) circle (  1.96);

\path[draw=drawColor,line width= 0.4pt,line join=round,line cap=round,fill=fillColor] (131.64,147.44) circle (  1.96);

\path[draw=drawColor,line width= 0.4pt,line join=round,line cap=round,fill=fillColor] (131.64,153.56) circle (  1.96);

\path[draw=drawColor,line width= 0.4pt,line join=round,line cap=round,fill=fillColor] (131.64,154.69) circle (  1.96);

\path[draw=drawColor,line width= 0.4pt,line join=round,line cap=round,fill=fillColor] (131.64,154.69) circle (  1.96);

\path[draw=drawColor,line width= 0.4pt,line join=round,line cap=round,fill=fillColor] (131.64,151.06) circle (  1.96);

\path[draw=drawColor,line width= 0.4pt,line join=round,line cap=round,fill=fillColor] (131.64,154.69) circle (  1.96);

\path[draw=drawColor,line width= 0.4pt,line join=round,line cap=round,fill=fillColor] (131.64,154.69) circle (  1.96);

\path[draw=drawColor,line width= 0.4pt,line join=round,line cap=round,fill=fillColor] (131.64,153.56) circle (  1.96);

\path[draw=drawColor,line width= 0.4pt,line join=round,line cap=round,fill=fillColor] (131.64,150.16) circle (  1.96);

\path[draw=drawColor,line width= 0.4pt,line join=round,line cap=round,fill=fillColor] (131.64,147.44) circle (  1.96);

\path[draw=drawColor,line width= 0.4pt,line join=round,line cap=round,fill=fillColor] (131.64,154.41) circle (  1.96);

\path[draw=drawColor,line width= 0.4pt,line join=round,line cap=round,fill=fillColor] (131.64,147.19) circle (  1.96);

\path[draw=drawColor,line width= 0.4pt,line join=round,line cap=round,fill=fillColor] (131.64,149.90) circle (  1.96);

\path[draw=drawColor,line width= 0.4pt,line join=round,line cap=round,fill=fillColor] (131.64,147.19) circle (  1.96);

\path[draw=drawColor,line width= 0.4pt,line join=round,line cap=round,fill=fillColor] (131.64,154.41) circle (  1.96);

\path[draw=drawColor,line width= 0.4pt,line join=round,line cap=round,fill=fillColor] (131.64,147.19) circle (  1.96);

\path[draw=drawColor,line width= 0.4pt,line join=round,line cap=round,fill=fillColor] (131.64,154.41) circle (  1.96);

\path[draw=drawColor,line width= 0.4pt,line join=round,line cap=round,fill=fillColor] (131.64,149.60) circle (  1.96);

\path[draw=drawColor,line width= 0.4pt,line join=round,line cap=round,fill=fillColor] (131.64,149.90) circle (  1.96);

\path[draw=drawColor,line width= 0.4pt,line join=round,line cap=round,fill=fillColor] (131.64,151.52) circle (  1.96);

\path[draw=drawColor,line width= 0.4pt,line join=round,line cap=round,fill=fillColor] (131.64,151.52) circle (  1.96);

\path[draw=drawColor,line width= 0.4pt,line join=round,line cap=round,fill=fillColor] (131.64,154.41) circle (  1.96);

\path[draw=drawColor,line width= 0.4pt,line join=round,line cap=round,fill=fillColor] (131.64,154.41) circle (  1.96);

\path[draw=drawColor,line width= 0.4pt,line join=round,line cap=round,fill=fillColor] (131.64,147.19) circle (  1.96);

\path[draw=drawColor,line width= 0.4pt,line join=round,line cap=round,fill=fillColor] (131.64,147.67) circle (  1.96);

\path[draw=drawColor,line width= 0.4pt,line join=round,line cap=round,fill=fillColor] (131.64,147.19) circle (  1.96);

\path[draw=drawColor,line width= 0.4pt,line join=round,line cap=round,fill=fillColor] (131.64,147.19) circle (  1.96);

\path[draw=drawColor,line width= 0.4pt,line join=round,line cap=round,fill=fillColor] (131.64,148.63) circle (  1.96);

\path[draw=drawColor,line width= 0.4pt,line join=round,line cap=round,fill=fillColor] (131.64,153.69) circle (  1.96);

\path[draw=drawColor,line width= 0.4pt,line join=round,line cap=round,fill=fillColor] (131.64,147.19) circle (  1.96);

\path[draw=drawColor,line width= 0.4pt,line join=round,line cap=round,fill=fillColor] (131.64,154.41) circle (  1.96);

\path[draw=drawColor,line width= 0.4pt,line join=round,line cap=round,fill=fillColor] (131.64,154.41) circle (  1.96);

\path[draw=drawColor,line width= 0.4pt,line join=round,line cap=round,fill=fillColor] (131.64,147.19) circle (  1.96);

\path[draw=drawColor,line width= 0.4pt,line join=round,line cap=round,fill=fillColor] (131.64,154.41) circle (  1.96);

\path[draw=drawColor,line width= 0.4pt,line join=round,line cap=round,fill=fillColor] (131.64,147.19) circle (  1.96);

\path[draw=drawColor,line width= 0.4pt,line join=round,line cap=round,fill=fillColor] (131.64,147.19) circle (  1.96);

\path[draw=drawColor,line width= 0.4pt,line join=round,line cap=round,fill=fillColor] (131.64,151.52) circle (  1.96);

\path[draw=drawColor,line width= 0.4pt,line join=round,line cap=round,fill=fillColor] (131.64,147.19) circle (  1.96);

\path[draw=drawColor,line width= 0.4pt,line join=round,line cap=round,fill=fillColor] (131.64,151.52) circle (  1.96);

\path[draw=drawColor,line width= 0.4pt,line join=round,line cap=round,fill=fillColor] (131.64,154.41) circle (  1.96);

\path[draw=drawColor,line width= 0.4pt,line join=round,line cap=round,fill=fillColor] (131.64,147.19) circle (  1.96);

\path[draw=drawColor,line width= 0.4pt,line join=round,line cap=round,fill=fillColor] (131.64,148.13) circle (  1.96);

\path[draw=drawColor,line width= 0.4pt,line join=round,line cap=round,fill=fillColor] (131.64,154.41) circle (  1.96);

\path[draw=drawColor,line width= 0.4pt,line join=round,line cap=round,fill=fillColor] (131.64,147.19) circle (  1.96);

\path[draw=drawColor,line width= 0.4pt,line join=round,line cap=round,fill=fillColor] (131.64,149.60) circle (  1.96);

\path[draw=drawColor,line width= 0.4pt,line join=round,line cap=round,fill=fillColor] (131.64,147.67) circle (  1.96);

\path[draw=drawColor,line width= 0.4pt,line join=round,line cap=round,fill=fillColor] (131.64,154.41) circle (  1.96);

\path[draw=drawColor,line width= 0.4pt,line join=round,line cap=round,fill=fillColor] (131.64,147.19) circle (  1.96);

\path[draw=drawColor,line width= 0.4pt,line join=round,line cap=round,fill=fillColor] (131.64,147.19) circle (  1.96);

\path[draw=drawColor,line width= 0.4pt,line join=round,line cap=round,fill=fillColor] (131.64,153.69) circle (  1.96);

\path[draw=drawColor,line width= 0.4pt,line join=round,line cap=round,fill=fillColor] (131.64,151.52) circle (  1.96);

\path[draw=drawColor,line width= 0.4pt,line join=round,line cap=round,fill=fillColor] (131.64,147.19) circle (  1.96);

\path[draw=drawColor,line width= 0.4pt,line join=round,line cap=round,fill=fillColor] (131.64,149.36) circle (  1.96);

\path[draw=drawColor,line width= 0.4pt,line join=round,line cap=round,fill=fillColor] (131.64,151.52) circle (  1.96);

\path[draw=drawColor,line width= 0.4pt,line join=round,line cap=round,fill=fillColor] (131.64,149.60) circle (  1.96);

\path[draw=drawColor,line width= 0.4pt,line join=round,line cap=round,fill=fillColor] (131.64,147.19) circle (  1.96);

\path[draw=drawColor,line width= 0.4pt,line join=round,line cap=round,fill=fillColor] (131.64,154.41) circle (  1.96);

\path[draw=drawColor,line width= 0.4pt,line join=round,line cap=round,fill=fillColor] (131.64,147.19) circle (  1.96);

\path[draw=drawColor,line width= 0.4pt,line join=round,line cap=round,fill=fillColor] (131.64,147.19) circle (  1.96);

\path[draw=drawColor,line width= 0.4pt,line join=round,line cap=round,fill=fillColor] (131.64,147.19) circle (  1.96);

\path[draw=drawColor,line width= 0.4pt,line join=round,line cap=round,fill=fillColor] (131.64,147.19) circle (  1.96);

\path[draw=drawColor,line width= 0.4pt,line join=round,line cap=round,fill=fillColor] (131.64,154.41) circle (  1.96);

\path[draw=drawColor,line width= 0.4pt,line join=round,line cap=round,fill=fillColor] (131.64,151.52) circle (  1.96);

\path[draw=drawColor,line width= 0.4pt,line join=round,line cap=round,fill=fillColor] (131.64,146.32) circle (  1.96);

\path[draw=drawColor,line width= 0.4pt,line join=round,line cap=round,fill=fillColor] (131.64,147.19) circle (  1.96);

\path[draw=drawColor,line width= 0.4pt,line join=round,line cap=round,fill=fillColor] (131.64,149.60) circle (  1.96);

\path[draw=drawColor,line width= 0.4pt,line join=round,line cap=round,fill=fillColor] (131.64,151.52) circle (  1.96);

\path[draw=drawColor,line width= 0.4pt,line join=round,line cap=round,fill=fillColor] (131.64,149.60) circle (  1.96);

\path[draw=drawColor,line width= 0.4pt,line join=round,line cap=round,fill=fillColor] (131.64,147.19) circle (  1.96);

\path[draw=drawColor,line width= 0.4pt,line join=round,line cap=round,fill=fillColor] (131.64,151.52) circle (  1.96);

\path[draw=drawColor,line width= 0.4pt,line join=round,line cap=round,fill=fillColor] (131.64,154.41) circle (  1.96);

\path[draw=drawColor,line width= 0.4pt,line join=round,line cap=round,fill=fillColor] (131.64,154.41) circle (  1.96);

\path[draw=drawColor,line width= 0.4pt,line join=round,line cap=round,fill=fillColor] (131.64,151.13) circle (  1.96);

\path[draw=drawColor,line width= 0.4pt,line join=round,line cap=round,fill=fillColor] (131.64,151.52) circle (  1.96);

\path[draw=drawColor,line width= 0.4pt,line join=round,line cap=round,fill=fillColor] (131.64,149.90) circle (  1.96);

\path[draw=drawColor,line width= 0.4pt,line join=round,line cap=round,fill=fillColor] (131.64,154.41) circle (  1.96);

\path[draw=drawColor,line width= 0.4pt,line join=round,line cap=round,fill=fillColor] (131.64,154.41) circle (  1.96);

\path[draw=drawColor,line width= 0.4pt,line join=round,line cap=round,fill=fillColor] (131.64,149.90) circle (  1.96);

\path[draw=drawColor,line width= 0.4pt,line join=round,line cap=round,fill=fillColor] (131.64,154.41) circle (  1.96);

\path[draw=drawColor,line width= 0.4pt,line join=round,line cap=round,fill=fillColor] (131.64,149.16) circle (  1.96);

\path[draw=drawColor,line width= 0.4pt,line join=round,line cap=round,fill=fillColor] (131.64,151.13) circle (  1.96);

\path[draw=drawColor,line width= 0.4pt,line join=round,line cap=round,fill=fillColor] (131.64,147.19) circle (  1.96);

\path[draw=drawColor,line width= 0.4pt,line join=round,line cap=round,fill=fillColor] (131.64,147.19) circle (  1.96);

\path[draw=drawColor,line width= 0.4pt,line join=round,line cap=round,fill=fillColor] (131.64,149.05) circle (  1.96);

\path[draw=drawColor,line width= 0.4pt,line join=round,line cap=round,fill=fillColor] (131.64,149.60) circle (  1.96);

\path[draw=drawColor,line width= 0.4pt,line join=round,line cap=round,fill=fillColor] (131.64,147.19) circle (  1.96);

\path[draw=drawColor,line width= 0.4pt,line join=round,line cap=round,fill=fillColor] (131.64,154.41) circle (  1.96);

\path[draw=drawColor,line width= 0.4pt,line join=round,line cap=round,fill=fillColor] (131.64,148.05) circle (  1.96);

\path[draw=drawColor,line width= 0.4pt,line join=round,line cap=round,fill=fillColor] (131.64,147.19) circle (  1.96);

\path[draw=drawColor,line width= 0.4pt,line join=round,line cap=round,fill=fillColor] (131.64,150.14) circle (  1.96);

\path[draw=drawColor,line width= 0.4pt,line join=round,line cap=round,fill=fillColor] (131.64,152.41) circle (  1.96);

\path[draw=drawColor,line width= 0.4pt,line join=round,line cap=round,fill=fillColor] (131.64,150.14) circle (  1.96);

\path[draw=drawColor,line width= 0.4pt,line join=round,line cap=round,fill=fillColor] (131.64,152.66) circle (  1.96);

\path[draw=drawColor,line width= 0.4pt,line join=round,line cap=round,fill=fillColor] (131.64,150.14) circle (  1.96);

\path[draw=drawColor,line width= 0.4pt,line join=round,line cap=round,fill=fillColor] (131.64,150.14) circle (  1.96);

\path[draw=drawColor,line width= 0.4pt,line join=round,line cap=round,fill=fillColor] (131.64,154.67) circle (  1.96);

\path[draw=drawColor,line width= 0.4pt,line join=round,line cap=round,fill=fillColor] (131.64,146.91) circle (  1.96);

\path[draw=drawColor,line width= 0.4pt,line join=round,line cap=round,fill=fillColor] (131.64,150.14) circle (  1.96);

\path[draw=drawColor,line width= 0.4pt,line join=round,line cap=round,fill=fillColor] (131.64,147.24) circle (  1.96);

\path[draw=drawColor,line width= 0.4pt,line join=round,line cap=round,fill=fillColor] (131.64,151.48) circle (  1.96);

\path[draw=drawColor,line width= 0.4pt,line join=round,line cap=round,fill=fillColor] (131.64,148.26) circle (  1.96);

\path[draw=drawColor,line width= 0.4pt,line join=round,line cap=round,fill=fillColor] (131.64,150.14) circle (  1.96);

\path[draw=drawColor,line width= 0.4pt,line join=round,line cap=round,fill=fillColor] (131.64,150.14) circle (  1.96);

\path[draw=drawColor,line width= 0.4pt,line join=round,line cap=round,fill=fillColor] (131.64,150.14) circle (  1.96);

\path[draw=drawColor,line width= 0.4pt,line join=round,line cap=round,fill=fillColor] (131.64,151.65) circle (  1.96);

\path[draw=drawColor,line width= 0.4pt,line join=round,line cap=round,fill=fillColor] (131.64,154.67) circle (  1.96);

\path[draw=drawColor,line width= 0.4pt,line join=round,line cap=round,fill=fillColor] (131.64,147.88) circle (  1.96);

\path[draw=drawColor,line width= 0.4pt,line join=round,line cap=round,fill=fillColor] (131.64,148.95) circle (  1.96);

\path[draw=drawColor,line width= 0.4pt,line join=round,line cap=round,fill=fillColor] (131.64,146.91) circle (  1.96);

\path[draw=drawColor,line width= 0.4pt,line join=round,line cap=round,fill=fillColor] (131.64,154.67) circle (  1.96);

\path[draw=drawColor,line width= 0.4pt,line join=round,line cap=round,fill=fillColor] (131.64,154.67) circle (  1.96);

\path[draw=drawColor,line width= 0.4pt,line join=round,line cap=round,fill=fillColor] (131.64,146.91) circle (  1.96);

\path[draw=drawColor,line width= 0.4pt,line join=round,line cap=round,fill=fillColor] (131.64,154.67) circle (  1.96);

\path[draw=drawColor,line width= 0.4pt,line join=round,line cap=round,fill=fillColor] (131.64,152.98) circle (  1.96);

\path[draw=drawColor,line width= 0.4pt,line join=round,line cap=round,fill=fillColor] (131.64,154.67) circle (  1.96);

\path[draw=drawColor,line width= 0.4pt,line join=round,line cap=round,fill=fillColor] (131.64,150.14) circle (  1.96);

\path[draw=drawColor,line width= 0.4pt,line join=round,line cap=round,fill=fillColor] (131.64,150.14) circle (  1.96);

\path[draw=drawColor,line width= 0.4pt,line join=round,line cap=round,fill=fillColor] (131.64,150.14) circle (  1.96);

\path[draw=drawColor,line width= 0.4pt,line join=round,line cap=round,fill=fillColor] (131.64,150.14) circle (  1.96);

\path[draw=drawColor,line width= 0.4pt,line join=round,line cap=round,fill=fillColor] (131.64,154.67) circle (  1.96);

\path[draw=drawColor,line width= 0.4pt,line join=round,line cap=round,fill=fillColor] (131.64,150.14) circle (  1.96);

\path[draw=drawColor,line width= 0.4pt,line join=round,line cap=round,fill=fillColor] (131.64,150.14) circle (  1.96);

\path[draw=drawColor,line width= 0.4pt,line join=round,line cap=round,fill=fillColor] (131.64,152.66) circle (  1.96);

\path[draw=drawColor,line width= 0.4pt,line join=round,line cap=round,fill=fillColor] (131.64,150.14) circle (  1.96);

\path[draw=drawColor,line width= 0.4pt,line join=round,line cap=round,fill=fillColor] (131.64,150.14) circle (  1.96);

\path[draw=drawColor,line width= 0.4pt,line join=round,line cap=round,fill=fillColor] (131.64,152.66) circle (  1.96);

\path[draw=drawColor,line width= 0.4pt,line join=round,line cap=round,fill=fillColor] (131.64,150.14) circle (  1.96);

\path[draw=drawColor,line width= 0.4pt,line join=round,line cap=round,fill=fillColor] (131.64,147.85) circle (  1.96);

\path[draw=drawColor,line width= 0.4pt,line join=round,line cap=round,fill=fillColor] (131.64,146.91) circle (  1.96);

\path[draw=drawColor,line width= 0.4pt,line join=round,line cap=round,fill=fillColor] (131.64,153.38) circle (  1.96);

\path[draw=drawColor,line width= 0.4pt,line join=round,line cap=round,fill=fillColor] (131.64,148.26) circle (  1.96);

\path[draw=drawColor,line width= 0.4pt,line join=round,line cap=round,fill=fillColor] (131.64,150.14) circle (  1.96);

\path[draw=drawColor,line width= 0.4pt,line join=round,line cap=round,fill=fillColor] (131.64,154.67) circle (  1.96);

\path[draw=drawColor,line width= 0.4pt,line join=round,line cap=round,fill=fillColor] (131.64,154.67) circle (  1.96);

\path[draw=drawColor,line width= 0.4pt,line join=round,line cap=round,fill=fillColor] (131.64,154.67) circle (  1.96);

\path[draw=drawColor,line width= 0.4pt,line join=round,line cap=round,fill=fillColor] (131.64,150.14) circle (  1.96);

\path[draw=drawColor,line width= 0.4pt,line join=round,line cap=round,fill=fillColor] (131.64,150.14) circle (  1.96);

\path[draw=drawColor,line width= 0.4pt,line join=round,line cap=round,fill=fillColor] (131.64,150.14) circle (  1.96);

\path[draw=drawColor,line width= 0.4pt,line join=round,line cap=round,fill=fillColor] (131.64,151.05) circle (  1.96);

\path[draw=drawColor,line width= 0.4pt,line join=round,line cap=round,fill=fillColor] (131.64,150.14) circle (  1.96);

\path[draw=drawColor,line width= 0.4pt,line join=round,line cap=round,fill=fillColor] (131.64,150.14) circle (  1.96);

\path[draw=drawColor,line width= 0.4pt,line join=round,line cap=round,fill=fillColor] (131.64,150.14) circle (  1.96);

\path[draw=drawColor,line width= 0.4pt,line join=round,line cap=round,fill=fillColor] (131.64,152.66) circle (  1.96);

\path[draw=drawColor,line width= 0.4pt,line join=round,line cap=round,fill=fillColor] (131.64,151.65) circle (  1.96);

\path[draw=drawColor,line width= 0.4pt,line join=round,line cap=round,fill=fillColor] (131.64,146.91) circle (  1.96);

\path[draw=drawColor,line width= 0.4pt,line join=round,line cap=round,fill=fillColor] (131.64,150.14) circle (  1.96);

\path[draw=drawColor,line width= 0.4pt,line join=round,line cap=round,fill=fillColor] (131.64,147.68) circle (  1.96);

\path[draw=drawColor,line width= 0.4pt,line join=round,line cap=round,fill=fillColor] (131.64,149.55) circle (  1.96);

\path[draw=drawColor,line width= 0.4pt,line join=round,line cap=round,fill=fillColor] (131.64,149.55) circle (  1.96);

\path[draw=drawColor,line width= 0.4pt,line join=round,line cap=round,fill=fillColor] (131.64,152.92) circle (  1.96);

\path[draw=drawColor,line width= 0.4pt,line join=round,line cap=round,fill=fillColor] (131.64,152.92) circle (  1.96);

\path[draw=drawColor,line width= 0.4pt,line join=round,line cap=round,fill=fillColor] (131.64,152.92) circle (  1.96);

\path[draw=drawColor,line width= 0.4pt,line join=round,line cap=round,fill=fillColor] (131.64,152.92) circle (  1.96);

\path[draw=drawColor,line width= 0.4pt,line join=round,line cap=round,fill=fillColor] (131.64,152.92) circle (  1.96);

\path[draw=drawColor,line width= 0.4pt,line join=round,line cap=round,fill=fillColor] (131.64,150.95) circle (  1.96);

\path[draw=drawColor,line width= 0.4pt,line join=round,line cap=round,fill=fillColor] (131.64,148.20) circle (  1.96);

\path[draw=drawColor,line width= 0.4pt,line join=round,line cap=round,fill=fillColor] (131.64,152.92) circle (  1.96);

\path[draw=drawColor,line width= 0.4pt,line join=round,line cap=round,fill=fillColor] (131.64,148.20) circle (  1.96);

\path[draw=drawColor,line width= 0.4pt,line join=round,line cap=round,fill=fillColor] (131.64,147.68) circle (  1.96);

\path[draw=drawColor,line width= 0.4pt,line join=round,line cap=round,fill=fillColor] (131.64,147.68) circle (  1.96);

\path[draw=drawColor,line width= 0.4pt,line join=round,line cap=round,fill=fillColor] (131.64,149.55) circle (  1.96);

\path[draw=drawColor,line width= 0.4pt,line join=round,line cap=round,fill=fillColor] (131.64,154.94) circle (  1.96);

\path[draw=drawColor,line width= 0.4pt,line join=round,line cap=round,fill=fillColor] (131.64,152.92) circle (  1.96);

\path[draw=drawColor,line width= 0.4pt,line join=round,line cap=round,fill=fillColor] (131.64,152.92) circle (  1.96);

\path[draw=drawColor,line width= 0.4pt,line join=round,line cap=round,fill=fillColor] (131.64,152.92) circle (  1.96);

\path[draw=drawColor,line width= 0.4pt,line join=round,line cap=round,fill=fillColor] (131.64,151.35) circle (  1.96);

\path[draw=drawColor,line width= 0.4pt,line join=round,line cap=round,fill=fillColor] (131.64,152.92) circle (  1.96);

\path[draw=drawColor,line width= 0.4pt,line join=round,line cap=round,fill=fillColor] (131.64,152.92) circle (  1.96);

\path[draw=drawColor,line width= 0.4pt,line join=round,line cap=round,fill=fillColor] (131.64,149.55) circle (  1.96);

\path[draw=drawColor,line width= 0.4pt,line join=round,line cap=round,fill=fillColor] (131.64,146.32) circle (  1.96);

\path[draw=drawColor,line width= 0.4pt,line join=round,line cap=round,fill=fillColor] (131.64,152.92) circle (  1.96);

\path[draw=drawColor,line width= 0.4pt,line join=round,line cap=round,fill=fillColor] (131.64,152.92) circle (  1.96);

\path[draw=drawColor,line width= 0.4pt,line join=round,line cap=round,fill=fillColor] (131.64,152.92) circle (  1.96);

\path[draw=drawColor,line width= 0.4pt,line join=round,line cap=round,fill=fillColor] (131.64,151.94) circle (  1.96);

\path[draw=drawColor,line width= 0.4pt,line join=round,line cap=round,fill=fillColor] (131.64,152.05) circle (  1.96);

\path[draw=drawColor,line width= 0.4pt,line join=round,line cap=round,fill=fillColor] (131.64,148.20) circle (  1.96);

\path[draw=drawColor,line width= 0.4pt,line join=round,line cap=round,fill=fillColor] (131.64,151.53) circle (  1.96);

\path[draw=drawColor,line width= 0.4pt,line join=round,line cap=round,fill=fillColor] (131.64,148.20) circle (  1.96);

\path[draw=drawColor,line width= 0.4pt,line join=round,line cap=round,fill=fillColor] (131.64,151.53) circle (  1.96);

\path[draw=drawColor,line width= 0.4pt,line join=round,line cap=round,fill=fillColor] (131.64,148.20) circle (  1.96);

\path[draw=drawColor,line width= 0.4pt,line join=round,line cap=round,fill=fillColor] (131.64,148.20) circle (  1.96);

\path[draw=drawColor,line width= 0.4pt,line join=round,line cap=round,fill=fillColor] (131.64,149.02) circle (  1.96);

\path[draw=drawColor,line width= 0.4pt,line join=round,line cap=round,fill=fillColor] (131.64,149.02) circle (  1.96);

\path[draw=drawColor,line width= 0.4pt,line join=round,line cap=round,fill=fillColor] (131.64,149.02) circle (  1.96);

\path[draw=drawColor,line width= 0.4pt,line join=round,line cap=round,fill=fillColor] (131.64,152.36) circle (  1.96);

\path[draw=drawColor,line width= 0.4pt,line join=round,line cap=round,fill=fillColor] (131.64,146.35) circle (  1.96);

\path[draw=drawColor,line width= 0.4pt,line join=round,line cap=round,fill=fillColor] (131.64,150.54) circle (  1.96);

\path[draw=drawColor,line width= 0.4pt,line join=round,line cap=round,fill=fillColor] (131.64,150.10) circle (  1.96);

\path[draw=drawColor,line width= 0.4pt,line join=round,line cap=round,fill=fillColor] (131.64,149.86) circle (  1.96);

\path[draw=drawColor,line width= 0.4pt,line join=round,line cap=round,fill=fillColor] (131.64,153.79) circle (  1.96);

\path[draw=drawColor,line width= 0.4pt,line join=round,line cap=round,fill=fillColor] (131.64,149.02) circle (  1.96);

\path[draw=drawColor,line width= 0.4pt,line join=round,line cap=round,fill=fillColor] (131.64,149.02) circle (  1.96);

\path[draw=drawColor,line width= 0.4pt,line join=round,line cap=round,fill=fillColor] (131.64,154.87) circle (  1.96);

\path[draw=drawColor,line width= 0.4pt,line join=round,line cap=round,fill=fillColor] (131.64,153.48) circle (  1.96);

\path[draw=drawColor,line width= 0.4pt,line join=round,line cap=round,fill=fillColor] (131.64,146.79) circle (  1.96);

\path[draw=drawColor,line width= 0.4pt,line join=round,line cap=round,fill=fillColor] (131.64,146.79) circle (  1.96);

\path[draw=drawColor,line width= 0.4pt,line join=round,line cap=round,fill=fillColor] (131.64,153.79) circle (  1.96);

\path[draw=drawColor,line width= 0.4pt,line join=round,line cap=round,fill=fillColor] (131.64,153.79) circle (  1.96);

\path[draw=drawColor,line width= 0.4pt,line join=round,line cap=round,fill=fillColor] (131.64,153.79) circle (  1.96);

\path[draw=drawColor,line width= 0.4pt,line join=round,line cap=round,fill=fillColor] (131.64,153.79) circle (  1.96);

\path[draw=drawColor,line width= 0.4pt,line join=round,line cap=round,fill=fillColor] (131.64,149.02) circle (  1.96);

\path[draw=drawColor,line width= 0.4pt,line join=round,line cap=round,fill=fillColor] (131.64,152.36) circle (  1.96);

\path[draw=drawColor,line width= 0.4pt,line join=round,line cap=round,fill=fillColor] (131.64,149.02) circle (  1.96);

\path[draw=drawColor,line width= 0.4pt,line join=round,line cap=round,fill=fillColor] (131.64,148.48) circle (  1.96);

\path[draw=drawColor,line width= 0.4pt,line join=round,line cap=round,fill=fillColor] (131.64,149.02) circle (  1.96);

\path[draw=drawColor,line width= 0.4pt,line join=round,line cap=round,fill=fillColor] (131.64,150.36) circle (  1.96);

\path[draw=drawColor,line width= 0.4pt,line join=round,line cap=round,fill=fillColor] (131.64,146.35) circle (  1.96);

\path[draw=drawColor,line width= 0.4pt,line join=round,line cap=round,fill=fillColor] (131.64,152.36) circle (  1.96);

\path[draw=drawColor,line width= 0.4pt,line join=round,line cap=round,fill=fillColor] (131.64,153.79) circle (  1.96);

\path[draw=drawColor,line width= 0.4pt,line join=round,line cap=round,fill=fillColor] (131.64,149.86) circle (  1.96);

\path[draw=drawColor,line width= 0.4pt,line join=round,line cap=round,fill=fillColor] (131.64,153.79) circle (  1.96);

\path[draw=drawColor,line width= 0.4pt,line join=round,line cap=round,fill=fillColor] (131.64,147.35) circle (  1.96);

\path[draw=drawColor,line width= 0.4pt,line join=round,line cap=round,fill=fillColor] (131.64,149.02) circle (  1.96);

\path[draw=drawColor,line width= 0.4pt,line join=round,line cap=round,fill=fillColor] (131.64,149.86) circle (  1.96);

\path[draw=drawColor,line width= 0.4pt,line join=round,line cap=round,fill=fillColor] (131.64,153.79) circle (  1.96);

\path[draw=drawColor,line width= 0.4pt,line join=round,line cap=round,fill=fillColor] (131.64,150.54) circle (  1.96);

\path[draw=drawColor,line width= 0.4pt,line join=round,line cap=round,fill=fillColor] (131.64,153.79) circle (  1.96);

\path[draw=drawColor,line width= 0.4pt,line join=round,line cap=round,fill=fillColor] (131.64,149.02) circle (  1.96);

\path[draw=drawColor,line width= 0.4pt,line join=round,line cap=round,fill=fillColor] (131.64,152.36) circle (  1.96);

\path[draw=drawColor,line width= 0.4pt,line join=round,line cap=round,fill=fillColor] (131.64,150.54) circle (  1.96);

\path[draw=drawColor,line width= 0.4pt,line join=round,line cap=round,fill=fillColor] (131.64,149.02) circle (  1.96);

\path[draw=drawColor,line width= 0.4pt,line join=round,line cap=round,fill=fillColor] (131.64,149.02) circle (  1.96);

\path[draw=drawColor,line width= 0.4pt,line join=round,line cap=round,fill=fillColor] (131.64,151.80) circle (  1.96);

\path[draw=drawColor,line width= 0.4pt,line join=round,line cap=round,fill=fillColor] (131.64,146.35) circle (  1.96);

\path[draw=drawColor,line width= 0.4pt,line join=round,line cap=round,fill=fillColor] (131.64,148.14) circle (  1.96);

\path[draw=drawColor,line width= 0.4pt,line join=round,line cap=round,fill=fillColor] (131.64,153.79) circle (  1.96);

\path[draw=drawColor,line width= 0.4pt,line join=round,line cap=round,fill=fillColor] (131.64,151.50) circle (  1.96);

\path[draw=drawColor,line width= 0.4pt,line join=round,line cap=round,fill=fillColor] (131.64,147.17) circle (  1.96);

\path[draw=drawColor,line width= 0.4pt,line join=round,line cap=round,fill=fillColor] (131.64,154.39) circle (  1.96);

\path[draw=drawColor,line width= 0.4pt,line join=round,line cap=round,fill=fillColor] (131.64,147.17) circle (  1.96);

\path[draw=drawColor,line width= 0.4pt,line join=round,line cap=round,fill=fillColor] (131.64,147.17) circle (  1.96);

\path[draw=drawColor,line width= 0.4pt,line join=round,line cap=round,fill=fillColor] (131.64,150.59) circle (  1.96);

\path[draw=drawColor,line width= 0.4pt,line join=round,line cap=round,fill=fillColor] (131.64,147.17) circle (  1.96);

\path[draw=drawColor,line width= 0.4pt,line join=round,line cap=round,fill=fillColor] (131.64,154.96) circle (  1.96);

\path[draw=drawColor,line width= 0.4pt,line join=round,line cap=round,fill=fillColor] (131.64,147.17) circle (  1.96);

\path[draw=drawColor,line width= 0.4pt,line join=round,line cap=round,fill=fillColor] (131.64,147.17) circle (  1.96);

\path[draw=drawColor,line width= 0.4pt,line join=round,line cap=round,fill=fillColor] (131.64,153.48) circle (  1.96);

\path[draw=drawColor,line width= 0.4pt,line join=round,line cap=round,fill=fillColor] (131.64,147.17) circle (  1.96);

\path[draw=drawColor,line width= 0.4pt,line join=round,line cap=round,fill=fillColor] (131.64,147.17) circle (  1.96);

\path[draw=drawColor,line width= 0.4pt,line join=round,line cap=round,fill=fillColor] (131.64,147.17) circle (  1.96);

\path[draw=drawColor,line width= 0.4pt,line join=round,line cap=round,fill=fillColor] (131.64,147.17) circle (  1.96);

\path[draw=drawColor,line width= 0.4pt,line join=round,line cap=round,fill=fillColor] (131.64,147.17) circle (  1.96);

\path[draw=drawColor,line width= 0.4pt,line join=round,line cap=round,fill=fillColor] (131.64,151.50) circle (  1.96);

\path[draw=drawColor,line width= 0.4pt,line join=round,line cap=round,fill=fillColor] (131.64,146.88) circle (  1.96);

\path[draw=drawColor,line width= 0.4pt,line join=round,line cap=round,fill=fillColor] (131.64,147.17) circle (  1.96);

\path[draw=drawColor,line width= 0.4pt,line join=round,line cap=round,fill=fillColor] (131.64,147.17) circle (  1.96);

\path[draw=drawColor,line width= 0.4pt,line join=round,line cap=round,fill=fillColor] (131.64,147.17) circle (  1.96);

\path[draw=drawColor,line width= 0.4pt,line join=round,line cap=round,fill=fillColor] (131.64,146.88) circle (  1.96);

\path[draw=drawColor,line width= 0.4pt,line join=round,line cap=round,fill=fillColor] (131.64,151.50) circle (  1.96);

\path[draw=drawColor,line width= 0.4pt,line join=round,line cap=round,fill=fillColor] (131.64,147.17) circle (  1.96);

\path[draw=drawColor,line width= 0.4pt,line join=round,line cap=round,fill=fillColor] (131.64,147.17) circle (  1.96);

\path[draw=drawColor,line width= 0.4pt,line join=round,line cap=round,fill=fillColor] (131.64,147.17) circle (  1.96);

\path[draw=drawColor,line width= 0.4pt,line join=round,line cap=round,fill=fillColor] (131.64,147.17) circle (  1.96);

\path[draw=drawColor,line width= 0.4pt,line join=round,line cap=round,fill=fillColor] (131.64,151.50) circle (  1.96);

\path[draw=drawColor,line width= 0.4pt,line join=round,line cap=round,fill=fillColor] (131.64,153.66) circle (  1.96);

\path[draw=drawColor,line width= 0.4pt,line join=round,line cap=round,fill=fillColor] (131.64,147.17) circle (  1.96);

\path[draw=drawColor,line width= 0.4pt,line join=round,line cap=round,fill=fillColor] (131.64,151.50) circle (  1.96);

\path[draw=drawColor,line width= 0.4pt,line join=round,line cap=round,fill=fillColor] (131.64,151.50) circle (  1.96);

\path[draw=drawColor,line width= 0.4pt,line join=round,line cap=round,fill=fillColor] (131.64,147.17) circle (  1.96);

\path[draw=drawColor,line width= 0.4pt,line join=round,line cap=round,fill=fillColor] (131.64,151.50) circle (  1.96);

\path[draw=drawColor,line width= 0.4pt,line join=round,line cap=round,fill=fillColor] (131.64,149.76) circle (  1.96);

\path[draw=drawColor,line width= 0.4pt,line join=round,line cap=round,fill=fillColor] (131.64,154.39) circle (  1.96);

\path[draw=drawColor,line width= 0.4pt,line join=round,line cap=round,fill=fillColor] (131.64,149.02) circle (  1.96);

\path[draw=drawColor,line width= 0.4pt,line join=round,line cap=round,fill=fillColor] (131.64,151.50) circle (  1.96);

\path[draw=drawColor,line width= 0.4pt,line join=round,line cap=round,fill=fillColor] (131.64,151.50) circle (  1.96);

\path[draw=drawColor,line width= 0.4pt,line join=round,line cap=round,fill=fillColor] (131.64,147.17) circle (  1.96);

\path[draw=drawColor,line width= 0.4pt,line join=round,line cap=round,fill=fillColor] (131.64,148.03) circle (  1.96);

\path[draw=drawColor,line width= 0.4pt,line join=round,line cap=round,fill=fillColor] (131.64,146.88) circle (  1.96);

\path[draw=drawColor,line width= 0.4pt,line join=round,line cap=round,fill=fillColor] (131.64,151.50) circle (  1.96);

\path[draw=drawColor,line width= 0.4pt,line join=round,line cap=round,fill=fillColor] (131.64,149.76) circle (  1.96);

\path[draw=drawColor,line width= 0.4pt,line join=round,line cap=round,fill=fillColor] (131.64,151.50) circle (  1.96);

\path[draw=drawColor,line width= 0.4pt,line join=round,line cap=round,fill=fillColor] (131.64,153.54) circle (  1.96);

\path[draw=drawColor,line width= 0.4pt,line join=round,line cap=round,fill=fillColor] (131.64,151.50) circle (  1.96);

\path[draw=drawColor,line width= 0.4pt,line join=round,line cap=round,fill=fillColor] (131.64,147.17) circle (  1.96);

\path[draw=drawColor,line width= 0.4pt,line join=round,line cap=round,fill=fillColor] (131.64,147.17) circle (  1.96);

\path[draw=drawColor,line width= 0.4pt,line join=round,line cap=round,fill=fillColor] (131.64,152.36) circle (  1.96);

\path[draw=drawColor,line width= 0.4pt,line join=round,line cap=round,fill=fillColor] (131.64,147.17) circle (  1.96);

\path[draw=drawColor,line width= 0.4pt,line join=round,line cap=round,fill=fillColor] (131.64,154.39) circle (  1.96);

\path[draw=drawColor,line width= 0.4pt,line join=round,line cap=round,fill=fillColor] (131.64,147.17) circle (  1.96);

\path[draw=drawColor,line width= 0.4pt,line join=round,line cap=round,fill=fillColor] (131.64,153.07) circle (  1.96);

\path[draw=drawColor,line width= 0.4pt,line join=round,line cap=round,fill=fillColor] (131.64,154.96) circle (  1.96);

\path[draw=drawColor,line width= 0.4pt,line join=round,line cap=round,fill=fillColor] (131.64,154.39) circle (  1.96);

\path[draw=drawColor,line width= 0.4pt,line join=round,line cap=round,fill=fillColor] (131.64,147.17) circle (  1.96);

\path[draw=drawColor,line width= 0.4pt,line join=round,line cap=round,fill=fillColor] (131.64,150.51) circle (  1.96);

\path[draw=drawColor,line width= 0.4pt,line join=round,line cap=round,fill=fillColor] (131.64,151.50) circle (  1.96);

\path[draw=drawColor,line width= 0.4pt,line join=round,line cap=round,fill=fillColor] (131.64,147.17) circle (  1.96);

\path[draw=drawColor,line width= 0.4pt,line join=round,line cap=round,fill=fillColor] (131.64,147.17) circle (  1.96);

\path[draw=drawColor,line width= 0.4pt,line join=round,line cap=round,fill=fillColor] (131.64,147.17) circle (  1.96);

\path[draw=drawColor,line width= 0.4pt,line join=round,line cap=round,fill=fillColor] (131.64,147.17) circle (  1.96);

\path[draw=drawColor,line width= 0.4pt,line join=round,line cap=round,fill=fillColor] (131.64,151.50) circle (  1.96);

\path[draw=drawColor,line width= 0.4pt,line join=round,line cap=round,fill=fillColor] (131.64,151.50) circle (  1.96);

\path[draw=drawColor,line width= 0.4pt,line join=round,line cap=round,fill=fillColor] (131.64,147.17) circle (  1.96);

\path[draw=drawColor,line width= 0.4pt,line join=round,line cap=round,fill=fillColor] (131.64,149.02) circle (  1.96);

\path[draw=drawColor,line width= 0.4pt,line join=round,line cap=round,fill=fillColor] (131.64,148.61) circle (  1.96);

\path[draw=drawColor,line width= 0.4pt,line join=round,line cap=round,fill=fillColor] (131.64,147.17) circle (  1.96);

\path[draw=drawColor,line width= 0.4pt,line join=round,line cap=round,fill=fillColor] (131.64,146.88) circle (  1.96);

\path[draw=drawColor,line width= 0.4pt,line join=round,line cap=round,fill=fillColor] (131.64,151.50) circle (  1.96);

\path[draw=drawColor,line width= 0.4pt,line join=round,line cap=round,fill=fillColor] (131.64,149.76) circle (  1.96);

\path[draw=drawColor,line width= 0.4pt,line join=round,line cap=round,fill=fillColor] (131.64,149.76) circle (  1.96);

\path[draw=drawColor,line width= 0.4pt,line join=round,line cap=round,fill=fillColor] (131.64,147.17) circle (  1.96);

\path[draw=drawColor,line width= 0.4pt,line join=round,line cap=round,fill=fillColor] (131.64,147.17) circle (  1.96);

\path[draw=drawColor,line width= 0.4pt,line join=round,line cap=round,fill=fillColor] (131.64,152.36) circle (  1.96);

\path[draw=drawColor,line width= 0.4pt,line join=round,line cap=round,fill=fillColor] (131.64,154.96) circle (  1.96);

\path[draw=drawColor,line width= 0.4pt,line join=round,line cap=round,fill=fillColor] (131.64,154.96) circle (  1.96);

\path[draw=drawColor,line width= 0.4pt,line join=round,line cap=round,fill=fillColor] (131.64,150.41) circle (  1.96);

\path[draw=drawColor,line width= 0.4pt,line join=round,line cap=round,fill=fillColor] (131.64,153.66) circle (  1.96);

\path[draw=drawColor,line width= 0.4pt,line join=round,line cap=round,fill=fillColor] (131.64,151.50) circle (  1.96);

\path[draw=drawColor,line width= 0.4pt,line join=round,line cap=round,fill=fillColor] (131.64,151.50) circle (  1.96);

\path[draw=drawColor,line width= 0.4pt,line join=round,line cap=round,fill=fillColor] (131.64,146.88) circle (  1.96);

\path[draw=drawColor,line width= 0.4pt,line join=round,line cap=round,fill=fillColor] (131.64,147.17) circle (  1.96);

\path[draw=drawColor,line width= 0.4pt,line join=round,line cap=round,fill=fillColor] (131.64,149.19) circle (  1.96);

\path[draw=drawColor,line width= 0.4pt,line join=round,line cap=round,fill=fillColor] (131.64,154.39) circle (  1.96);

\path[draw=drawColor,line width= 0.4pt,line join=round,line cap=round,fill=fillColor] (131.64,147.17) circle (  1.96);

\path[draw=drawColor,line width= 0.4pt,line join=round,line cap=round,fill=fillColor] (131.64,151.50) circle (  1.96);

\path[draw=drawColor,line width= 0.4pt,line join=round,line cap=round,fill=fillColor] (131.64,150.41) circle (  1.96);

\path[draw=drawColor,line width= 0.4pt,line join=round,line cap=round,fill=fillColor] (131.64,147.17) circle (  1.96);

\path[draw=drawColor,line width= 0.4pt,line join=round,line cap=round,fill=fillColor] (131.64,147.17) circle (  1.96);

\path[draw=drawColor,line width= 0.4pt,line join=round,line cap=round,fill=fillColor] (131.64,151.50) circle (  1.96);

\path[draw=drawColor,line width= 0.4pt,line join=round,line cap=round,fill=fillColor] (131.64,147.17) circle (  1.96);

\path[draw=drawColor,line width= 0.4pt,line join=round,line cap=round,fill=fillColor] (131.64,151.50) circle (  1.96);

\path[draw=drawColor,line width= 0.4pt,line join=round,line cap=round,fill=fillColor] (131.64,151.50) circle (  1.96);

\path[draw=drawColor,line width= 0.4pt,line join=round,line cap=round,fill=fillColor] (131.64,146.30) circle (  1.96);

\path[draw=drawColor,line width= 0.4pt,line join=round,line cap=round,fill=fillColor] (131.64,148.73) circle (  1.96);

\path[draw=drawColor,line width= 0.4pt,line join=round,line cap=round,fill=fillColor] (131.64,147.17) circle (  1.96);

\path[draw=drawColor,line width= 0.4pt,line join=round,line cap=round,fill=fillColor] (131.64,151.50) circle (  1.96);

\path[draw=drawColor,line width= 0.4pt,line join=round,line cap=round,fill=fillColor] (131.64,147.17) circle (  1.96);

\path[draw=drawColor,line width= 0.4pt,line join=round,line cap=round,fill=fillColor] (131.64,154.39) circle (  1.96);

\path[draw=drawColor,line width= 0.4pt,line join=round,line cap=round,fill=fillColor] (131.64,151.50) circle (  1.96);

\path[draw=drawColor,line width= 0.4pt,line join=round,line cap=round,fill=fillColor] (131.64,147.17) circle (  1.96);

\path[draw=drawColor,line width= 0.4pt,line join=round,line cap=round,fill=fillColor] (131.64,147.17) circle (  1.96);

\path[draw=drawColor,line width= 0.4pt,line join=round,line cap=round,fill=fillColor] (131.64,154.96) circle (  1.96);

\path[draw=drawColor,line width= 0.4pt,line join=round,line cap=round,fill=fillColor] (131.64,147.17) circle (  1.96);

\path[draw=drawColor,line width= 0.4pt,line join=round,line cap=round,fill=fillColor] (131.64,147.17) circle (  1.96);

\path[draw=drawColor,line width= 0.4pt,line join=round,line cap=round,fill=fillColor] (131.64,146.30) circle (  1.96);

\path[draw=drawColor,line width= 0.4pt,line join=round,line cap=round,fill=fillColor] (131.64,147.17) circle (  1.96);

\path[draw=drawColor,line width= 0.4pt,line join=round,line cap=round,fill=fillColor] (131.64,147.17) circle (  1.96);

\path[draw=drawColor,line width= 0.4pt,line join=round,line cap=round,fill=fillColor] (131.64,147.17) circle (  1.96);

\path[draw=drawColor,line width= 0.4pt,line join=round,line cap=round,fill=fillColor] (131.64,147.17) circle (  1.96);

\path[draw=drawColor,line width= 0.4pt,line join=round,line cap=round,fill=fillColor] (131.64,151.50) circle (  1.96);

\path[draw=drawColor,line width= 0.4pt,line join=round,line cap=round,fill=fillColor] (131.64,149.76) circle (  1.96);

\path[draw=drawColor,line width= 0.4pt,line join=round,line cap=round,fill=fillColor] (131.64,147.17) circle (  1.96);

\path[draw=drawColor,line width= 0.4pt,line join=round,line cap=round,fill=fillColor] (131.64,151.50) circle (  1.96);

\path[draw=drawColor,line width= 0.4pt,line join=round,line cap=round,fill=fillColor] (131.64,153.52) circle (  1.96);

\path[draw=drawColor,line width= 0.4pt,line join=round,line cap=round,fill=fillColor] (131.64,149.76) circle (  1.96);

\path[draw=drawColor,line width= 0.4pt,line join=round,line cap=round,fill=fillColor] (131.64,151.50) circle (  1.96);

\path[draw=drawColor,line width= 0.4pt,line join=round,line cap=round,fill=fillColor] (131.64,147.17) circle (  1.96);

\path[draw=drawColor,line width= 0.4pt,line join=round,line cap=round,fill=fillColor] (131.64,154.39) circle (  1.96);

\path[draw=drawColor,line width= 0.4pt,line join=round,line cap=round,fill=fillColor] (131.64,148.03) circle (  1.96);

\path[draw=drawColor,line width= 0.4pt,line join=round,line cap=round,fill=fillColor] (131.64,151.50) circle (  1.96);

\path[draw=drawColor,line width= 0.4pt,line join=round,line cap=round,fill=fillColor] (131.64,147.17) circle (  1.96);

\path[draw=drawColor,line width= 0.4pt,line join=round,line cap=round,fill=fillColor] (131.64,147.17) circle (  1.96);

\path[draw=drawColor,line width= 0.4pt,line join=round,line cap=round,fill=fillColor] (131.64,147.17) circle (  1.96);

\path[draw=drawColor,line width= 0.4pt,line join=round,line cap=round,fill=fillColor] (131.64,153.66) circle (  1.96);

\path[draw=drawColor,line width= 0.4pt,line join=round,line cap=round,fill=fillColor] (131.64,151.50) circle (  1.96);

\path[draw=drawColor,line width= 0.4pt,line join=round,line cap=round,fill=fillColor] (131.64,147.17) circle (  1.96);

\path[draw=drawColor,line width= 0.6pt,line join=round] (131.64,118.36) -- (131.64,146.02);

\path[draw=drawColor,line width= 0.6pt,line join=round] (131.64, 99.86) -- (131.64, 82.37);
\definecolor{fillColor}{RGB}{228,26,28}

\path[draw=drawColor,line width= 0.6pt,line join=round,line cap=round,fill=fillColor] (122.75,118.36) --
	(122.75, 99.86) --
	(140.52, 99.86) --
	(140.52,118.36) --
	(122.75,118.36) --
	cycle;

\path[draw=drawColor,line width= 1.1pt,line join=round] (122.75,108.28) -- (140.52,108.28);
\definecolor{fillColor}{gray}{0.20}

\path[draw=drawColor,line width= 0.4pt,line join=round,line cap=round,fill=fillColor] (155.33,151.05) circle (  1.96);

\path[draw=drawColor,line width= 0.4pt,line join=round,line cap=round,fill=fillColor] (155.33,150.22) circle (  1.96);

\path[draw=drawColor,line width= 0.4pt,line join=round,line cap=round,fill=fillColor] (155.33,153.23) circle (  1.96);

\path[draw=drawColor,line width= 0.4pt,line join=round,line cap=round,fill=fillColor] (155.33,153.23) circle (  1.96);

\path[draw=drawColor,line width= 0.6pt,line join=round] (155.33,118.13) -- (155.33,148.50);

\path[draw=drawColor,line width= 0.6pt,line join=round] (155.33, 96.96) -- (155.33, 83.21);
\definecolor{fillColor}{RGB}{55,126,184}

\path[draw=drawColor,line width= 0.6pt,line join=round,line cap=round,fill=fillColor] (146.44,118.13) --
	(146.44, 96.96) --
	(164.21, 96.96) --
	(164.21,118.13) --
	(146.44,118.13) --
	cycle;

\path[draw=drawColor,line width= 1.1pt,line join=round] (146.44,105.42) -- (164.21,105.42);

\path[draw=drawColor,line width= 0.6pt,line join=round] (179.02,135.24) -- (179.02,154.87);

\path[draw=drawColor,line width= 0.6pt,line join=round] (179.02,112.38) -- (179.02, 86.14);
\definecolor{fillColor}{RGB}{77,175,74}

\path[draw=drawColor,line width= 0.6pt,line join=round,line cap=round,fill=fillColor] (170.13,135.24) --
	(170.13,112.38) --
	(187.90,112.38) --
	(187.90,135.24) --
	(170.13,135.24) --
	cycle;

\path[draw=drawColor,line width= 1.1pt,line join=round] (170.13,121.53) -- (187.90,121.53);
\end{scope}
\begin{scope}
\path[clip] (198.73, 78.54) rectangle (274.54,158.60);
\definecolor{drawColor}{RGB}{255,255,255}

\path[draw=drawColor,line width= 0.3pt,line join=round] (198.73, 92.57) --
	(274.54, 92.57);

\path[draw=drawColor,line width= 0.3pt,line join=round] (198.73,113.37) --
	(274.54,113.37);

\path[draw=drawColor,line width= 0.3pt,line join=round] (198.73,134.17) --
	(274.54,134.17);

\path[draw=drawColor,line width= 0.3pt,line join=round] (198.73,154.96) --
	(274.54,154.96);

\path[draw=drawColor,line width= 0.6pt,line join=round] (198.73, 82.18) --
	(274.54, 82.18);

\path[draw=drawColor,line width= 0.6pt,line join=round] (198.73,102.97) --
	(274.54,102.97);

\path[draw=drawColor,line width= 0.6pt,line join=round] (198.73,123.77) --
	(274.54,123.77);

\path[draw=drawColor,line width= 0.6pt,line join=round] (198.73,144.57) --
	(274.54,144.57);

\path[draw=drawColor,line width= 0.6pt,line join=round] (212.94, 78.54) --
	(212.94,158.60);

\path[draw=drawColor,line width= 0.6pt,line join=round] (236.64, 78.54) --
	(236.64,158.60);

\path[draw=drawColor,line width= 0.6pt,line join=round] (260.33, 78.54) --
	(260.33,158.60);
\definecolor{drawColor}{gray}{0.20}

\path[draw=drawColor,line width= 0.6pt,line join=round] (212.94,131.55) -- (212.94,154.96);

\path[draw=drawColor,line width= 0.6pt,line join=round] (212.94,108.71) -- (212.94, 82.68);
\definecolor{fillColor}{RGB}{228,26,28}

\path[draw=drawColor,line width= 0.6pt,line join=round,line cap=round,fill=fillColor] (204.06,131.55) --
	(204.06,108.71) --
	(221.83,108.71) --
	(221.83,131.55) --
	(204.06,131.55) --
	cycle;

\path[draw=drawColor,line width= 1.1pt,line join=round] (204.06,119.98) -- (221.83,119.98);

\path[draw=drawColor,line width= 0.6pt,line join=round] (236.64,125.78) -- (236.64,152.03);

\path[draw=drawColor,line width= 0.6pt,line join=round] (236.64,103.09) -- (236.64, 84.92);
\definecolor{fillColor}{RGB}{55,126,184}

\path[draw=drawColor,line width= 0.6pt,line join=round,line cap=round,fill=fillColor] (227.75,125.78) --
	(227.75,103.09) --
	(245.52,103.09) --
	(245.52,125.78) --
	(227.75,125.78) --
	cycle;

\path[draw=drawColor,line width= 1.1pt,line join=round] (227.75,113.65) -- (245.52,113.65);

\path[draw=drawColor,line width= 0.6pt,line join=round] (260.33,141.13) -- (260.33,154.67);

\path[draw=drawColor,line width= 0.6pt,line join=round] (260.33,112.50) -- (260.33, 94.12);
\definecolor{fillColor}{RGB}{77,175,74}

\path[draw=drawColor,line width= 0.6pt,line join=round,line cap=round,fill=fillColor] (251.44,141.13) --
	(251.44,112.50) --
	(269.21,112.50) --
	(269.21,141.13) --
	(251.44,141.13) --
	cycle;

\path[draw=drawColor,line width= 1.1pt,line join=round] (251.44,127.13) -- (269.21,127.13);
\end{scope}
\begin{scope}
\path[clip] (280.04, 78.54) rectangle (355.85,158.60);
\definecolor{drawColor}{RGB}{255,255,255}

\path[draw=drawColor,line width= 0.3pt,line join=round] (280.04, 92.57) --
	(355.85, 92.57);

\path[draw=drawColor,line width= 0.3pt,line join=round] (280.04,113.37) --
	(355.85,113.37);

\path[draw=drawColor,line width= 0.3pt,line join=round] (280.04,134.17) --
	(355.85,134.17);

\path[draw=drawColor,line width= 0.3pt,line join=round] (280.04,154.96) --
	(355.85,154.96);

\path[draw=drawColor,line width= 0.6pt,line join=round] (280.04, 82.18) --
	(355.85, 82.18);

\path[draw=drawColor,line width= 0.6pt,line join=round] (280.04,102.97) --
	(355.85,102.97);

\path[draw=drawColor,line width= 0.6pt,line join=round] (280.04,123.77) --
	(355.85,123.77);

\path[draw=drawColor,line width= 0.6pt,line join=round] (280.04,144.57) --
	(355.85,144.57);

\path[draw=drawColor,line width= 0.6pt,line join=round] (294.25, 78.54) --
	(294.25,158.60);

\path[draw=drawColor,line width= 0.6pt,line join=round] (317.95, 78.54) --
	(317.95,158.60);

\path[draw=drawColor,line width= 0.6pt,line join=round] (341.64, 78.54) --
	(341.64,158.60);
\definecolor{drawColor}{gray}{0.20}

\path[draw=drawColor,line width= 0.6pt,line join=round] (294.25,137.88) -- (294.25,154.96);

\path[draw=drawColor,line width= 0.6pt,line join=round] (294.25,114.61) -- (294.25, 82.74);
\definecolor{fillColor}{RGB}{228,26,28}

\path[draw=drawColor,line width= 0.6pt,line join=round,line cap=round,fill=fillColor] (285.37,137.88) --
	(285.37,114.61) --
	(303.14,114.61) --
	(303.14,137.88) --
	(285.37,137.88) --
	cycle;

\path[draw=drawColor,line width= 1.1pt,line join=round] (285.37,125.50) -- (303.14,125.50);

\path[draw=drawColor,line width= 0.6pt,line join=round] (317.95,137.41) -- (317.95,154.67);

\path[draw=drawColor,line width= 0.6pt,line join=round] (317.95,110.82) -- (317.95, 85.97);
\definecolor{fillColor}{RGB}{55,126,184}

\path[draw=drawColor,line width= 0.6pt,line join=round,line cap=round,fill=fillColor] (309.06,137.41) --
	(309.06,110.82) --
	(326.83,110.82) --
	(326.83,137.41) --
	(309.06,137.41) --
	cycle;

\path[draw=drawColor,line width= 1.1pt,line join=round] (309.06,125.83) -- (326.83,125.83);

\path[draw=drawColor,line width= 0.6pt,line join=round] (341.64,141.15) -- (341.64,154.23);

\path[draw=drawColor,line width= 0.6pt,line join=round] (341.64,118.41) -- (341.64, 95.74);
\definecolor{fillColor}{RGB}{77,175,74}

\path[draw=drawColor,line width= 0.6pt,line join=round,line cap=round,fill=fillColor] (332.75,141.15) --
	(332.75,118.41) --
	(350.52,118.41) --
	(350.52,141.15) --
	(332.75,141.15) --
	cycle;

\path[draw=drawColor,line width= 1.1pt,line join=round] (332.75,129.37) -- (350.52,129.37);
\end{scope}
\begin{scope}
\path[clip] ( 36.11,158.60) rectangle (111.92,175.17);
\definecolor{drawColor}{RGB}{0,0,0}

\node[text=drawColor,anchor=base,inner sep=0pt, outer sep=0pt, scale=  0.70] at ( 74.02,163.86) {No Calificados};
\end{scope}
\begin{scope}
\path[clip] (117.42,158.60) rectangle (193.23,175.17);
\definecolor{drawColor}{RGB}{0,0,0}

\node[text=drawColor,anchor=base,inner sep=0pt, outer sep=0pt, scale=  0.70] at (155.33,163.86) {Operativos};
\end{scope}
\begin{scope}
\path[clip] (198.73,158.60) rectangle (274.54,175.17);
\definecolor{drawColor}{RGB}{0,0,0}

\node[text=drawColor,anchor=base,inner sep=0pt, outer sep=0pt, scale=  0.70] at (236.64,163.86) {Técnicos};
\end{scope}
\begin{scope}
\path[clip] (280.04,158.60) rectangle (355.85,175.17);
\definecolor{drawColor}{RGB}{0,0,0}

\node[text=drawColor,anchor=base,inner sep=0pt, outer sep=0pt, scale=  0.70] at (317.95,163.86) {Profesionales};
\end{scope}
\begin{scope}
\path[clip] (  0.00,  0.00) rectangle (361.35,180.67);
\definecolor{drawColor}{gray}{0.20}

\path[draw=drawColor,line width= 0.6pt,line join=round] ( 50.33, 75.79) --
	( 50.33, 78.54);

\path[draw=drawColor,line width= 0.6pt,line join=round] ( 74.02, 75.79) --
	( 74.02, 78.54);

\path[draw=drawColor,line width= 0.6pt,line join=round] ( 97.71, 75.79) --
	( 97.71, 78.54);
\end{scope}
\begin{scope}
\path[clip] (  0.00,  0.00) rectangle (361.35,180.67);
\definecolor{drawColor}{RGB}{0,0,0}

\node[text=drawColor,rotate= 90.00,anchor=base,inner sep=0pt, outer sep=0pt, scale=  0.60] at ( 54.39, 45.78) {Nació y vive};

\node[text=drawColor,rotate= 90.00,anchor=base,inner sep=0pt, outer sep=0pt, scale=  0.60] at ( 63.89, 45.78) {en Centro};

\node[text=drawColor,rotate= 90.00,anchor=base,inner sep=0pt, outer sep=0pt, scale=  0.60] at ( 78.08, 45.78) {Nació en Centro};

\node[text=drawColor,rotate= 90.00,anchor=base,inner sep=0pt, outer sep=0pt, scale=  0.60] at ( 87.58, 45.78) {vive en Norte};

\node[text=drawColor,rotate= 90.00,anchor=base,inner sep=0pt, outer sep=0pt, scale=  0.60] at (101.77, 45.78) {Nació en Centro};

\node[text=drawColor,rotate= 90.00,anchor=base,inner sep=0pt, outer sep=0pt, scale=  0.60] at (111.27, 45.78) {vive en Sur};
\end{scope}
\begin{scope}
\path[clip] (  0.00,  0.00) rectangle (361.35,180.67);
\definecolor{drawColor}{gray}{0.20}

\path[draw=drawColor,line width= 0.6pt,line join=round] (131.64, 75.79) --
	(131.64, 78.54);

\path[draw=drawColor,line width= 0.6pt,line join=round] (155.33, 75.79) --
	(155.33, 78.54);

\path[draw=drawColor,line width= 0.6pt,line join=round] (179.02, 75.79) --
	(179.02, 78.54);
\end{scope}
\begin{scope}
\path[clip] (  0.00,  0.00) rectangle (361.35,180.67);
\definecolor{drawColor}{RGB}{0,0,0}

\node[text=drawColor,rotate= 90.00,anchor=base,inner sep=0pt, outer sep=0pt, scale=  0.60] at (135.70, 45.78) {Nació y vive};

\node[text=drawColor,rotate= 90.00,anchor=base,inner sep=0pt, outer sep=0pt, scale=  0.60] at (145.20, 45.78) {en Centro};

\node[text=drawColor,rotate= 90.00,anchor=base,inner sep=0pt, outer sep=0pt, scale=  0.60] at (159.39, 45.78) {Nació en Centro};

\node[text=drawColor,rotate= 90.00,anchor=base,inner sep=0pt, outer sep=0pt, scale=  0.60] at (168.89, 45.78) {vive en Norte};

\node[text=drawColor,rotate= 90.00,anchor=base,inner sep=0pt, outer sep=0pt, scale=  0.60] at (183.08, 45.78) {Nació en Centro};

\node[text=drawColor,rotate= 90.00,anchor=base,inner sep=0pt, outer sep=0pt, scale=  0.60] at (192.58, 45.78) {vive en Sur};
\end{scope}
\begin{scope}
\path[clip] (  0.00,  0.00) rectangle (361.35,180.67);
\definecolor{drawColor}{gray}{0.20}

\path[draw=drawColor,line width= 0.6pt,line join=round] (212.94, 75.79) --
	(212.94, 78.54);

\path[draw=drawColor,line width= 0.6pt,line join=round] (236.64, 75.79) --
	(236.64, 78.54);

\path[draw=drawColor,line width= 0.6pt,line join=round] (260.33, 75.79) --
	(260.33, 78.54);
\end{scope}
\begin{scope}
\path[clip] (  0.00,  0.00) rectangle (361.35,180.67);
\definecolor{drawColor}{RGB}{0,0,0}

\node[text=drawColor,rotate= 90.00,anchor=base,inner sep=0pt, outer sep=0pt, scale=  0.60] at (217.01, 45.78) {Nació y vive};

\node[text=drawColor,rotate= 90.00,anchor=base,inner sep=0pt, outer sep=0pt, scale=  0.60] at (226.51, 45.78) {en Centro};

\node[text=drawColor,rotate= 90.00,anchor=base,inner sep=0pt, outer sep=0pt, scale=  0.60] at (240.70, 45.78) {Nació en Centro};

\node[text=drawColor,rotate= 90.00,anchor=base,inner sep=0pt, outer sep=0pt, scale=  0.60] at (250.20, 45.78) {vive en Norte};

\node[text=drawColor,rotate= 90.00,anchor=base,inner sep=0pt, outer sep=0pt, scale=  0.60] at (264.39, 45.78) {Nació en Centro};

\node[text=drawColor,rotate= 90.00,anchor=base,inner sep=0pt, outer sep=0pt, scale=  0.60] at (273.89, 45.78) {vive en Sur};
\end{scope}
\begin{scope}
\path[clip] (  0.00,  0.00) rectangle (361.35,180.67);
\definecolor{drawColor}{gray}{0.20}

\path[draw=drawColor,line width= 0.6pt,line join=round] (294.25, 75.79) --
	(294.25, 78.54);

\path[draw=drawColor,line width= 0.6pt,line join=round] (317.95, 75.79) --
	(317.95, 78.54);

\path[draw=drawColor,line width= 0.6pt,line join=round] (341.64, 75.79) --
	(341.64, 78.54);
\end{scope}
\begin{scope}
\path[clip] (  0.00,  0.00) rectangle (361.35,180.67);
\definecolor{drawColor}{RGB}{0,0,0}

\node[text=drawColor,rotate= 90.00,anchor=base,inner sep=0pt, outer sep=0pt, scale=  0.60] at (298.32, 45.78) {Nació y vive};

\node[text=drawColor,rotate= 90.00,anchor=base,inner sep=0pt, outer sep=0pt, scale=  0.60] at (307.82, 45.78) {en Centro};

\node[text=drawColor,rotate= 90.00,anchor=base,inner sep=0pt, outer sep=0pt, scale=  0.60] at (322.01, 45.78) {Nació en Centro};

\node[text=drawColor,rotate= 90.00,anchor=base,inner sep=0pt, outer sep=0pt, scale=  0.60] at (331.51, 45.78) {vive en Norte};

\node[text=drawColor,rotate= 90.00,anchor=base,inner sep=0pt, outer sep=0pt, scale=  0.60] at (345.70, 45.78) {Nació en Centro};

\node[text=drawColor,rotate= 90.00,anchor=base,inner sep=0pt, outer sep=0pt, scale=  0.60] at (355.20, 45.78) {vive en Sur};
\end{scope}
\begin{scope}
\path[clip] (  0.00,  0.00) rectangle (361.35,180.67);
\definecolor{drawColor}{RGB}{0,0,0}

\node[text=drawColor,anchor=base east,inner sep=0pt, outer sep=0pt, scale=  0.88] at ( 31.16, 79.15) {0};

\node[text=drawColor,anchor=base east,inner sep=0pt, outer sep=0pt, scale=  0.88] at ( 31.16, 99.94) {200};

\node[text=drawColor,anchor=base east,inner sep=0pt, outer sep=0pt, scale=  0.88] at ( 31.16,120.74) {400};

\node[text=drawColor,anchor=base east,inner sep=0pt, outer sep=0pt, scale=  0.88] at ( 31.16,141.54) {600};
\end{scope}
\begin{scope}
\path[clip] (  0.00,  0.00) rectangle (361.35,180.67);
\definecolor{drawColor}{gray}{0.20}

\path[draw=drawColor,line width= 0.6pt,line join=round] ( 33.36, 82.18) --
	( 36.11, 82.18);

\path[draw=drawColor,line width= 0.6pt,line join=round] ( 33.36,102.97) --
	( 36.11,102.97);

\path[draw=drawColor,line width= 0.6pt,line join=round] ( 33.36,123.77) --
	( 36.11,123.77);

\path[draw=drawColor,line width= 0.6pt,line join=round] ( 33.36,144.57) --
	( 36.11,144.57);
\end{scope}
\begin{scope}
\path[clip] (  0.00,  0.00) rectangle (361.35,180.67);
\definecolor{drawColor}{RGB}{0,0,0}

\node[text=drawColor,rotate= 90.00,anchor=base,inner sep=0pt, outer sep=0pt, scale=  0.6] at ( 13.08,118.57) {Ingreso laboral por hora};
\end{scope}
\end{tikzpicture}
  
\end{center}
\begin{flushleft}
\begin{scriptsize}
Fuente: Elaboración propia en base a EPH.\\
Nota: Los profesionales y técnicos son considerados de calificación alta mientras que los operativos y no calificados son considerados de calificación baja. Los migrantes están definidos como personas que vivían hace cinco años en otra provincia. Los nativos están definidos como personas que nacieron y viven en la misma provincia. Las estimaciones corresponden al período desde el segundo trimestre de 2016 hasta el cuarto trimestre de 2019.
\end{scriptsize}
\end{flushleft}
\end{figure}

\newpage
\subsection{Estado civil y organización familiar}

El estado civil y la organización familiar de las personas impacta directamente en las ganancias y los costos de migrar. Una persona soltera y sin hijos a su cargo tendrá que tener en consideración muchos menos factores en la decisión del éxodo que una persona con un proyecto familiar en marcha. Por esta razón es necesario analizar este determinante para los nativos y los migrantes en las distintas regiones con el fin de reconocer si existen heterogeneidades en los distintos grupos.

\begin{figure}[htbp!]
\begin{center}
\caption{\\Estado civil y situación familiar de los nativos y migrantes}
\label{figure:estadociv_mig}
% Created by tikzDevice version 0.12.3.1 on 2021-05-25 00:28:46
% !TEX encoding = UTF-8 Unicode
\begin{tikzpicture}[x=1pt,y=1pt]
\definecolor{fillColor}{RGB}{255,255,255}
\path[use as bounding box,fill=fillColor,fill opacity=0.00] (0,0) rectangle (433.62,289.08);
\begin{scope}
\path[clip] (  0.00,  0.00) rectangle (433.62,289.08);
\definecolor{drawColor}{RGB}{255,255,255}
\definecolor{fillColor}{RGB}{255,255,255}

\path[draw=drawColor,line width= 0.6pt,line join=round,line cap=round,fill=fillColor] (  0.00,  0.00) rectangle (433.62,289.08);
\end{scope}
\begin{scope}
\path[clip] ( 43.44, 30.69) rectangle (127.69,267.01);
\definecolor{drawColor}{RGB}{255,255,255}

\path[draw=drawColor,line width= 0.3pt,line join=round] ( 43.44, 68.28) --
	(127.69, 68.28);

\path[draw=drawColor,line width= 0.3pt,line join=round] ( 43.44,121.99) --
	(127.69,121.99);

\path[draw=drawColor,line width= 0.3pt,line join=round] ( 43.44,175.70) --
	(127.69,175.70);

\path[draw=drawColor,line width= 0.3pt,line join=round] ( 43.44,229.41) --
	(127.69,229.41);

\path[draw=drawColor,line width= 0.6pt,line join=round] ( 43.44, 41.43) --
	(127.69, 41.43);

\path[draw=drawColor,line width= 0.6pt,line join=round] ( 43.44, 95.14) --
	(127.69, 95.14);

\path[draw=drawColor,line width= 0.6pt,line join=round] ( 43.44,148.85) --
	(127.69,148.85);

\path[draw=drawColor,line width= 0.6pt,line join=round] ( 43.44,202.56) --
	(127.69,202.56);

\path[draw=drawColor,line width= 0.6pt,line join=round] ( 43.44,256.27) --
	(127.69,256.27);

\path[draw=drawColor,line width= 0.6pt,line join=round] ( 66.42, 30.69) --
	( 66.42,267.01);

\path[draw=drawColor,line width= 0.6pt,line join=round] (104.71, 30.69) --
	(104.71,267.01);
\definecolor{fillColor}{RGB}{228,26,28}

\path[fill=fillColor] ( 51.10,162.56) rectangle ( 81.74,256.27);
\definecolor{fillColor}{RGB}{55,126,184}

\path[fill=fillColor] ( 51.10,125.55) rectangle ( 81.74,162.56);
\definecolor{fillColor}{RGB}{77,175,74}

\path[fill=fillColor] ( 51.10, 81.06) rectangle ( 81.74,125.55);
\definecolor{fillColor}{RGB}{166,206,227}

\path[fill=fillColor] ( 51.10, 41.43) rectangle ( 81.74, 81.06);
\definecolor{fillColor}{RGB}{228,26,28}

\path[fill=fillColor] ( 89.40,168.44) rectangle (120.03,256.27);
\definecolor{fillColor}{RGB}{55,126,184}

\path[fill=fillColor] ( 89.40,150.12) rectangle (120.03,168.44);
\definecolor{fillColor}{RGB}{77,175,74}

\path[fill=fillColor] ( 89.40, 65.38) rectangle (120.03,150.12);
\definecolor{fillColor}{RGB}{166,206,227}

\path[fill=fillColor] ( 89.40, 41.43) rectangle (120.03, 65.38);
\definecolor{drawColor}{RGB}{0,0,0}

\node[text=drawColor,anchor=base,inner sep=0pt, outer sep=0pt, scale=  0.60] at ( 66.42,197.10) {43.62{\%}};

\node[text=drawColor,anchor=base,inner sep=0pt, outer sep=0pt, scale=  0.60] at ( 66.42,137.42) {17.23{\%}};

\node[text=drawColor,anchor=base,inner sep=0pt, outer sep=0pt, scale=  0.60] at ( 66.42, 95.92) {20.71{\%}};

\node[text=drawColor,anchor=base,inner sep=0pt, outer sep=0pt, scale=  0.60] at ( 66.42, 54.34) {18.45{\%}};

\node[text=drawColor,anchor=base,inner sep=0pt, outer sep=0pt, scale=  0.60] at (104.71,200.63) {40.88{\%}};

\node[text=drawColor,anchor=base,inner sep=0pt, outer sep=0pt, scale=  0.60] at (104.71,154.51) {8.53{\%}};

\node[text=drawColor,anchor=base,inner sep=0pt, outer sep=0pt, scale=  0.60] at (104.71, 96.34) {39.44{\%}};

\node[text=drawColor,anchor=base,inner sep=0pt, outer sep=0pt, scale=  0.60] at (104.71, 48.07) {11.15{\%}};
\end{scope}
\begin{scope}
\path[clip] (133.19, 30.69) rectangle (217.43,267.01);
\definecolor{drawColor}{RGB}{255,255,255}

\path[draw=drawColor,line width= 0.3pt,line join=round] (133.19, 68.28) --
	(217.43, 68.28);

\path[draw=drawColor,line width= 0.3pt,line join=round] (133.19,121.99) --
	(217.43,121.99);

\path[draw=drawColor,line width= 0.3pt,line join=round] (133.19,175.70) --
	(217.43,175.70);

\path[draw=drawColor,line width= 0.3pt,line join=round] (133.19,229.41) --
	(217.43,229.41);

\path[draw=drawColor,line width= 0.6pt,line join=round] (133.19, 41.43) --
	(217.43, 41.43);

\path[draw=drawColor,line width= 0.6pt,line join=round] (133.19, 95.14) --
	(217.43, 95.14);

\path[draw=drawColor,line width= 0.6pt,line join=round] (133.19,148.85) --
	(217.43,148.85);

\path[draw=drawColor,line width= 0.6pt,line join=round] (133.19,202.56) --
	(217.43,202.56);

\path[draw=drawColor,line width= 0.6pt,line join=round] (133.19,256.27) --
	(217.43,256.27);

\path[draw=drawColor,line width= 0.6pt,line join=round] (156.16, 30.69) --
	(156.16,267.01);

\path[draw=drawColor,line width= 0.6pt,line join=round] (194.46, 30.69) --
	(194.46,267.01);
\definecolor{fillColor}{RGB}{228,26,28}

\path[fill=fillColor] (140.85,166.53) rectangle (171.48,256.27);
\definecolor{fillColor}{RGB}{55,126,184}

\path[fill=fillColor] (140.85,132.78) rectangle (171.48,166.53);
\definecolor{fillColor}{RGB}{77,175,74}

\path[fill=fillColor] (140.85, 78.58) rectangle (171.48,132.78);
\definecolor{fillColor}{RGB}{166,206,227}

\path[fill=fillColor] (140.85, 41.43) rectangle (171.48, 78.58);
\definecolor{fillColor}{RGB}{228,26,28}

\path[fill=fillColor] (179.14,169.88) rectangle (209.78,256.27);
\definecolor{fillColor}{RGB}{55,126,184}

\path[fill=fillColor] (179.14,148.60) rectangle (209.78,169.88);
\definecolor{fillColor}{RGB}{77,175,74}

\path[fill=fillColor] (179.14, 65.62) rectangle (209.78,148.60);
\definecolor{fillColor}{RGB}{166,206,227}

\path[fill=fillColor] (179.14, 41.43) rectangle (209.78, 65.62);
\definecolor{drawColor}{RGB}{0,0,0}

\node[text=drawColor,anchor=base,inner sep=0pt, outer sep=0pt, scale=  0.60] at (156.16,199.49) {41.77{\%}};

\node[text=drawColor,anchor=base,inner sep=0pt, outer sep=0pt, scale=  0.60] at (156.16,143.34) {15.71{\%}};

\node[text=drawColor,anchor=base,inner sep=0pt, outer sep=0pt, scale=  0.60] at (156.16, 97.32) {25.22{\%}};

\node[text=drawColor,anchor=base,inner sep=0pt, outer sep=0pt, scale=  0.60] at (156.16, 53.35) {17.29{\%}};

\node[text=drawColor,anchor=base,inner sep=0pt, outer sep=0pt, scale=  0.60] at (194.46,201.50) {40.21{\%}};

\node[text=drawColor,anchor=base,inner sep=0pt, outer sep=0pt, scale=  0.60] at (194.46,154.18) {9.91{\%}};

\node[text=drawColor,anchor=base,inner sep=0pt, outer sep=0pt, scale=  0.60] at (194.46, 95.87) {38.63{\%}};

\node[text=drawColor,anchor=base,inner sep=0pt, outer sep=0pt, scale=  0.60] at (194.46, 48.17) {11.26{\%}};
\end{scope}
\begin{scope}
\path[clip] (222.93, 30.69) rectangle (307.18,267.01);
\definecolor{drawColor}{RGB}{255,255,255}

\path[draw=drawColor,line width= 0.3pt,line join=round] (222.93, 68.28) --
	(307.18, 68.28);

\path[draw=drawColor,line width= 0.3pt,line join=round] (222.93,121.99) --
	(307.18,121.99);

\path[draw=drawColor,line width= 0.3pt,line join=round] (222.93,175.70) --
	(307.18,175.70);

\path[draw=drawColor,line width= 0.3pt,line join=round] (222.93,229.41) --
	(307.18,229.41);

\path[draw=drawColor,line width= 0.6pt,line join=round] (222.93, 41.43) --
	(307.18, 41.43);

\path[draw=drawColor,line width= 0.6pt,line join=round] (222.93, 95.14) --
	(307.18, 95.14);

\path[draw=drawColor,line width= 0.6pt,line join=round] (222.93,148.85) --
	(307.18,148.85);

\path[draw=drawColor,line width= 0.6pt,line join=round] (222.93,202.56) --
	(307.18,202.56);

\path[draw=drawColor,line width= 0.6pt,line join=round] (222.93,256.27) --
	(307.18,256.27);

\path[draw=drawColor,line width= 0.6pt,line join=round] (245.91, 30.69) --
	(245.91,267.01);

\path[draw=drawColor,line width= 0.6pt,line join=round] (284.20, 30.69) --
	(284.20,267.01);
\definecolor{fillColor}{RGB}{228,26,28}

\path[fill=fillColor] (230.59,173.00) rectangle (261.23,256.27);
\definecolor{fillColor}{RGB}{55,126,184}

\path[fill=fillColor] (230.59,126.31) rectangle (261.23,173.00);
\definecolor{fillColor}{RGB}{77,175,74}

\path[fill=fillColor] (230.59, 85.07) rectangle (261.23,126.31);
\definecolor{fillColor}{RGB}{166,206,227}

\path[fill=fillColor] (230.59, 41.43) rectangle (261.23, 85.07);
\definecolor{fillColor}{RGB}{228,26,28}

\path[fill=fillColor] (268.89,170.72) rectangle (299.52,256.27);
\definecolor{fillColor}{RGB}{55,126,184}

\path[fill=fillColor] (268.89,139.78) rectangle (299.52,170.72);
\definecolor{fillColor}{RGB}{77,175,74}

\path[fill=fillColor] (268.89, 69.04) rectangle (299.52,139.78);
\definecolor{fillColor}{RGB}{166,206,227}

\path[fill=fillColor] (268.89, 41.43) rectangle (299.52, 69.04);
\definecolor{drawColor}{RGB}{0,0,0}

\node[text=drawColor,anchor=base,inner sep=0pt, outer sep=0pt, scale=  0.60] at (245.91,203.37) {38.76{\%}};

\node[text=drawColor,anchor=base,inner sep=0pt, outer sep=0pt, scale=  0.60] at (245.91,142.05) {21.73{\%}};

\node[text=drawColor,anchor=base,inner sep=0pt, outer sep=0pt, scale=  0.60] at (245.91, 98.63) {19.20{\%}};

\node[text=drawColor,anchor=base,inner sep=0pt, outer sep=0pt, scale=  0.60] at (245.91, 55.94) {20.31{\%}};

\node[text=drawColor,anchor=base,inner sep=0pt, outer sep=0pt, scale=  0.60] at (284.20,202.00) {39.82{\%}};

\node[text=drawColor,anchor=base,inner sep=0pt, outer sep=0pt, scale=  0.60] at (284.20,149.21) {14.40{\%}};

\node[text=drawColor,anchor=base,inner sep=0pt, outer sep=0pt, scale=  0.60] at (284.20, 94.39) {32.93{\%}};

\node[text=drawColor,anchor=base,inner sep=0pt, outer sep=0pt, scale=  0.60] at (284.20, 49.53) {12.85{\%}};
\end{scope}
\begin{scope}
\path[clip] ( 43.44,267.01) rectangle (127.69,283.58);
\definecolor{drawColor}{gray}{0.10}

\node[text=drawColor,anchor=base,inner sep=0pt, outer sep=0pt, scale=  0.70] at ( 85.57,272.26) {\textbf{Región 1}};
\end{scope}
\begin{scope}
\path[clip] (133.19,267.01) rectangle (217.43,283.58);
\definecolor{drawColor}{gray}{0.10}

\node[text=drawColor,anchor=base,inner sep=0pt, outer sep=0pt, scale=  0.70] at (175.31,272.26) {\textbf{Región 2}};
\end{scope}
\begin{scope}
\path[clip] (222.93,267.01) rectangle (307.18,283.58);
\definecolor{drawColor}{gray}{0.10}

\node[text=drawColor,anchor=base,inner sep=0pt, outer sep=0pt, scale=  0.70] at (265.06,272.26) {\textbf{Región 3}};
\end{scope}
\begin{scope}
\path[clip] (  0.00,  0.00) rectangle (433.62,289.08);
\definecolor{drawColor}{gray}{0.20}

\path[draw=drawColor,line width= 0.6pt,line join=round] ( 66.42, 27.94) --
	( 66.42, 30.69);

\path[draw=drawColor,line width= 0.6pt,line join=round] (104.71, 27.94) --
	(104.71, 30.69);
\end{scope}
\begin{scope}
\path[clip] (  0.00,  0.00) rectangle (433.62,289.08);
\definecolor{drawColor}{RGB}{0,0,0}

\node[text=drawColor,anchor=base,inner sep=0pt, outer sep=0pt, scale=  0.70] at ( 66.42, 19.68) {Migrantes};

\node[text=drawColor,anchor=base,inner sep=0pt, outer sep=0pt, scale=  0.70] at (104.71, 19.68) {Nativos};
\end{scope}
\begin{scope}
\path[clip] (  0.00,  0.00) rectangle (433.62,289.08);
\definecolor{drawColor}{gray}{0.20}

\path[draw=drawColor,line width= 0.6pt,line join=round] (156.16, 27.94) --
	(156.16, 30.69);

\path[draw=drawColor,line width= 0.6pt,line join=round] (194.46, 27.94) --
	(194.46, 30.69);
\end{scope}
\begin{scope}
\path[clip] (  0.00,  0.00) rectangle (433.62,289.08);
\definecolor{drawColor}{RGB}{0,0,0}

\node[text=drawColor,anchor=base,inner sep=0pt, outer sep=0pt, scale=  0.70] at (156.16, 19.68) {Migrantes};

\node[text=drawColor,anchor=base,inner sep=0pt, outer sep=0pt, scale=  0.70] at (194.46, 19.68) {Nativos};
\end{scope}
\begin{scope}
\path[clip] (  0.00,  0.00) rectangle (433.62,289.08);
\definecolor{drawColor}{gray}{0.20}

\path[draw=drawColor,line width= 0.6pt,line join=round] (245.91, 27.94) --
	(245.91, 30.69);

\path[draw=drawColor,line width= 0.6pt,line join=round] (284.20, 27.94) --
	(284.20, 30.69);
\end{scope}
\begin{scope}
\path[clip] (  0.00,  0.00) rectangle (433.62,289.08);
\definecolor{drawColor}{RGB}{0,0,0}

\node[text=drawColor,anchor=base,inner sep=0pt, outer sep=0pt, scale=  0.70] at (245.91, 19.68) {Migrantes};

\node[text=drawColor,anchor=base,inner sep=0pt, outer sep=0pt, scale=  0.70] at (284.20, 19.68) {Nativos};
\end{scope}
\begin{scope}
\path[clip] (  0.00,  0.00) rectangle (433.62,289.08);
\definecolor{drawColor}{RGB}{0,0,0}

\node[text=drawColor,anchor=base east,inner sep=0pt, outer sep=0pt, scale=  0.88] at ( 38.49, 38.40) {0{\%}};

\node[text=drawColor,anchor=base east,inner sep=0pt, outer sep=0pt, scale=  0.88] at ( 38.49, 92.11) {25{\%}};

\node[text=drawColor,anchor=base east,inner sep=0pt, outer sep=0pt, scale=  0.88] at ( 38.49,145.82) {50{\%}};

\node[text=drawColor,anchor=base east,inner sep=0pt, outer sep=0pt, scale=  0.88] at ( 38.49,199.53) {75{\%}};

\node[text=drawColor,anchor=base east,inner sep=0pt, outer sep=0pt, scale=  0.88] at ( 38.49,253.24) {100{\%}};
\end{scope}
\begin{scope}
\path[clip] (  0.00,  0.00) rectangle (433.62,289.08);
\definecolor{drawColor}{gray}{0.20}

\path[draw=drawColor,line width= 0.6pt,line join=round] ( 40.69, 41.43) --
	( 43.44, 41.43);

\path[draw=drawColor,line width= 0.6pt,line join=round] ( 40.69, 95.14) --
	( 43.44, 95.14);

\path[draw=drawColor,line width= 0.6pt,line join=round] ( 40.69,148.85) --
	( 43.44,148.85);

\path[draw=drawColor,line width= 0.6pt,line join=round] ( 40.69,202.56) --
	( 43.44,202.56);

\path[draw=drawColor,line width= 0.6pt,line join=round] ( 40.69,256.27) --
	( 43.44,256.27);
\end{scope}
\begin{scope}
\path[clip] (  0.00,  0.00) rectangle (433.62,289.08);
\definecolor{fillColor}{RGB}{255,255,255}

\path[fill=fillColor] (318.18,106.83) rectangle (428.12,190.86);
\end{scope}
\begin{scope}
\path[clip] (  0.00,  0.00) rectangle (433.62,289.08);
\definecolor{fillColor}{gray}{0.95}

\path[fill=fillColor] (323.68,155.69) rectangle (338.13,170.15);
\end{scope}
\begin{scope}
\path[clip] (  0.00,  0.00) rectangle (433.62,289.08);
\definecolor{fillColor}{RGB}{228,26,28}

\path[fill=fillColor] (324.39,156.41) rectangle (337.42,169.44);
\end{scope}
\begin{scope}
\path[clip] (  0.00,  0.00) rectangle (433.62,289.08);
\definecolor{fillColor}{gray}{0.95}

\path[fill=fillColor] (323.68,141.24) rectangle (338.13,155.69);
\end{scope}
\begin{scope}
\path[clip] (  0.00,  0.00) rectangle (433.62,289.08);
\definecolor{fillColor}{RGB}{55,126,184}

\path[fill=fillColor] (324.39,141.95) rectangle (337.42,154.98);
\end{scope}
\begin{scope}
\path[clip] (  0.00,  0.00) rectangle (433.62,289.08);
\definecolor{fillColor}{gray}{0.95}

\path[fill=fillColor] (323.68,126.79) rectangle (338.13,141.24);
\end{scope}
\begin{scope}
\path[clip] (  0.00,  0.00) rectangle (433.62,289.08);
\definecolor{fillColor}{RGB}{77,175,74}

\path[fill=fillColor] (324.39,127.50) rectangle (337.42,140.53);
\end{scope}
\begin{scope}
\path[clip] (  0.00,  0.00) rectangle (433.62,289.08);
\definecolor{fillColor}{gray}{0.95}

\path[fill=fillColor] (323.68,112.33) rectangle (338.13,126.79);
\end{scope}
\begin{scope}
\path[clip] (  0.00,  0.00) rectangle (433.62,289.08);
\definecolor{fillColor}{RGB}{166,206,227}

\path[fill=fillColor] (324.39,113.04) rectangle (337.42,126.07);
\end{scope}
\begin{scope}
\path[clip] (  0.00,  0.00) rectangle (433.62,289.08);
\definecolor{drawColor}{RGB}{0,0,0}

\node[text=drawColor,anchor=base west,inner sep=0pt, outer sep=0pt, scale=  0.60] at (343.63,159.89) {Casado/a con hijo/a};
\end{scope}
\begin{scope}
\path[clip] (  0.00,  0.00) rectangle (433.62,289.08);
\definecolor{drawColor}{RGB}{0,0,0}

\node[text=drawColor,anchor=base west,inner sep=0pt, outer sep=0pt, scale=  0.60] at (343.63,145.44) {Casado/a sin hijo/a};
\end{scope}
\begin{scope}
\path[clip] (  0.00,  0.00) rectangle (433.62,289.08);
\definecolor{drawColor}{RGB}{0,0,0}

\node[text=drawColor,anchor=base west,inner sep=0pt, outer sep=0pt, scale=  0.60] at (343.63,130.98) {Soltero/a con hijo/a};
\end{scope}
\begin{scope}
\path[clip] (  0.00,  0.00) rectangle (433.62,289.08);
\definecolor{drawColor}{RGB}{0,0,0}

\node[text=drawColor,anchor=base west,inner sep=0pt, outer sep=0pt, scale=  0.60] at (343.63,116.53) {Soltero/a sin hijo/a};
\end{scope}
\end{tikzpicture}
 
\end{center}
\begin{flushleft}
\begin{scriptsize}
Fuente: Elaboración propia en base a EPH.\\
Nota: Solamente se tienen en cuenta nativos y migrantes mayores a 18 años. Los migrantes están definidos como personas que vivían hace cinco años en otra provincia. Los nativos están definidos como personas que nacieron y viven en la misma provincia. Las estimaciones corresponden al período desde el segundo trimestre de 2016 hasta el cuarto trimestre de 2019.
\end{scriptsize}
\end{flushleft}
\end{figure}

En la Figura \ref{figure:estadociv_mig} se puede observar como los porcentajes de personas solteras sin hijos son mayores para los migrantes que para los nativos en las tres regiones. En términos relativos, es mucho más frecuente encontar un nativo que este soltero y con hijos que un migrante, lo que indica una importancia elevada de los hijos en la decisión migratoria en las familias con padres solteros.

En el caso de las personas que se encuentran en pareja sin hijos existe una diferencia entre los migrantes y los nativos, siendo más frecuente encontrar a migrantes en esta situación en las tres regiones. Esta relación indica que la condición de pareja, sin tener ningun hijo a su cargo, pareciera reforzar la decisión migratoria.

\subsection{Nivel educativo}

El nivel educativo de las personas es un factor extremadamente importante en la decisión migratoria, sobre todo considerando la posibilidad de obtener mejores ofertas laborales o mayores retornos  en provincias.

\begin{figure}[htbp!]
\begin{center}
\caption{\\Estado civil y situación familiar de los nativos y migrantes}
\label{figure:niveled_mig}
% Created by tikzDevice version 0.12.3.1 on 2021-07-03 12:16:23
% !TEX encoding = UTF-8 Unicode
\begin{tikzpicture}[x=1pt,y=1pt]
\definecolor{fillColor}{RGB}{255,255,255}
\path[use as bounding box,fill=fillColor,fill opacity=0.00] (0,0) rectangle (433.62,289.08);
\begin{scope}
\path[clip] (  0.00,  0.00) rectangle (433.62,289.08);
\definecolor{drawColor}{RGB}{255,255,255}
\definecolor{fillColor}{RGB}{255,255,255}

\path[draw=drawColor,line width= 0.6pt,line join=round,line cap=round,fill=fillColor] (  0.00,  0.00) rectangle (433.62,289.08);
\end{scope}
\begin{scope}
\path[clip] ( 43.44, 30.69) rectangle (134.22,267.01);
\definecolor{drawColor}{RGB}{255,255,255}

\path[draw=drawColor,line width= 0.3pt,line join=round] ( 43.44, 68.28) --
	(134.22, 68.28);

\path[draw=drawColor,line width= 0.3pt,line join=round] ( 43.44,121.99) --
	(134.22,121.99);

\path[draw=drawColor,line width= 0.3pt,line join=round] ( 43.44,175.70) --
	(134.22,175.70);

\path[draw=drawColor,line width= 0.3pt,line join=round] ( 43.44,229.41) --
	(134.22,229.41);

\path[draw=drawColor,line width= 0.6pt,line join=round] ( 43.44, 41.43) --
	(134.22, 41.43);

\path[draw=drawColor,line width= 0.6pt,line join=round] ( 43.44, 95.14) --
	(134.22, 95.14);

\path[draw=drawColor,line width= 0.6pt,line join=round] ( 43.44,148.85) --
	(134.22,148.85);

\path[draw=drawColor,line width= 0.6pt,line join=round] ( 43.44,202.56) --
	(134.22,202.56);

\path[draw=drawColor,line width= 0.6pt,line join=round] ( 43.44,256.27) --
	(134.22,256.27);

\path[draw=drawColor,line width= 0.6pt,line join=round] ( 68.20, 30.69) --
	( 68.20,267.01);

\path[draw=drawColor,line width= 0.6pt,line join=round] (109.46, 30.69) --
	(109.46,267.01);
\definecolor{fillColor}{RGB}{228,26,28}

\path[fill=fillColor] ( 51.70,189.37) rectangle ( 84.71,256.27);
\definecolor{fillColor}{RGB}{55,126,184}

\path[fill=fillColor] ( 51.70, 95.92) rectangle ( 84.71,189.37);
\definecolor{fillColor}{RGB}{77,175,74}

\path[fill=fillColor] ( 51.70, 41.43) rectangle ( 84.71, 95.92);
\definecolor{fillColor}{RGB}{228,26,28}

\path[fill=fillColor] ( 92.96,205.52) rectangle (125.97,256.27);
\definecolor{fillColor}{RGB}{55,126,184}

\path[fill=fillColor] ( 92.96,121.48) rectangle (125.97,205.52);
\definecolor{fillColor}{RGB}{77,175,74}

\path[fill=fillColor] ( 92.96, 41.43) rectangle (125.97,121.48);
\definecolor{drawColor}{RGB}{0,0,0}

\node[text=drawColor,anchor=base,inner sep=0pt, outer sep=0pt, scale=  0.70] at ( 68.20,213.19) {31.14{\%}};

\node[text=drawColor,anchor=base,inner sep=0pt, outer sep=0pt, scale=  0.70] at ( 68.20,130.36) {43.50{\%}};

\node[text=drawColor,anchor=base,inner sep=0pt, outer sep=0pt, scale=  0.70] at ( 68.20, 60.28) {25.36{\%}};

\node[text=drawColor,anchor=base,inner sep=0pt, outer sep=0pt, scale=  0.70] at (109.46,222.88) {23.62{\%}};

\node[text=drawColor,anchor=base,inner sep=0pt, outer sep=0pt, scale=  0.70] at (109.46,152.15) {39.12{\%}};

\node[text=drawColor,anchor=base,inner sep=0pt, outer sep=0pt, scale=  0.70] at (109.46, 70.51) {37.26{\%}};
\end{scope}
\begin{scope}
\path[clip] (139.72, 30.69) rectangle (230.50,267.01);
\definecolor{drawColor}{RGB}{255,255,255}

\path[draw=drawColor,line width= 0.3pt,line join=round] (139.72, 68.28) --
	(230.50, 68.28);

\path[draw=drawColor,line width= 0.3pt,line join=round] (139.72,121.99) --
	(230.50,121.99);

\path[draw=drawColor,line width= 0.3pt,line join=round] (139.72,175.70) --
	(230.50,175.70);

\path[draw=drawColor,line width= 0.3pt,line join=round] (139.72,229.41) --
	(230.50,229.41);

\path[draw=drawColor,line width= 0.6pt,line join=round] (139.72, 41.43) --
	(230.50, 41.43);

\path[draw=drawColor,line width= 0.6pt,line join=round] (139.72, 95.14) --
	(230.50, 95.14);

\path[draw=drawColor,line width= 0.6pt,line join=round] (139.72,148.85) --
	(230.50,148.85);

\path[draw=drawColor,line width= 0.6pt,line join=round] (139.72,202.56) --
	(230.50,202.56);

\path[draw=drawColor,line width= 0.6pt,line join=round] (139.72,256.27) --
	(230.50,256.27);

\path[draw=drawColor,line width= 0.6pt,line join=round] (164.48, 30.69) --
	(164.48,267.01);

\path[draw=drawColor,line width= 0.6pt,line join=round] (205.74, 30.69) --
	(205.74,267.01);
\definecolor{fillColor}{RGB}{228,26,28}

\path[fill=fillColor] (147.97,184.34) rectangle (180.99,256.27);
\definecolor{fillColor}{RGB}{55,126,184}

\path[fill=fillColor] (147.97, 89.55) rectangle (180.99,184.34);
\definecolor{fillColor}{RGB}{77,175,74}

\path[fill=fillColor] (147.97, 41.43) rectangle (180.99, 89.55);
\definecolor{fillColor}{RGB}{228,26,28}

\path[fill=fillColor] (189.24,215.57) rectangle (222.25,256.27);
\definecolor{fillColor}{RGB}{55,126,184}

\path[fill=fillColor] (189.24,132.09) rectangle (222.25,215.57);
\definecolor{fillColor}{RGB}{77,175,74}

\path[fill=fillColor] (189.24, 41.43) rectangle (222.25,132.09);
\definecolor{drawColor}{RGB}{0,0,0}

\node[text=drawColor,anchor=base,inner sep=0pt, outer sep=0pt, scale=  0.70] at (164.48,210.17) {33.48{\%}};

\node[text=drawColor,anchor=base,inner sep=0pt, outer sep=0pt, scale=  0.70] at (164.48,124.53) {44.12{\%}};

\node[text=drawColor,anchor=base,inner sep=0pt, outer sep=0pt, scale=  0.70] at (164.48, 57.74) {22.40{\%}};

\node[text=drawColor,anchor=base,inner sep=0pt, outer sep=0pt, scale=  0.70] at (205.74,228.91) {18.94{\%}};

\node[text=drawColor,anchor=base,inner sep=0pt, outer sep=0pt, scale=  0.70] at (205.74,162.54) {38.86{\%}};

\node[text=drawColor,anchor=base,inner sep=0pt, outer sep=0pt, scale=  0.70] at (205.74, 74.75) {42.20{\%}};
\end{scope}
\begin{scope}
\path[clip] (236.00, 30.69) rectangle (326.78,267.01);
\definecolor{drawColor}{RGB}{255,255,255}

\path[draw=drawColor,line width= 0.3pt,line join=round] (236.00, 68.28) --
	(326.78, 68.28);

\path[draw=drawColor,line width= 0.3pt,line join=round] (236.00,121.99) --
	(326.78,121.99);

\path[draw=drawColor,line width= 0.3pt,line join=round] (236.00,175.70) --
	(326.78,175.70);

\path[draw=drawColor,line width= 0.3pt,line join=round] (236.00,229.41) --
	(326.78,229.41);

\path[draw=drawColor,line width= 0.6pt,line join=round] (236.00, 41.43) --
	(326.78, 41.43);

\path[draw=drawColor,line width= 0.6pt,line join=round] (236.00, 95.14) --
	(326.78, 95.14);

\path[draw=drawColor,line width= 0.6pt,line join=round] (236.00,148.85) --
	(326.78,148.85);

\path[draw=drawColor,line width= 0.6pt,line join=round] (236.00,202.56) --
	(326.78,202.56);

\path[draw=drawColor,line width= 0.6pt,line join=round] (236.00,256.27) --
	(326.78,256.27);

\path[draw=drawColor,line width= 0.6pt,line join=round] (260.76, 30.69) --
	(260.76,267.01);

\path[draw=drawColor,line width= 0.6pt,line join=round] (302.02, 30.69) --
	(302.02,267.01);
\definecolor{fillColor}{RGB}{228,26,28}

\path[fill=fillColor] (244.25,171.28) rectangle (277.26,256.27);
\definecolor{fillColor}{RGB}{55,126,184}

\path[fill=fillColor] (244.25, 90.05) rectangle (277.26,171.28);
\definecolor{fillColor}{RGB}{77,175,74}

\path[fill=fillColor] (244.25, 41.43) rectangle (277.26, 90.05);
\definecolor{fillColor}{RGB}{228,26,28}

\path[fill=fillColor] (285.52,221.81) rectangle (318.53,256.27);
\definecolor{fillColor}{RGB}{55,126,184}

\path[fill=fillColor] (285.52,138.33) rectangle (318.53,221.81);
\definecolor{fillColor}{RGB}{77,175,74}

\path[fill=fillColor] (285.52, 41.43) rectangle (318.53,138.33);
\definecolor{drawColor}{RGB}{0,0,0}

\node[text=drawColor,anchor=base,inner sep=0pt, outer sep=0pt, scale=  0.70] at (260.76,202.33) {39.56{\%}};

\node[text=drawColor,anchor=base,inner sep=0pt, outer sep=0pt, scale=  0.70] at (260.76,119.60) {37.81{\%}};

\node[text=drawColor,anchor=base,inner sep=0pt, outer sep=0pt, scale=  0.70] at (260.76, 57.94) {22.63{\%}};

\node[text=drawColor,anchor=base,inner sep=0pt, outer sep=0pt, scale=  0.70] at (302.02,232.65) {16.04{\%}};

\node[text=drawColor,anchor=base,inner sep=0pt, outer sep=0pt, scale=  0.70] at (302.02,168.78) {38.86{\%}};

\node[text=drawColor,anchor=base,inner sep=0pt, outer sep=0pt, scale=  0.70] at (302.02, 77.25) {45.10{\%}};
\end{scope}
\begin{scope}
\path[clip] ( 43.44,267.01) rectangle (134.22,283.58);
\definecolor{drawColor}{gray}{0.10}

\node[text=drawColor,anchor=base,inner sep=0pt, outer sep=0pt, scale=  0.88] at ( 88.83,272.26) {\textbf{Región Centro}};
\end{scope}
\begin{scope}
\path[clip] (139.72,267.01) rectangle (230.50,283.58);
\definecolor{drawColor}{gray}{0.10}

\node[text=drawColor,anchor=base,inner sep=0pt, outer sep=0pt, scale=  0.88] at (185.11,272.26) {\textbf{Región Norte}};
\end{scope}
\begin{scope}
\path[clip] (236.00,267.01) rectangle (326.78,283.58);
\definecolor{drawColor}{gray}{0.10}

\node[text=drawColor,anchor=base,inner sep=0pt, outer sep=0pt, scale=  0.88] at (281.39,272.26) {\textbf{Región Sur}};
\end{scope}
\begin{scope}
\path[clip] (  0.00,  0.00) rectangle (433.62,289.08);
\definecolor{drawColor}{gray}{0.20}

\path[draw=drawColor,line width= 0.6pt,line join=round] ( 68.20, 27.94) --
	( 68.20, 30.69);

\path[draw=drawColor,line width= 0.6pt,line join=round] (109.46, 27.94) --
	(109.46, 30.69);
\end{scope}
\begin{scope}
\path[clip] (  0.00,  0.00) rectangle (433.62,289.08);
\definecolor{drawColor}{RGB}{0,0,0}

\node[text=drawColor,anchor=base,inner sep=0pt, outer sep=0pt, scale=  0.70] at ( 68.20, 19.68) {Migrantes};

\node[text=drawColor,anchor=base,inner sep=0pt, outer sep=0pt, scale=  0.70] at (109.46, 19.68) {Nativos};
\end{scope}
\begin{scope}
\path[clip] (  0.00,  0.00) rectangle (433.62,289.08);
\definecolor{drawColor}{gray}{0.20}

\path[draw=drawColor,line width= 0.6pt,line join=round] (164.48, 27.94) --
	(164.48, 30.69);

\path[draw=drawColor,line width= 0.6pt,line join=round] (205.74, 27.94) --
	(205.74, 30.69);
\end{scope}
\begin{scope}
\path[clip] (  0.00,  0.00) rectangle (433.62,289.08);
\definecolor{drawColor}{RGB}{0,0,0}

\node[text=drawColor,anchor=base,inner sep=0pt, outer sep=0pt, scale=  0.70] at (164.48, 19.68) {Migrantes};

\node[text=drawColor,anchor=base,inner sep=0pt, outer sep=0pt, scale=  0.70] at (205.74, 19.68) {Nativos};
\end{scope}
\begin{scope}
\path[clip] (  0.00,  0.00) rectangle (433.62,289.08);
\definecolor{drawColor}{gray}{0.20}

\path[draw=drawColor,line width= 0.6pt,line join=round] (260.76, 27.94) --
	(260.76, 30.69);

\path[draw=drawColor,line width= 0.6pt,line join=round] (302.02, 27.94) --
	(302.02, 30.69);
\end{scope}
\begin{scope}
\path[clip] (  0.00,  0.00) rectangle (433.62,289.08);
\definecolor{drawColor}{RGB}{0,0,0}

\node[text=drawColor,anchor=base,inner sep=0pt, outer sep=0pt, scale=  0.70] at (260.76, 19.68) {Migrantes};

\node[text=drawColor,anchor=base,inner sep=0pt, outer sep=0pt, scale=  0.70] at (302.02, 19.68) {Nativos};
\end{scope}
\begin{scope}
\path[clip] (  0.00,  0.00) rectangle (433.62,289.08);
\definecolor{drawColor}{RGB}{0,0,0}

\node[text=drawColor,anchor=base east,inner sep=0pt, outer sep=0pt, scale=  0.88] at ( 38.49, 38.40) {0{\%}};

\node[text=drawColor,anchor=base east,inner sep=0pt, outer sep=0pt, scale=  0.88] at ( 38.49, 92.11) {25{\%}};

\node[text=drawColor,anchor=base east,inner sep=0pt, outer sep=0pt, scale=  0.88] at ( 38.49,145.82) {50{\%}};

\node[text=drawColor,anchor=base east,inner sep=0pt, outer sep=0pt, scale=  0.88] at ( 38.49,199.53) {75{\%}};

\node[text=drawColor,anchor=base east,inner sep=0pt, outer sep=0pt, scale=  0.88] at ( 38.49,253.24) {100{\%}};
\end{scope}
\begin{scope}
\path[clip] (  0.00,  0.00) rectangle (433.62,289.08);
\definecolor{drawColor}{gray}{0.20}

\path[draw=drawColor,line width= 0.6pt,line join=round] ( 40.69, 41.43) --
	( 43.44, 41.43);

\path[draw=drawColor,line width= 0.6pt,line join=round] ( 40.69, 95.14) --
	( 43.44, 95.14);

\path[draw=drawColor,line width= 0.6pt,line join=round] ( 40.69,148.85) --
	( 43.44,148.85);

\path[draw=drawColor,line width= 0.6pt,line join=round] ( 40.69,202.56) --
	( 43.44,202.56);

\path[draw=drawColor,line width= 0.6pt,line join=round] ( 40.69,256.27) --
	( 43.44,256.27);
\end{scope}
\begin{scope}
\path[clip] (  0.00,  0.00) rectangle (433.62,289.08);
\definecolor{fillColor}{RGB}{255,255,255}

\path[fill=fillColor] (337.78,109.83) rectangle (428.12,187.87);
\end{scope}
\begin{scope}
\path[clip] (  0.00,  0.00) rectangle (433.62,289.08);
\definecolor{fillColor}{gray}{0.95}

\path[fill=fillColor] (343.28,149.88) rectangle (357.73,167.15);
\end{scope}
\begin{scope}
\path[clip] (  0.00,  0.00) rectangle (433.62,289.08);
\definecolor{fillColor}{RGB}{228,26,28}

\path[fill=fillColor] (343.99,150.59) rectangle (357.02,166.44);
\end{scope}
\begin{scope}
\path[clip] (  0.00,  0.00) rectangle (433.62,289.08);
\definecolor{fillColor}{gray}{0.95}

\path[fill=fillColor] (343.28,132.60) rectangle (357.73,149.88);
\end{scope}
\begin{scope}
\path[clip] (  0.00,  0.00) rectangle (433.62,289.08);
\definecolor{fillColor}{RGB}{55,126,184}

\path[fill=fillColor] (343.99,133.31) rectangle (357.02,149.17);
\end{scope}
\begin{scope}
\path[clip] (  0.00,  0.00) rectangle (433.62,289.08);
\definecolor{fillColor}{gray}{0.95}

\path[fill=fillColor] (343.28,115.33) rectangle (357.73,132.60);
\end{scope}
\begin{scope}
\path[clip] (  0.00,  0.00) rectangle (433.62,289.08);
\definecolor{fillColor}{RGB}{77,175,74}

\path[fill=fillColor] (343.99,116.04) rectangle (357.02,131.89);
\end{scope}
\begin{scope}
\path[clip] (  0.00,  0.00) rectangle (433.62,289.08);
\definecolor{drawColor}{RGB}{0,0,0}

\node[text=drawColor,anchor=base west,inner sep=0pt, outer sep=0pt, scale=  0.70] at (363.23,160.24) {Nivel educativo};

\node[text=drawColor,anchor=base west,inner sep=0pt, outer sep=0pt, scale=  0.70] at (363.23,150.73) {elevado};
\end{scope}
\begin{scope}
\path[clip] (  0.00,  0.00) rectangle (433.62,289.08);
\definecolor{drawColor}{RGB}{0,0,0}

\node[text=drawColor,anchor=base west,inner sep=0pt, outer sep=0pt, scale=  0.70] at (363.23,142.96) {Nivel educativo};

\node[text=drawColor,anchor=base west,inner sep=0pt, outer sep=0pt, scale=  0.70] at (363.23,133.46) {medio};
\end{scope}
\begin{scope}
\path[clip] (  0.00,  0.00) rectangle (433.62,289.08);
\definecolor{drawColor}{RGB}{0,0,0}

\node[text=drawColor,anchor=base west,inner sep=0pt, outer sep=0pt, scale=  0.70] at (363.23,125.69) {Nivel educativo};

\node[text=drawColor,anchor=base west,inner sep=0pt, outer sep=0pt, scale=  0.70] at (363.23,116.18) {Bajo};
\end{scope}
\end{tikzpicture}
 
\end{center}
\begin{flushleft}
\begin{scriptsize}
Fuente: Elaboración propia en base a EPH.\\
Nota: El nivel educativo bajo está compuesto por las personas sin intrucción o con primaria completa, el nivel educativo medio por personas con secundario completo, y el nivel educativo alto por personas con nivel terciario o universitario. Solamente se tienen en cuenta nativos y migrantes mayores a 25 años. Los migrantes están definidos como personas que vivían hace cinco años en otra provincia. Los nativos están definidos como personas que nacieron y viven en la misma provincia. Las estimaciones corresponden al período desde el segundo trimestre de 2016 hasta el cuarto trimestre de 2019.
\end{scriptsize}
\end{flushleft}
\end{figure}

En la Figura \ref{figure:niveled_mig} se puede observar las diferencias en los niveles educativos de los nativos y migrantes mayores a 25 años, dentro del nivel educativo bajo se encuentran aquellos sin instrucción o cuyo nivel máximo de educación es el primario, el nivel educativo medio está compuesto por aquellos cuyo nivel educativo maximo alcanzado es el secundario, y por último los que componen el grupo de nivel educativo alto son aquellos cuyo nivel educativo máximo alcanzado es la educación superior o universitaria.

Las regiones Norte, Centro y Sur se caracterizan por una mayor proporción de migrantes con nivel educativo alto y una menor proporción de migrantes con nivel educativo bajo, en comparación con la población nativa. Esto brinda una panorama de que en todas las regiones los migrantes poseen un mayor nivel educativo que los nativos. Los elevados niveles de instrucción en conjunto con la mayor calificación de los migrantes, dan indicios de que la migración interna en Argentina está marcada por un éxodo de mano de obra con un elevado nivel de capital humano.


\newpage
\section{Factores determinantes de la migración}
\subsection{Modelo}
El análisis de los factores determinantes de la migración  encuentra su base en que las personas se desplazan a partir de una decisión racional que maximiza su utilidad. En esta decisión intervienen una serie de factores socioeconómicos que hacen más o menos propenso que el individuo decida permanecer en su localidad de origen o migrar hacia alguna otra región.

Para cuantificar el impacto de los distintos factores en la decisión migratoria, se recurre a la formulación de dos modelo de probabilidad, un Logit Binomial y un Logit Multinomial. El modelo logit binomial será utilizado como un modelo de elección de dos alternativas mutuamente excluyentes, la de migrar o no migrar. Por otro lado, en el modelo Logit Multinomial los individuos se enfrentan a $j$ alternativas mutuamente excluyentes que no siguen un orden específico \parencite{greene_econometric_2018}, estas son la decisión de no migrar, la de migrar hacia la región Centro, la de migrar hacia la región Norte o la de migrar hacia la región Sur.

El Modelo Logit Multinomial es un generalización del modelo Logit Binomial, donde existen más de dos alternativas mutuamente excluyentes. A continuación se realizará el desarrollo del caso general, teniendo en cuenta que el modelo Logit Binomial puede ser considerado un caso particular en el cual $j=2$.

Este modelo se basa en la teoría de la utilidad aleatoria \parencite{domencich_urban_1975} que permite fundamentar teóricamente el modelo de elección discreta de los individuos en el caso de que se enfrente a un conjunto finito de alternativas mutuamente excluyente, como es el caso de la decisión del éxodo.

Según esta teoría la utilidad de los $i$ individuos que se enfrentan a $j$ alternativas estan representadas por una función $U_{ij}$(\ref{eq:utility}). Esta función posee un componente observable $\beta_{j} \textbf{x}_{i}$ y un componente inobservable $\varepsilon_{ij}$. El componente observable está determinado por un vector de factores socio-económicos determinantes de la migración $\textbf{x}_{i}$ y un vector de parámetros $\beta_{j}$, mientras que el componente inobservable sigue una distribución  Gumbel independiente e identicamente distribuida.
\begin {center}
\begin{equation}\label{eq:utility}
U_{ij}=\beta_{j} \textbf{x}_{i}+\varepsilon_{ij}
\end{equation}
\end {center}

Asumiendo que el individuo $i$ es racional y maximiza su utilidad en la elección de las  alternativas, si este selecciona una alternativa $m$ sera tal que la utilidad $U_{im}$ es la máxima entre las $j$ alternativas. Cabe recordad que La utilidad del individuo posee un componente estocástico, por ello es necesaria la introducción de la probabilidad en la ocurrencia de dicha elección, debido a que no todos los componentes son observables. Entonces se puede especificar que  la probabilidad de la selección de una alternativa $m$ para el individuo $i$ es:
\begin {center}
\begin{equation}\label{eq:prob}
P_{im}=P{(Y_{i}=m)} =P(U_{im}>U_{ij}) \ \ \forall \ \ j\neq m
\end{equation}
\end {center}

Como se menciono anteriormente, si se supone que el componente aleatorio $\varepsilon_{ij}$ sigue una una distribución Gumbel independiente e identicamente distribuida, se puede generalizar que la probabilidad de que un individuo $i$ elija una alternativa $j$ de las $n$ alternativas está dada por:
\begin {center}
\begin{equation}\label{eq:multinom}
P{(Y_{i}=j)}=\frac{e^{\ \beta_{j} \textbf{x}_{i}}}{\sum_{k=0}^{n-1}e^{\ \beta_{k} \textbf{x}_{i}}}
\end{equation}
\end {center}

Para evitar el problema de indeterminación en la estimación de los parámetros $\beta_{j}$, se normaliza el modelo representado por la ecuación (\ref{eq:multinom}) asignando el valor de cero para todos los parámetros de la alternativa correspondiente a ``No migrar" ($\beta_{0}=0$). Es así que la estimación de la probabildiad en el modelo logit multinomial será relativa a esta ``alternativa base"  de no migrar \parencite{coxhead_migration_2015}.
Luego de esa normalización, se pueden definir las probabilidades de las $n$ alternativas de la siguiente forma:
\begin {center}
\begin{equation}\label{eq:multinom_2}
P{(Y_{i}=j)}=\frac{e^{\ \beta_{j} \textbf{x}_{i}}}{1+\sum_{k=1}^{n-1}e^{\ \beta_{k} \textbf{x}_{i}}} \ \ \Leftrightarrow \ \ j=1,...,(n-1)
\end{equation}
\begin{equation}\label{eq:multinom_3}
P{(Y_{i}=0)}=\frac{1}{1+\sum_{k=1}^{n-1}e^{\ \beta_{k} \textbf{x}_{i}}} \ \ \Leftrightarrow \ \ j=0
\end{equation}
\end {center}

En donde se tiene que cumplir necesariamente la condición:
\begin {center}
\begin{equation}\label{eq:multinom_4}
\sum_{j=0}^{n-1}P{(Y_{i}=j)}\equiv1
\end{equation}
\end {center}

La ecuación \ref{eq:multinom_3} está definida para la probabilidad de elección de la ``alternativa base" (no migrar), mientras que la ecuación \ref{eq:multinom_2} está definida para las restantes alternativas. A través de estas ecuaciones se puede definir el ratio de probabilidades (\textit{odds}) de que un individuo $i$ elija migrar hacia alguna de las regiones ($j\neq0$) en comparación con la decisión de no migrar ($j=0$), la cual está dada por la siguiente expresión:
\begin{center}
\begin{equation}\label{eq:odds}
\frac{P{(Y_{i}=j)}}{P{(Y_{i}=0)}}=e^{\ \beta_{j} \textbf{x}_{i}} \ \ \forall \ \ j\neq0
\end{equation}
\end {center}
Tomando logaritmos neperianos para ambos términos de la ecuación \ref{eq:odds} se llega a la expresión:
\begin{center}
\begin{equation}\label{eq:logodds}
\log\frac{P{(Y_{i}=j)}}{P{(Y_{i}=0)}}=\ \beta_{j} \textbf{x}_{i} \ \ \forall \ \ j\neq0
\end{equation}
\end {center}

En el primer término de la ecuación \ref{eq:logodds} se encuentra el logaritmo del ratio de probabilidades (\textit{log-odds}) y en el segundo término se encuentra el vector de parametros $\beta_{j}$ acompañado del vector de las características individuales $\textbf{x}_{i}$. 

La lienalidad del segundo término implica que la dirección del impacto al logaritmo del ratio de probabilidades ante  un cambio en las características individuales del individuo estará dado por el signo del parametro $\beta_{j}$ correspondiente a esa característica específica. Si el parámetro  $\beta_{jc}$ de una característica $\textbf{x}_{ic}$ es positivo y ese $\textbf{x}_{ic}$  aumenta su valor, el logaritmo del ratio de probabilidades tambien se verá aumentado, y considerando que el logaritmo es una función monótona estrictamente concava (creciente), se puede demostrar que un aumento del logaritmo del ratio de probabilidades implica un aumento del ratio de probabilidades.

A modo de resumen, el signo del parámetro $\beta_{jc}$ indica la dirección del impacto de un cambio en la característica $\textbf{x}_{ic}$ en la probabilidad de la elección de la alternativa $j$ con respecto a la ``alternativa base''. Para el caso específico tratado en este modelo, será el sentido del impacto del cambio de un determinante en la probabilidad de una persona de migrar hacia la región $j$ con respecto a no migrar.

\subsection{Base de datos y variables}
La totalidad de los datos referidos a los flujos migratorios y a las características socio económicas de los individuos tienen como fuente un pool de cortes transversales  construido a partir de la Encuesta Permanente de Hogares (EPH). Esta encuesta es relevada por el Instituto de Estadísticas y Censos de la República Argentina, en particular se utilizan aquellas comprendidas entre el segundo trimestre del año 2016 y el cuarto trimestre del año 2019.

Para la estimación de los determinantes de la migración se toma como unidad de análisis al Jefe de Hogar, con el fin de condensar la decisión familiar de la migración en un solo sujeto, y reconocer la importancia de los vinculos y las ganancias familiares en las decisiones migratorias de los sujetos.

Para este análisis no se tiene en consideración a los migrantes intraprovinciales ni tampoco a los migrantes internacionales.

Las variables dependientes de los modelos estan representadas por las alternativas que puede tomar la persona con respecto la decisión y el destino del acto migratorio. Estas varían dependiendo del tipo de modelo que se tome a consideración para la estimación:

\textbf{Variable dependiente del Modelo Logit Binomial:}
\begin{itemize}
\item No migrar.
\item Migrar.
\end{itemize}

\textbf{Variable dependiente del Modelo Logit Multinomial:}
\begin{itemize}
\item No migrar.
\item Migrar hacia la Región Sur.
\item Migrar hacia la Región Norte.
\item Migrar hacia la Región Centro.
\end{itemize}

En cuanto a los regresores o variables independientes del modelo se encuentran:
\begin{itemize}
\item Hombre: Esta variable binaria indica el sexo de la persona, toma el valor de uno para los hombres y cero para las mujeres (grupo base).
\item Pobre: Esta variable binaria indica la incidencia en la pobreza de la persona en el momento del relevamiento, toma el valor de uno para las personas pobres y cero para las no pobres (grupo base).
\item Subsidio: Esta variable binaria indica si la persona recibe o no  subsidios o ayudas sociales en dinero al momento del relevamiento, toma el valor de uno para las personas que reciben esta transferencia económica y cero para las que no lo reciben (grupo base).
\item Ocupado baja calificación y ocupado de alta calificación: Estas variables binarias son del tipo categórica y representan a tres categorías, en donde el grupo base esta representado por las personas inactivas o desocupadas, el grupo de ocupados de baja calificación está compuesto por trabajadores con puestos no calificados u operativos y el grupo de ocupados de alta calificación está compuesto por personas que poseen un trabajo con calificación profesional o técnica.
\item Casado: Esta variable binaria indica si la persona se encuentra en pareja (casado o unido) o esta soltera al momento del relevamiento, toma el valor de uno para las personas que estan en pareja y cero para las que estan solteras (grupo base).
\item Hijos: Esta variable binaria indica si la persona tiene hijos a su cargo o no los tiene, toma el valor de uno para las personas que tienen al menos un hijo a su cargo y cero para los que no tienen ningun hijo a su cargo (grupo base). 
%\item Soltero con hijo, casado con hijo y casado sin hijo: Estas variables binarias son del tipo categórica y representan a cuatro categorías, en donde el grupo base esta representado por las personas solteras sin hijos. Otra forma en la que normalmente se presentan las variables categóricas en estos casos es en forma de interacciones de binarias, sin embargo se desistió la utilización de las mismas debido a que presentan conflictos en la interpretación de los parámetros estimados cuando los modelos son no lineales, para un tratamiento completo del tema ver  \textcite{ai_interaction_2003}.
\item Propietario: Esta variable binaria indica la condición de tenencia de la vivienda de la persona de la persona en el momento del relevamiento, toma el valor de uno para las personas que son dueñas de la vivienda que habitan y cero para las que no lo son (grupo base).
\item Edad y Edad$^{2}$: Estas variables continuas indican la edad  y la edad elevada al cuadrado del individuo, la transformación cuadrática se realiza con el fin de captar relaciones no lineales de la variable edad. 
\item Educación media y educación alta: Estas variables binarias son del tipo categórica y representan a tres categorías, en donde el grupo base esta representado por las personas que como mucho poseen un nivel primario de educación, el grupo de personas con educación media está compuesto por aquellas que lograron terminar el secundario como nivel máximo de educación y el grupo con educación alta está compuesto por todos aquellos que lograron un título superior o universitario de educación.
\item Región Centro y Sur: Estas variables binarias son del tipo categórica y representan a tres categorías en donde habitan las personas, en donde el grupo base esta representado por las personas que habitan en la Región Norte.
\end{itemize}

\subsection{Estimación del modelo}
En la estimación de los determinantes de la migración se procedera a un abordaje desde dos niveles de generalidad. En primer lugar se intenta determinar si existen determinantes de la migración que son comunes a todos los migrantes internos, independientemente de su región de destino. En segundo lugar, se intenta probar si estos determinantes presentan heterogeneidades dependiendo de la región elegida como destino de la migración, con el fin de encontrar vinculaciones en los patrones migratorios y las características socioeconómicas de las provincias receptoras del éxodo.

En las siguientes subsecciones se estiman dos modelos de probabilidad no lineal. Para el caso de los determinantes generales de la migración se utiliza el modelo Logit Binomial, mientras que para los determinantes de la migración diferenciados por regiones de destino se utiliza un modelo Logit Multinomial.
\subsubsection{Determinantes generales de la migración}
El resultado de las estimación de los parámetros del modelo se encuentra resumido en el Cuadro \ref{cuadro:estimacion_general} a través de tres submodelos.
Los tres submodelos comparten la misma variable dependiente, se diferencian entre ellos en la cantidad y generalidad de los regresores.

En el \textit{``Modelo A''} se incluyen los determinantes más generales de las migraciones, que son el sexo de la persona, la edad y su organización familiar.
En el \textit{``Modelo B''} se adicionan al modelo general determinantes de la situación económica, patrimonial, laboral y educativa de las personas.
En el \textit{``Modelo C''} se adiciona en última instancia una variable de control relativa a la región donde habita la persona.

Cabe aclarar que en la interpretación de los mismos se utiliza implicitamente o explicitamente la comparación contra el grupo base, que está conformado por la alternativa de no migrar.

\textbf{Resultados}

El sexo y la edad son factores determinantes de la migración interna para los tres submodelos estimados. Ser hombre aumenta la probabilidad de ser migrante y la edad tiene una relación negativa con la probabilidad de ser un migrante, sin embargo, existe una relación cuadrática significativa, lo que genera que cada año que aumente la edad aportará cada vez menos a la disminución de  la probabilidad de ser un migrante interno.

La incidencia en la pobreza es un factor determinante significativo en la migración de las personas, este se mantiene robusto a pesar de agregar los controles regionales en el ``Modelo C''. Las personas pobres tienen menos probabilidad de ser migrantes internos.

La recepción de susidios o ayudas por parte del gobierno, instituciones, iglesias, etc, no tiene un efecto significativo en la probabilidad de ser un migrante interno para ninguno de los submodelos estimados. 

La probabilidad de migrar internamente se reduce en presencia de un trabajo de alta o baja calificación en relación a una persona que está desocupada o inactiva.

Otro factor relevante a la hora de analizar los factores determinantes de la decisión del éxodo es la organización familiar. El hecho de tener hijos disminuye la probabilidad de ser migrante en comparación con una persona sin hijos, este efecto se mantiene significativo y con el mismo signo en los tres submodelos estimados. Estar casado para el ``Modelo A'' pareciera disminuir la probabilidad de migrar, sin embargo, agregando más variables de control en los sucesivos modelos hace que esta variable deje de ser un determinante significativo en la decisión del éxodo.

Las personas que son propietarias de su residencia tienen una menor probabilidad de ser migrantes que  las personas que no lo son.

Por último se analiza el efecto que tiene la educación en la probabilidad de ser migrante. La probabilidad de ser migrante aumenta si la persona posee un nivel educativo medio o alto en relación con una persona con un nivel educativo bajo, este efecto se mantiene significativo para los dos submodelos en donde esta variable se encuentra incluida.

\begin{table}[!htbp] \centering \scriptsize 
  \caption{\\Estimación de los determinantes generales de la migración} 
  \label{cuadro:estimacion_general} 
\begin{tabular}{@{\extracolsep{5pt}}lccc} 
\\[-1.8ex]\hline 
\hline \\[-1.8ex] 
 & \multicolumn{3}{c}{\textit{Variable dependiente:}} \\ 
\cline{2-4} 
\\[-1.8ex] & \multicolumn{3}{c}{Inmigrante} \\ 
\\[-1.8ex] & Modelo A & Modelo B & Modelo C\\ 
\hline \\[-1.8ex] 

Hombre & 0.113$^{***}$ & 0.353$^{***}$ & 0.397$^{***}$ \\ 
  & (0.035) & (0.036) & (0.037) \\ 
  & & & \\ 
 Casado & $-$0.080$^{**}$ & $-$0.026 & $-$0.043 \\ 
  & (0.038) & (0.039) & (0.040) \\ 
  & & & \\ 
 Hijos & $-$0.922$^{***}$ & $-$0.477$^{***}$ & $-$0.509$^{***}$ \\ 
  & (0.038) & (0.042) & (0.042) \\ 
  & & & \\ 
 Edad & $-$0.145$^{***}$ & $-$0.101$^{***}$ & $-$0.108$^{***}$ \\ 
  & (0.006) & (0.007) & (0.007) \\ 
  & & & \\ 
 Edad$^2$ & 0.001$^{***}$ & 0.0004$^{***}$ & 0.001$^{***}$ \\ 
  & (0.0001) & (0.0001) & (0.0001) \\ 
  & & & \\ 
 Pobre &  & $-$0.707$^{***}$ & $-$0.647$^{***}$ \\ 
  &  & (0.037) & (0.038) \\ 
  & & & \\ 
 Subsidios &  & $-$0.043 & $-$0.023 \\ 
  &  & (0.065) & (0.065) \\ 
  & & & \\ 
 Ocupado calif. baja &  & $-$0.918$^{***}$ & $-$0.955$^{***}$ \\ 
  &  & (0.043) & (0.044) \\ 
  & & & \\ 
 Ocupado calif. alta &  & $-$0.667$^{***}$ & $-$0.667$^{***}$ \\ 
  &  & (0.053) & (0.053) \\ 
  & & & \\ 
 Propietario &  & $-$1.746$^{***}$ & $-$1.737$^{***}$ \\ 
  &  & (0.051) & (0.052) \\ 
  & & & \\ 
 Educación media &  & 0.556$^{***}$ & 0.648$^{***}$ \\ 
  &  & (0.047) & (0.047) \\ 
  & & & \\ 
 Educación alta &  & 0.908$^{***}$ & 0.991$^{***}$ \\ 
  &  & (0.057) & (0.058) \\ 
  & & & \\ 
 Región Sur &  &  & 1.026$^{***}$ \\ 
  &  &  & (0.040) \\ 
  & & & \\ 
 Región Centro &  &  & $-$0.494$^{***}$ \\ 
  &  &  & (0.040) \\ 
  & & & \\ 
 Constante & 1.085$^{***}$ & 0.397$^{***}$ & 0.389$^{***}$ \\ 
  & (0.115) & (0.135) & (0.136) \\ 
  & & & \\ 
\hline \\[-1.8ex] 
Observations & 218,834 & 218,834 & 218,834 \\ 
Log Likelihood & $-$17,715.010 & $-$16,408.520 & $-$15,823.660 \\ 
Akaike Inf. Crit. & 35,442.020 & 32,843.050 & 31,677.320 \\ 
\hline 
\hline \\[-1.8ex] 
\textit{Nota:}  & \multicolumn{3}{r}{$^{*}$p$<$0.1; $^{**}$p$<$0.05; $^{***}$p$<$0.01} \\ 
\end{tabular} 
\end{table} 



\subsubsection{Determinantes regionales de la migración}
El resultado de las estimación de los parámetros del modelo se encuentra resumido en el Cuadro \ref{cuadro:estimacion}, en el se puede ver los coeficientes $\beta_{j}$ estimados para cada una de las alternativas del éxodo, cabe aclarar que en la interpretación de los mismos se utiliza implicitamente o explicitamente la comparación contra el grupo base, que está conformado por la alternativa de no migrar, es decir,se entienden de la alternativa de migrar hacia la región $j$ en comparación con la opción de no migrar.

\textbf{Resultados}

El sexo es un factor determinante de la migración hacia las tres regiones, en donde ser hombre aumenta la probabilidad de ser migrante, sin embargo, no es un factor determinante significativo si la migración es hacia la región 3. La edad tiene una relación positiva con la probabilidad de migrar hacia cualquiera de las tres regiones, lo que indica que la probabilidad de migrar aumenta conforme la persona es mayor, sin embargo existe una relación cuadrática significativa, lo que genera que cada año que aumente la edad aportará cada vez menos en el aumento de  la probabilidad de ser migrante hacia cualquiera de las regiones.

La incidencia en la pobreza es un factor determinante significativo en la migración de las personas, las personas pobres tienen menos probabilidad de ser migrantes con destino hacia la región 1 y 2, sin embargo esta relación se invierte si consideramos como destino de la migración a la región 3, en donde ser pobre aumenta la probabilidad de ser un migrante hacia la región 3.

La recepción de susidios o ayudas por parte del gobierno, instituciones, iglesias, etc, aumenta la probabibilidad de ser migrante con destino hacia la región 2 en relación a una persona que no recibe ningun tipo de transferencia económica (grupo base), sin embargo, para los migrantes que tienen como destino las regiones 1 y 3 la recepción de alguna transferencia económica  disminuye la probabilidad de ser migrante.

La probabilidad de migrar hacia cualquiera de las dos regiones de las personas que nacieron en la región 1 se reduce en presencia de un trabajo de alta calificación en relación a una persona que está desocupada o inactiva (en adelante grupo base). Para los migrantes que tienen como destino a la región 2 se encuentra  un efecto con igual signo si la persona posee un trabajo de baja calificación, en donde la probabilidad de migrar se reduce en comparación con el grupo base, lo cual indica que una persona  con un trabajo de baja calificación que nació en la región 1 tiene menos probabilidad de migrar hacia la región 2 que una persona desocupada o inactiva.

La probabilidad de migrar hacia cualquiera de las tres regiones se reduce en presencia de un trabajo de alta o baja calificación en relación a una persona que está desocupada o inactiva (grupo base), en este sentido no existen efectos con signos diferenciados dependiendo de la región de destino.

Otro factor relevante a la hora de analizar los factores determinantes de la decisión del éxodo es la organización familiar, en este caso, ser soltero con hijos disminuye la probabilidad de ser migrante en comparación con una persona soltera sin hijos (en adelante grupo base), independientemente de cual sea su región de destino, efecto similar se puede encontrar en las personas que están en pareja (casados o unidos) y tienen hijos, las cuales tienen una menor probabilidad de ser migrantes en comparación con el grupo base para las tres regiones de destino. Por último, para el caso de las personas que estan en parejas y no tienen hijos la probabilidad de ser migrante aumenta con respecto al grupo base si la región de destino es la 1 o la 2, pero esta  probabilidad  disminuye si la región de destino es la 3.

La condición de ser propietario de la vivienda tiene efectos diferenciados en la probabilidad de que una persona migre hacia alguna de las regiones, mientras ser propietario de la vivienda aumenta la probabilidad de ser un migrante con destino a la región 2 en relación con las personas que no son propietarias de las viviendas (en adelante grupo base), la probabilidad de ser migrante disminuye para una persona que es propietaria de su vivienda en relación al grupo base si la región de destino es la 1 o la 3.

Por último se analiza el efecto que tiene la educación en la probabilidad de ser migrante. La probabilidad de ser migrante aumenta si la persona posee un nivel educativo medio o alto en relación con una persona con un nivel educativo bajo (en adelante grupo base), tanto para los que deciden migrar hacia la región 1 como hacia la region 2, sin embargo, la probabilidad de ser migrante con un nivel educativo medio o alto disminuye en relación al grupo base si al región de destino es la 3.

\begin{table}[!htbp] \centering \footnotesize 
  \caption{Estimación de los determinantes regionales de la migración} 
  \label{cuadro:estimacion_regional} 
\begin{tabular}{@{\extracolsep{5pt}}lccc} 
\\[-1.8ex]\hline 
\hline \\[-1.8ex] 
 & \multicolumn{3}{c}{\textit{Variable dependiente:}} \\ 
\cline{2-4} 
\\[-1.8ex] & Región Sur & Región Norte & Región Centro \\ 
\hline \\[-1.8ex] 
 Hombre & 0.205$^{***}$ & 0.485$^{***}$ & 0.266$^{***}$ \\ 
  & (0.070) & (0.059) & (0.071) \\ 
  & & & \\ 
 Pobre & $-$0.847$^{***}$ & $-$0.683$^{***}$ & $-$0.494$^{***}$ \\ 
  & (0.072) & (0.062) & (0.071) \\ 
  & & & \\ 
 Subsidio & $-$0.034 & 0.191$^{**}$ & 0.007 \\ 
  & (0.065) & (0.095) & (0.032) \\ 
  & & & \\ 
 Ocupado calif. baja & $-$0.450$^{***}$ & $-$1.143$^{***}$ & $-$0.873$^{***}$ \\ 
  & (0.059) & (0.068) & (0.061) \\ 
  & & & \\ 
 Ocupado calif. alta & $-$0.284$^{***}$ & $-$0.856$^{***}$ & $-$0.563$^{***}$ \\ 
  & (0.043) & (0.079) & (0.048) \\ 
  & & & \\ 
 Casado & 0.329$^{***}$ & $-$0.242$^{***}$ & $-$0.197$^{**}$ \\ 
  & (0.079) & (0.066) & (0.083) \\ 
  & & & \\ 
 Hijos & $-$0.324$^{***}$ & $-$0.432$^{***}$ & $-$0.774$^{***}$ \\ 
  & (0.076) & (0.068) & (0.081) \\ 
  & & & \\ 
 Propietario & $-$2.531$^{***}$ & $-$1.339$^{***}$ & $-$1.541$^{***}$ \\ 
  & (0.026) & (0.076) & (0.097) \\ 
  & & & \\ 
 Edad & $-$0.030$^{***}$ & $-$0.079$^{***}$ & $-$0.168$^{***}$ \\ 
  & (0.005) & (0.004) & (0.004) \\ 
  & & & \\ 
 Edad$^2$ & $-$0.0003$^{***}$ & 0.0002$^{***}$ & 0.001$^{***}$ \\ 
  & (0.0001) & (0.0001) & (0.0001) \\ 
  & & & \\ 
 Educación media & 0.094 & 0.929$^{***}$ & 0.569$^{***}$ \\ 
  & (0.066) & (0.047) & (0.053) \\ 
  & & & \\ 
  Educación alta & 0.849$^{***}$ & 1.122$^{***}$ & 0.527$^{***}$ \\ 
  & (0.060) & (0.032) & (0.036) \\ 
  & & & \\ 
 Constante & $-$2.608$^{***}$ & $-$1.338$^{***}$ & 0.356$^{***}$ \\ 
  & (0.006) & (0.010) & (0.009) \\ 
  & & & \\ 
\hline \\[-1.8ex] 
Akaike Inf. Crit. & 34,715.440 & 34,715.440 & 34,715.440 \\ 
\hline 
\hline \\[-1.8ex] 
\textit{Nota:}  & \multicolumn{3}{r}{$^{*}$p$<$0.1; $^{**}$p$<$0.05; $^{***}$p$<$0.01} \\ 
\end{tabular} 
\end{table} 



%%%%
\newpage
\printbibliography[title={Bibliografía}, heading=bibintoc]


\end{document}